%/////////////////////////////////////////////////////////////////////////////
%
% Definition des Dokuments
%
%/////////////////////////////////////////////////////////////////////////////
\documentclass[
11pt,				% Schriftgröße
a4paper,			% Seitengröße
DIV12,		     	% Satzspiegel
liststotoc,			% Inhalsverzeichnis in Inhaltsverzeichnis
bibliography=totoc, % Quellenverzeichnis in Inhaltsverzeichnis
listof=entryprefix, % Abbildungsverzeichnis in Inhaltsverzeichnis
listot=entryprefix, % Tabellenverzeichnis in Inhaltsverzeichnis
appendixprefix=true,
pointlessnumbers,
%	DIV=calc,		% Satzspiegel autom. berechnen, widerspricht sich mit vorherigem
%	twocolumn,		% Zweispaltiger Text
	oneside,		% Doppelseitig, während dem Arbeiten ausschalten, dann besser zum arbeiten, für digitale Variante auch ausschalten, 
%	BROC=10mm		% Bindeausgleich, Achtung links oder rechts anpassen
]{scrbook}

% Zeilenabstand
\usepackage{setspace}
\usepackage{dirtree}
% Absatzeinrückung und -abstand
\parindent 	= 0pt
\parskip 	= 6pt

% Dateicodierung
\usepackage[utf8]{inputenc} 

% Spracheinstellung
\usepackage[ngerman]{babel}
\usepackage[ngerman]{translator}

%Besondere Trennungen
\hyphenation{De-zi-mal-tren-nung}

% Zeichencodierung
\usepackage[T1]{fontenc}
\usepackage{titlesec}

% Schriften einstellen
\usepackage{lmodern}		%Vektorschrift, keine Pixel
\usepackage{microtype}		%Verbessert microtypographie, Verbessert auch warnings 	(overfull boxes)
\renewcommand{\familydefault}{\sfdefault} 	% {1}Standard Schrift, {2}entfernt Serifen, 													Schrift dann ähnlich Calibri
\usepackage{sansmath}		% Mathemodus ohne Serifen
%\sansmath					% Mathemodus aktivieren

\setcounter{secnumdepth}{3}
\setcounter{tocdepth}{3}

% Bilder einbinden
\usepackage{graphicx}
\usepackage{subfigure} 
\graphicspath{{Bilder/}}  % können mehrere Pfade eingebunden werden, von aktuellen Dokument ausgehend


% Dynamisch verlinktes Dokument
\usepackage[%pdfborder={0 0 0 }, 
colorlinks = true,
linkcolor = blue,
pdftitle={Titel der Abschlussarbeit},
pdfsubject={},
pdfauthor={Euer Name},
pdfkeywords={},	
]{hyperref}		%[]Art und Weise für die Anzeige


%------------- Literatur ------------------------------------
\bibliographystyle{unsrt}

\usepackage{color}
\definecolor{middlegray}{rgb}{0.5,0.5,0.5}
\definecolor{lightgray}{rgb}{0.8,0.8,0.8}
\definecolor{orange}{rgb}{0.8,0.3,0.3}
\definecolor{yac}{rgb}{0.6,0.6,0.1}
\usepackage{listings}
 \lstset{
	language=C++,
	basicstyle=\ttfamily,
	keywordstyle=\color{blue}\ttfamily,
	stringstyle=\color{red}\ttfamily,
	commentstyle=\color{green}\ttfamily,
	morecomment=[l][\color{blue}]{\#},
	backgroundcolor=\ttfamily\color{white},
	frame = single 
}

%--------------Große Tabellen---------------------------------------
\usepackage{longtable}
%------------------------Mathe----------------------------
\usepackage{amssymb}
\usepackage{nicefrac}
\usepackage{amsfonts}
\usepackage{amsmath}
%-----------------------------------------------------------


%Befehle von Klaus
\newcommand{\x}{\mathbf{x}}
\newcommand{\y}{\mathbf{y}}
\newcommand{\A}{\mathbf{A}}
\newcommand{\B}{\mathbf{B}}
\newcommand{\C}{\mathbf{C}}
\newcommand{\PP}{\mathbf{P}}
\newcommand{\tp}{^{\mathrm{T}}}
\newcommand{\mat}{\boldsymbol}
\newcommand{\xxi}{\boldsymbol{\xi}}

% entweder erste Spalte fett oder Trennstrich, nicht beides

\usepackage{epsf}


%--------------- Kopf und Fusszeile --------------
\usepackage{scrpage2}
\pagestyle{scrheadings}
\clearscrheadfoot
\ifoot{}
\ofoot{\pagemark}

\automark[section]{chapter}
\ihead{\headmark}

\setheadsepline{0.2pt}
\usepackage{caption}
\usepackage{graphicx}
%\renewcommand{\thechapter}{\arabic{chapter}}
%\renewcommand{\thesection}{\arabic{section}}
%\usepackage{titlesec}
%------------------Hier beginnt das Dokument---------------------------------
%-----------------------------------------------------------------------------
%-----------------------------------------------------------------------------
\begin{document}
	
\chapter*{User Manual-Generic Aircraft Simulation}

\section{Benötigte Software}
\begin{itemize}
	\item Für die Ausführung der Simulation wird Micrsoft Visual Studio 2017 oder neuer benötigt
	\subitem Download: https://visualstudio.microsoft.com/de/downloads/
	\item  Für die Auswertung der Simulation wird mindestens Matlab 2018a oder  mit Python 3.6 benötigt
	\subitem Download: https://de.mathworks.com/downloads/
	\subitem Es wird empfohlen für Python mittels Anaconda zu installieren:
	\subitem Download: https://anaconda.org/
	\item Für die Kompilierung der Code-Dokumentation wird Doxygen benötigt
	\subitem Download: http://www.stack.nl/~dimitri/doxygen/
\end{itemize}
\section{Start-Up und Auswertung}
\begin{itemize}
	\item Im Hauptverzeichnis kann die Solution \textbf{GenericFlightSimulation.sln} in Visual Studio geöffnet werden
	\item Nach dem die Solution in Visual Studio geladen wurde, muss im Projektmappenexplorer \textbf{Executive} als Startprojekt festlegt werden
	\item Der Nutzer kann selbst entscheiden, ob er das Programm als Debug oder Release kompiliert
	\item Nach Beendigung der Simulation können die Ergebnisse visualisiert werden
	\subitem -Matlab: Im Verzeichnis Matlab das File \textbf{Evaluation\_Simulation.m} ausführen
	\subitem -Python: Im Verzeichnis Python das File \textbf{Evaluation.py} ausführen
\end{itemize}
\section{Einstellungen der Simulation}
\begin{itemize}
	\item Im Verzeichnis Input befinden sich alle Dateien, die Parameter für die Simulation bereitstellen
	\item Das File \textbf{Simulation.dat} dient als Steuerung der Simulation
	\subitem -Auswahl der Modelle und der Ausbaustufe
	\item Die erzeugten Output Files befinden sich im Verzeichnis Output
\end{itemize}
\section{Durchführung von Unit- und Modultests}
\begin{itemize}
	\item Unit-Tests: In Visual Studio im Reiter unter Test/Testeinstellung/Standardmäßige Prozessorarchitektur x64 auswählen
	\item Im Reiter Test/Ausführen/ Alle Tests auswählen
	\item Modul-Tests: In Visual Studio im Projektmappenexplorer \textbf{ModuleTests} als Startprojekt festlegen
	\subitem -Als Debug oder Release kompilieren 
	\subitem -Evaluierung der Testergebnisse nur unter Matlab möglich
	\subitem -Im Verzeichnis Matlab das File: \textbf{Evaluation\_ModuleTests} ausführen
\end{itemize} 



\end{document}