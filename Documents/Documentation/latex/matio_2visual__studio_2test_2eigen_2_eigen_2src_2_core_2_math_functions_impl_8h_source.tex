\hypertarget{matio_2visual__studio_2test_2eigen_2_eigen_2src_2_core_2_math_functions_impl_8h_source}{}\section{matio/visual\+\_\+studio/test/eigen/\+Eigen/src/\+Core/\+Math\+Functions\+Impl.h}
\label{matio_2visual__studio_2test_2eigen_2_eigen_2src_2_core_2_math_functions_impl_8h_source}\index{Math\+Functions\+Impl.\+h@{Math\+Functions\+Impl.\+h}}

\begin{DoxyCode}
00001 \textcolor{comment}{// This file is part of Eigen, a lightweight C++ template library}
00002 \textcolor{comment}{// for linear algebra.}
00003 \textcolor{comment}{//}
00004 \textcolor{comment}{// Copyright (C) 2014 Pedro Gonnet (pedro.gonnet@gmail.com)}
00005 \textcolor{comment}{// Copyright (C) 2016 Gael Guennebaud <gael.guennebaud@inria.fr>}
00006 \textcolor{comment}{//}
00007 \textcolor{comment}{// This Source Code Form is subject to the terms of the Mozilla}
00008 \textcolor{comment}{// Public License v. 2.0. If a copy of the MPL was not distributed}
00009 \textcolor{comment}{// with this file, You can obtain one at http://mozilla.org/MPL/2.0/.}
00010 
00011 \textcolor{preprocessor}{#ifndef EIGEN\_MATHFUNCTIONSIMPL\_H}
00012 \textcolor{preprocessor}{#define EIGEN\_MATHFUNCTIONSIMPL\_H}
00013 
00014 \textcolor{keyword}{namespace }\hyperlink{namespace_eigen}{Eigen} \{
00015 
00016 \textcolor{keyword}{namespace }\hyperlink{namespaceinternal}{internal} \{
00017 
00025 \textcolor{keyword}{template}<\textcolor{keyword}{typename} T>
00026 \hyperlink{group___sparse_core___module_class_eigen_1_1_triplet}{T} generic\_fast\_tanh\_float(\textcolor{keyword}{const} \hyperlink{group___sparse_core___module_class_eigen_1_1_triplet}{T}& a\_x)
00027 \{
00028   \textcolor{comment}{// Clamp the inputs to the range [-9, 9] since anything outside}
00029   \textcolor{comment}{// this range is +/-1.0f in single-precision.}
00030   \textcolor{keyword}{const} \hyperlink{group___sparse_core___module_class_eigen_1_1_triplet}{T} plus\_9 = pset1<T>(9.f);
00031   \textcolor{keyword}{const} \hyperlink{group___sparse_core___module_class_eigen_1_1_triplet}{T} minus\_9 = pset1<T>(-9.f);
00032   \textcolor{comment}{// NOTE GCC prior to 6.3 might improperly optimize this max/min}
00033   \textcolor{comment}{//      step such that if a\_x is nan, x will be either 9 or -9,}
00034   \textcolor{comment}{//      and tanh will return 1 or -1 instead of nan.}
00035   \textcolor{comment}{//      This is supposed to be fixed in gcc6.3,}
00036   \textcolor{comment}{//      see: https://gcc.gnu.org/bugzilla/show\_bug.cgi?id=72867}
00037   \textcolor{keyword}{const} \hyperlink{group___sparse_core___module_class_eigen_1_1_triplet}{T} x = pmax(minus\_9,pmin(plus\_9,a\_x));
00038   \textcolor{comment}{// The monomial coefficients of the numerator polynomial (odd).}
00039   \textcolor{keyword}{const} \hyperlink{group___sparse_core___module_class_eigen_1_1_triplet}{T} alpha\_1 = pset1<T>(4.89352455891786e-03f);
00040   \textcolor{keyword}{const} \hyperlink{group___sparse_core___module_class_eigen_1_1_triplet}{T} alpha\_3 = pset1<T>(6.37261928875436e-04f);
00041   \textcolor{keyword}{const} \hyperlink{group___sparse_core___module_class_eigen_1_1_triplet}{T} alpha\_5 = pset1<T>(1.48572235717979e-05f);
00042   \textcolor{keyword}{const} \hyperlink{group___sparse_core___module_class_eigen_1_1_triplet}{T} alpha\_7 = pset1<T>(5.12229709037114e-08f);
00043   \textcolor{keyword}{const} \hyperlink{group___sparse_core___module_class_eigen_1_1_triplet}{T} alpha\_9 = pset1<T>(-8.60467152213735e-11f);
00044   \textcolor{keyword}{const} \hyperlink{group___sparse_core___module_class_eigen_1_1_triplet}{T} alpha\_11 = pset1<T>(2.00018790482477e-13f);
00045   \textcolor{keyword}{const} \hyperlink{group___sparse_core___module_class_eigen_1_1_triplet}{T} alpha\_13 = pset1<T>(-2.76076847742355e-16f);
00046 
00047   \textcolor{comment}{// The monomial coefficients of the denominator polynomial (even).}
00048   \textcolor{keyword}{const} \hyperlink{group___sparse_core___module_class_eigen_1_1_triplet}{T} beta\_0 = pset1<T>(4.89352518554385e-03f);
00049   \textcolor{keyword}{const} \hyperlink{group___sparse_core___module_class_eigen_1_1_triplet}{T} beta\_2 = pset1<T>(2.26843463243900e-03f);
00050   \textcolor{keyword}{const} \hyperlink{group___sparse_core___module_class_eigen_1_1_triplet}{T} beta\_4 = pset1<T>(1.18534705686654e-04f);
00051   \textcolor{keyword}{const} \hyperlink{group___sparse_core___module_class_eigen_1_1_triplet}{T} beta\_6 = pset1<T>(1.19825839466702e-06f);
00052 
00053   \textcolor{comment}{// Since the polynomials are odd/even, we need x^2.}
00054   \textcolor{keyword}{const} \hyperlink{group___sparse_core___module_class_eigen_1_1_triplet}{T} x2 = pmul(x, x);
00055 
00056   \textcolor{comment}{// Evaluate the numerator polynomial p.}
00057   \hyperlink{group___sparse_core___module_class_eigen_1_1_triplet}{T} p = pmadd(x2, alpha\_13, alpha\_11);
00058   p = pmadd(x2, p, alpha\_9);
00059   p = pmadd(x2, p, alpha\_7);
00060   p = pmadd(x2, p, alpha\_5);
00061   p = pmadd(x2, p, alpha\_3);
00062   p = pmadd(x2, p, alpha\_1);
00063   p = pmul(x, p);
00064 
00065   \textcolor{comment}{// Evaluate the denominator polynomial p.}
00066   \hyperlink{group___sparse_core___module_class_eigen_1_1_triplet}{T} q = pmadd(x2, beta\_6, beta\_4);
00067   q = pmadd(x2, q, beta\_2);
00068   q = pmadd(x2, q, beta\_0);
00069 
00070   \textcolor{comment}{// Divide the numerator by the denominator.}
00071   \textcolor{keywordflow}{return} pdiv(p, q);
00072 \}
00073 
00074 \} \textcolor{comment}{// end namespace internal}
00075 
00076 \} \textcolor{comment}{// end namespace Eigen}
00077 
00078 \textcolor{preprocessor}{#endif // EIGEN\_MATHFUNCTIONSIMPL\_H}
\end{DoxyCode}
