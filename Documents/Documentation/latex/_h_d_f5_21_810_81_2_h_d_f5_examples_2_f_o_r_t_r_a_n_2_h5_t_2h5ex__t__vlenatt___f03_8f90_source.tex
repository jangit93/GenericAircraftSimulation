\hypertarget{_h_d_f5_21_810_81_2_h_d_f5_examples_2_f_o_r_t_r_a_n_2_h5_t_2h5ex__t__vlenatt___f03_8f90_source}{}\section{H\+D\+F5/1.10.1/\+H\+D\+F5\+Examples/\+F\+O\+R\+T\+R\+A\+N/\+H5\+T/h5ex\+\_\+t\+\_\+vlenatt\+\_\+\+F03.f90}
\label{_h_d_f5_21_810_81_2_h_d_f5_examples_2_f_o_r_t_r_a_n_2_h5_t_2h5ex__t__vlenatt___f03_8f90_source}\index{h5ex\+\_\+t\+\_\+vlenatt\+\_\+\+F03.\+f90@{h5ex\+\_\+t\+\_\+vlenatt\+\_\+\+F03.\+f90}}

\begin{DoxyCode}
00001 \textcolor{comment}{!************************************************************}
00002 \textcolor{comment}{!}
00003 \textcolor{comment}{!  This example shows how to read and write variable-length}
00004 \textcolor{comment}{!  datatypes to an attribute.  The program first writes two}
00005 \textcolor{comment}{!  variable-length integer arrays to the attribute then}
00006 \textcolor{comment}{!  closes the file.  Next, it reopens the file, reads back}
00007 \textcolor{comment}{!  the data, and outputs it to the screen.}
00008 \textcolor{comment}{!}
00009 \textcolor{comment}{!  This file is intended for use with HDF5 Library version 1.8}
00010 \textcolor{comment}{!  with --enable-fortran2003 }
00011 \textcolor{comment}{!}
00012 \textcolor{comment}{! ************************************************************/}
00013 
00014 \textcolor{keyword}{PROGRAM} main
00015 
00016   \textcolor{keywordtype}{USE }hdf5
00017   \textcolor{keywordtype}{USE }iso\_c\_binding
00018   
00019   \textcolor{keywordtype}{IMPLICIT NONE}
00020 
00021   \textcolor{keywordtype}{CHARACTER(LEN=21)}, \textcolor{keywordtype}{PARAMETER} :: filename  = \textcolor{stringliteral}{"h5ex\_t\_vlenatt\_F03.h5"}
00022   \textcolor{keywordtype}{CHARACTER(LEN=3)} , \textcolor{keywordtype}{PARAMETER} :: dataset   = \textcolor{stringliteral}{"DS1"}
00023   \textcolor{keywordtype}{CHARACTER(LEN=2)} , \textcolor{keywordtype}{PARAMETER} :: attribute = \textcolor{stringliteral}{"A1"}
00024   \textcolor{keywordtype}{INTEGER}, \textcolor{keywordtype}{PARAMETER} :: len0 = 3
00025   \textcolor{keywordtype}{INTEGER}, \textcolor{keywordtype}{PARAMETER} :: len1 = 12
00026 
00027   \textcolor{keywordtype}{INTEGER(HID\_T)}  :: \hyperlink{structfile}{file}, filetype, memtype, space, dset, attr \textcolor{comment}{! Handles}
00028   \textcolor{keywordtype}{INTEGER} :: hdferr
00029   \textcolor{keywordtype}{INTEGER(HSIZE\_T)}, \textcolor{keywordtype}{DIMENSION(1:2)}   :: maxdims
00030   \textcolor{keywordtype}{INTEGER} :: i, j
00031 
00032   \textcolor{comment}{! vl data}
00033   \textcolor{keyword}{TYPE} \hyperlink{structvl}{vl}
00034      \textcolor{keywordtype}{INTEGER}, \textcolor{keywordtype}{DIMENSION(:)}, \textcolor{keywordtype}{POINTER} :: data
00035 \textcolor{keyword}{  END TYPE }\hyperlink{structvl}{vl}
00036   \textcolor{keywordtype}{TYPE}(\hyperlink{structvl}{vl}), \textcolor{keywordtype}{DIMENSION(:)}, \textcolor{keywordtype}{ALLOCATABLE} :: ptr
00037 
00038   \textcolor{keywordtype}{TYPE}(\hyperlink{structhvl__t}{hvl\_t}), \textcolor{keywordtype}{DIMENSION(1:2)}, \textcolor{keywordtype}{TARGET} :: wdata \textcolor{comment}{! Array of vlen structures}
00039   \textcolor{keywordtype}{TYPE}(\hyperlink{structhvl__t}{hvl\_t}), \textcolor{keywordtype}{DIMENSION(1:2)}, \textcolor{keywordtype}{TARGET} :: rdata \textcolor{comment}{! Pointer to vlen structures}
00040 
00041   \textcolor{keywordtype}{INTEGER(hsize\_t)}, \textcolor{keywordtype}{DIMENSION(1:1)} :: dims = (/2/)
00042   \textcolor{keywordtype}{INTEGER}, \textcolor{keywordtype}{DIMENSION(:)}, \textcolor{keywordtype}{POINTER} :: ptr\_r 
00043   \textcolor{keywordtype}{TYPE}(c\_ptr) :: f\_ptr
00044   
00045   \textcolor{comment}{!}
00046   \textcolor{comment}{! Initialize FORTRAN interface.}
00047   \textcolor{comment}{!}
00048   \textcolor{keyword}{CALL }h5open\_f(hdferr)
00049   \textcolor{comment}{!}
00050   \textcolor{comment}{! Initialize variable-length data.  wdata(1) is a countdown of}
00051   \textcolor{comment}{! length LEN0, wdata(2) is a Fibonacci sequence of length LEN1.}
00052   \textcolor{comment}{!}
00053   wdata(1)%len = len0
00054   wdata(2)%len = len1
00055 
00056   \textcolor{keyword}{ALLOCATE}( ptr(1:2) )
00057   \textcolor{keyword}{ALLOCATE}( ptr(1)%data(1:wdata(1)%len) )
00058   \textcolor{keyword}{ALLOCATE}( ptr(2)%data(1:wdata(2)%len) )
00059 
00060   \textcolor{keywordflow}{DO} i=1, wdata(1)%len
00061      ptr(1)%data(i) = wdata(1)%len - i + 1 \textcolor{comment}{! 3 2 1}
00062 \textcolor{keywordflow}{  ENDDO}
00063   wdata(1)%p = c\_loc(ptr(1)%data(1))
00064 
00065   ptr(2)%data(1:2) = 1
00066   \textcolor{keywordflow}{DO} i = 3, wdata(2)%len
00067      ptr(2)%data(i) = ptr(2)%data(i-1) + ptr(2)%data(i-2) \textcolor{comment}{! (1 1 2 3 5 8 etc.)}
00068 \textcolor{keywordflow}{  ENDDO}
00069   wdata(2)%p = c\_loc(ptr(2)%data(1))
00070 
00071   \textcolor{comment}{!}
00072   \textcolor{comment}{! Create a new file using the default properties.}
00073   \textcolor{comment}{!}
00074   \textcolor{keyword}{CALL }h5fcreate\_f(filename, h5f\_acc\_trunc\_f, \hyperlink{structfile}{file}, hdferr)
00075   \textcolor{comment}{!}
00076   \textcolor{comment}{! Create variable-length datatype for file and memory.}
00077   \textcolor{comment}{!}
00078   \textcolor{keyword}{CALL }h5tvlen\_create\_f(h5t\_std\_i32le, filetype, hdferr)
00079   \textcolor{keyword}{CALL }h5tvlen\_create\_f(h5t\_native\_integer, memtype, hdferr)
00080   \textcolor{comment}{!}
00081   \textcolor{comment}{! Create dataset with a null dataspace.}
00082   \textcolor{comment}{!}
00083   \textcolor{keyword}{CALL }h5screate\_f(h5s\_null\_f, space, hdferr)
00084   \textcolor{keyword}{CALL }h5dcreate\_f(\hyperlink{structfile}{file}, dataset, h5t\_std\_i32le, space, dset, hdferr)
00085   \textcolor{keyword}{CALL }h5sclose\_f(space, hdferr)  \textcolor{comment}{!}
00086   \textcolor{comment}{! Create dataspace.}
00087   \textcolor{comment}{!}
00088   \textcolor{keyword}{CALL }h5screate\_simple\_f(1, dims, space, hdferr)
00089 
00090   \textcolor{comment}{!}
00091   \textcolor{comment}{! Create the attribute and write the variable-length data to it}
00092   \textcolor{comment}{!}
00093   \textcolor{keyword}{CALL }h5acreate\_f(dset, attribute, filetype, space, attr, hdferr)
00094 
00095   f\_ptr = c\_loc(wdata(1))
00096   \textcolor{keyword}{CALL }h5awrite\_f(attr, memtype, f\_ptr, hdferr)
00097 
00098   \textcolor{comment}{!}
00099   \textcolor{comment}{! Close and release resources.  Note the use of H5Dvlen\_reclaim}
00100   \textcolor{comment}{! removes the need to manually deallocate the previously allocated}
00101   \textcolor{comment}{! data.}
00102   \textcolor{comment}{!}
00103   \textcolor{keyword}{CALL }h5aclose\_f(attr , hdferr)
00104   \textcolor{keyword}{CALL }h5dclose\_f(dset , hdferr)
00105   \textcolor{keyword}{CALL }h5sclose\_f(space, hdferr)
00106   \textcolor{keyword}{CALL }h5tclose\_f(filetype, hdferr)
00107   \textcolor{keyword}{CALL }h5tclose\_f(memtype, hdferr)
00108   \textcolor{keyword}{CALL }h5fclose\_f(\hyperlink{structfile}{file} , hdferr)
00109   \textcolor{keyword}{DEALLOCATE}(ptr)
00110   \textcolor{comment}{!}
00111   \textcolor{comment}{! Now we begin the read section of this example.}
00112 
00113   \textcolor{comment}{!}
00114   \textcolor{comment}{! Open file and dataset, and attribute.}
00115   \textcolor{comment}{!}
00116   \textcolor{keyword}{CALL }h5fopen\_f(filename, h5f\_acc\_rdonly\_f, \hyperlink{structfile}{file}, hdferr)
00117   \textcolor{keyword}{CALL }h5dopen\_f(\hyperlink{structfile}{file}, dataset, dset, hdferr)
00118   \textcolor{keyword}{CALL }h5aopen\_f(dset, attribute, attr, hdferr)
00119   \textcolor{comment}{!}
00120   \textcolor{comment}{! Get dataspace and allocate memory for array of vlen structures.}
00121   \textcolor{comment}{! This does not actually allocate memory for the vlen data, that}
00122   \textcolor{comment}{! will be done by the library.}
00123   \textcolor{comment}{!}
00124   \textcolor{keyword}{CALL }h5aget\_space\_f(attr, space, hdferr)
00125   \textcolor{keyword}{CALL }h5sget\_simple\_extent\_dims\_f(space, dims, maxdims, hdferr)
00126   \textcolor{comment}{!}
00127   \textcolor{comment}{! Create the memory datatype.}
00128   \textcolor{comment}{!}
00129   \textcolor{keyword}{CALL }h5tvlen\_create\_f(h5t\_native\_integer, memtype, hdferr)
00130   \textcolor{comment}{!}
00131   \textcolor{comment}{! Read the data.}
00132   \textcolor{comment}{!}
00133   f\_ptr = c\_loc(rdata(1))
00134   \textcolor{keyword}{CALL }h5aread\_f(attr, memtype, f\_ptr, hdferr)
00135   \textcolor{comment}{!}
00136   \textcolor{comment}{! Output the variable-length data to the screen.}
00137   \textcolor{comment}{!}
00138   \textcolor{keywordflow}{DO} i = 1, dims(1)
00139      \textcolor{keyword}{WRITE}(*,\textcolor{stringliteral}{'(A,"(",I0,"):",/,"\{")'}, advance=\textcolor{stringliteral}{"no"}) attribute,i
00140      \textcolor{keyword}{CALL }c\_f\_pointer(rdata(i)%p, ptr\_r, [rdata(i)%len] )
00141      \textcolor{keywordflow}{DO} j = 1, rdata(i)%len
00142         \textcolor{keyword}{WRITE}(*,\textcolor{stringliteral}{'(1X,I0)'}, advance=\textcolor{stringliteral}{'no'}) ptr\_r(j)
00143         \textcolor{keywordflow}{IF} ( j .LT. rdata(i)%len) \textcolor{keyword}{WRITE}(*,\textcolor{stringliteral}{'(",")'}, advance=\textcolor{stringliteral}{'no'})
00144 \textcolor{keywordflow}{     ENDDO}
00145      \textcolor{keyword}{WRITE}(*,\textcolor{stringliteral}{'( " \}")'})
00146 \textcolor{keywordflow}{  ENDDO}
00147   \textcolor{comment}{!}
00148   \textcolor{comment}{! Close and release resources.}
00149   \textcolor{comment}{!}
00150   \textcolor{keyword}{CALL }h5dvlen\_reclaim\_f(memtype, space, h5p\_default\_f, f\_ptr, hdferr)
00151   \textcolor{keyword}{CALL }h5aclose\_f(attr , hdferr)
00152   \textcolor{keyword}{CALL }h5dclose\_f(dset , hdferr)
00153   \textcolor{keyword}{CALL }h5sclose\_f(space, hdferr)
00154   \textcolor{keyword}{CALL }h5tclose\_f(memtype, hdferr)
00155   \textcolor{keyword}{CALL }h5fclose\_f(\hyperlink{structfile}{file} , hdferr)
00156 
00157 \textcolor{keyword}{END PROGRAM }main
\end{DoxyCode}
