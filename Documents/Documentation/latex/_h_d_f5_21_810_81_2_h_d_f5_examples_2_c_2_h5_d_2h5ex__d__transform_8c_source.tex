\hypertarget{_h_d_f5_21_810_81_2_h_d_f5_examples_2_c_2_h5_d_2h5ex__d__transform_8c_source}{}\section{H\+D\+F5/1.10.1/\+H\+D\+F5\+Examples/\+C/\+H5\+D/h5ex\+\_\+d\+\_\+transform.c}
\label{_h_d_f5_21_810_81_2_h_d_f5_examples_2_c_2_h5_d_2h5ex__d__transform_8c_source}\index{h5ex\+\_\+d\+\_\+transform.\+c@{h5ex\+\_\+d\+\_\+transform.\+c}}

\begin{DoxyCode}
00001 \textcolor{comment}{/************************************************************}
00002 \textcolor{comment}{}
00003 \textcolor{comment}{  This example shows how to read and write data to a dataset}
00004 \textcolor{comment}{  using a data transform expression.  The program first}
00005 \textcolor{comment}{  writes integers to a dataset using the transform}
00006 \textcolor{comment}{  expression TRANSFORM, then closes the file.  Next, it}
00007 \textcolor{comment}{  reopens the file, reads back the data without a transform,}
00008 \textcolor{comment}{  and outputs the data to the screen.  Finally it reads the}
00009 \textcolor{comment}{  data using the transform expression RTRANSFORM and outputs}
00010 \textcolor{comment}{  the results to the screen.}
00011 \textcolor{comment}{}
00012 \textcolor{comment}{  This file is intended for use with HDF5 Library version 1.8}
00013 \textcolor{comment}{}
00014 \textcolor{comment}{ ************************************************************/}
00015 
00016 \textcolor{preprocessor}{#include "hdf5.h"}
00017 \textcolor{preprocessor}{#include <stdio.h>}
00018 \textcolor{preprocessor}{#include <stdlib.h>}
00019 
00020 \textcolor{preprocessor}{#define FILE            "h5ex\_d\_transform.h5"}
00021 \textcolor{preprocessor}{#define DATASET         "DS1"}
00022 \textcolor{preprocessor}{#define DIM0            4}
00023 \textcolor{preprocessor}{#define DIM1            7}
00024 \textcolor{preprocessor}{#define TRANSFORM       "x+1"}
00025 \textcolor{preprocessor}{#define RTRANSFORM      "x-1"}
00026 
00027 \textcolor{keywordtype}{int}
00028 main (\textcolor{keywordtype}{void})
00029 \{
00030     hid\_t           \hyperlink{structfile}{file}, space, dset, dxpl;
00031                                                 \textcolor{comment}{/* Handles */}
00032     herr\_t          status;
00033     hsize\_t         dims[2] = \{DIM0, DIM1\};
00034     \textcolor{keywordtype}{int}             wdata[DIM0][DIM1],          \textcolor{comment}{/* Write buffer */}
00035                     rdata[DIM0][DIM1],          \textcolor{comment}{/* Read buffer */}
00036                     i, j;
00037 
00038     \textcolor{comment}{/*}
00039 \textcolor{comment}{     * Initialize data.}
00040 \textcolor{comment}{     */}
00041     \textcolor{keywordflow}{for} (i=0; i<DIM0; i++)
00042         \textcolor{keywordflow}{for} (j=0; j<DIM1; j++)
00043             wdata[i][j] = i * j - j;
00044 
00045     \textcolor{comment}{/*}
00046 \textcolor{comment}{     * Output the data to the screen.}
00047 \textcolor{comment}{     */}
00048     printf (\textcolor{stringliteral}{"Original Data:\(\backslash\)n"});
00049     \textcolor{keywordflow}{for} (i=0; i<DIM0; i++) \{
00050         printf (\textcolor{stringliteral}{" ["});
00051         \textcolor{keywordflow}{for} (j=0; j<DIM1; j++)
00052             printf (\textcolor{stringliteral}{" %3d"}, wdata[i][j]);
00053         printf (\textcolor{stringliteral}{"]\(\backslash\)n"});
00054     \}
00055 
00056     \textcolor{comment}{/*}
00057 \textcolor{comment}{     * Create a new file using the default properties.}
00058 \textcolor{comment}{     */}
00059     file = H5Fcreate (FILE, H5F\_ACC\_TRUNC, H5P\_DEFAULT, H5P\_DEFAULT);
00060 
00061     \textcolor{comment}{/*}
00062 \textcolor{comment}{     * Create dataspace.  Setting maximum size to NULL sets the maximum}
00063 \textcolor{comment}{     * size to be the current size.}
00064 \textcolor{comment}{     */}
00065     space = H5Screate\_simple (2, dims, NULL);
00066 
00067     \textcolor{comment}{/*}
00068 \textcolor{comment}{     * Create the dataset transfer property list and define the}
00069 \textcolor{comment}{     * transform expression.}
00070 \textcolor{comment}{     */}
00071     dxpl = H5Pcreate (H5P\_DATASET\_XFER);
00072     status = H5Pset\_data\_transform (dxpl, TRANSFORM);
00073 
00074     \textcolor{comment}{/*}
00075 \textcolor{comment}{     * Create the dataset using the default properties.  Unfortunately}
00076 \textcolor{comment}{     * we must save as a native type or the transform operation will}
00077 \textcolor{comment}{     * fail.}
00078 \textcolor{comment}{     */}
00079     dset = H5Dcreate (file, DATASET, H5T\_NATIVE\_INT, space, H5P\_DEFAULT,
00080                 H5P\_DEFAULT, H5P\_DEFAULT);
00081 
00082     \textcolor{comment}{/*}
00083 \textcolor{comment}{     * Write the data to the dataset using the dataset transfer}
00084 \textcolor{comment}{     * property list.}
00085 \textcolor{comment}{     */}
00086     status = H5Dwrite (dset, H5T\_NATIVE\_INT, H5S\_ALL, H5S\_ALL, dxpl, wdata[0]);
00087 
00088     \textcolor{comment}{/*}
00089 \textcolor{comment}{     * Close and release resources.}
00090 \textcolor{comment}{     */}
00091     status = H5Pclose (dxpl);
00092     status = H5Dclose (dset);
00093     status = H5Sclose (space);
00094     status = H5Fclose (file);
00095 
00096 
00097     \textcolor{comment}{/*}
00098 \textcolor{comment}{     * Now we begin the read section of this example.}
00099 \textcolor{comment}{     */}
00100 
00101     \textcolor{comment}{/*}
00102 \textcolor{comment}{     * Open file and dataset using the default properties.}
00103 \textcolor{comment}{     */}
00104     file = H5Fopen (FILE, H5F\_ACC\_RDONLY, H5P\_DEFAULT);
00105     dset = H5Dopen (file, DATASET, H5P\_DEFAULT);
00106 
00107     \textcolor{comment}{/*}
00108 \textcolor{comment}{     * Read the data using the default properties.}
00109 \textcolor{comment}{     */}
00110     status = H5Dread (dset, H5T\_NATIVE\_INT, H5S\_ALL, H5S\_ALL, H5P\_DEFAULT,
00111                 rdata[0]);
00112 
00113     \textcolor{comment}{/*}
00114 \textcolor{comment}{     * Output the data to the screen.}
00115 \textcolor{comment}{     */}
00116     printf (\textcolor{stringliteral}{"\(\backslash\)nData as written with transform \(\backslash\)"%s\(\backslash\)":\(\backslash\)n"}, TRANSFORM);
00117     \textcolor{keywordflow}{for} (i=0; i<DIM0; i++) \{
00118         printf (\textcolor{stringliteral}{" ["});
00119         \textcolor{keywordflow}{for} (j=0; j<DIM1; j++)
00120             printf (\textcolor{stringliteral}{" %3d"}, rdata[i][j]);
00121         printf (\textcolor{stringliteral}{"]\(\backslash\)n"});
00122     \}
00123 
00124     \textcolor{comment}{/*}
00125 \textcolor{comment}{     * Create the dataset transfer property list and define the}
00126 \textcolor{comment}{     * transform expression.}
00127 \textcolor{comment}{     */}
00128     dxpl = H5Pcreate (H5P\_DATASET\_XFER);
00129     status = H5Pset\_data\_transform (dxpl, RTRANSFORM);
00130 
00131     \textcolor{comment}{/*}
00132 \textcolor{comment}{     * Read the data using the dataset transfer property list.}
00133 \textcolor{comment}{     */}
00134     status = H5Dread (dset, H5T\_NATIVE\_INT, H5S\_ALL, H5S\_ALL, dxpl, rdata[0]);
00135 
00136     \textcolor{comment}{/*}
00137 \textcolor{comment}{     * Output the data to the screen.}
00138 \textcolor{comment}{     */}
00139     printf (\textcolor{stringliteral}{"\(\backslash\)nData as written with transform \(\backslash\)"%s\(\backslash\)" and read with transform \(\backslash\)"%s\(\backslash\)":\(\backslash\)n"},
00140                 TRANSFORM, RTRANSFORM);
00141     \textcolor{keywordflow}{for} (i=0; i<DIM0; i++) \{
00142         printf (\textcolor{stringliteral}{" ["});
00143         \textcolor{keywordflow}{for} (j=0; j<DIM1; j++)
00144             printf (\textcolor{stringliteral}{" %3d"}, rdata[i][j]);
00145         printf (\textcolor{stringliteral}{"]\(\backslash\)n"});
00146     \}
00147 
00148     \textcolor{comment}{/*}
00149 \textcolor{comment}{     * Close and release resources.}
00150 \textcolor{comment}{     */}
00151     status = H5Pclose (dxpl);
00152     status = H5Dclose (dset);
00153     status = H5Fclose (file);
00154 
00155     \textcolor{keywordflow}{return} 0;
00156 \}
\end{DoxyCode}
