\hypertarget{_h_d_f5_21_810_81_2_h_d_f5_examples_2_c_2_h5_t_2h5ex__t__float_8c_source}{}\section{H\+D\+F5/1.10.1/\+H\+D\+F5\+Examples/\+C/\+H5\+T/h5ex\+\_\+t\+\_\+float.c}
\label{_h_d_f5_21_810_81_2_h_d_f5_examples_2_c_2_h5_t_2h5ex__t__float_8c_source}\index{h5ex\+\_\+t\+\_\+float.\+c@{h5ex\+\_\+t\+\_\+float.\+c}}

\begin{DoxyCode}
00001 \textcolor{comment}{/************************************************************}
00002 \textcolor{comment}{}
00003 \textcolor{comment}{  This example shows how to read and write integer datatypes}
00004 \textcolor{comment}{  to a dataset.  The program first writes integers to a}
00005 \textcolor{comment}{  dataset with a dataspace of DIM0xDIM1, then closes the}
00006 \textcolor{comment}{  file.  Next, it reopens the file, reads back the data, and}
00007 \textcolor{comment}{  outputs it to the screen.}
00008 \textcolor{comment}{}
00009 \textcolor{comment}{  This file is intended for use with HDF5 Library version 1.8}
00010 \textcolor{comment}{}
00011 \textcolor{comment}{ ************************************************************/}
00012 
00013 \textcolor{preprocessor}{#include "hdf5.h"}
00014 \textcolor{preprocessor}{#include <stdio.h>}
00015 \textcolor{preprocessor}{#include <stdlib.h>}
00016 
00017 \textcolor{preprocessor}{#define FILE            "h5ex\_t\_float.h5"}
00018 \textcolor{preprocessor}{#define DATASET         "DS1"}
00019 \textcolor{preprocessor}{#define DIM0            4}
00020 \textcolor{preprocessor}{#define DIM1            7}
00021 
00022 \textcolor{keywordtype}{int}
00023 main (\textcolor{keywordtype}{void})
00024 \{
00025     hid\_t       \hyperlink{structfile}{file}, space, dset;          \textcolor{comment}{/* Handles */}
00026     herr\_t      status;
00027     hsize\_t     dims[2] = \{DIM0, DIM1\};
00028     \textcolor{keywordtype}{double}      wdata[DIM0][DIM1],          \textcolor{comment}{/* Write buffer */}
00029                 **rdata;                    \textcolor{comment}{/* Read buffer */}
00030     \textcolor{keywordtype}{int}         ndims,
00031                 i, j;
00032 
00033     \textcolor{comment}{/*}
00034 \textcolor{comment}{     * Initialize data.}
00035 \textcolor{comment}{     */}
00036     \textcolor{keywordflow}{for} (i=0; i<DIM0; i++)
00037         \textcolor{keywordflow}{for} (j=0; j<DIM1; j++)
00038             wdata[i][j] = (\textcolor{keywordtype}{double}) i / (j + 0.5) + j;
00039 
00040     \textcolor{comment}{/*}
00041 \textcolor{comment}{     * Create a new file using the default properties.}
00042 \textcolor{comment}{     */}
00043     file = H5Fcreate (FILE, H5F\_ACC\_TRUNC, H5P\_DEFAULT, H5P\_DEFAULT);
00044 
00045     \textcolor{comment}{/*}
00046 \textcolor{comment}{     * Create dataspace.  Setting maximum size to NULL sets the maximum}
00047 \textcolor{comment}{     * size to be the current size.}
00048 \textcolor{comment}{     */}
00049     space = H5Screate\_simple (2, dims, NULL);
00050 
00051     \textcolor{comment}{/*}
00052 \textcolor{comment}{     * Create the dataset and write the floating point data to it.  In}
00053 \textcolor{comment}{     * this example we will save the data as 64 bit little endian IEEE}
00054 \textcolor{comment}{     * floating point numbers, regardless of the native type.  The HDF5}
00055 \textcolor{comment}{     * library automatically converts between different floating point}
00056 \textcolor{comment}{     * types.}
00057 \textcolor{comment}{     */}
00058     dset = H5Dcreate (file, DATASET, H5T\_IEEE\_F64LE, space, H5P\_DEFAULT,
00059                 H5P\_DEFAULT, H5P\_DEFAULT);
00060     status = H5Dwrite (dset, H5T\_NATIVE\_DOUBLE, H5S\_ALL, H5S\_ALL, H5P\_DEFAULT,
00061                 wdata[0]);
00062 
00063     \textcolor{comment}{/*}
00064 \textcolor{comment}{     * Close and release resources.}
00065 \textcolor{comment}{     */}
00066     status = H5Dclose (dset);
00067     status = H5Sclose (space);
00068     status = H5Fclose (file);
00069 
00070 
00071     \textcolor{comment}{/*}
00072 \textcolor{comment}{     * Now we begin the read section of this example.  Here we assume}
00073 \textcolor{comment}{     * the dataset has the same name and rank, but can have any size.}
00074 \textcolor{comment}{     * Therefore we must allocate a new array to read in data using}
00075 \textcolor{comment}{     * malloc().}
00076 \textcolor{comment}{     */}
00077 
00078     \textcolor{comment}{/*}
00079 \textcolor{comment}{     * Open file and dataset.}
00080 \textcolor{comment}{     */}
00081     file = H5Fopen (FILE, H5F\_ACC\_RDONLY, H5P\_DEFAULT);
00082     dset = H5Dopen (file, DATASET, H5P\_DEFAULT);
00083 
00084     \textcolor{comment}{/*}
00085 \textcolor{comment}{     * Get dataspace and allocate memory for read buffer.  This is a}
00086 \textcolor{comment}{     * two dimensional dataset so the dynamic allocation must be done}
00087 \textcolor{comment}{     * in steps.}
00088 \textcolor{comment}{     */}
00089     space = H5Dget\_space (dset);
00090     ndims = H5Sget\_simple\_extent\_dims (space, dims, NULL);
00091 
00092     \textcolor{comment}{/*}
00093 \textcolor{comment}{     * Allocate array of pointers to rows.}
00094 \textcolor{comment}{     */}
00095     rdata = (\textcolor{keywordtype}{double} **) malloc (dims[0] * \textcolor{keyword}{sizeof} (\textcolor{keywordtype}{double} *));
00096 
00097     \textcolor{comment}{/*}
00098 \textcolor{comment}{     * Allocate space for floating point data.}
00099 \textcolor{comment}{     */}
00100     rdata[0] = (\textcolor{keywordtype}{double} *) malloc (dims[0] * dims[1] * \textcolor{keyword}{sizeof} (\textcolor{keywordtype}{double}));
00101 
00102     \textcolor{comment}{/*}
00103 \textcolor{comment}{     * Set the rest of the pointers to rows to the correct addresses.}
00104 \textcolor{comment}{     */}
00105     \textcolor{keywordflow}{for} (i=1; i<dims[0]; i++)
00106         rdata[i] = rdata[0] + i * dims[1];
00107 
00108     \textcolor{comment}{/*}
00109 \textcolor{comment}{     * Read the data.}
00110 \textcolor{comment}{     */}
00111     status = H5Dread (dset, H5T\_NATIVE\_DOUBLE, H5S\_ALL, H5S\_ALL, H5P\_DEFAULT,
00112                 rdata[0]);
00113 
00114     \textcolor{comment}{/*}
00115 \textcolor{comment}{     * Output the data to the screen.}
00116 \textcolor{comment}{     */}
00117     printf (\textcolor{stringliteral}{"%s:\(\backslash\)n"}, DATASET);
00118     \textcolor{keywordflow}{for} (i=0; i<dims[0]; i++) \{
00119         printf (\textcolor{stringliteral}{" ["});
00120         \textcolor{keywordflow}{for} (j=0; j<dims[1]; j++)
00121             printf (\textcolor{stringliteral}{" %6.4f"}, rdata[i][j]);
00122         printf (\textcolor{stringliteral}{"]\(\backslash\)n"});
00123     \}
00124 
00125     \textcolor{comment}{/*}
00126 \textcolor{comment}{     * Close and release resources.}
00127 \textcolor{comment}{     */}
00128     free (rdata[0]);
00129     free (rdata);
00130     status = H5Dclose (dset);
00131     status = H5Sclose (space);
00132     status = H5Fclose (file);
00133 
00134     \textcolor{keywordflow}{return} 0;
00135 \}
\end{DoxyCode}
