\hypertarget{eigen_2lapack_2slamch_8f_source}{}\section{eigen/lapack/slamch.f}
\label{eigen_2lapack_2slamch_8f_source}\index{slamch.\+f@{slamch.\+f}}

\begin{DoxyCode}
00001 \textcolor{comment}{*> \(\backslash\)brief \(\backslash\)b SLAMCH}
00002 \textcolor{comment}{*}
00003 \textcolor{comment}{*  =========== DOCUMENTATION ===========}
00004 \textcolor{comment}{*}
00005 \textcolor{comment}{* Online html documentation available at }
00006 \textcolor{comment}{*            http://www.netlib.org/lapack/explore-html/ }
00007 \textcolor{comment}{*}
00008 \textcolor{comment}{*  Definition:}
00009 \textcolor{comment}{*  ===========}
00010 \textcolor{comment}{*}
00011 \textcolor{comment}{*      REAL             FUNCTION SLAMCH( CMACH )}
00012 \textcolor{comment}{*}
00013 \textcolor{comment}{*     .. Scalar Arguments ..}
00014 \textcolor{comment}{*      CHARACTER          CMACH}
00015 \textcolor{comment}{*     ..}
00016 \textcolor{comment}{*  }
00017 \textcolor{comment}{*}
00018 \textcolor{comment}{*> \(\backslash\)par Purpose:}
00019 \textcolor{comment}{*  =============}
00020 \textcolor{comment}{*>}
00021 \textcolor{comment}{*> \(\backslash\)verbatim}
00022 \textcolor{comment}{*>}
00023 \textcolor{comment}{*> SLAMCH determines single precision machine parameters.}
00024 \textcolor{comment}{*> \(\backslash\)endverbatim}
00025 \textcolor{comment}{*}
00026 \textcolor{comment}{*  Arguments:}
00027 \textcolor{comment}{*  ==========}
00028 \textcolor{comment}{*}
00029 \textcolor{comment}{*> \(\backslash\)param[in] CMACH}
00030 \textcolor{comment}{*> \(\backslash\)verbatim}
00031 \textcolor{comment}{*>          Specifies the value to be returned by SLAMCH:}
00032 \textcolor{comment}{*>          = 'E' or 'e',   SLAMCH := eps}
00033 \textcolor{comment}{*>          = 'S' or 's ,   SLAMCH := sfmin}
00034 \textcolor{comment}{*>          = 'B' or 'b',   SLAMCH := base}
00035 \textcolor{comment}{*>          = 'P' or 'p',   SLAMCH := eps*base}
00036 \textcolor{comment}{*>          = 'N' or 'n',   SLAMCH := t}
00037 \textcolor{comment}{*>          = 'R' or 'r',   SLAMCH := rnd}
00038 \textcolor{comment}{*>          = 'M' or 'm',   SLAMCH := emin}
00039 \textcolor{comment}{*>          = 'U' or 'u',   SLAMCH := rmin}
00040 \textcolor{comment}{*>          = 'L' or 'l',   SLAMCH := emax}
00041 \textcolor{comment}{*>          = 'O' or 'o',   SLAMCH := rmax}
00042 \textcolor{comment}{*>          where}
00043 \textcolor{comment}{*>          eps   = relative machine precision}
00044 \textcolor{comment}{*>          sfmin = safe minimum, such that 1/sfmin does not overflow}
00045 \textcolor{comment}{*>          base  = base of the machine}
00046 \textcolor{comment}{*>          prec  = eps*base}
00047 \textcolor{comment}{*>          t     = number of (base) digits in the mantissa}
00048 \textcolor{comment}{*>          rnd   = 1.0 when rounding occurs in addition, 0.0 otherwise}
00049 \textcolor{comment}{*>          emin  = minimum exponent before (gradual) underflow}
00050 \textcolor{comment}{*>          rmin  = underflow threshold - base**(emin-1)}
00051 \textcolor{comment}{*>          emax  = largest exponent before overflow}
00052 \textcolor{comment}{*>          rmax  = overflow threshold  - (base**emax)*(1-eps)}
00053 \textcolor{comment}{*> \(\backslash\)endverbatim}
00054 \textcolor{comment}{*}
00055 \textcolor{comment}{*  Authors:}
00056 \textcolor{comment}{*  ========}
00057 \textcolor{comment}{*}
00058 \textcolor{comment}{*> \(\backslash\)author Univ. of Tennessee }
00059 \textcolor{comment}{*> \(\backslash\)author Univ. of California Berkeley }
00060 \textcolor{comment}{*> \(\backslash\)author Univ. of Colorado Denver }
00061 \textcolor{comment}{*> \(\backslash\)author NAG Ltd. }
00062 \textcolor{comment}{*}
00063 \textcolor{comment}{*> \(\backslash\)date November 2011}
00064 \textcolor{comment}{*}
00065 \textcolor{comment}{*> \(\backslash\)ingroup auxOTHERauxiliary}
00066 \textcolor{comment}{*}
00067 \textcolor{comment}{*  =====================================================================}
00068 \textcolor{keyword}{      REAL             }\textcolor{keyword}{FUNCTION }slamch( CMACH )
00069 \textcolor{comment}{*}
00070 \textcolor{comment}{*  -- LAPACK auxiliary routine (version 3.4.0) --}
00071 \textcolor{comment}{*  -- LAPACK is a software package provided by Univ. of Tennessee,    --}
00072 \textcolor{comment}{*  -- Univ. of California Berkeley, Univ. of Colorado Denver and NAG Ltd..--}
00073 \textcolor{comment}{*     November 2011}
00074 \textcolor{comment}{*}
00075 \textcolor{comment}{*     .. Scalar Arguments ..}
00076       \textcolor{keywordtype}{CHARACTER}          cmach
00077 \textcolor{comment}{*     ..}
00078 \textcolor{comment}{*}
00079 \textcolor{comment}{* =====================================================================}
00080 \textcolor{comment}{*}
00081 \textcolor{comment}{*     .. Parameters ..}
00082       \textcolor{keywordtype}{REAL}               one, zero
00083       parameter( one = 1.0e+0, zero = 0.0e+0 )
00084 \textcolor{comment}{*     ..}
00085 \textcolor{comment}{*     .. Local Scalars ..}
00086       \textcolor{keywordtype}{REAL}               rnd, eps, sfmin, small, rmach
00087 \textcolor{comment}{*     ..}
00088 \textcolor{comment}{*     .. External Functions ..}
00089       \textcolor{keywordtype}{LOGICAL}            lsame
00090       \textcolor{keywordtype}{EXTERNAL}           lsame
00091 \textcolor{comment}{*     ..}
00092 \textcolor{comment}{*     .. Intrinsic Functions ..}
00093       \textcolor{keywordtype}{INTRINSIC}          digits, epsilon, huge, maxexponent,
00094      $                   minexponent, radix, tiny
00095 \textcolor{comment}{*     ..}
00096 \textcolor{comment}{*     .. Executable Statements ..}
00097 \textcolor{comment}{*}
00098 \textcolor{comment}{*}
00099 \textcolor{comment}{*     Assume rounding, not chopping. Always.}
00100 \textcolor{comment}{*}
00101       rnd = one
00102 \textcolor{comment}{*}
00103       \textcolor{keywordflow}{IF}( one.EQ.rnd ) \textcolor{keywordflow}{THEN}
00104          eps = epsilon(zero) * 0.5
00105       \textcolor{keywordflow}{ELSE}
00106          eps = epsilon(zero)
00107 \textcolor{keywordflow}{      END IF}
00108 \textcolor{comment}{*}
00109       \textcolor{keywordflow}{IF}( lsame( cmach, \textcolor{stringliteral}{'E'} ) ) \textcolor{keywordflow}{THEN}
00110          rmach = eps
00111       \textcolor{keywordflow}{ELSE} \textcolor{keywordflow}{IF}( lsame( cmach, \textcolor{stringliteral}{'S'} ) ) \textcolor{keywordflow}{THEN}
00112          sfmin = tiny(zero)
00113          small = one / huge(zero)
00114          \textcolor{keywordflow}{IF}( small.GE.sfmin ) \textcolor{keywordflow}{THEN}
00115 \textcolor{comment}{*}
00116 \textcolor{comment}{*           Use SMALL plus a bit, to avoid the possibility of rounding}
00117 \textcolor{comment}{*           causing overflow when computing  1/sfmin.}
00118 \textcolor{comment}{*}
00119             sfmin = small*( one+eps )
00120 \textcolor{keywordflow}{         END IF}
00121          rmach = sfmin
00122       \textcolor{keywordflow}{ELSE} \textcolor{keywordflow}{IF}( lsame( cmach, \textcolor{stringliteral}{'B'} ) ) \textcolor{keywordflow}{THEN}
00123          rmach = radix(zero)
00124       \textcolor{keywordflow}{ELSE} \textcolor{keywordflow}{IF}( lsame( cmach, \textcolor{stringliteral}{'P'} ) ) \textcolor{keywordflow}{THEN}
00125          rmach = eps * radix(zero)
00126       \textcolor{keywordflow}{ELSE} \textcolor{keywordflow}{IF}( lsame( cmach, \textcolor{stringliteral}{'N'} ) ) \textcolor{keywordflow}{THEN}
00127          rmach = digits(zero)
00128       \textcolor{keywordflow}{ELSE} \textcolor{keywordflow}{IF}( lsame( cmach, \textcolor{stringliteral}{'R'} ) ) \textcolor{keywordflow}{THEN}
00129          rmach = rnd
00130       \textcolor{keywordflow}{ELSE} \textcolor{keywordflow}{IF}( lsame( cmach, \textcolor{stringliteral}{'M'} ) ) \textcolor{keywordflow}{THEN}
00131          rmach = minexponent(zero)
00132       \textcolor{keywordflow}{ELSE} \textcolor{keywordflow}{IF}( lsame( cmach, \textcolor{stringliteral}{'U'} ) ) \textcolor{keywordflow}{THEN}
00133          rmach = tiny(zero)
00134       \textcolor{keywordflow}{ELSE} \textcolor{keywordflow}{IF}( lsame( cmach, \textcolor{stringliteral}{'L'} ) ) \textcolor{keywordflow}{THEN}
00135          rmach = maxexponent(zero)
00136       \textcolor{keywordflow}{ELSE} \textcolor{keywordflow}{IF}( lsame( cmach, \textcolor{stringliteral}{'O'} ) ) \textcolor{keywordflow}{THEN}
00137          rmach = huge(zero)
00138       \textcolor{keywordflow}{ELSE}
00139          rmach = zero
00140 \textcolor{keywordflow}{      END IF}
00141 \textcolor{comment}{*}
00142       slamch = rmach
00143       \textcolor{keywordflow}{RETURN}
00144 \textcolor{comment}{*}
00145 \textcolor{comment}{*     End of SLAMCH}
00146 \textcolor{comment}{*}
00147 \textcolor{keyword}{      END}
00148 \textcolor{comment}{************************************************************************}
00149 \textcolor{comment}{*> \(\backslash\)brief \(\backslash\)b SLAMC3}
00150 \textcolor{comment}{*> \(\backslash\)details}
00151 \textcolor{comment}{*> \(\backslash\)b Purpose:}
00152 \textcolor{comment}{*> \(\backslash\)verbatim}
00153 \textcolor{comment}{*> SLAMC3  is intended to force  A  and  B  to be stored prior to doing}
00154 \textcolor{comment}{*> the addition of  A  and  B ,  for use in situations where optimizers}
00155 \textcolor{comment}{*> might hold one of these in a register.}
00156 \textcolor{comment}{*> \(\backslash\)endverbatim}
00157 \textcolor{comment}{*> \(\backslash\)author LAPACK is a software package provided by Univ. of Tennessee, Univ. of California Berkeley, Univ.
       of Colorado Denver and NAG Ltd..}
00158 \textcolor{comment}{*> \(\backslash\)date November 2011}
00159 \textcolor{comment}{*> \(\backslash\)ingroup auxOTHERauxiliary}
00160 \textcolor{comment}{*>}
00161 \textcolor{comment}{*> \(\backslash\)param[in] A}
00162 \textcolor{comment}{*> \(\backslash\)verbatim}
00163 \textcolor{comment}{*> \(\backslash\)endverbatim}
00164 \textcolor{comment}{*>}
00165 \textcolor{comment}{*> \(\backslash\)param[in] B}
00166 \textcolor{comment}{*> \(\backslash\)verbatim}
00167 \textcolor{comment}{*>          The values A and B.}
00168 \textcolor{comment}{*> \(\backslash\)endverbatim}
00169 \textcolor{comment}{*>}
00170 \textcolor{comment}{*}
00171 \textcolor{keyword}{      REAL             }\textcolor{keyword}{FUNCTION }slamc3( A, B )
00172 \textcolor{comment}{*}
00173 \textcolor{comment}{*  -- LAPACK auxiliary routine (version 3.4.0) --}
00174 \textcolor{comment}{*     Univ. of Tennessee, Univ. of California Berkeley and NAG Ltd..}
00175 \textcolor{comment}{*     November 2010}
00176 \textcolor{comment}{*}
00177 \textcolor{comment}{*     .. Scalar Arguments ..}
00178       \textcolor{keywordtype}{REAL}               a, b
00179 \textcolor{comment}{*     ..}
00180 \textcolor{comment}{* =====================================================================}
00181 \textcolor{comment}{*}
00182 \textcolor{comment}{*     .. Executable Statements ..}
00183 \textcolor{comment}{*}
00184       slamc3 = a + b
00185 \textcolor{comment}{*}
00186       \textcolor{keywordflow}{RETURN}
00187 \textcolor{comment}{*}
00188 \textcolor{comment}{*     End of SLAMC3}
00189 \textcolor{comment}{*}
00190 \textcolor{keyword}{      END}
00191 \textcolor{comment}{*}
00192 \textcolor{comment}{************************************************************************}
\end{DoxyCode}
