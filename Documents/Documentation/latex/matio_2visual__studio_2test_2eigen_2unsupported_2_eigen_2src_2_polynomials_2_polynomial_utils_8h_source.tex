\hypertarget{matio_2visual__studio_2test_2eigen_2unsupported_2_eigen_2src_2_polynomials_2_polynomial_utils_8h_source}{}\section{matio/visual\+\_\+studio/test/eigen/unsupported/\+Eigen/src/\+Polynomials/\+Polynomial\+Utils.h}
\label{matio_2visual__studio_2test_2eigen_2unsupported_2_eigen_2src_2_polynomials_2_polynomial_utils_8h_source}\index{Polynomial\+Utils.\+h@{Polynomial\+Utils.\+h}}

\begin{DoxyCode}
00001 \textcolor{comment}{// This file is part of Eigen, a lightweight C++ template library}
00002 \textcolor{comment}{// for linear algebra.}
00003 \textcolor{comment}{//}
00004 \textcolor{comment}{// Copyright (C) 2010 Manuel Yguel <manuel.yguel@gmail.com>}
00005 \textcolor{comment}{//}
00006 \textcolor{comment}{// This Source Code Form is subject to the terms of the Mozilla}
00007 \textcolor{comment}{// Public License v. 2.0. If a copy of the MPL was not distributed}
00008 \textcolor{comment}{// with this file, You can obtain one at http://mozilla.org/MPL/2.0/.}
00009 
00010 \textcolor{preprocessor}{#ifndef EIGEN\_POLYNOMIAL\_UTILS\_H}
00011 \textcolor{preprocessor}{#define EIGEN\_POLYNOMIAL\_UTILS\_H}
00012 
00013 \textcolor{keyword}{namespace }\hyperlink{namespace_eigen}{Eigen} \{ 
00014 
00026 \textcolor{keyword}{template} <\textcolor{keyword}{typename} Polynomials, \textcolor{keyword}{typename} T>
00027 \textcolor{keyword}{inline}
00028 \hyperlink{group___sparse_core___module_class_eigen_1_1_triplet}{T} \hyperlink{namespace_eigen_aadbf059bc28ce1cf94c57c1454633d40}{poly\_eval\_horner}( \textcolor{keyword}{const} Polynomials& poly, \textcolor{keyword}{const} \hyperlink{group___sparse_core___module_class_eigen_1_1_triplet}{T}& x )
00029 \{
00030   \hyperlink{group___sparse_core___module_class_eigen_1_1_triplet}{T} val=poly[poly.size()-1];
00031   \textcolor{keywordflow}{for}(DenseIndex i=poly.size()-2; i>=0; --i )\{
00032     val = val*x + poly[i]; \}
00033   \textcolor{keywordflow}{return} val;
00034 \}
00035 
00044 \textcolor{keyword}{template} <\textcolor{keyword}{typename} Polynomials, \textcolor{keyword}{typename} T>
00045 \textcolor{keyword}{inline}
00046 \hyperlink{group___sparse_core___module_class_eigen_1_1_triplet}{T} \hyperlink{namespace_eigen_adb64ffddaa9e83634e3ab0e3fd3664f5}{poly\_eval}( \textcolor{keyword}{const} Polynomials& poly, \textcolor{keyword}{const} \hyperlink{group___sparse_core___module_class_eigen_1_1_triplet}{T}& x )
00047 \{
00048   \textcolor{keyword}{typedef} \textcolor{keyword}{typename} NumTraits<T>::Real Real;
00049 
00050   \textcolor{keywordflow}{if}( numext::abs2( x ) <= Real(1) )\{
00051     \textcolor{keywordflow}{return} \hyperlink{namespace_eigen_aadbf059bc28ce1cf94c57c1454633d40}{poly\_eval\_horner}( poly, x ); \}
00052   \textcolor{keywordflow}{else}
00053   \{
00054     \hyperlink{group___sparse_core___module_class_eigen_1_1_triplet}{T} val=poly[0];
00055     \hyperlink{group___sparse_core___module_class_eigen_1_1_triplet}{T} inv\_x = \hyperlink{group___sparse_core___module_class_eigen_1_1_triplet}{T}(1)/x;
00056     \textcolor{keywordflow}{for}( DenseIndex i=1; i<poly.size(); ++i )\{
00057       val = val*inv\_x + poly[i]; \}
00058 
00059     \textcolor{keywordflow}{return} numext::pow(x,(\hyperlink{group___sparse_core___module_class_eigen_1_1_triplet}{T})(poly.size()-1)) * val;
00060   \}
00061 \}
00062 
00073 \textcolor{keyword}{template} <\textcolor{keyword}{typename} Polynomial>
00074 \textcolor{keyword}{inline}
00075 \textcolor{keyword}{typename} NumTraits<typename Polynomial::Scalar>::Real \hyperlink{namespace_eigen_ac90ec4513aa09bb8ad54daa209322d03}{cauchy\_max\_bound}( \textcolor{keyword}{const} Polynomial& 
      poly )
00076 \{
00077   \textcolor{keyword}{using} std::abs;
00078   \textcolor{keyword}{typedef} \textcolor{keyword}{typename} Polynomial::Scalar Scalar;
00079   \textcolor{keyword}{typedef} \textcolor{keyword}{typename} NumTraits<Scalar>::Real Real;
00080 
00081   eigen\_assert( Scalar(0) != poly[poly.size()-1] );
00082   \textcolor{keyword}{const} Scalar inv\_leading\_coeff = Scalar(1)/poly[poly.size()-1];
00083   Real cb(0);
00084 
00085   \textcolor{keywordflow}{for}( DenseIndex i=0; i<poly.size()-1; ++i )\{
00086     cb += abs(poly[i]*inv\_leading\_coeff); \}
00087   \textcolor{keywordflow}{return} cb + Real(1);
00088 \}
00089 
00096 \textcolor{keyword}{template} <\textcolor{keyword}{typename} Polynomial>
00097 \textcolor{keyword}{inline}
00098 \textcolor{keyword}{typename} NumTraits<typename Polynomial::Scalar>::Real \hyperlink{namespace_eigen_a43f0af310d5cc131eb5e806f241af951}{cauchy\_min\_bound}( \textcolor{keyword}{const} Polynomial& 
      poly )
00099 \{
00100   \textcolor{keyword}{using} std::abs;
00101   \textcolor{keyword}{typedef} \textcolor{keyword}{typename} Polynomial::Scalar Scalar;
00102   \textcolor{keyword}{typedef} \textcolor{keyword}{typename} NumTraits<Scalar>::Real Real;
00103 
00104   DenseIndex i=0;
00105   \textcolor{keywordflow}{while}( i<poly.size()-1 && Scalar(0) == poly(i) )\{ ++i; \}
00106   \textcolor{keywordflow}{if}( poly.size()-1 == i )\{
00107     \textcolor{keywordflow}{return} Real(1); \}
00108 
00109   \textcolor{keyword}{const} Scalar inv\_min\_coeff = Scalar(1)/poly[i];
00110   Real cb(1);
00111   \textcolor{keywordflow}{for}( DenseIndex j=i+1; j<poly.size(); ++j )\{
00112     cb += abs(poly[j]*inv\_min\_coeff); \}
00113   \textcolor{keywordflow}{return} Real(1)/cb;
00114 \}
00115 
00126 \textcolor{keyword}{template} <\textcolor{keyword}{typename} RootVector, \textcolor{keyword}{typename} Polynomial>
00127 \textcolor{keywordtype}{void} \hyperlink{namespace_eigen_afbc3648f7ef67db3d5d04454fc1257fd}{roots\_to\_monicPolynomial}( \textcolor{keyword}{const} RootVector& rv, Polynomial& poly )
00128 \{
00129 
00130   \textcolor{keyword}{typedef} \textcolor{keyword}{typename} Polynomial::Scalar Scalar;
00131 
00132   poly.setZero( rv.size()+1 );
00133   poly[0] = -rv[0]; poly[1] = Scalar(1);
00134   \textcolor{keywordflow}{for}( DenseIndex i=1; i< rv.size(); ++i )
00135   \{
00136     \textcolor{keywordflow}{for}( DenseIndex j=i+1; j>0; --j )\{ poly[j] = poly[j-1] - rv[i]*poly[j]; \}
00137     poly[0] = -rv[i]*poly[0];
00138   \}
00139 \}
00140 
00141 \} \textcolor{comment}{// end namespace Eigen}
00142 
00143 \textcolor{preprocessor}{#endif // EIGEN\_POLYNOMIAL\_UTILS\_H}
\end{DoxyCode}
