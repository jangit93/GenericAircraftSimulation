\hypertarget{matio_2visual__studio_2test_2eigen_2test_2adjoint_8cpp_source}{}\section{matio/visual\+\_\+studio/test/eigen/test/adjoint.cpp}
\label{matio_2visual__studio_2test_2eigen_2test_2adjoint_8cpp_source}\index{adjoint.\+cpp@{adjoint.\+cpp}}

\begin{DoxyCode}
00001 \textcolor{comment}{// This file is part of Eigen, a lightweight C++ template library}
00002 \textcolor{comment}{// for linear algebra.}
00003 \textcolor{comment}{//}
00004 \textcolor{comment}{// Copyright (C) 2006-2008 Benoit Jacob <jacob.benoit.1@gmail.com>}
00005 \textcolor{comment}{//}
00006 \textcolor{comment}{// This Source Code Form is subject to the terms of the Mozilla}
00007 \textcolor{comment}{// Public License v. 2.0. If a copy of the MPL was not distributed}
00008 \textcolor{comment}{// with this file, You can obtain one at http://mozilla.org/MPL/2.0/.}
00009 
00010 \textcolor{preprocessor}{#define EIGEN\_NO\_STATIC\_ASSERT}
00011 
00012 \textcolor{preprocessor}{#include "main.h"}
00013 
00014 \textcolor{keyword}{template}<\textcolor{keywordtype}{bool} IsInteger> \textcolor{keyword}{struct }\hyperlink{structadjoint__specific}{adjoint\_specific};
00015 
00016 \textcolor{keyword}{template}<> \textcolor{keyword}{struct }\hyperlink{structadjoint__specific}{adjoint\_specific}<true> \{
00017   \textcolor{keyword}{template}<\textcolor{keyword}{typename} Vec, \textcolor{keyword}{typename} Mat, \textcolor{keyword}{typename} Scalar>
00018   \textcolor{keyword}{static} \textcolor{keywordtype}{void} run(\textcolor{keyword}{const} \hyperlink{group___core___module_class_eigen_1_1_matrix}{Vec}& v1, \textcolor{keyword}{const} \hyperlink{group___core___module_class_eigen_1_1_matrix}{Vec}& v2, \hyperlink{group___core___module_class_eigen_1_1_matrix}{Vec}& v3, \textcolor{keyword}{const} \hyperlink{group___core___module}{Mat}& square, Scalar s1, Scalar 
      s2) \{
00019     VERIFY(test\_isApproxWithRef((s1 * v1 + s2 * v2).dot(v3),     numext::conj(s1) * v1.dot(v3) + 
      numext::conj(s2) * v2.dot(v3), 0));
00020     VERIFY(test\_isApproxWithRef(v3.dot(s1 * v1 + s2 * v2),       s1*v3.dot(v1)+s2*v3.dot(v2), 0));
00021     
00022     \textcolor{comment}{// check compatibility of dot and adjoint}
00023     VERIFY(test\_isApproxWithRef(v1.dot(square * v2), (square.adjoint() * v1).dot(v2), 0));
00024   \}
00025 \};
00026 
00027 \textcolor{keyword}{template}<> \textcolor{keyword}{struct }\hyperlink{structadjoint__specific}{adjoint\_specific}<false> \{
00028   \textcolor{keyword}{template}<\textcolor{keyword}{typename} Vec, \textcolor{keyword}{typename} Mat, \textcolor{keyword}{typename} Scalar>
00029   \textcolor{keyword}{static} \textcolor{keywordtype}{void} run(\textcolor{keyword}{const} \hyperlink{group___core___module_class_eigen_1_1_matrix}{Vec}& v1, \textcolor{keyword}{const} \hyperlink{group___core___module_class_eigen_1_1_matrix}{Vec}& v2, \hyperlink{group___core___module_class_eigen_1_1_matrix}{Vec}& v3, \textcolor{keyword}{const} \hyperlink{group___core___module}{Mat}& square, Scalar s1, Scalar 
      s2) \{
00030     \textcolor{keyword}{typedef} \textcolor{keyword}{typename} \hyperlink{group___core___module_struct_eigen_1_1_num_traits}{NumTraits<Scalar>::Real} RealScalar;
00031     \textcolor{keyword}{using} std::abs;
00032     
00033     RealScalar ref = \hyperlink{group___core___module_struct_eigen_1_1_num_traits}{NumTraits<Scalar>::IsInteger} ? RealScalar(0) : (
      \hyperlink{namespacestd}{std}::max)((s1 * v1 + s2 * v2).norm(),v3.norm());
00034     VERIFY(test\_isApproxWithRef((s1 * v1 + s2 * v2).dot(v3),     numext::conj(s1) * v1.dot(v3) + 
      numext::conj(s2) * v2.dot(v3), ref));
00035     VERIFY(test\_isApproxWithRef(v3.dot(s1 * v1 + s2 * v2),       s1*v3.dot(v1)+s2*v3.dot(v2), ref));
00036   
00037     VERIFY\_IS\_APPROX(v1.squaredNorm(),                v1.norm() * v1.norm());
00038     \textcolor{comment}{// check normalized() and normalize()}
00039     VERIFY\_IS\_APPROX(v1, v1.norm() * v1.normalized());
00040     v3 = v1;
00041     v3.normalize();
00042     VERIFY\_IS\_APPROX(v1, v1.norm() * v3);
00043     VERIFY\_IS\_APPROX(v3, v1.normalized());
00044     VERIFY\_IS\_APPROX(v3.norm(), RealScalar(1));
00045 
00046     \textcolor{comment}{// check null inputs}
00047     VERIFY\_IS\_APPROX((v1*0).normalized(), (v1*0));
00048 \textcolor{preprocessor}{#if (!EIGEN\_ARCH\_i386) || defined(EIGEN\_VECTORIZE)}
00049     RealScalar very\_small = (std::numeric\_limits<RealScalar>::min)();
00050     VERIFY( (v1*very\_small).norm() == 0 );
00051     VERIFY\_IS\_APPROX((v1*very\_small).normalized(), (v1*very\_small));
00052     v3 = v1*very\_small;
00053     v3.normalize();
00054     VERIFY\_IS\_APPROX(v3, (v1*very\_small));
00055 \textcolor{preprocessor}{#endif}
00056     
00057     \textcolor{comment}{// check compatibility of dot and adjoint}
00058     ref = \hyperlink{group___core___module_struct_eigen_1_1_num_traits}{NumTraits<Scalar>::IsInteger} ? 0 : (std::max)((std::max)(v1.norm(),v2
      .norm()),(std::max)((square * v2).norm(),(square.adjoint() * v1).norm()));
00059     VERIFY(internal::isMuchSmallerThan(abs(v1.dot(square * v2) - (square.adjoint() * v1).dot(v2)), ref, 
      test\_precision<Scalar>()));
00060     
00061     \textcolor{comment}{// check that Random().normalized() works: tricky as the random xpr must be evaluated by}
00062     \textcolor{comment}{// normalized() in order to produce a consistent result.}
00063     VERIFY\_IS\_APPROX(Vec::Random(v1.size()).normalized().norm(), RealScalar(1));
00064   \}
00065 \};
00066 
00067 \textcolor{keyword}{template}<\textcolor{keyword}{typename} MatrixType> \textcolor{keywordtype}{void} adjoint(\textcolor{keyword}{const} MatrixType& m)
00068 \{
00069   \textcolor{comment}{/* this test covers the following files:}
00070 \textcolor{comment}{     Transpose.h Conjugate.h Dot.h}
00071 \textcolor{comment}{  */}
00072   \textcolor{keyword}{using} std::abs;
00073   \textcolor{keyword}{typedef} \textcolor{keyword}{typename} MatrixType::Index \hyperlink{namespace_eigen_a62e77e0933482dafde8fe197d9a2cfde}{Index};
00074   \textcolor{keyword}{typedef} \textcolor{keyword}{typename} MatrixType::Scalar Scalar;
00075   \textcolor{keyword}{typedef} \textcolor{keyword}{typename} \hyperlink{group___core___module_struct_eigen_1_1_num_traits}{NumTraits<Scalar>::Real} RealScalar;
00076   \textcolor{keyword}{typedef} \hyperlink{group___core___module_class_eigen_1_1_matrix}{Matrix<Scalar, MatrixType::RowsAtCompileTime, 1>} 
      \hyperlink{struct_vector_type}{VectorType};
00077   \textcolor{keyword}{typedef} 
      \hyperlink{group___core___module_class_eigen_1_1_matrix}{Matrix<Scalar, MatrixType::RowsAtCompileTime, MatrixType::RowsAtCompileTime>}
       SquareMatrixType;
00078   \textcolor{keyword}{const} Index PacketSize = internal::packet\_traits<Scalar>::size;
00079   
00080   Index rows = m.rows();
00081   Index cols = m.cols();
00082 
00083   MatrixType m1 = MatrixType::Random(rows, cols),
00084              m2 = MatrixType::Random(rows, cols),
00085              m3(rows, cols),
00086              square = SquareMatrixType::Random(rows, rows);
00087   VectorType v1 = VectorType::Random(rows),
00088              v2 = VectorType::Random(rows),
00089              v3 = VectorType::Random(rows),
00090              vzero = VectorType::Zero(rows);
00091 
00092   Scalar s1 = internal::random<Scalar>(),
00093          s2 = internal::random<Scalar>();
00094 
00095   \textcolor{comment}{// check basic compatibility of adjoint, transpose, conjugate}
00096   VERIFY\_IS\_APPROX(m1.transpose().conjugate().adjoint(),    m1);
00097   VERIFY\_IS\_APPROX(m1.adjoint().conjugate().transpose(),    m1);
00098 
00099   \textcolor{comment}{// check multiplicative behavior}
00100   VERIFY\_IS\_APPROX((m1.adjoint() * m2).adjoint(),           m2.adjoint() * m1);
00101   VERIFY\_IS\_APPROX((s1 * m1).adjoint(),                     numext::conj(s1) * m1.adjoint());
00102 
00103   \textcolor{comment}{// check basic properties of dot, squaredNorm}
00104   VERIFY\_IS\_APPROX(numext::conj(v1.dot(v2)),               v2.dot(v1));
00105   VERIFY\_IS\_APPROX(numext::real(v1.dot(v1)),               v1.squaredNorm());
00106   
00107   \hyperlink{structadjoint__specific}{adjoint\_specific<NumTraits<Scalar>::IsInteger}>::run(v1, v2, 
      v3, square, s1, s2);
00108   
00109   VERIFY\_IS\_MUCH\_SMALLER\_THAN(abs(vzero.dot(v1)),  static\_cast<RealScalar>(1));
00110   
00111   \textcolor{comment}{// like in testBasicStuff, test operator() to check const-qualification}
00112   Index r = internal::random<Index>(0, rows-1),
00113       c = internal::random<Index>(0, cols-1);
00114   VERIFY\_IS\_APPROX(m1.conjugate()(r,c), numext::conj(m1(r,c)));
00115   VERIFY\_IS\_APPROX(m1.adjoint()(c,r), numext::conj(m1(r,c)));
00116 
00117   \textcolor{comment}{// check inplace transpose}
00118   m3 = m1;
00119   m3.transposeInPlace();
00120   VERIFY\_IS\_APPROX(m3,m1.transpose());
00121   m3.transposeInPlace();
00122   VERIFY\_IS\_APPROX(m3,m1);
00123   
00124   \textcolor{keywordflow}{if}(PacketSize<m3.rows() && PacketSize<m3.cols())
00125   \{
00126     m3 = m1;
00127     Index i = internal::random<Index>(0,m3.rows()-PacketSize);
00128     Index j = internal::random<Index>(0,m3.cols()-PacketSize);
00129     m3.template block<PacketSize,PacketSize>(i,j).transposeInPlace();
00130     VERIFY\_IS\_APPROX( (m3.template block<PacketSize,PacketSize>(i,j)), (m1.template 
      block<PacketSize,PacketSize>(i,j).transpose()) );
00131     m3.template block<PacketSize,PacketSize>(i,j).transposeInPlace();
00132     VERIFY\_IS\_APPROX(m3,m1);
00133   \}
00134 
00135   \textcolor{comment}{// check inplace adjoint}
00136   m3 = m1;
00137   m3.adjointInPlace();
00138   VERIFY\_IS\_APPROX(m3,m1.adjoint());
00139   m3.transposeInPlace();
00140   VERIFY\_IS\_APPROX(m3,m1.conjugate());
00141 
00142   \textcolor{comment}{// check mixed dot product}
00143   \textcolor{keyword}{typedef} \hyperlink{group___core___module_class_eigen_1_1_matrix}{Matrix<RealScalar, MatrixType::RowsAtCompileTime, 1>}
       RealVectorType;
00144   RealVectorType rv1 = RealVectorType::Random(rows);
00145   VERIFY\_IS\_APPROX(v1.dot(rv1.template cast<Scalar>()), v1.dot(rv1));
00146   VERIFY\_IS\_APPROX(rv1.template cast<Scalar>().dot(v1), rv1.dot(v1));
00147 \}
00148 
00149 \textcolor{keywordtype}{void} test\_adjoint()
00150 \{
00151   \textcolor{keywordflow}{for}(\textcolor{keywordtype}{int} i = 0; i < g\_repeat; i++) \{
00152     CALL\_SUBTEST\_1( adjoint(\hyperlink{group___core___module_class_eigen_1_1_matrix}{Matrix<float, 1, 1>}()) );
00153     CALL\_SUBTEST\_2( adjoint(Matrix3d()) );
00154     CALL\_SUBTEST\_3( adjoint(Matrix4f()) );
00155     
00156     CALL\_SUBTEST\_4( adjoint(MatrixXcf(internal::random<int>(1,EIGEN\_TEST\_MAX\_SIZE/2), internal::random<int>
      (1,EIGEN\_TEST\_MAX\_SIZE/2))) );
00157     CALL\_SUBTEST\_5( adjoint(MatrixXi(internal::random<int>(1,EIGEN\_TEST\_MAX\_SIZE), internal::random<int>(1,
      EIGEN\_TEST\_MAX\_SIZE))) );
00158     CALL\_SUBTEST\_6( adjoint(MatrixXf(internal::random<int>(1,EIGEN\_TEST\_MAX\_SIZE), internal::random<int>(1,
      EIGEN\_TEST\_MAX\_SIZE))) );
00159     
00160     \textcolor{comment}{// Complement for 128 bits vectorization:}
00161     CALL\_SUBTEST\_8( adjoint(Matrix2d()) );
00162     CALL\_SUBTEST\_9( adjoint(\hyperlink{group___core___module_class_eigen_1_1_matrix}{Matrix<int,4,4>}()) );
00163     
00164     \textcolor{comment}{// 256 bits vectorization:}
00165     CALL\_SUBTEST\_10( adjoint(\hyperlink{group___core___module_class_eigen_1_1_matrix}{Matrix<float,8,8>}()) );
00166     CALL\_SUBTEST\_11( adjoint(\hyperlink{group___core___module_class_eigen_1_1_matrix}{Matrix<double,4,4>}()) );
00167     CALL\_SUBTEST\_12( adjoint(\hyperlink{group___core___module_class_eigen_1_1_matrix}{Matrix<int,8,8>}()) );
00168   \}
00169   \textcolor{comment}{// test a large static matrix only once}
00170   CALL\_SUBTEST\_7( adjoint(\hyperlink{group___core___module_class_eigen_1_1_matrix}{Matrix<float, 100, 100>}()) );
00171 
00172 \textcolor{preprocessor}{#ifdef EIGEN\_TEST\_PART\_13}
00173   \{
00174     MatrixXcf a(10,10), b(10,10);
00175     VERIFY\_RAISES\_ASSERT(a = a.transpose());
00176     VERIFY\_RAISES\_ASSERT(a = a.transpose() + b);
00177     VERIFY\_RAISES\_ASSERT(a = b + a.transpose());
00178     VERIFY\_RAISES\_ASSERT(a = a.conjugate().transpose());
00179     VERIFY\_RAISES\_ASSERT(a = a.adjoint());
00180     VERIFY\_RAISES\_ASSERT(a = a.adjoint() + b);
00181     VERIFY\_RAISES\_ASSERT(a = b + a.adjoint());
00182 
00183     \textcolor{comment}{// no assertion should be triggered for these cases:}
00184     a.transpose() = a.transpose();
00185     a.transpose() += a.transpose();
00186     a.transpose() += a.transpose() + b;
00187     a.transpose() = a.adjoint();
00188     a.transpose() += a.adjoint();
00189     a.transpose() += a.adjoint() + b;
00190 
00191     \textcolor{comment}{// regression tests for check\_for\_aliasing}
00192     MatrixXd c(10,10);
00193     c = 1.0 * MatrixXd::Ones(10,10) + c;
00194     c = MatrixXd::Ones(10,10) * 1.0 + c;
00195     c = c + MatrixXd::Ones(10,10) .cwiseProduct( MatrixXd::Zero(10,10) );
00196     c = MatrixXd::Ones(10,10) * MatrixXd::Zero(10,10);
00197   \}
00198 \textcolor{preprocessor}{#endif}
00199 \}
00200 
\end{DoxyCode}
