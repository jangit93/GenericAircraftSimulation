\hypertarget{visual__studio_2_h_d_f5_21_810_81_2_h_d_f5_examples_2_c_2_h5_t_2h5ex__t__commit_8c_source}{}\section{visual\+\_\+studio/\+H\+D\+F5/1.10.1/\+H\+D\+F5\+Examples/\+C/\+H5\+T/h5ex\+\_\+t\+\_\+commit.c}
\label{visual__studio_2_h_d_f5_21_810_81_2_h_d_f5_examples_2_c_2_h5_t_2h5ex__t__commit_8c_source}\index{h5ex\+\_\+t\+\_\+commit.\+c@{h5ex\+\_\+t\+\_\+commit.\+c}}

\begin{DoxyCode}
00001 \textcolor{comment}{/************************************************************}
00002 \textcolor{comment}{}
00003 \textcolor{comment}{  This example shows how to commit a named datatype to a}
00004 \textcolor{comment}{  file, and read back that datatype.  The program first}
00005 \textcolor{comment}{  defines a compound datatype, commits it to a file, then}
00006 \textcolor{comment}{  closes the file.  Next, it reopens the file, opens the}
00007 \textcolor{comment}{  datatype, and outputs the names of its fields to the}
00008 \textcolor{comment}{  screen.}
00009 \textcolor{comment}{}
00010 \textcolor{comment}{  This file is intended for use with HDF5 Library version 1.8}
00011 \textcolor{comment}{}
00012 \textcolor{comment}{ ************************************************************/}
00013 
00014 \textcolor{preprocessor}{#include "hdf5.h"}
00015 \textcolor{preprocessor}{#include <stdio.h>}
00016 \textcolor{preprocessor}{#include <stdlib.h>}
00017 
00018 \textcolor{preprocessor}{#define FILE            "h5ex\_t\_commit.h5"}
00019 \textcolor{preprocessor}{#define DATATYPE        "Sensor\_Type"}
00020 
00021 \textcolor{keywordtype}{int}
00022 main (\textcolor{keywordtype}{void})
00023 \{
00024     hid\_t           \hyperlink{structfile}{file}, filetype, strtype;
00025                                             \textcolor{comment}{/* Handles */}
00026     herr\_t          status;
00027     H5T\_class\_t     typeclass;
00028     \textcolor{keywordtype}{char}            *name;
00029     \textcolor{keywordtype}{int}             nmembs,
00030                     i;
00031 
00032     \textcolor{comment}{/*}
00033 \textcolor{comment}{     * Create a new file using the default properties.}
00034 \textcolor{comment}{     */}
00035     file = H5Fcreate (FILE, H5F\_ACC\_TRUNC, H5P\_DEFAULT, H5P\_DEFAULT);
00036 
00037     \textcolor{comment}{/*}
00038 \textcolor{comment}{     * Create variable-length string datatype.}
00039 \textcolor{comment}{     */}
00040     strtype = H5Tcopy (H5T\_C\_S1);
00041     status = H5Tset\_size (strtype, H5T\_VARIABLE);
00042 
00043     \textcolor{comment}{/*}
00044 \textcolor{comment}{     * Create the compound datatype.  Because the standard types we are}
00045 \textcolor{comment}{     * using may have different sizes than the corresponding native}
00046 \textcolor{comment}{     * types, we must manually calculate the offset of each member.}
00047 \textcolor{comment}{     */}
00048     filetype = H5Tcreate (H5T\_COMPOUND, 8 + \textcolor{keyword}{sizeof} (\textcolor{keywordtype}{char} *) + 8 + 8);
00049     status = H5Tinsert (filetype, \textcolor{stringliteral}{"Serial number"}, 0, H5T\_STD\_I64BE);
00050     status = H5Tinsert (filetype, \textcolor{stringliteral}{"Location"}, 8, strtype);
00051     status = H5Tinsert (filetype, \textcolor{stringliteral}{"Temperature (F)"}, 8 + \textcolor{keyword}{sizeof} (\textcolor{keywordtype}{char} *),
00052                 H5T\_IEEE\_F64BE);
00053     status = H5Tinsert (filetype, \textcolor{stringliteral}{"Pressure (inHg)"}, 8 + \textcolor{keyword}{sizeof} (\textcolor{keywordtype}{char} *) + 8,
00054                 H5T\_IEEE\_F64BE);
00055 
00056     \textcolor{comment}{/*}
00057 \textcolor{comment}{     * Commit the compound datatype to the file, creating a named}
00058 \textcolor{comment}{     * datatype.}
00059 \textcolor{comment}{     */}
00060     status = H5Tcommit (file, DATATYPE, filetype, H5P\_DEFAULT,
00061                 H5P\_DEFAULT, H5P\_DEFAULT);
00062 
00063     \textcolor{comment}{/*}
00064 \textcolor{comment}{     * Close and release resources.}
00065 \textcolor{comment}{     */}
00066     status = H5Tclose (filetype);
00067     status = H5Tclose (strtype);
00068     status = H5Fclose (file);
00069 
00070 
00071     \textcolor{comment}{/*}
00072 \textcolor{comment}{     * Now we begin the read section of this example.}
00073 \textcolor{comment}{     */}
00074 
00075     \textcolor{comment}{/*}
00076 \textcolor{comment}{     * Open file.}
00077 \textcolor{comment}{     */}
00078     file = H5Fopen (FILE, H5F\_ACC\_RDONLY, H5P\_DEFAULT);
00079 
00080     \textcolor{comment}{/*}
00081 \textcolor{comment}{     * Open the named datatype.}
00082 \textcolor{comment}{     */}
00083     filetype = H5Topen (file, DATATYPE, H5P\_DEFAULT);
00084 
00085     \textcolor{comment}{/*}
00086 \textcolor{comment}{     * Output the data to the screen.}
00087 \textcolor{comment}{     */}
00088     printf (\textcolor{stringliteral}{"Named datatype: %s:\(\backslash\)n"}, DATATYPE);
00089     \textcolor{comment}{/*}
00090 \textcolor{comment}{     * Get datatype class.  If it isn't compound, we won't print}
00091 \textcolor{comment}{     * anything.}
00092 \textcolor{comment}{     */}
00093     typeclass = H5Tget\_class (filetype);
00094     \textcolor{keywordflow}{if} (typeclass == H5T\_COMPOUND) \{
00095         printf (\textcolor{stringliteral}{"   Class: H5T\_COMPOUND\(\backslash\)n"});
00096         nmembs = H5Tget\_nmembers (filetype);
00097         \textcolor{comment}{/*}
00098 \textcolor{comment}{         * Iterate over compound datatype members.}
00099 \textcolor{comment}{         */}
00100         \textcolor{keywordflow}{for} (i=0; i<nmembs; i++) \{
00101             \textcolor{comment}{/*}
00102 \textcolor{comment}{             * Get the member name and print it.  Note that}
00103 \textcolor{comment}{             * H5Tget\_member\_name allocates space for the string in}
00104 \textcolor{comment}{             * name, so we must H5free\_memory() it after use.}
00105 \textcolor{comment}{             */}
00106             name = H5Tget\_member\_name (filetype, i);
00107             printf (\textcolor{stringliteral}{"   %s\(\backslash\)n"}, name);
00108             H5free\_memory (name);
00109         \}
00110     \}
00111 
00112     \textcolor{comment}{/*}
00113 \textcolor{comment}{     * Close and release resources.}
00114 \textcolor{comment}{     */}
00115     status = H5Tclose (filetype);
00116     status = H5Fclose (file);
00117 
00118     \textcolor{keywordflow}{return} 0;
00119 \}
\end{DoxyCode}
