\hypertarget{visual__studio_2_h_d_f5_21_810_81_2_h_d_f5_examples_2_f_o_r_t_r_a_n_2_h5_d_2h5ex__d__soint_8f90_source}{}\section{visual\+\_\+studio/\+H\+D\+F5/1.10.1/\+H\+D\+F5\+Examples/\+F\+O\+R\+T\+R\+A\+N/\+H5\+D/h5ex\+\_\+d\+\_\+soint.f90}
\label{visual__studio_2_h_d_f5_21_810_81_2_h_d_f5_examples_2_f_o_r_t_r_a_n_2_h5_d_2h5ex__d__soint_8f90_source}\index{h5ex\+\_\+d\+\_\+soint.\+f90@{h5ex\+\_\+d\+\_\+soint.\+f90}}

\begin{DoxyCode}
00001 \textcolor{comment}{!************************************************************}
00002 \textcolor{comment}{!}
00003 \textcolor{comment}{!  This example shows how to read and write data to a dataset}
00004 \textcolor{comment}{!  using the Scale-Offset filter.  The program first checks}
00005 \textcolor{comment}{!  if the Scale-Offset filter is available, then if it is it}
00006 \textcolor{comment}{!  writes integers to a dataset using Scale-Offset, then}
00007 \textcolor{comment}{!  closes the file Next, it reopens the file, reads back the}
00008 \textcolor{comment}{!  data, and outputs the type of filter and the maximum value}
00009 \textcolor{comment}{!  in the dataset to the screen.}
00010 \textcolor{comment}{!}
00011 \textcolor{comment}{!  This file is intended for use with HDF5 Library verion 1.8}
00012 \textcolor{comment}{!}
00013 \textcolor{comment}{!************************************************************}
00014 \textcolor{keyword}{PROGRAM} main
00015 
00016   \textcolor{keywordtype}{USE }hdf5
00017 
00018   \textcolor{keywordtype}{IMPLICIT NONE}
00019 
00020   \textcolor{keywordtype}{CHARACTER(LEN=15)}, \textcolor{keywordtype}{PARAMETER} :: filename = \textcolor{stringliteral}{"h5ex\_d\_soint.h5"}
00021   \textcolor{keywordtype}{CHARACTER(LEN=3)} , \textcolor{keywordtype}{PARAMETER} :: dataset  = \textcolor{stringliteral}{"DS1"}
00022   \textcolor{keywordtype}{INTEGER}          , \textcolor{keywordtype}{PARAMETER} :: dim0     = 32
00023   \textcolor{keywordtype}{INTEGER}          , \textcolor{keywordtype}{PARAMETER} :: dim1     = 64
00024   \textcolor{keywordtype}{INTEGER}          , \textcolor{keywordtype}{PARAMETER} :: chunk0   = 4
00025   \textcolor{keywordtype}{INTEGER}          , \textcolor{keywordtype}{PARAMETER} :: chunk1   = 8
00026 
00027   \textcolor{keywordtype}{INTEGER} :: hdferr
00028   \textcolor{keywordtype}{LOGICAL} :: avail
00029   \textcolor{keywordtype}{INTEGER(HID\_T)}  :: \hyperlink{structfile}{file}, space, dset, dtype, dcpl \textcolor{comment}{! Handles}
00030 
00031   \textcolor{keywordtype}{INTEGER}, \textcolor{keywordtype}{DIMENSION(1:1)} :: cd\_values
00032   \textcolor{keywordtype}{INTEGER} :: filter\_id
00033   \textcolor{keywordtype}{INTEGER} :: filter\_info\_both
00034   \textcolor{keywordtype}{INTEGER(HSIZE\_T)}, \textcolor{keywordtype}{DIMENSION(1:2)} :: dims = (/dim0, dim1/), chunk =(/chunk0,chunk1/)
00035   \textcolor{keywordtype}{INTEGER(SIZE\_T)} :: nelmts
00036   \textcolor{keywordtype}{INTEGER} :: flags, filter\_info
00037   \textcolor{keywordtype}{INTEGER}, \textcolor{keywordtype}{DIMENSION(1:dim0, 1:dim1)} :: wdata, & \textcolor{comment}{! Write buffer }
00038                                                  rdata    \textcolor{comment}{! Read buffer}
00039   \textcolor{keywordtype}{INTEGER} :: max, min
00040   \textcolor{keywordtype}{INTEGER} :: i, j
00041   \textcolor{keywordtype}{INTEGER}, \textcolor{keywordtype}{PARAMETER} :: maxchrlen = 80
00042   \textcolor{keywordtype}{CHARACTER(LEN=MaxChrLen)} :: name
00043   \textcolor{comment}{!}
00044   \textcolor{comment}{! Initialize FORTRAN interface.}
00045   \textcolor{comment}{!}
00046   \textcolor{keyword}{CALL }h5open\_f(hdferr)
00047   \textcolor{comment}{! }
00048   \textcolor{comment}{! Check if Scale-Offset compression is available and can be used}
00049   \textcolor{comment}{! for both compression and decompression.  Normally we do not}
00050   \textcolor{comment}{! perform error checking in these examples for the sake of}
00051   \textcolor{comment}{! clarity, but in this case we will make an exception because this}
00052   \textcolor{comment}{! filter is an optional part of the hdf5 library.}
00053   \textcolor{comment}{!  }
00054   \textcolor{keyword}{CALL }h5zfilter\_avail\_f(h5z\_filter\_scaleoffset\_f, avail, hdferr)
00055 
00056   \textcolor{keywordflow}{IF} (.NOT.avail) \textcolor{keywordflow}{THEN}
00057      \textcolor{keyword}{WRITE}(*,\textcolor{stringliteral}{'("Scale-Offset filter not available.",/)'})
00058      stop
00059 \textcolor{keywordflow}{  ENDIF}
00060 
00061   \textcolor{keyword}{CALL }h5zget\_filter\_info\_f(h5z\_filter\_scaleoffset\_f, filter\_info, hdferr)
00062 
00063   filter\_info\_both=ior(h5z\_filter\_encode\_enabled\_f,h5z\_filter\_decode\_enabled\_f)
00064   \textcolor{keywordflow}{IF} (filter\_info .NE. filter\_info\_both) \textcolor{keywordflow}{THEN}
00065      \textcolor{keyword}{WRITE}(*,\textcolor{stringliteral}{'("Scale-Offset filter not available for encoding and decoding.",/)'})
00066      stop
00067 \textcolor{keywordflow}{  ENDIF}
00068   \textcolor{comment}{!}
00069   \textcolor{comment}{! Initialize data.}
00070   \textcolor{comment}{!}
00071   \textcolor{keywordflow}{DO} i = 1, dim0
00072      \textcolor{keywordflow}{DO} j = 1, dim1
00073         wdata(i,j) = (i-1)*(j-1)-(j-1)
00074 \textcolor{keywordflow}{     ENDDO}
00075 \textcolor{keywordflow}{  ENDDO}
00076 
00077   \textcolor{comment}{!}
00078   \textcolor{comment}{! Find the maximum value in the dataset, to verify that it was}
00079   \textcolor{comment}{! read correctly.}
00080   \textcolor{comment}{!}
00081   max = maxval(wdata)
00082   min = minval(wdata)
00083   \textcolor{comment}{!}
00084   \textcolor{comment}{! Print the maximum value.}
00085   \textcolor{comment}{!}
00086   \textcolor{keyword}{WRITE}(*,\textcolor{stringliteral}{'("Maximum value in write buffer is: ",i4)'}) max
00087   \textcolor{keyword}{WRITE}(*,\textcolor{stringliteral}{'("Minimum value in write buffer is: ",i4)'}) min
00088   \textcolor{comment}{!}
00089   \textcolor{comment}{! Create a new file using the default properties.}
00090   \textcolor{comment}{!}
00091   \textcolor{keyword}{CALL }h5fcreate\_f(filename, h5f\_acc\_trunc\_f, \hyperlink{structfile}{file}, hdferr)
00092   \textcolor{comment}{!}
00093   \textcolor{comment}{! Create dataspace.  Setting size to be the current size.}
00094   \textcolor{comment}{!}
00095   \textcolor{keyword}{CALL }h5screate\_simple\_f(2, dims, space, hdferr)
00096   \textcolor{comment}{!}
00097   \textcolor{comment}{! Create the dataset creation property list, add the Scale-Offset}
00098   \textcolor{comment}{! filter and set the chunk size.}
00099   \textcolor{comment}{!}
00100   \textcolor{keyword}{CALL }h5pcreate\_f(h5p\_dataset\_create\_f, dcpl, hdferr)
00101   \textcolor{keyword}{CALL }h5pset\_scaleoffset\_f(dcpl, h5z\_so\_int\_f, h5z\_so\_int\_minbits\_default\_f, hdferr)
00102   \textcolor{keyword}{CALL }h5pset\_chunk\_f(dcpl, 2, chunk, hdferr)
00103   \textcolor{comment}{!}
00104   \textcolor{comment}{! Create the dataset.}
00105   \textcolor{comment}{!}
00106   \textcolor{keyword}{CALL }h5dcreate\_f(\hyperlink{structfile}{file}, dataset, h5t\_std\_i32le, space, dset, hdferr, dcpl)
00107   \textcolor{comment}{!}
00108   \textcolor{comment}{! Write the data to the dataset.}
00109   \textcolor{comment}{!}
00110   \textcolor{keyword}{CALL }h5dwrite\_f(dset, h5t\_native\_integer, wdata, dims, hdferr)
00111   \textcolor{comment}{!}
00112   \textcolor{comment}{! Close and release resources.}
00113   \textcolor{comment}{!}
00114   \textcolor{keyword}{CALL }h5pclose\_f(dcpl , hdferr)
00115   \textcolor{keyword}{CALL }h5dclose\_f(dset , hdferr)
00116   \textcolor{keyword}{CALL }h5sclose\_f(space, hdferr)
00117   \textcolor{keyword}{CALL }h5fclose\_f(\hyperlink{structfile}{file} , hdferr)
00118   \textcolor{comment}{!}
00119   \textcolor{comment}{! Now we begin the read section of this example.}
00120   \textcolor{comment}{!}
00121   \textcolor{comment}{!}
00122   \textcolor{comment}{! Open file and dataset using the default properties.}
00123   \textcolor{comment}{!}
00124   \textcolor{keyword}{CALL }h5fopen\_f(filename, h5f\_acc\_rdonly\_f, \hyperlink{structfile}{file}, hdferr)
00125   \textcolor{keyword}{CALL }h5dopen\_f (\hyperlink{structfile}{file}, dataset, dset, hdferr)
00126   \textcolor{comment}{!}
00127   \textcolor{comment}{! Retrieve dataset creation property list.}
00128   \textcolor{comment}{!}
00129   \textcolor{keyword}{CALL }h5dget\_create\_plist\_f(dset, dcpl, hdferr)
00130   \textcolor{comment}{!}
00131   \textcolor{comment}{! Retrieve and print the filter type.  Here we only retrieve the}
00132   \textcolor{comment}{! first filter because we know that we only added one filter.}
00133   \textcolor{comment}{!}
00134   nelmts = 1
00135   \textcolor{keyword}{CALL }h5pget\_filter\_f(dcpl, 0, flags, nelmts, cd\_values, maxchrlen, name, filter\_id, hdferr)
00136   \textcolor{keyword}{WRITE}(*,\textcolor{stringliteral}{'("Filter type is: ")'}, advance=\textcolor{stringliteral}{'NO'})
00137   \textcolor{keywordflow}{IF}(filter\_id.EQ.h5z\_filter\_deflate\_f)\textcolor{keywordflow}{THEN}
00138      \textcolor{keyword}{WRITE}(*,\textcolor{stringliteral}{'(T2,"H5Z\_FILTER\_DEFLATE\_F")'})
00139   \textcolor{keywordflow}{ELSE} \textcolor{keywordflow}{IF}(filter\_id.EQ.h5z\_filter\_shuffle\_f)\textcolor{keywordflow}{THEN}
00140      \textcolor{keyword}{WRITE}(*,\textcolor{stringliteral}{'(T2,"H5Z\_FILTER\_SHUFFLE\_F")'})
00141   \textcolor{keywordflow}{ELSE} \textcolor{keywordflow}{IF}(filter\_id.EQ.h5z\_filter\_fletcher32\_f)\textcolor{keywordflow}{THEN}
00142      \textcolor{keyword}{WRITE}(*,\textcolor{stringliteral}{'(T2,"H5Z\_FILTER\_FLETCHER32\_F")'})
00143   \textcolor{keywordflow}{ELSE} \textcolor{keywordflow}{IF}(filter\_id.EQ.h5z\_filter\_szip\_f)\textcolor{keywordflow}{THEN}
00144      \textcolor{keyword}{WRITE}(*,\textcolor{stringliteral}{'(T2,"H5Z\_FILTER\_SZIP\_F")'})
00145   \textcolor{keywordflow}{ELSE} \textcolor{keywordflow}{IF}(filter\_id.EQ.h5z\_filter\_nbit\_f)\textcolor{keywordflow}{THEN}
00146      \textcolor{keyword}{WRITE}(*,\textcolor{stringliteral}{'(T2,"H5Z\_FILTER\_NBIT\_F")'})
00147   \textcolor{keywordflow}{ELSE} \textcolor{keywordflow}{IF}(filter\_id.EQ.h5z\_filter\_scaleoffset\_f)\textcolor{keywordflow}{THEN}
00148      \textcolor{keyword}{WRITE}(*,\textcolor{stringliteral}{'(T2,"H5Z\_FILTER\_SCALEOFFSET\_F")'})
00149 \textcolor{keywordflow}{  ENDIF}
00150   \textcolor{comment}{!}
00151   \textcolor{comment}{! Read the data using the default properties.}
00152   \textcolor{comment}{!}
00153   \textcolor{keyword}{CALL }h5dread\_f(dset, h5t\_native\_integer, rdata, dims, hdferr)
00154   \textcolor{comment}{!}
00155   \textcolor{comment}{!}
00156   \textcolor{comment}{! Find the maximum and minimum value in the dataset, to verify that it was}
00157   \textcolor{comment}{! read correctly.}
00158   \textcolor{comment}{!}
00159   max = maxval(rdata)
00160   min = minval(rdata)
00161   \textcolor{comment}{!}
00162   \textcolor{comment}{! Print the maximum value.}
00163   \textcolor{comment}{!}
00164   \textcolor{keyword}{WRITE}(*,\textcolor{stringliteral}{'("Maximum value in ",A," is: ",i4)'}) dataset,max
00165   \textcolor{keyword}{WRITE}(*,\textcolor{stringliteral}{'("Minimum value in ",A," is: ",i4)'}) dataset,min
00166   \textcolor{comment}{!}
00167   \textcolor{comment}{! Close and release resources.}
00168   \textcolor{comment}{!}
00169   \textcolor{keyword}{CALL }h5pclose\_f(dcpl , hdferr)
00170   \textcolor{keyword}{CALL }h5dclose\_f(dset , hdferr)
00171   \textcolor{keyword}{CALL }h5fclose\_f(\hyperlink{structfile}{file} , hdferr)
00172 
00173 \textcolor{keyword}{END PROGRAM }main
\end{DoxyCode}
