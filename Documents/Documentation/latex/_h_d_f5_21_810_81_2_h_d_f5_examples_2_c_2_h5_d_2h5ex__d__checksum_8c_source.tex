\hypertarget{_h_d_f5_21_810_81_2_h_d_f5_examples_2_c_2_h5_d_2h5ex__d__checksum_8c_source}{}\section{H\+D\+F5/1.10.1/\+H\+D\+F5\+Examples/\+C/\+H5\+D/h5ex\+\_\+d\+\_\+checksum.c}
\label{_h_d_f5_21_810_81_2_h_d_f5_examples_2_c_2_h5_d_2h5ex__d__checksum_8c_source}\index{h5ex\+\_\+d\+\_\+checksum.\+c@{h5ex\+\_\+d\+\_\+checksum.\+c}}

\begin{DoxyCode}
00001 \textcolor{comment}{/************************************************************}
00002 \textcolor{comment}{}
00003 \textcolor{comment}{  This example shows how to read and write data to a dataset}
00004 \textcolor{comment}{  using the Fletcher32 checksum filter.  The program first}
00005 \textcolor{comment}{  checks if the Fletcher32 filter is available, then if it}
00006 \textcolor{comment}{  is it writes integers to a dataset using Fletcher32, then}
00007 \textcolor{comment}{  closes the file.  Next, it reopens the file, reads back}
00008 \textcolor{comment}{  the data, checks if the filter detected an error and}
00009 \textcolor{comment}{  outputs the type of filter and the maximum value in the}
00010 \textcolor{comment}{  dataset to the screen.}
00011 \textcolor{comment}{}
00012 \textcolor{comment}{  This file is intended for use with HDF5 Library version 1.8}
00013 \textcolor{comment}{}
00014 \textcolor{comment}{ ************************************************************/}
00015 
00016 \textcolor{preprocessor}{#include "hdf5.h"}
00017 \textcolor{preprocessor}{#include <stdio.h>}
00018 \textcolor{preprocessor}{#include <stdlib.h>}
00019 
00020 \textcolor{preprocessor}{#define FILE            "h5ex\_d\_checksum.h5"}
00021 \textcolor{preprocessor}{#define DATASET         "DS1"}
00022 \textcolor{preprocessor}{#define DIM0            32}
00023 \textcolor{preprocessor}{#define DIM1            64}
00024 \textcolor{preprocessor}{#define CHUNK0          4}
00025 \textcolor{preprocessor}{#define CHUNK1          8}
00026 
00027 \textcolor{keywordtype}{int}
00028 main (\textcolor{keywordtype}{void})
00029 \{
00030     hid\_t           \hyperlink{structfile}{file}, space, dset, dcpl;
00031                                                 \textcolor{comment}{/* Handles */}
00032     herr\_t          status;
00033     htri\_t          avail;
00034     H5Z\_filter\_t    filter\_type;
00035     hsize\_t         dims[2] = \{DIM0, DIM1\},
00036                     chunk[2] = \{CHUNK0, CHUNK1\};
00037     \textcolor{keywordtype}{size\_t}          nelmts;
00038     \textcolor{keywordtype}{unsigned} \textcolor{keywordtype}{int}    flags,
00039                     filter\_info;
00040     \textcolor{keywordtype}{int}             wdata[DIM0][DIM1],          \textcolor{comment}{/* Write buffer */}
00041                     rdata[DIM0][DIM1],          \textcolor{comment}{/* Read buffer */}
00042                     max,
00043                     i, j;
00044 
00045     \textcolor{comment}{/*}
00046 \textcolor{comment}{     * Check if the Fletcher32 filter is available and can be used for}
00047 \textcolor{comment}{     * both encoding and decoding.  Normally we do not perform error}
00048 \textcolor{comment}{     * checking in these examples for the sake of clarity, but in this}
00049 \textcolor{comment}{     * case we will make an exception because this filter is an}
00050 \textcolor{comment}{     * optional part of the hdf5 library.}
00051 \textcolor{comment}{     */}
00052     avail = H5Zfilter\_avail(H5Z\_FILTER\_FLETCHER32);
00053     \textcolor{keywordflow}{if} (!avail) \{
00054         printf (\textcolor{stringliteral}{"Fletcher32 filter not available.\(\backslash\)n"});
00055         \textcolor{keywordflow}{return} 1;
00056     \}
00057     status = H5Zget\_filter\_info (H5Z\_FILTER\_FLETCHER32, &filter\_info);
00058     \textcolor{keywordflow}{if} ( !(filter\_info & H5Z\_FILTER\_CONFIG\_ENCODE\_ENABLED) ||
00059                 !(filter\_info & H5Z\_FILTER\_CONFIG\_DECODE\_ENABLED) ) \{
00060         printf (\textcolor{stringliteral}{"Fletcher32 filter not available for encoding and decoding.\(\backslash\)n"});
00061         \textcolor{keywordflow}{return} 1;
00062     \}
00063 
00064     \textcolor{comment}{/*}
00065 \textcolor{comment}{     * Initialize data.}
00066 \textcolor{comment}{     */}
00067     \textcolor{keywordflow}{for} (i=0; i<DIM0; i++)
00068         \textcolor{keywordflow}{for} (j=0; j<DIM1; j++)
00069             wdata[i][j] = i * j - j;
00070 
00071     \textcolor{comment}{/*}
00072 \textcolor{comment}{     * Create a new file using the default properties.}
00073 \textcolor{comment}{     */}
00074     file = H5Fcreate (FILE, H5F\_ACC\_TRUNC, H5P\_DEFAULT, H5P\_DEFAULT);
00075 
00076     \textcolor{comment}{/*}
00077 \textcolor{comment}{     * Create dataspace.  Setting maximum size to NULL sets the maximum}
00078 \textcolor{comment}{     * size to be the current size.}
00079 \textcolor{comment}{     */}
00080     space = H5Screate\_simple (2, dims, NULL);
00081 
00082     \textcolor{comment}{/*}
00083 \textcolor{comment}{     * Create the dataset creation property list, add the Fletcher32 filter}
00084 \textcolor{comment}{     * and set the chunk size.}
00085 \textcolor{comment}{     */}
00086     dcpl = H5Pcreate (H5P\_DATASET\_CREATE);
00087     status = H5Pset\_fletcher32 (dcpl);
00088     status = H5Pset\_chunk (dcpl, 2, chunk);
00089 
00090     \textcolor{comment}{/*}
00091 \textcolor{comment}{     * Create the dataset.}
00092 \textcolor{comment}{     */}
00093     dset = H5Dcreate (file, DATASET, H5T\_STD\_I32LE, space, H5P\_DEFAULT, dcpl,
00094                 H5P\_DEFAULT);
00095 
00096     \textcolor{comment}{/*}
00097 \textcolor{comment}{     * Write the data to the dataset.}
00098 \textcolor{comment}{     */}
00099     status = H5Dwrite (dset, H5T\_NATIVE\_INT, H5S\_ALL, H5S\_ALL, H5P\_DEFAULT,
00100                 wdata[0]);
00101 
00102     \textcolor{comment}{/*}
00103 \textcolor{comment}{     * Close and release resources.}
00104 \textcolor{comment}{     */}
00105     status = H5Pclose (dcpl);
00106     status = H5Dclose (dset);
00107     status = H5Sclose (space);
00108     status = H5Fclose (file);
00109 
00110 
00111     \textcolor{comment}{/*}
00112 \textcolor{comment}{     * Now we begin the read section of this example.}
00113 \textcolor{comment}{     */}
00114 
00115     \textcolor{comment}{/*}
00116 \textcolor{comment}{     * Open file and dataset using the default properties.}
00117 \textcolor{comment}{     */}
00118     file = H5Fopen (FILE, H5F\_ACC\_RDONLY, H5P\_DEFAULT);
00119     dset = H5Dopen (file, DATASET, H5P\_DEFAULT);
00120 
00121     \textcolor{comment}{/*}
00122 \textcolor{comment}{     * Retrieve dataset creation property list.}
00123 \textcolor{comment}{     */}
00124     dcpl = H5Dget\_create\_plist (dset);
00125 
00126     \textcolor{comment}{/*}
00127 \textcolor{comment}{     * Retrieve and print the filter type.  Here we only retrieve the}
00128 \textcolor{comment}{     * first filter because we know that we only added one filter.}
00129 \textcolor{comment}{     */}
00130     nelmts = 0;
00131     filter\_type = H5Pget\_filter (dcpl, 0, &flags, &nelmts, NULL, 0, NULL,
00132                 &filter\_info);
00133     printf (\textcolor{stringliteral}{"Filter type is: "});
00134     \textcolor{keywordflow}{switch} (filter\_type) \{
00135         \textcolor{keywordflow}{case} H5Z\_FILTER\_DEFLATE:
00136             printf (\textcolor{stringliteral}{"H5Z\_FILTER\_DEFLATE\(\backslash\)n"});
00137             \textcolor{keywordflow}{break};
00138         \textcolor{keywordflow}{case} H5Z\_FILTER\_SHUFFLE:
00139             printf (\textcolor{stringliteral}{"H5Z\_FILTER\_SHUFFLE\(\backslash\)n"});
00140             \textcolor{keywordflow}{break};
00141         \textcolor{keywordflow}{case} H5Z\_FILTER\_FLETCHER32:
00142             printf (\textcolor{stringliteral}{"H5Z\_FILTER\_FLETCHER32\(\backslash\)n"});
00143             \textcolor{keywordflow}{break};
00144         \textcolor{keywordflow}{case} H5Z\_FILTER\_SZIP:
00145             printf (\textcolor{stringliteral}{"H5Z\_FILTER\_SZIP\(\backslash\)n"});
00146             \textcolor{keywordflow}{break};
00147         \textcolor{keywordflow}{case} H5Z\_FILTER\_NBIT:
00148             printf (\textcolor{stringliteral}{"H5Z\_FILTER\_NBIT\(\backslash\)n"});
00149             \textcolor{keywordflow}{break};
00150         \textcolor{keywordflow}{case} H5Z\_FILTER\_SCALEOFFSET:
00151             printf (\textcolor{stringliteral}{"H5Z\_FILTER\_SCALEOFFSET\(\backslash\)n"});
00152     \}
00153 
00154     \textcolor{comment}{/*}
00155 \textcolor{comment}{     * Read the data using the default properties.}
00156 \textcolor{comment}{     */}
00157     status = H5Dread (dset, H5T\_NATIVE\_INT, H5S\_ALL, H5S\_ALL, H5P\_DEFAULT,
00158                 rdata[0]);
00159 
00160     \textcolor{comment}{/*}
00161 \textcolor{comment}{     * Check if the read was successful.  Normally we do not perform}
00162 \textcolor{comment}{     * error checking in these examples for the sake of clarity, but in}
00163 \textcolor{comment}{     * this case we will make an exception because this is how the}
00164 \textcolor{comment}{     * fletcher32 checksum filter reports data errors.}
00165 \textcolor{comment}{     */}
00166     \textcolor{keywordflow}{if} (status<0) \{
00167         fprintf (stderr, \textcolor{stringliteral}{"Dataset read failed!\(\backslash\)n"});
00168         status = H5Pclose (dcpl);
00169         status = H5Dclose (dset);
00170         status = H5Fclose (file);
00171         \textcolor{keywordflow}{return} 2;
00172     \}
00173 
00174     \textcolor{comment}{/*}
00175 \textcolor{comment}{     * Find the maximum value in the dataset, to verify that it was}
00176 \textcolor{comment}{     * read correctly.}
00177 \textcolor{comment}{     */}
00178     max = rdata[0][0];
00179     \textcolor{keywordflow}{for} (i=0; i<DIM0; i++)
00180         \textcolor{keywordflow}{for} (j=0; j<DIM1; j++)
00181             \textcolor{keywordflow}{if} (max < rdata[i][j])
00182                 max = rdata[i][j];
00183 
00184     \textcolor{comment}{/*}
00185 \textcolor{comment}{     * Print the maximum value.}
00186 \textcolor{comment}{     */}
00187     printf (\textcolor{stringliteral}{"Maximum value in %s is: %d\(\backslash\)n"}, DATASET, max);
00188 
00189     \textcolor{comment}{/*}
00190 \textcolor{comment}{     * Close and release resources.}
00191 \textcolor{comment}{     */}
00192     status = H5Pclose (dcpl);
00193     status = H5Dclose (dset);
00194     status = H5Fclose (file);
00195 
00196     \textcolor{keywordflow}{return} 0;
00197 \}
\end{DoxyCode}
