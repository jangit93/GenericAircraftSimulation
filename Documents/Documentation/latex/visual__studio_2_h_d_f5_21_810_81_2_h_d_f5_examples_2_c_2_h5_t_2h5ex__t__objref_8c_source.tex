\hypertarget{visual__studio_2_h_d_f5_21_810_81_2_h_d_f5_examples_2_c_2_h5_t_2h5ex__t__objref_8c_source}{}\section{visual\+\_\+studio/\+H\+D\+F5/1.10.1/\+H\+D\+F5\+Examples/\+C/\+H5\+T/h5ex\+\_\+t\+\_\+objref.c}
\label{visual__studio_2_h_d_f5_21_810_81_2_h_d_f5_examples_2_c_2_h5_t_2h5ex__t__objref_8c_source}\index{h5ex\+\_\+t\+\_\+objref.\+c@{h5ex\+\_\+t\+\_\+objref.\+c}}

\begin{DoxyCode}
00001 \textcolor{comment}{/************************************************************}
00002 \textcolor{comment}{}
00003 \textcolor{comment}{  This example shows how to read and write object references}
00004 \textcolor{comment}{  to a dataset.  The program first creates objects in the}
00005 \textcolor{comment}{  file and writes references to those objects to a dataset}
00006 \textcolor{comment}{  with a dataspace of DIM0, then closes the file.  Next, it}
00007 \textcolor{comment}{  reopens the file, dereferences the references, and outputs}
00008 \textcolor{comment}{  the names of their targets to the screen.}
00009 \textcolor{comment}{}
00010 \textcolor{comment}{  This file is intended for use with HDF5 Library version 1.8}
00011 \textcolor{comment}{}
00012 \textcolor{comment}{ ************************************************************/}
00013 
00014 \textcolor{preprocessor}{#include "hdf5.h"}
00015 \textcolor{preprocessor}{#include <stdio.h>}
00016 \textcolor{preprocessor}{#include <stdlib.h>}
00017 
00018 \textcolor{preprocessor}{#define FILE            "h5ex\_t\_objref.h5"}
00019 \textcolor{preprocessor}{#define DATASET         "DS1"}
00020 \textcolor{preprocessor}{#define DIM0            2}
00021 
00022 \textcolor{keywordtype}{int}
00023 main (\textcolor{keywordtype}{void})
00024 \{
00025     hid\_t       \hyperlink{structfile}{file}, space, dset, obj;     \textcolor{comment}{/* Handles */}
00026     herr\_t      status;
00027     hsize\_t     dims[1] = \{DIM0\};
00028     hobj\_ref\_t  wdata[DIM0],                \textcolor{comment}{/* Write buffer */}
00029                 *rdata;                     \textcolor{comment}{/* Read buffer */}
00030     H5O\_type\_t  objtype;
00031     ssize\_t     size;
00032     \textcolor{keywordtype}{char}        *name;
00033     \textcolor{keywordtype}{int}         ndims,
00034                 i;
00035 
00036     \textcolor{comment}{/*}
00037 \textcolor{comment}{     * Create a new file using the default properties.}
00038 \textcolor{comment}{     */}
00039     file = H5Fcreate (FILE, H5F\_ACC\_TRUNC, H5P\_DEFAULT, H5P\_DEFAULT);
00040 
00041     \textcolor{comment}{/*}
00042 \textcolor{comment}{     * Create a dataset with a null dataspace.}
00043 \textcolor{comment}{     */}
00044     space = H5Screate (H5S\_NULL);
00045     obj = H5Dcreate (file, \textcolor{stringliteral}{"DS2"}, H5T\_STD\_I32LE, space, H5P\_DEFAULT,
00046                 H5P\_DEFAULT, H5P\_DEFAULT);
00047     status = H5Dclose (obj);
00048     status = H5Sclose (space);
00049 
00050     \textcolor{comment}{/*}
00051 \textcolor{comment}{     * Create a group.}
00052 \textcolor{comment}{     */}
00053     obj = H5Gcreate (file, \textcolor{stringliteral}{"G1"}, H5P\_DEFAULT, H5P\_DEFAULT, H5P\_DEFAULT);
00054     status = H5Gclose (obj);
00055 
00056     \textcolor{comment}{/*}
00057 \textcolor{comment}{     * Create references to the previously created objects.  Passing -1}
00058 \textcolor{comment}{     * as space\_id causes this parameter to be ignored.  Other values}
00059 \textcolor{comment}{     * besides valid dataspaces result in an error.}
00060 \textcolor{comment}{     */}
00061     status = H5Rcreate (&wdata[0], file, \textcolor{stringliteral}{"G1"}, H5R\_OBJECT, -1);
00062     status = H5Rcreate (&wdata[1], file, \textcolor{stringliteral}{"DS2"}, H5R\_OBJECT, -1);
00063 
00064     \textcolor{comment}{/*}
00065 \textcolor{comment}{     * Create dataspace.  Setting maximum size to NULL sets the maximum}
00066 \textcolor{comment}{     * size to be the current size.}
00067 \textcolor{comment}{     */}
00068     space = H5Screate\_simple (1, dims, NULL);
00069 
00070     \textcolor{comment}{/*}
00071 \textcolor{comment}{     * Create the dataset and write the object references to it.}
00072 \textcolor{comment}{     */}
00073     dset = H5Dcreate (file, DATASET, H5T\_STD\_REF\_OBJ, space, H5P\_DEFAULT,
00074                 H5P\_DEFAULT, H5P\_DEFAULT);
00075     status = H5Dwrite (dset, H5T\_STD\_REF\_OBJ, H5S\_ALL, H5S\_ALL, H5P\_DEFAULT,
00076                 wdata);
00077 
00078     \textcolor{comment}{/*}
00079 \textcolor{comment}{     * Close and release resources.}
00080 \textcolor{comment}{     */}
00081     status = H5Dclose (dset);
00082     status = H5Sclose (space);
00083     status = H5Fclose (file);
00084 
00085 
00086     \textcolor{comment}{/*}
00087 \textcolor{comment}{     * Now we begin the read section of this example.  Here we assume}
00088 \textcolor{comment}{     * the dataset has the same name and rank, but can have any size.}
00089 \textcolor{comment}{     * Therefore we must allocate a new array to read in data using}
00090 \textcolor{comment}{     * malloc().}
00091 \textcolor{comment}{     */}
00092 
00093     \textcolor{comment}{/*}
00094 \textcolor{comment}{     * Open file and dataset.}
00095 \textcolor{comment}{     */}
00096     file = H5Fopen (FILE, H5F\_ACC\_RDONLY, H5P\_DEFAULT);
00097     dset = H5Dopen (file, DATASET, H5P\_DEFAULT);
00098 
00099     \textcolor{comment}{/*}
00100 \textcolor{comment}{     * Get dataspace and allocate memory for read buffer.}
00101 \textcolor{comment}{     */}
00102     space = H5Dget\_space (dset);
00103     ndims = H5Sget\_simple\_extent\_dims (space, dims, NULL);
00104     rdata = (hobj\_ref\_t *) malloc (dims[0] * \textcolor{keyword}{sizeof} (hobj\_ref\_t));
00105 
00106     \textcolor{comment}{/*}
00107 \textcolor{comment}{     * Read the data.}
00108 \textcolor{comment}{     */}
00109     status = H5Dread (dset, H5T\_STD\_REF\_OBJ, H5S\_ALL, H5S\_ALL, H5P\_DEFAULT,
00110                 rdata);
00111 
00112     \textcolor{comment}{/*}
00113 \textcolor{comment}{     * Output the data to the screen.}
00114 \textcolor{comment}{     */}
00115     \textcolor{keywordflow}{for} (i=0; i<dims[0]; i++) \{
00116         printf (\textcolor{stringliteral}{"%s[%d]:\(\backslash\)n  ->"}, DATASET, i);
00117 
00118         \textcolor{comment}{/*}
00119 \textcolor{comment}{         * Open the referenced object, get its name and type.}
00120 \textcolor{comment}{         */}
00121         obj = H5Rdereference (dset, H5P\_DEFAULT, H5R\_OBJECT, &rdata[i]);
00122         status = H5Rget\_obj\_type (dset, H5R\_OBJECT, &rdata[i], &objtype);
00123 
00124         \textcolor{comment}{/*}
00125 \textcolor{comment}{         * Get the length of the name, allocate space, then retrieve}
00126 \textcolor{comment}{         * the name.}
00127 \textcolor{comment}{         */}
00128         size = 1 + H5Iget\_name (obj, NULL, 0);
00129         name = (\textcolor{keywordtype}{char} *) malloc (size);
00130         size = H5Iget\_name (obj, name, size);
00131 
00132         \textcolor{comment}{/*}
00133 \textcolor{comment}{         * Print the object type and close the object.}
00134 \textcolor{comment}{         */}
00135         \textcolor{keywordflow}{switch} (objtype) \{
00136             \textcolor{keywordflow}{case} H5O\_TYPE\_GROUP:
00137                 printf (\textcolor{stringliteral}{"Group"});
00138                 \textcolor{keywordflow}{break};
00139             \textcolor{keywordflow}{case} H5O\_TYPE\_DATASET:
00140                 printf (\textcolor{stringliteral}{"Dataset"});
00141                 \textcolor{keywordflow}{break};
00142             \textcolor{keywordflow}{case} H5O\_TYPE\_NAMED\_DATATYPE:
00143                 printf (\textcolor{stringliteral}{"Named Datatype"});
00144                 \textcolor{keywordflow}{break};
00145             \textcolor{keywordflow}{case} H5O\_TYPE\_UNKNOWN:
00146             \textcolor{keywordflow}{case} H5O\_TYPE\_NTYPES:
00147                 printf (\textcolor{stringliteral}{"Unknown"});
00148         \}
00149         status = H5Oclose (obj);
00150 
00151         \textcolor{comment}{/*}
00152 \textcolor{comment}{         * Print the name and deallocate space for the name.}
00153 \textcolor{comment}{         */}
00154         printf (\textcolor{stringliteral}{": %s\(\backslash\)n"}, name);
00155         free (name);
00156     \}
00157 
00158     \textcolor{comment}{/*}
00159 \textcolor{comment}{     * Close and release resources.}
00160 \textcolor{comment}{     */}
00161     free (rdata);
00162     status = H5Dclose (dset);
00163     status = H5Sclose (space);
00164     status = H5Fclose (file);
00165 
00166     \textcolor{keywordflow}{return} 0;
00167 \}
\end{DoxyCode}
