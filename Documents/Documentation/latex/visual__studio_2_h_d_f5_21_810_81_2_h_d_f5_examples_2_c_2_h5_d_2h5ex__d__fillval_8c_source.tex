\hypertarget{visual__studio_2_h_d_f5_21_810_81_2_h_d_f5_examples_2_c_2_h5_d_2h5ex__d__fillval_8c_source}{}\section{visual\+\_\+studio/\+H\+D\+F5/1.10.1/\+H\+D\+F5\+Examples/\+C/\+H5\+D/h5ex\+\_\+d\+\_\+fillval.c}
\label{visual__studio_2_h_d_f5_21_810_81_2_h_d_f5_examples_2_c_2_h5_d_2h5ex__d__fillval_8c_source}\index{h5ex\+\_\+d\+\_\+fillval.\+c@{h5ex\+\_\+d\+\_\+fillval.\+c}}

\begin{DoxyCode}
00001 \textcolor{comment}{/************************************************************}
00002 \textcolor{comment}{}
00003 \textcolor{comment}{  This example shows how to set the fill value for a}
00004 \textcolor{comment}{  dataset.  The program first sets the fill value to}
00005 \textcolor{comment}{  FILLVAL, creates a dataset with dimensions of DIM0xDIM1,}
00006 \textcolor{comment}{  reads from the uninitialized dataset, and outputs the}
00007 \textcolor{comment}{  contents to the screen.  Next, it writes integers to the}
00008 \textcolor{comment}{  dataset, reads the data back, and outputs it to the}
00009 \textcolor{comment}{  screen.  Finally it extends the dataset, reads from it,}
00010 \textcolor{comment}{  and outputs the result to the screen.}
00011 \textcolor{comment}{}
00012 \textcolor{comment}{  This file is intended for use with HDF5 Library version 1.8}
00013 \textcolor{comment}{}
00014 \textcolor{comment}{ ************************************************************/}
00015 
00016 \textcolor{preprocessor}{#include "hdf5.h"}
00017 \textcolor{preprocessor}{#include <stdio.h>}
00018 \textcolor{preprocessor}{#include <stdlib.h>}
00019 
00020 \textcolor{preprocessor}{#define FILE            "h5ex\_d\_fillval.h5"}
00021 \textcolor{preprocessor}{#define DATASET         "DS1"}
00022 \textcolor{preprocessor}{#define DIM0            4}
00023 \textcolor{preprocessor}{#define DIM1            7}
00024 \textcolor{preprocessor}{#define EDIM0           6}
00025 \textcolor{preprocessor}{#define EDIM1           10}
00026 \textcolor{preprocessor}{#define CHUNK0          4}
00027 \textcolor{preprocessor}{#define CHUNK1          4}
00028 \textcolor{preprocessor}{#define FILLVAL         99}
00029 
00030 \textcolor{keywordtype}{int}
00031 main (\textcolor{keywordtype}{void})
00032 \{
00033     hid\_t           \hyperlink{structfile}{file}, space, dset, dcpl;    \textcolor{comment}{/* Handles */}
00034     herr\_t          status;
00035     hsize\_t         dims[2] = \{DIM0, DIM1\},
00036                     extdims[2] = \{EDIM0, EDIM1\},
00037                     maxdims[2] = \{H5S\_UNLIMITED, H5S\_UNLIMITED\},
00038                     chunk[2] = \{CHUNK0, CHUNK1\};
00039     \textcolor{keywordtype}{int}             wdata[DIM0][DIM1],          \textcolor{comment}{/* Write buffer */}
00040                     rdata[DIM0][DIM1],          \textcolor{comment}{/* Read buffer */}
00041                     rdata2[EDIM0][EDIM1],       \textcolor{comment}{/* Read buffer for}
00042 \textcolor{comment}{                                                   extenstion */}
00043                     fillval,
00044                     i, j;
00045 
00046     \textcolor{comment}{/*}
00047 \textcolor{comment}{     * Initialize data.}
00048 \textcolor{comment}{     */}
00049     \textcolor{keywordflow}{for} (i=0; i<DIM0; i++)
00050         \textcolor{keywordflow}{for} (j=0; j<DIM1; j++)
00051             wdata[i][j] = i * j - j;
00052 
00053     \textcolor{comment}{/*}
00054 \textcolor{comment}{     * Create a new file using the default properties.}
00055 \textcolor{comment}{     */}
00056     file = H5Fcreate (FILE, H5F\_ACC\_TRUNC, H5P\_DEFAULT, H5P\_DEFAULT);
00057 
00058     \textcolor{comment}{/*}
00059 \textcolor{comment}{     * Create dataspace with unlimited dimensions.}
00060 \textcolor{comment}{     */}
00061     space = H5Screate\_simple (2, dims, maxdims);
00062 
00063     \textcolor{comment}{/*}
00064 \textcolor{comment}{     * Create the dataset creation property list, and set the chunk}
00065 \textcolor{comment}{     * size.}
00066 \textcolor{comment}{     */}
00067     dcpl = H5Pcreate (H5P\_DATASET\_CREATE);
00068     status = H5Pset\_chunk (dcpl, 2, chunk);
00069 
00070     \textcolor{comment}{/*}
00071 \textcolor{comment}{     * Set the fill value for the dataset.}
00072 \textcolor{comment}{     */}
00073     fillval = FILLVAL;
00074     status = H5Pset\_fill\_value (dcpl, H5T\_NATIVE\_INT, &fillval);
00075 
00076     \textcolor{comment}{/*}
00077 \textcolor{comment}{     * Set the allocation time to "early".  This way we can be sure}
00078 \textcolor{comment}{     * that reading from the dataset immediately after creation will}
00079 \textcolor{comment}{     * return the fill value.}
00080 \textcolor{comment}{     */}
00081     status = H5Pset\_alloc\_time (dcpl, H5D\_ALLOC\_TIME\_EARLY);
00082 
00083     \textcolor{comment}{/*}
00084 \textcolor{comment}{     * Create the dataset using the dataset creation property list.}
00085 \textcolor{comment}{     */}
00086     dset = H5Dcreate (file, DATASET, H5T\_STD\_I32LE, space, H5P\_DEFAULT, dcpl,
00087                 H5P\_DEFAULT);
00088 
00089     \textcolor{comment}{/*}
00090 \textcolor{comment}{     * Read values from the dataset, which has not been written to yet.}
00091 \textcolor{comment}{     */}
00092     status = H5Dread (dset, H5T\_NATIVE\_INT, H5S\_ALL, H5S\_ALL, H5P\_DEFAULT,
00093                 rdata[0]);
00094 
00095     \textcolor{comment}{/*}
00096 \textcolor{comment}{     * Output the data to the screen.}
00097 \textcolor{comment}{     */}
00098     printf (\textcolor{stringliteral}{"Dataset before being written to:\(\backslash\)n"});
00099     \textcolor{keywordflow}{for} (i=0; i<dims[0]; i++) \{
00100         printf (\textcolor{stringliteral}{" ["});
00101         \textcolor{keywordflow}{for} (j=0; j<dims[1]; j++)
00102             printf (\textcolor{stringliteral}{" %3d"}, rdata[i][j]);
00103         printf (\textcolor{stringliteral}{"]\(\backslash\)n"});
00104     \}
00105 
00106     \textcolor{comment}{/*}
00107 \textcolor{comment}{     * Write the data to the dataset.}
00108 \textcolor{comment}{     */}
00109     status = H5Dwrite (dset, H5T\_NATIVE\_INT, H5S\_ALL, H5S\_ALL, H5P\_DEFAULT,
00110                 wdata[0]);
00111 
00112     \textcolor{comment}{/*}
00113 \textcolor{comment}{     * Read the data back.}
00114 \textcolor{comment}{     */}
00115     status = H5Dread (dset, H5T\_NATIVE\_INT, H5S\_ALL, H5S\_ALL, H5P\_DEFAULT,
00116                 rdata[0]);
00117 
00118     \textcolor{comment}{/*}
00119 \textcolor{comment}{     * Output the data to the screen.}
00120 \textcolor{comment}{     */}
00121     printf (\textcolor{stringliteral}{"\(\backslash\)nDataset after being written to:\(\backslash\)n"});
00122     \textcolor{keywordflow}{for} (i=0; i<dims[0]; i++) \{
00123         printf (\textcolor{stringliteral}{" ["});
00124         \textcolor{keywordflow}{for} (j=0; j<dims[1]; j++)
00125             printf (\textcolor{stringliteral}{" %3d"}, rdata[i][j]);
00126         printf (\textcolor{stringliteral}{"]\(\backslash\)n"});
00127     \}
00128 
00129     \textcolor{comment}{/*}
00130 \textcolor{comment}{     * Extend the dataset.}
00131 \textcolor{comment}{     */}
00132     status = H5Dset\_extent (dset, extdims);
00133 
00134     \textcolor{comment}{/*}
00135 \textcolor{comment}{     * Read from the extended dataset.}
00136 \textcolor{comment}{     */}
00137     status = H5Dread (dset, H5T\_NATIVE\_INT, H5S\_ALL, H5S\_ALL, H5P\_DEFAULT,
00138                 rdata2[0]);
00139 
00140     \textcolor{comment}{/*}
00141 \textcolor{comment}{     * Output the data to the screen.}
00142 \textcolor{comment}{     */}
00143     printf (\textcolor{stringliteral}{"\(\backslash\)nDataset after extension:\(\backslash\)n"});
00144     \textcolor{keywordflow}{for} (i=0; i<extdims[0]; i++) \{
00145         printf (\textcolor{stringliteral}{" ["});
00146         \textcolor{keywordflow}{for} (j=0; j<extdims[1]; j++)
00147             printf (\textcolor{stringliteral}{" %3d"}, rdata2[i][j]);
00148         printf (\textcolor{stringliteral}{"]\(\backslash\)n"});
00149     \}
00150 
00151     \textcolor{comment}{/*}
00152 \textcolor{comment}{     * Close and release resources.}
00153 \textcolor{comment}{     */}
00154     status = H5Pclose (dcpl);
00155     status = H5Dclose (dset);
00156     status = H5Sclose (space);
00157     status = H5Fclose (file);
00158 
00159     \textcolor{keywordflow}{return} 0;
00160 \}
\end{DoxyCode}
