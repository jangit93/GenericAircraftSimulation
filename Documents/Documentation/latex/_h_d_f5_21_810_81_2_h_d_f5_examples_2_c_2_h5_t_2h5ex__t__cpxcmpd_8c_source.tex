\hypertarget{_h_d_f5_21_810_81_2_h_d_f5_examples_2_c_2_h5_t_2h5ex__t__cpxcmpd_8c_source}{}\section{H\+D\+F5/1.10.1/\+H\+D\+F5\+Examples/\+C/\+H5\+T/h5ex\+\_\+t\+\_\+cpxcmpd.c}
\label{_h_d_f5_21_810_81_2_h_d_f5_examples_2_c_2_h5_t_2h5ex__t__cpxcmpd_8c_source}\index{h5ex\+\_\+t\+\_\+cpxcmpd.\+c@{h5ex\+\_\+t\+\_\+cpxcmpd.\+c}}

\begin{DoxyCode}
00001 \textcolor{comment}{/************************************************************}
00002 \textcolor{comment}{}
00003 \textcolor{comment}{  This example shows how to read and write a complex}
00004 \textcolor{comment}{  compound datatype to a dataset.  The program first writes}
00005 \textcolor{comment}{  complex compound structures to a dataset with a dataspace}
00006 \textcolor{comment}{  of DIM0, then closes the file.  Next, it reopens the file,}
00007 \textcolor{comment}{  reads back selected fields in the structure, and outputs}
00008 \textcolor{comment}{  them to the screen.}
00009 \textcolor{comment}{}
00010 \textcolor{comment}{  Unlike the other datatype examples, in this example we}
00011 \textcolor{comment}{  save to the file using native datatypes to simplify the}
00012 \textcolor{comment}{  type definitions here.  To save using standard types you}
00013 \textcolor{comment}{  must manually calculate the sizes and offsets of compound}
00014 \textcolor{comment}{  types as shown in h5ex\_t\_cmpd.c, and convert enumerated}
00015 \textcolor{comment}{  values as shown in h5ex\_t\_enum.c.}
00016 \textcolor{comment}{}
00017 \textcolor{comment}{  The datatype defined here consists of a compound}
00018 \textcolor{comment}{  containing a variable-length list of compound types, as}
00019 \textcolor{comment}{  well as a variable-length string, enumeration, double}
00020 \textcolor{comment}{  array, object reference and region reference.  The nested}
00021 \textcolor{comment}{  compound type contains an int, variable-length string and}
00022 \textcolor{comment}{  two doubles.}
00023 \textcolor{comment}{}
00024 \textcolor{comment}{  This file is intended for use with HDF5 Library version 1.8}
00025 \textcolor{comment}{}
00026 \textcolor{comment}{ ************************************************************/}
00027 
00028 \textcolor{preprocessor}{#include "hdf5.h"}
00029 \textcolor{preprocessor}{#include <stdio.h>}
00030 \textcolor{preprocessor}{#include <stdlib.h>}
00031 
00032 \textcolor{preprocessor}{#define FILE            "h5ex\_t\_cpxcmpd.h5"}
00033 \textcolor{preprocessor}{#define DATASET         "DS1"}
00034 \textcolor{preprocessor}{#define DIM0            2}
00035 
00036 \textcolor{keyword}{typedef} \textcolor{keyword}{struct }\{
00037     \textcolor{keywordtype}{int}     serial\_no;
00038     \textcolor{keywordtype}{char}    *location;
00039     \textcolor{keywordtype}{double}  temperature;
00040     \textcolor{keywordtype}{double}  pressure;
00041 \} \hyperlink{structsensor__t}{sensor\_t};                                 \textcolor{comment}{/* Nested compound type */}
00042 
00043 \textcolor{keyword}{typedef} \textcolor{keyword}{enum} \{
00044     RED,
00045     GREEN,
00046     BLUE
00047 \} color\_t  ;                                \textcolor{comment}{/* Enumerated type */}
00048 
\Hypertarget{_h_d_f5_21_810_81_2_h_d_f5_examples_2_c_2_h5_t_2h5ex__t__cpxcmpd_8c_source_l00049}\hyperlink{structvehicle__t}{00049} \textcolor{keyword}{typedef} \textcolor{keyword}{struct }\{
00050     \hyperlink{structhvl__t}{hvl\_t}               sensors;
00051     \textcolor{keywordtype}{char}                *name;
00052     color\_t             color;
00053     \textcolor{keywordtype}{double}              location[3];
00054     hobj\_ref\_t          group;
00055     hdset\_reg\_ref\_t     surveyed\_areas;
00056 \} \hyperlink{structvehicle__t}{vehicle\_t};                                \textcolor{comment}{/* Main compound type */}
00057 
\Hypertarget{_h_d_f5_21_810_81_2_h_d_f5_examples_2_c_2_h5_t_2h5ex__t__cpxcmpd_8c_source_l00058}\hyperlink{structrvehicle__t}{00058} \textcolor{keyword}{typedef} \textcolor{keyword}{struct }\{
00059     \hyperlink{structhvl__t}{hvl\_t}       sensors;
00060     \textcolor{keywordtype}{char}        *name;
00061 \} \hyperlink{structrvehicle__t}{rvehicle\_t};                               \textcolor{comment}{/* Read type */}
00062 
00063 \textcolor{keywordtype}{int}
00064 main (\textcolor{keywordtype}{void})
00065 \{
00066     hid\_t       \hyperlink{structfile}{file}, vehicletype, colortype, sensortype, sensorstype, loctype,
00067                 strtype, rvehicletype, rsensortype, rsensorstype, space, dset,
00068                 group;
00069                                             \textcolor{comment}{/* Handles */}
00070     herr\_t      status;
00071     hsize\_t     dims[1] = \{DIM0\},
00072                 adims[1] = \{3\},
00073                 adims2[2] = \{32, 32\},
00074                 start[2] = \{8, 26\},
00075                 count[2] = \{4, 3\},
00076                 coords[3][2] = \{ \{3, 2\},
00077                                  \{3, 3\},
00078                                  \{4, 4\} \};
00079     \hyperlink{structvehicle__t}{vehicle\_t}   wdata[2];                   \textcolor{comment}{/* Write buffer */}
00080     \hyperlink{structrvehicle__t}{rvehicle\_t}  *rdata;                     \textcolor{comment}{/* Read buffer */}
00081     color\_t     val;
00082     \hyperlink{structsensor__t}{sensor\_t}    *ptr;
00083     \textcolor{keywordtype}{double}      wdata2[32][32];
00084     \textcolor{keywordtype}{int}         ndims,
00085                 i, j;
00086 
00087     \textcolor{comment}{/*}
00088 \textcolor{comment}{     * Create a new file using the default properties.}
00089 \textcolor{comment}{     */}
00090     file = H5Fcreate (FILE, H5F\_ACC\_TRUNC, H5P\_DEFAULT, H5P\_DEFAULT);
00091 
00092     \textcolor{comment}{/*}
00093 \textcolor{comment}{     * Create dataset to use for region references.}
00094 \textcolor{comment}{     */}
00095     \textcolor{keywordflow}{for} (i=0; i<32; i++)
00096         \textcolor{keywordflow}{for} (j=0; j<32; j++)
00097             wdata2[i][j]= 70. + 0.1 * (i - 16.) + 0.1 * (j - 16.);
00098     space = H5Screate\_simple (2, adims2, NULL);
00099     dset = H5Dcreate (file, \textcolor{stringliteral}{"Ambient\_Temperature"}, H5T\_NATIVE\_DOUBLE, space,
00100                 H5P\_DEFAULT, H5P\_DEFAULT, H5P\_DEFAULT);
00101     status = H5Dwrite (dset, H5T\_NATIVE\_DOUBLE, H5S\_ALL, H5S\_ALL, H5P\_DEFAULT,
00102                 wdata2[0]);
00103     status = H5Dclose (dset);
00104 
00105     \textcolor{comment}{/*}
00106 \textcolor{comment}{     * Create groups to use for object references.}
00107 \textcolor{comment}{     */}
00108     group = H5Gcreate (file, \textcolor{stringliteral}{"Land\_Vehicles"}, H5P\_DEFAULT, H5P\_DEFAULT,
00109                 H5P\_DEFAULT);
00110     status = H5Gclose (group);
00111     group = H5Gcreate (file, \textcolor{stringliteral}{"Air\_Vehicles"}, H5P\_DEFAULT, H5P\_DEFAULT,
00112                 H5P\_DEFAULT);
00113     status = H5Gclose (group);
00114 
00115     \textcolor{comment}{/*}
00116 \textcolor{comment}{     * Initialize variable-length compound in the first data element.}
00117 \textcolor{comment}{     */}
00118     wdata[0].sensors.len = 4;
00119     ptr = (\hyperlink{structsensor__t}{sensor\_t} *) malloc (wdata[0].sensors.len * sizeof (\hyperlink{structsensor__t}{sensor\_t}));
00120     ptr[0].serial\_no = 1153;
00121     ptr[0].location = \textcolor{stringliteral}{"Exterior (static)"};
00122     ptr[0].temperature = 53.23;
00123     ptr[0].pressure = 24.57;
00124     ptr[1].serial\_no = 1184;
00125     ptr[1].location = \textcolor{stringliteral}{"Intake"};
00126     ptr[1].temperature = 55.12;
00127     ptr[1].pressure = 22.95;
00128     ptr[2].serial\_no = 1027;
00129     ptr[2].location = \textcolor{stringliteral}{"Intake manifold"};
00130     ptr[2].temperature = 103.55;
00131     ptr[2].pressure = 31.23;
00132     ptr[3].serial\_no = 1313;
00133     ptr[3].location = \textcolor{stringliteral}{"Exhaust manifold"};
00134     ptr[3].temperature = 1252.89;
00135     ptr[3].pressure = 84.11;
00136     wdata[0].sensors.p = (\textcolor{keywordtype}{void} *) ptr;
00137 
00138     \textcolor{comment}{/*}
00139 \textcolor{comment}{     * Initialize other fields in the first data element.}
00140 \textcolor{comment}{     */}
00141     wdata[0].name = \textcolor{stringliteral}{"Airplane"};
00142     wdata[0].color = GREEN;
00143     wdata[0].location[0] = -103234.21;
00144     wdata[0].location[1] = 422638.78;
00145     wdata[0].location[2] = 5996.43;
00146     status = H5Rcreate (&wdata[0].group, file, \textcolor{stringliteral}{"Air\_Vehicles"}, H5R\_OBJECT, -1);
00147     status = H5Sselect\_elements (space, H5S\_SELECT\_SET, 3, coords[0]);
00148     status = H5Rcreate (&wdata[0].surveyed\_areas, file, \textcolor{stringliteral}{"Ambient\_Temperature"},
00149                 H5R\_DATASET\_REGION, space);
00150 
00151     \textcolor{comment}{/*}
00152 \textcolor{comment}{     * Initialize variable-length compound in the second data element.}
00153 \textcolor{comment}{     */}
00154     wdata[1].sensors.len = 1;
00155     ptr = (\hyperlink{structsensor__t}{sensor\_t} *) malloc (wdata[1].sensors.len * sizeof (\hyperlink{structsensor__t}{sensor\_t}));
00156     ptr[0].serial\_no = 3244;
00157     ptr[0].location = \textcolor{stringliteral}{"Roof"};
00158     ptr[0].temperature = 83.82;
00159     ptr[0].pressure = 29.92;
00160     wdata[1].sensors.p = (\textcolor{keywordtype}{void} *) ptr;
00161 
00162     \textcolor{comment}{/*}
00163 \textcolor{comment}{     * Initialize other fields in the second data element.}
00164 \textcolor{comment}{     */}
00165     wdata[1].name = \textcolor{stringliteral}{"Automobile"};
00166     wdata[1].color = RED;
00167     wdata[1].location[0] = 326734.36;
00168     wdata[1].location[1] = 221568.23;
00169     wdata[1].location[2] = 432.36;
00170     status = H5Rcreate (&wdata[1].group, file, \textcolor{stringliteral}{"Land\_Vehicles"}, H5R\_OBJECT, -1);
00171     status = H5Sselect\_hyperslab (space, H5S\_SELECT\_SET, start, NULL, count,
00172                 NULL);
00173     status = H5Rcreate (&wdata[1].surveyed\_areas, file, \textcolor{stringliteral}{"Ambient\_Temperature"},
00174                 H5R\_DATASET\_REGION, space);
00175 
00176     status = H5Sclose (space);
00177 
00178     \textcolor{comment}{/*}
00179 \textcolor{comment}{     * Create variable-length string datatype.}
00180 \textcolor{comment}{     */}
00181     strtype = H5Tcopy (H5T\_C\_S1);
00182     status = H5Tset\_size (strtype, H5T\_VARIABLE);
00183 
00184     \textcolor{comment}{/*}
00185 \textcolor{comment}{     * Create the nested compound datatype.}
00186 \textcolor{comment}{     */}
00187     sensortype = H5Tcreate (H5T\_COMPOUND, \textcolor{keyword}{sizeof} (\hyperlink{structsensor__t}{sensor\_t}));
00188     status = H5Tinsert (sensortype, \textcolor{stringliteral}{"Serial number"},
00189                 HOFFSET (\hyperlink{structsensor__t}{sensor\_t}, serial\_no), H5T\_NATIVE\_INT);
00190     status = H5Tinsert (sensortype, \textcolor{stringliteral}{"Location"}, HOFFSET (\hyperlink{structsensor__t}{sensor\_t}, location),
00191                 strtype);
00192     status = H5Tinsert (sensortype, \textcolor{stringliteral}{"Temperature (F)"},
00193                 HOFFSET (\hyperlink{structsensor__t}{sensor\_t}, temperature), H5T\_NATIVE\_DOUBLE);
00194     status = H5Tinsert (sensortype, \textcolor{stringliteral}{"Pressure (inHg)"},
00195                 HOFFSET (\hyperlink{structsensor__t}{sensor\_t}, pressure), H5T\_NATIVE\_DOUBLE);
00196 
00197     \textcolor{comment}{/*}
00198 \textcolor{comment}{     * Create the variable-length datatype.}
00199 \textcolor{comment}{     */}
00200     sensorstype = H5Tvlen\_create (sensortype);
00201 
00202     \textcolor{comment}{/*}
00203 \textcolor{comment}{     * Create the enumerated datatype.}
00204 \textcolor{comment}{     */}
00205     colortype = H5Tenum\_create (H5T\_NATIVE\_INT);
00206     val = (color\_t) RED;
00207     status = H5Tenum\_insert (colortype, \textcolor{stringliteral}{"Red"}, &val);
00208     val = (color\_t) GREEN;
00209     status = H5Tenum\_insert (colortype, \textcolor{stringliteral}{"Green"}, &val);
00210     val = (color\_t) BLUE;
00211     status = H5Tenum\_insert (colortype, \textcolor{stringliteral}{"Blue"}, &val);
00212 
00213     \textcolor{comment}{/*}
00214 \textcolor{comment}{     * Create the array datatype.}
00215 \textcolor{comment}{     */}
00216     loctype = H5Tarray\_create (H5T\_NATIVE\_DOUBLE, 1, adims);
00217 
00218     \textcolor{comment}{/*}
00219 \textcolor{comment}{     * Create the main compound datatype.}
00220 \textcolor{comment}{     */}
00221     vehicletype = H5Tcreate (H5T\_COMPOUND, \textcolor{keyword}{sizeof} (\hyperlink{structvehicle__t}{vehicle\_t}));
00222     status = H5Tinsert (vehicletype, \textcolor{stringliteral}{"Sensors"}, HOFFSET (\hyperlink{structvehicle__t}{vehicle\_t}, sensors),
00223                 sensorstype);
00224     status = H5Tinsert (vehicletype, \textcolor{stringliteral}{"Name"}, HOFFSET (\hyperlink{structvehicle__t}{vehicle\_t}, name),
00225                 strtype);
00226     status = H5Tinsert (vehicletype, \textcolor{stringliteral}{"Color"}, HOFFSET (\hyperlink{structvehicle__t}{vehicle\_t}, color),
00227                 colortype);
00228     status = H5Tinsert (vehicletype, \textcolor{stringliteral}{"Location"}, HOFFSET (\hyperlink{structvehicle__t}{vehicle\_t}, location),
00229                 loctype);
00230     status = H5Tinsert (vehicletype, \textcolor{stringliteral}{"Group"}, HOFFSET (\hyperlink{structvehicle__t}{vehicle\_t}, group),
00231                 H5T\_STD\_REF\_OBJ);
00232     status = H5Tinsert (vehicletype, \textcolor{stringliteral}{"Surveyed areas"},
00233                 HOFFSET (\hyperlink{structvehicle__t}{vehicle\_t}, surveyed\_areas), H5T\_STD\_REF\_DSETREG);
00234 
00235     \textcolor{comment}{/*}
00236 \textcolor{comment}{     * Create dataspace.  Setting maximum size to NULL sets the maximum}
00237 \textcolor{comment}{     * size to be the current size.}
00238 \textcolor{comment}{     */}
00239     space = H5Screate\_simple (1, dims, NULL);
00240 
00241     \textcolor{comment}{/*}
00242 \textcolor{comment}{     * Create the dataset and write the compound data to it.}
00243 \textcolor{comment}{     */}
00244     dset = H5Dcreate (file, DATASET, vehicletype, space, H5P\_DEFAULT, H5P\_DEFAULT,
00245                 H5P\_DEFAULT);
00246     status = H5Dwrite (dset, vehicletype, H5S\_ALL, H5S\_ALL, H5P\_DEFAULT, wdata);
00247 
00248     \textcolor{comment}{/*}
00249 \textcolor{comment}{     * Close and release resources.  Note that we cannot use}
00250 \textcolor{comment}{     * H5Dvlen\_reclaim as it would attempt to free() the string}
00251 \textcolor{comment}{     * constants used to initialize the name fields in wdata.  We must}
00252 \textcolor{comment}{     * therefore manually free() only the data previously allocated}
00253 \textcolor{comment}{     * through malloc().}
00254 \textcolor{comment}{     */}
00255     \textcolor{keywordflow}{for} (i=0; i<dims[0]; i++)
00256         free (wdata[i].sensors.p);
00257     status = H5Dclose (dset);
00258     status = H5Sclose (space);
00259     status = H5Tclose (strtype);
00260     status = H5Tclose (sensortype);
00261     status = H5Tclose (sensorstype);
00262     status = H5Tclose (colortype);
00263     status = H5Tclose (loctype);
00264     status = H5Tclose (vehicletype);
00265     status = H5Fclose (file);
00266 
00267 
00268     \textcolor{comment}{/*}
00269 \textcolor{comment}{     * Now we begin the read section of this example.  Here we assume}
00270 \textcolor{comment}{     * the dataset has the same name and rank, but can have any size.}
00271 \textcolor{comment}{     * Therefore we must allocate a new array to read in data using}
00272 \textcolor{comment}{     * malloc().  We will only read back the variable length strings.}
00273 \textcolor{comment}{     */}
00274 
00275     \textcolor{comment}{/*}
00276 \textcolor{comment}{     * Open file and dataset.}
00277 \textcolor{comment}{     */}
00278     file = H5Fopen (FILE, H5F\_ACC\_RDONLY, H5P\_DEFAULT);
00279     dset = H5Dopen (file, DATASET, H5P\_DEFAULT);
00280 
00281     \textcolor{comment}{/*}
00282 \textcolor{comment}{     * Create variable-length string datatype.}
00283 \textcolor{comment}{     */}
00284     strtype = H5Tcopy (H5T\_C\_S1);
00285     status = H5Tset\_size (strtype, H5T\_VARIABLE);
00286 
00287     \textcolor{comment}{/*}
00288 \textcolor{comment}{     * Create the nested compound datatype for reading.  Even though it}
00289 \textcolor{comment}{     * has only one field, it must still be defined as a compound type}
00290 \textcolor{comment}{     * so the library can match the correct field in the file type.}
00291 \textcolor{comment}{     * This matching is done by name.  However, we do not need to}
00292 \textcolor{comment}{     * define a structure for the read buffer as we can simply treat it}
00293 \textcolor{comment}{     * as a char *.}
00294 \textcolor{comment}{     */}
00295     rsensortype = H5Tcreate (H5T\_COMPOUND, \textcolor{keyword}{sizeof} (\textcolor{keywordtype}{char} *));
00296     status = H5Tinsert (rsensortype, \textcolor{stringliteral}{"Location"}, 0, strtype);
00297 
00298     \textcolor{comment}{/*}
00299 \textcolor{comment}{     * Create the variable-length datatype for reading.}
00300 \textcolor{comment}{     */}
00301     rsensorstype = H5Tvlen\_create (rsensortype);
00302 
00303     \textcolor{comment}{/*}
00304 \textcolor{comment}{     * Create the main compound datatype for reading.}
00305 \textcolor{comment}{     */}
00306     rvehicletype = H5Tcreate (H5T\_COMPOUND, \textcolor{keyword}{sizeof} (\hyperlink{structrvehicle__t}{rvehicle\_t}));
00307     status = H5Tinsert (rvehicletype, \textcolor{stringliteral}{"Sensors"}, HOFFSET (\hyperlink{structrvehicle__t}{rvehicle\_t}, sensors),
00308                 rsensorstype);
00309     status = H5Tinsert (rvehicletype, \textcolor{stringliteral}{"Name"}, HOFFSET (\hyperlink{structrvehicle__t}{rvehicle\_t}, name),
00310                 strtype);
00311 
00312     \textcolor{comment}{/*}
00313 \textcolor{comment}{     * Get dataspace and allocate memory for read buffer.}
00314 \textcolor{comment}{     */}
00315     space = H5Dget\_space (dset);
00316     ndims = H5Sget\_simple\_extent\_dims (space, dims, NULL);
00317     rdata = (\hyperlink{structrvehicle__t}{rvehicle\_t} *) malloc (dims[0] * \textcolor{keyword}{sizeof} (\hyperlink{structrvehicle__t}{rvehicle\_t}));
00318 
00319     \textcolor{comment}{/*}
00320 \textcolor{comment}{     * Read the data.}
00321 \textcolor{comment}{     */}
00322     status = H5Dread (dset, rvehicletype, H5S\_ALL, H5S\_ALL, H5P\_DEFAULT, rdata);
00323 
00324     \textcolor{comment}{/*}
00325 \textcolor{comment}{     * Output the data to the screen.}
00326 \textcolor{comment}{     */}
00327     \textcolor{keywordflow}{for} (i=0; i<dims[0]; i++) \{
00328         printf (\textcolor{stringliteral}{"%s[%d]:\(\backslash\)n"}, DATASET, i);
00329         printf (\textcolor{stringliteral}{"   Vehicle name :\(\backslash\)n      %s\(\backslash\)n"}, rdata[i].name);
00330         printf (\textcolor{stringliteral}{"   Sensor locations :\(\backslash\)n"});
00331         \textcolor{keywordflow}{for} (j=0; j<rdata[i].sensors.len; j++)
00332             printf (\textcolor{stringliteral}{"      %s\(\backslash\)n"}, ( (\textcolor{keywordtype}{char} **) rdata[i].sensors.p )[j] );
00333     \}
00334 
00335     \textcolor{comment}{/*}
00336 \textcolor{comment}{     * Close and release resources.  H5Dvlen\_reclaim will automatically}
00337 \textcolor{comment}{     * traverse the structure and free any vlen data (including}
00338 \textcolor{comment}{     * strings).}
00339 \textcolor{comment}{     */}
00340     status = H5Dvlen\_reclaim (rvehicletype, space, H5P\_DEFAULT, rdata);
00341     free (rdata);
00342     status = H5Dclose (dset);
00343     status = H5Sclose (space);
00344     status = H5Tclose (strtype);
00345     status = H5Tclose (rsensortype);
00346     status = H5Tclose (rsensorstype);
00347     status = H5Tclose (rvehicletype);
00348     status = H5Fclose (file);
00349 
00350     \textcolor{keywordflow}{return} 0;
00351 \}
\end{DoxyCode}
