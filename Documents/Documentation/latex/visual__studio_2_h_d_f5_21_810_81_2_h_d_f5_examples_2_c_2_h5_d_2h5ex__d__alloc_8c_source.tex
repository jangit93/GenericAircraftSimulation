\hypertarget{visual__studio_2_h_d_f5_21_810_81_2_h_d_f5_examples_2_c_2_h5_d_2h5ex__d__alloc_8c_source}{}\section{visual\+\_\+studio/\+H\+D\+F5/1.10.1/\+H\+D\+F5\+Examples/\+C/\+H5\+D/h5ex\+\_\+d\+\_\+alloc.c}
\label{visual__studio_2_h_d_f5_21_810_81_2_h_d_f5_examples_2_c_2_h5_d_2h5ex__d__alloc_8c_source}\index{h5ex\+\_\+d\+\_\+alloc.\+c@{h5ex\+\_\+d\+\_\+alloc.\+c}}

\begin{DoxyCode}
00001 \textcolor{comment}{/************************************************************}
00002 \textcolor{comment}{}
00003 \textcolor{comment}{  This example shows how to set the space allocation time}
00004 \textcolor{comment}{  for a dataset.  The program first creates two datasets,}
00005 \textcolor{comment}{  one with the default allocation time (late) and one with}
00006 \textcolor{comment}{  early allocation time, and displays whether each has been}
00007 \textcolor{comment}{  allocated and their allocation size.  Next, it writes data}
00008 \textcolor{comment}{  to the datasets, and again displays whether each has been}
00009 \textcolor{comment}{  allocated and their allocation size.}
00010 \textcolor{comment}{}
00011 \textcolor{comment}{  This file is intended for use with HDF5 Library version 1.8}
00012 \textcolor{comment}{}
00013 \textcolor{comment}{ ************************************************************/}
00014 
00015 \textcolor{preprocessor}{#include "hdf5.h"}
00016 \textcolor{preprocessor}{#include <stdio.h>}
00017 \textcolor{preprocessor}{#include <stdlib.h>}
00018 
00019 \textcolor{preprocessor}{#define FILE            "h5ex\_d\_alloc.h5"}
00020 \textcolor{preprocessor}{#define DATASET1        "DS1"}
00021 \textcolor{preprocessor}{#define DATASET2        "DS2"}
00022 \textcolor{preprocessor}{#define DIM0            4}
00023 \textcolor{preprocessor}{#define DIM1            7}
00024 \textcolor{preprocessor}{#define FILLVAL         99}
00025 
00026 \textcolor{keywordtype}{int}
00027 main (\textcolor{keywordtype}{void})
00028 \{
00029     hid\_t                   \hyperlink{structfile}{file}, space, dset1, dset2, dcpl;
00030                                                     \textcolor{comment}{/* Handles */}
00031     herr\_t                  status;
00032     H5D\_space\_status\_t      space\_status;
00033     hsize\_t                 dims[2] = \{DIM0, DIM1\},
00034                             storage\_size;
00035     \textcolor{keywordtype}{int}                     wdata[DIM0][DIM1],      \textcolor{comment}{/* Write buffer */}
00036                             i, j;
00037 
00038     \textcolor{comment}{/*}
00039 \textcolor{comment}{     * Initialize data.}
00040 \textcolor{comment}{     */}
00041     \textcolor{keywordflow}{for} (i=0; i<DIM0; i++)
00042         \textcolor{keywordflow}{for} (j=0; j<DIM1; j++)
00043             wdata[i][j] = i * j - j;
00044 
00045     \textcolor{comment}{/*}
00046 \textcolor{comment}{     * Create a new file using the default properties.}
00047 \textcolor{comment}{     */}
00048     file = H5Fcreate (FILE, H5F\_ACC\_TRUNC, H5P\_DEFAULT, H5P\_DEFAULT);
00049 
00050     \textcolor{comment}{/*}
00051 \textcolor{comment}{     * Create dataspace.  Setting maximum size to NULL sets the maximum}
00052 \textcolor{comment}{     * size to be the current size.}
00053 \textcolor{comment}{     */}
00054     space = H5Screate\_simple (2, dims, NULL);
00055 
00056     \textcolor{comment}{/*}
00057 \textcolor{comment}{     * Create the dataset creation property list, and set the chunk}
00058 \textcolor{comment}{     * size.}
00059 \textcolor{comment}{     */}
00060     dcpl = H5Pcreate (H5P\_DATASET\_CREATE);
00061 
00062     \textcolor{comment}{/*}
00063 \textcolor{comment}{     * Set the allocation time to "early".  This way we can be sure}
00064 \textcolor{comment}{     * that reading from the dataset immediately after creation will}
00065 \textcolor{comment}{     * return the fill value.}
00066 \textcolor{comment}{     */}
00067     status = H5Pset\_alloc\_time (dcpl, H5D\_ALLOC\_TIME\_EARLY);
00068 
00069     printf (\textcolor{stringliteral}{"Creating datasets...\(\backslash\)n"});
00070     printf (\textcolor{stringliteral}{"%s has allocation time H5D\_ALLOC\_TIME\_LATE\(\backslash\)n"}, DATASET1);
00071     printf (\textcolor{stringliteral}{"%s has allocation time H5D\_ALLOC\_TIME\_EARLY\(\backslash\)n\(\backslash\)n"}, DATASET2);
00072 
00073     \textcolor{comment}{/*}
00074 \textcolor{comment}{     * Create the dataset using the dataset creation property list.}
00075 \textcolor{comment}{     */}
00076     dset1 = H5Dcreate (file, DATASET1, H5T\_STD\_I32LE, space, H5P\_DEFAULT,
00077                 H5P\_DEFAULT, H5P\_DEFAULT);
00078     dset2 = H5Dcreate (file, DATASET2, H5T\_STD\_I32LE, space, H5P\_DEFAULT, dcpl,
00079                 H5P\_DEFAULT);
00080 
00081     \textcolor{comment}{/*}
00082 \textcolor{comment}{     * Retrieve and print space status and storage size for dset1.}
00083 \textcolor{comment}{     */}
00084     status = H5Dget\_space\_status (dset1, &space\_status);
00085     storage\_size = H5Dget\_storage\_size (dset1);
00086     printf (\textcolor{stringliteral}{"Space for %s has%sbeen allocated.\(\backslash\)n"}, DATASET1,
00087                 space\_status == H5D\_SPACE\_STATUS\_ALLOCATED ? \textcolor{stringliteral}{" "} : \textcolor{stringliteral}{" not "});
00088     printf (\textcolor{stringliteral}{"Storage size for %s is: %ld bytes.\(\backslash\)n"}, DATASET1,
00089                 (\textcolor{keywordtype}{long}) storage\_size);
00090 
00091     \textcolor{comment}{/*}
00092 \textcolor{comment}{     * Retrieve and print space status and storage size for dset2.}
00093 \textcolor{comment}{     */}
00094     status = H5Dget\_space\_status (dset2, &space\_status);
00095     storage\_size = H5Dget\_storage\_size (dset2);
00096     printf (\textcolor{stringliteral}{"Space for %s has%sbeen allocated.\(\backslash\)n"}, DATASET2,
00097                 space\_status == H5D\_SPACE\_STATUS\_ALLOCATED ? \textcolor{stringliteral}{" "} : \textcolor{stringliteral}{" not "});
00098     printf (\textcolor{stringliteral}{"Storage size for %s is: %ld bytes.\(\backslash\)n"}, DATASET2,
00099                 (\textcolor{keywordtype}{long}) storage\_size);
00100 
00101     printf (\textcolor{stringliteral}{"\(\backslash\)nWriting data...\(\backslash\)n\(\backslash\)n"});
00102 
00103     \textcolor{comment}{/*}
00104 \textcolor{comment}{     * Write the data to the datasets.}
00105 \textcolor{comment}{     */}
00106     status = H5Dwrite (dset1, H5T\_NATIVE\_INT, H5S\_ALL, H5S\_ALL, H5P\_DEFAULT,
00107                 wdata[0]);
00108     status = H5Dwrite (dset2, H5T\_NATIVE\_INT, H5S\_ALL, H5S\_ALL, H5P\_DEFAULT,
00109                 wdata[0]);
00110 
00111     \textcolor{comment}{/*}
00112 \textcolor{comment}{     * Retrieve and print space status and storage size for dset1.}
00113 \textcolor{comment}{     */}
00114     status = H5Dget\_space\_status (dset1, &space\_status);
00115     storage\_size = H5Dget\_storage\_size (dset1);
00116     printf (\textcolor{stringliteral}{"Space for %s has%sbeen allocated.\(\backslash\)n"}, DATASET1,
00117                 space\_status == H5D\_SPACE\_STATUS\_ALLOCATED ? \textcolor{stringliteral}{" "} : \textcolor{stringliteral}{" not "});
00118     printf (\textcolor{stringliteral}{"Storage size for %s is: %ld bytes.\(\backslash\)n"}, DATASET1,
00119                 (\textcolor{keywordtype}{long}) storage\_size);
00120 
00121     \textcolor{comment}{/*}
00122 \textcolor{comment}{     * Retrieve and print space status and storage size for dset2.}
00123 \textcolor{comment}{     */}
00124     status = H5Dget\_space\_status (dset2, &space\_status);
00125     storage\_size = H5Dget\_storage\_size (dset2);
00126     printf (\textcolor{stringliteral}{"Space for %s has%sbeen allocated.\(\backslash\)n"}, DATASET2,
00127                 space\_status == H5D\_SPACE\_STATUS\_ALLOCATED ? \textcolor{stringliteral}{" "} : \textcolor{stringliteral}{" not "});
00128     printf (\textcolor{stringliteral}{"Storage size for %s is: %ld bytes.\(\backslash\)n"}, DATASET2,
00129                 (\textcolor{keywordtype}{long}) storage\_size);
00130 
00131     \textcolor{comment}{/*}
00132 \textcolor{comment}{     * Close and release resources.}
00133 \textcolor{comment}{     */}
00134     status = H5Pclose (dcpl);
00135     status = H5Dclose (dset1);
00136     status = H5Dclose (dset2);
00137     status = H5Sclose (space);
00138     status = H5Fclose (file);
00139 
00140     \textcolor{keywordflow}{return} 0;
00141 \}
\end{DoxyCode}
