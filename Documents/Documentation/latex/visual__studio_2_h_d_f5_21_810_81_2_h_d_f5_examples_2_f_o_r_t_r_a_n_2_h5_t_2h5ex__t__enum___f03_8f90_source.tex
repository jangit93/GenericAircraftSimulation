\hypertarget{visual__studio_2_h_d_f5_21_810_81_2_h_d_f5_examples_2_f_o_r_t_r_a_n_2_h5_t_2h5ex__t__enum___f03_8f90_source}{}\section{visual\+\_\+studio/\+H\+D\+F5/1.10.1/\+H\+D\+F5\+Examples/\+F\+O\+R\+T\+R\+A\+N/\+H5\+T/h5ex\+\_\+t\+\_\+enum\+\_\+\+F03.f90}
\label{visual__studio_2_h_d_f5_21_810_81_2_h_d_f5_examples_2_f_o_r_t_r_a_n_2_h5_t_2h5ex__t__enum___f03_8f90_source}\index{h5ex\+\_\+t\+\_\+enum\+\_\+\+F03.\+f90@{h5ex\+\_\+t\+\_\+enum\+\_\+\+F03.\+f90}}

\begin{DoxyCode}
00001 \textcolor{comment}{!************************************************************}
00002 \textcolor{comment}{!}
00003 \textcolor{comment}{!  This example shows how to read and write enumerated}
00004 \textcolor{comment}{!  datatypes to a dataset.  The program first writes}
00005 \textcolor{comment}{!  enumerated values to a dataset with a dataspace of}
00006 \textcolor{comment}{!  DIM0xDIM1, then closes the file.  Next, it reopens the}
00007 \textcolor{comment}{!  file, reads back the data, and outputs it to the screen.}
00008 \textcolor{comment}{!}
00009 \textcolor{comment}{!  This file is intended for use with HDF5 Library version 1.8}
00010 \textcolor{comment}{!  with --enable-fortran2003}
00011 \textcolor{comment}{!}
00012 \textcolor{comment}{!************************************************************}
00013 \textcolor{keyword}{PROGRAM} main
00014 
00015   \textcolor{keywordtype}{USE }hdf5
00016   \textcolor{keywordtype}{USE }iso\_c\_binding
00017 
00018   \textcolor{keywordtype}{IMPLICIT NONE}
00019   \textcolor{keywordtype}{CHARACTER(LEN=19)}, \textcolor{keywordtype}{PARAMETER} :: filename  = \textcolor{stringliteral}{"h5ex\_t\_enum\_F03.h5"}
00020   \textcolor{keywordtype}{CHARACTER(LEN=3)} , \textcolor{keywordtype}{PARAMETER} :: dataset   = \textcolor{stringliteral}{"DS1"}
00021   \textcolor{keywordtype}{INTEGER}          , \textcolor{keywordtype}{PARAMETER} :: dim0      = 4
00022   \textcolor{keywordtype}{INTEGER}          , \textcolor{keywordtype}{PARAMETER} :: dim1      = 7
00023   \textcolor{keywordtype}{INTEGER(HID\_T)}               :: f\_baset  \textcolor{comment}{! File base type}
00024   \textcolor{keywordtype}{INTEGER(HID\_T)}               :: m\_baset  \textcolor{comment}{! Memory base type}
00025   \textcolor{keywordtype}{INTEGER(SIZE\_T)}  , \textcolor{keywordtype}{PARAMETER} :: name\_buf\_size = 16
00026 
00027 \textcolor{comment}{! Enumerated type}
00028   enum, bind(c)
00029      enumerator :: solid = 0, liquid, gas, plasma
00030 \textcolor{keyword}{  END }enum
00031 
00032   \textcolor{keywordtype}{INTEGER(HID\_T)} :: \hyperlink{structfile}{file}, filetype, memtype, space, dset \textcolor{comment}{! Handles}
00033   \textcolor{keywordtype}{INTEGER} :: hdferr
00034 
00035   \textcolor{keywordtype}{INTEGER(hsize\_t)},   \textcolor{keywordtype}{DIMENSION(1:2)} :: dims = (/dim0, dim1/)
00036   \textcolor{keywordtype}{INTEGER(kind(SOLID))}, \textcolor{keywordtype}{DIMENSION(1:dim0, 1:dim1)}, \textcolor{keywordtype}{TARGET} :: wdata \textcolor{comment}{! Write buffer}
00037   \textcolor{keywordtype}{INTEGER(kind(SOLID))}, \textcolor{keywordtype}{DIMENSION(:,:)}, \textcolor{keywordtype}{ALLOCATABLE}, \textcolor{keywordtype}{TARGET} :: rdata \textcolor{comment}{! Read buffer}
00038   \textcolor{keywordtype}{INTEGER(kind(SOLID))}, \textcolor{keywordtype}{TARGET} :: val
00039 
00040   \textcolor{keywordtype}{CHARACTER(LEN=6)}, \textcolor{keywordtype}{DIMENSION(1:4)} :: &
00041        names = (/\textcolor{stringliteral}{"SOLID "}, \textcolor{stringliteral}{"LIQUID"}, \textcolor{stringliteral}{"GAS   "}, \textcolor{stringliteral}{"PLASMA"}/)
00042   \textcolor{keywordtype}{CHARACTER(LEN=NAME\_BUF\_SIZE)} :: name
00043   \textcolor{keywordtype}{INTEGER(HSIZE\_T)}, \textcolor{keywordtype}{DIMENSION(1:2)} :: maxdims
00044   \textcolor{keywordtype}{INTEGER(kind(SOLID))} :: i, j
00045   \textcolor{keywordtype}{TYPE}(c\_ptr) :: f\_ptr
00046   \textcolor{comment}{!}
00047   \textcolor{comment}{! Initialize FORTRAN interface.}
00048   \textcolor{comment}{!}
00049   \textcolor{keyword}{CALL }h5open\_f(hdferr)
00050   \textcolor{comment}{!}
00051   \textcolor{comment}{! Initialize DATA.}
00052   \textcolor{comment}{!}
00053   f\_baset   = h5t\_std\_i16be      \textcolor{comment}{! File base type}
00054   m\_baset   = h5kind\_to\_type(kind(solid), h5\_integer\_kind) \textcolor{comment}{! Memory base type}
00055   \textcolor{keywordflow}{DO} i = 1, dim0
00056      \textcolor{keywordflow}{DO} j = 1, dim1 
00057         wdata(i,j) = mod( (j-1)*(i-1), plasma+1)
00058 \textcolor{keywordflow}{     ENDDO}
00059 \textcolor{keywordflow}{  ENDDO}
00060   \textcolor{comment}{!}
00061   \textcolor{comment}{! Create a new file using the default properties.}
00062   \textcolor{comment}{!}
00063   \textcolor{keyword}{CALL }h5fcreate\_f(filename, h5f\_acc\_trunc\_f, \hyperlink{structfile}{file}, hdferr)
00064   \textcolor{comment}{!}
00065   \textcolor{comment}{! Create the enumerated datatypes for file and memory.  This}
00066   \textcolor{comment}{! process is simplified IF native types are used for the file,}
00067   \textcolor{comment}{! as only one type must be defined.}
00068   \textcolor{comment}{!}
00069   \textcolor{keyword}{CALL }h5tenum\_create\_f (f\_baset, filetype, hdferr)
00070   \textcolor{keyword}{CALL }h5tenum\_create\_f (m\_baset, memtype, hdferr)
00071 
00072   \textcolor{keywordflow}{DO} i = solid, plasma
00073      \textcolor{comment}{!}
00074      \textcolor{comment}{! Insert enumerated value for memtype.}
00075      \textcolor{comment}{!}
00076      val = i
00077      \textcolor{keyword}{CALL }h5tenum\_insert\_f(memtype, trim(names(i+1)), val, hdferr)
00078      \textcolor{comment}{!}
00079      \textcolor{comment}{! Insert enumerated value for filetype.  We must first convert}
00080      \textcolor{comment}{! the numerical value val to the base type of the destination.}
00081      \textcolor{comment}{!}
00082      f\_ptr = c\_loc(val)
00083      \textcolor{keyword}{CALL }h5tconvert\_f (m\_baset, f\_baset, int(1,size\_t), f\_ptr, hdferr)
00084      \textcolor{keyword}{CALL }h5tenum\_insert\_f(filetype, trim(names(i+1)), val, hdferr)
00085 \textcolor{keywordflow}{  ENDDO}
00086   \textcolor{comment}{!}
00087   \textcolor{comment}{! Create dataspace.  Setting maximum size to be the current size.}
00088   \textcolor{comment}{!}
00089   \textcolor{keyword}{CALL }h5screate\_simple\_f(2, dims, space, hdferr)
00090   \textcolor{comment}{!}
00091   \textcolor{comment}{! Create the dataset and write the enumerated data to it.}
00092   \textcolor{comment}{! }
00093   \textcolor{keyword}{CALL }h5dcreate\_f(\hyperlink{structfile}{file}, dataset, filetype, space, dset, hdferr)
00094   f\_ptr = c\_loc(wdata(1,1))
00095   \textcolor{keyword}{CALL }h5dwrite\_f(dset, memtype, f\_ptr, hdferr)
00096   \textcolor{comment}{!}
00097   \textcolor{comment}{! Close and release resources.}
00098   \textcolor{comment}{!}
00099   \textcolor{keyword}{CALL }h5dclose\_f(dset , hdferr)
00100   \textcolor{keyword}{CALL }h5sclose\_f(space, hdferr)
00101   \textcolor{keyword}{CALL }h5tclose\_f(filetype, hdferr)
00102   \textcolor{keyword}{CALL }h5fclose\_f(\hyperlink{structfile}{file} , hdferr)
00103 
00104   \textcolor{comment}{!}
00105   \textcolor{comment}{! Now we begin the read section of this example.}
00106   \textcolor{comment}{!}
00107   \textcolor{comment}{! Open file and dataset.}
00108   \textcolor{comment}{!}
00109   \textcolor{keyword}{CALL }h5fopen\_f(filename, h5f\_acc\_rdonly\_f, \hyperlink{structfile}{file}, hdferr)
00110   \textcolor{keyword}{CALL }h5dopen\_f (\hyperlink{structfile}{file}, dataset, dset, hdferr)
00111   \textcolor{comment}{!}
00112   \textcolor{comment}{! Get dataspace and allocate memory for read buffer.}
00113   \textcolor{comment}{!}
00114   \textcolor{keyword}{CALL }h5dget\_space\_f(dset,space, hdferr)
00115   \textcolor{keyword}{CALL }h5sget\_simple\_extent\_dims\_f (space, dims, maxdims, hdferr)
00116   \textcolor{keyword}{ALLOCATE}(rdata(1:dims(1),1:dims(2)))
00117   \textcolor{comment}{!}
00118   \textcolor{comment}{! Read the data.}
00119   \textcolor{comment}{!}
00120   f\_ptr = c\_loc(rdata(1,1))
00121   \textcolor{keyword}{CALL }h5dread\_f(dset, memtype, f\_ptr, hdferr)
00122   \textcolor{comment}{!}
00123   \textcolor{comment}{! Output the data to the screen.}
00124   \textcolor{comment}{!}
00125   \textcolor{keyword}{WRITE}(*, \textcolor{stringliteral}{'(A,":")'}) dataset
00126   \textcolor{keywordflow}{DO} i=1, dims(1)
00127      \textcolor{keyword}{WRITE}(*,\textcolor{stringliteral}{'(" [")'}, advance=\textcolor{stringliteral}{'NO'})
00128      \textcolor{keywordflow}{DO} j = 1, dims(2)
00129         \textcolor{comment}{!}
00130         \textcolor{comment}{! Get the name of the enumeration member.}
00131         \textcolor{comment}{!}
00132         \textcolor{keyword}{CALL }h5tenum\_nameof\_f( memtype, rdata(i,j), name\_buf\_size, name, hdferr) 
00133         \textcolor{keyword}{WRITE}(*,\textcolor{stringliteral}{'(" ", A6," ")'}, advance=\textcolor{stringliteral}{'NO'}) trim(name)
00134 \textcolor{keywordflow}{     ENDDO}
00135      \textcolor{keyword}{WRITE}(*,\textcolor{stringliteral}{'("]")'})
00136 \textcolor{keywordflow}{  ENDDO}
00137   \textcolor{comment}{!}
00138   \textcolor{comment}{! Close and release resources.}
00139   \textcolor{comment}{!}
00140   \textcolor{keyword}{DEALLOCATE}(rdata)
00141   \textcolor{keyword}{CALL }h5dclose\_f(dset , hdferr)
00142   \textcolor{keyword}{CALL }h5sclose\_f(space, hdferr)
00143   \textcolor{keyword}{CALL }h5tclose\_f(memtype, hdferr)
00144   \textcolor{keyword}{CALL }h5fclose\_f(\hyperlink{structfile}{file} , hdferr)
00145 \textcolor{keyword}{END PROGRAM }main
\end{DoxyCode}
