\hypertarget{visual__studio_2_h_d_f5_21_810_81_2_h_d_f5_examples_2_f_o_r_t_r_a_n_2_h5_d_2h5ex__d__gzip_8f90_source}{}\section{visual\+\_\+studio/\+H\+D\+F5/1.10.1/\+H\+D\+F5\+Examples/\+F\+O\+R\+T\+R\+A\+N/\+H5\+D/h5ex\+\_\+d\+\_\+gzip.f90}
\label{visual__studio_2_h_d_f5_21_810_81_2_h_d_f5_examples_2_f_o_r_t_r_a_n_2_h5_d_2h5ex__d__gzip_8f90_source}\index{h5ex\+\_\+d\+\_\+gzip.\+f90@{h5ex\+\_\+d\+\_\+gzip.\+f90}}

\begin{DoxyCode}
00001 \textcolor{comment}{! ************************************************************}
00002 \textcolor{comment}{!}
00003 \textcolor{comment}{!  This example shows how to read and write data to a dataset}
00004 \textcolor{comment}{!  using gzip compression (also called zlib or deflate).  The}
00005 \textcolor{comment}{!  program first checks if gzip compression is available,}
00006 \textcolor{comment}{!  then if it is it writes integers to a dataset using gzip,}
00007 \textcolor{comment}{!  then closes the file.  Next, it reopens the file, reads}
00008 \textcolor{comment}{!  back the data, and outputs the type of compression and the}
00009 \textcolor{comment}{!  maximum value in the dataset to the screen.}
00010 \textcolor{comment}{!}
00011 \textcolor{comment}{!  This file is intended for use with HDF5 Library verion 1.8}
00012 \textcolor{comment}{!}
00013 \textcolor{comment}{! ************************************************************}
00014 \textcolor{keyword}{PROGRAM} main
00015 
00016   \textcolor{keywordtype}{USE }hdf5
00017 
00018   \textcolor{keywordtype}{IMPLICIT NONE}
00019 
00020   \textcolor{keywordtype}{CHARACTER(LEN=14)}, \textcolor{keywordtype}{PARAMETER} :: filename = \textcolor{stringliteral}{"h5ex\_d\_gzip.h5"}
00021   \textcolor{keywordtype}{CHARACTER(LEN=3)} , \textcolor{keywordtype}{PARAMETER} :: dataset  = \textcolor{stringliteral}{"DS1"}
00022   \textcolor{keywordtype}{INTEGER}          , \textcolor{keywordtype}{PARAMETER} :: dim0     = 32
00023   \textcolor{keywordtype}{INTEGER}          , \textcolor{keywordtype}{PARAMETER} :: dim1     = 64
00024   \textcolor{keywordtype}{INTEGER}          , \textcolor{keywordtype}{PARAMETER} :: chunk0   = 4
00025   \textcolor{keywordtype}{INTEGER}          , \textcolor{keywordtype}{PARAMETER} :: chunk1   = 8
00026 
00027   \textcolor{keywordtype}{INTEGER} :: hdferr
00028   \textcolor{keywordtype}{LOGICAL}         :: avail
00029   \textcolor{keywordtype}{INTEGER(HID\_T)}  :: \hyperlink{structfile}{file}, space, dset, dcpl \textcolor{comment}{! Handles}
00030   \textcolor{keywordtype}{INTEGER}, \textcolor{keywordtype}{DIMENSION(1:1)} :: cd\_values
00031   \textcolor{keywordtype}{INTEGER} :: filter\_id
00032   \textcolor{keywordtype}{INTEGER} :: filter\_info\_both
00033   \textcolor{keywordtype}{INTEGER(HSIZE\_T)}, \textcolor{keywordtype}{DIMENSION(1:2)} :: dims = (/dim0, dim1/), chunk =(/chunk0,chunk1/)
00034   \textcolor{keywordtype}{INTEGER(SIZE\_T)} :: nelmts
00035   \textcolor{keywordtype}{INTEGER} :: flags, filter\_info
00036   \textcolor{keywordtype}{INTEGER}, \textcolor{keywordtype}{DIMENSION(1:dim0, 1:dim1)} :: wdata, & \textcolor{comment}{! Write buffer }
00037                                         rdata    \textcolor{comment}{! Read buffer}
00038   \textcolor{keywordtype}{INTEGER} :: max, i, j
00039   \textcolor{keywordtype}{INTEGER(SIZE\_T)}, \textcolor{keywordtype}{PARAMETER} :: maxchrlen = 80
00040   \textcolor{keywordtype}{CHARACTER(LEN=MaxChrLen)} :: name
00041   \textcolor{comment}{!}
00042   \textcolor{comment}{! Initialize FORTRAN interface.}
00043   \textcolor{comment}{!}
00044   \textcolor{keyword}{CALL }h5open\_f(hdferr)
00045   \textcolor{comment}{! }
00046   \textcolor{comment}{!  Check if gzip compression is available and can be used for both}
00047   \textcolor{comment}{!  compression and decompression.  Normally we do not perform error}
00048   \textcolor{comment}{!  checking in these examples for the sake of clarity, but in this}
00049   \textcolor{comment}{!  case we will make an exception because this filter is an}
00050   \textcolor{comment}{!  optional part of the hdf5 library.}
00051   \textcolor{comment}{!  }
00052   \textcolor{keyword}{CALL }h5zfilter\_avail\_f(h5z\_filter\_deflate\_f, avail, hdferr)
00053 
00054   \textcolor{keywordflow}{IF} (.NOT.avail) \textcolor{keywordflow}{THEN}
00055      \textcolor{keyword}{WRITE}(*,\textcolor{stringliteral}{'("gzip filter not available.",/)'})
00056      stop
00057 \textcolor{keywordflow}{  ENDIF}
00058   \textcolor{keyword}{CALL }h5zget\_filter\_info\_f(h5z\_filter\_deflate\_f, filter\_info, hdferr)
00059 
00060   filter\_info\_both=ior(h5z\_filter\_encode\_enabled\_f,h5z\_filter\_decode\_enabled\_f)
00061   \textcolor{keywordflow}{IF} (filter\_info .NE. filter\_info\_both) \textcolor{keywordflow}{THEN}
00062      \textcolor{keyword}{WRITE}(*,\textcolor{stringliteral}{'("gzip filter not available for encoding and decoding.",/)'})
00063      stop
00064 \textcolor{keywordflow}{  ENDIF}
00065   \textcolor{comment}{!}
00066   \textcolor{comment}{! Initialize data.}
00067   \textcolor{comment}{!}
00068   \textcolor{keywordflow}{DO} i = 1, dim0
00069      \textcolor{keywordflow}{DO} j = 1, dim1
00070         wdata(i,j) = (i-2)*(j-1)
00071 \textcolor{keywordflow}{     ENDDO}
00072 \textcolor{keywordflow}{  ENDDO}
00073   \textcolor{comment}{!}
00074   \textcolor{comment}{! Create a new file using the default properties.}
00075   \textcolor{comment}{!}
00076   \textcolor{keyword}{CALL }h5fcreate\_f(filename, h5f\_acc\_trunc\_f, \hyperlink{structfile}{file}, hdferr)
00077   \textcolor{comment}{!}
00078   \textcolor{comment}{! Create dataspace.  Setting size to be the current size.}
00079   \textcolor{comment}{!}
00080   \textcolor{keyword}{CALL }h5screate\_simple\_f(2, dims, space, hdferr)
00081   \textcolor{comment}{!}
00082   \textcolor{comment}{! Create the dataset creation property list, add the gzip}
00083   \textcolor{comment}{! compression filter and set the chunk size.}
00084   \textcolor{comment}{!}
00085   \textcolor{keyword}{CALL }h5pcreate\_f(h5p\_dataset\_create\_f, dcpl, hdferr)
00086   \textcolor{keyword}{CALL }h5pset\_deflate\_f(dcpl, 9, hdferr)
00087   \textcolor{keyword}{CALL }h5pset\_chunk\_f(dcpl, 2, chunk, hdferr)
00088   \textcolor{comment}{!}
00089   \textcolor{comment}{! Create the dataset.}
00090   \textcolor{comment}{!}
00091   \textcolor{keyword}{CALL }h5dcreate\_f(\hyperlink{structfile}{file}, dataset, h5t\_std\_i32le, space, dset, hdferr, dcpl)
00092   \textcolor{comment}{!}
00093   \textcolor{comment}{! Write the data to the dataset.}
00094   \textcolor{comment}{!}
00095   \textcolor{keyword}{CALL }h5dwrite\_f(dset, h5t\_native\_integer, wdata, dims, hdferr)
00096   \textcolor{comment}{!}
00097   \textcolor{comment}{! Close and release resources.}
00098   \textcolor{comment}{!}
00099   \textcolor{keyword}{CALL }h5pclose\_f(dcpl , hdferr)
00100   \textcolor{keyword}{CALL }h5dclose\_f(dset , hdferr)
00101   \textcolor{keyword}{CALL }h5sclose\_f(space, hdferr)
00102   \textcolor{keyword}{CALL }h5fclose\_f(\hyperlink{structfile}{file} , hdferr)
00103   \textcolor{comment}{!}
00104   \textcolor{comment}{! Now we begin the read section of this example.}
00105   \textcolor{comment}{!}
00106   \textcolor{comment}{!}
00107   \textcolor{comment}{! Open file and dataset using the default properties.}
00108   \textcolor{comment}{!}
00109   \textcolor{keyword}{CALL }h5fopen\_f(filename, h5f\_acc\_rdonly\_f, \hyperlink{structfile}{file}, hdferr)
00110   \textcolor{keyword}{CALL }h5dopen\_f (\hyperlink{structfile}{file}, dataset, dset, hdferr)
00111   \textcolor{comment}{!}
00112   \textcolor{comment}{! Retrieve dataset creation property list.}
00113   \textcolor{comment}{!}
00114   \textcolor{keyword}{CALL }h5dget\_create\_plist\_f(dset, dcpl, hdferr)
00115   \textcolor{comment}{!}
00116   \textcolor{comment}{! Retrieve and print the filter type.  Here we only retrieve the}
00117   \textcolor{comment}{! first filter because we know that we only added one filter.}
00118   \textcolor{comment}{!}
00119   nelmts = 1
00120   \textcolor{keyword}{CALL }h5pget\_filter\_f(dcpl, 0, flags, nelmts, cd\_values, maxchrlen, name, filter\_id, hdferr)
00121   \textcolor{keyword}{WRITE}(*,\textcolor{stringliteral}{'("Filter type is: ")'}, advance=\textcolor{stringliteral}{'NO'})
00122   \textcolor{keywordflow}{IF}(filter\_id.EQ.h5z\_filter\_deflate\_f)\textcolor{keywordflow}{THEN}
00123      \textcolor{keyword}{WRITE}(*,\textcolor{stringliteral}{'(T2,"H5Z\_FILTER\_DEFLATE\_F")'})
00124   \textcolor{keywordflow}{ELSE} \textcolor{keywordflow}{IF}(filter\_id.EQ.h5z\_filter\_shuffle\_f)\textcolor{keywordflow}{THEN}
00125      \textcolor{keyword}{WRITE}(*,\textcolor{stringliteral}{'(T2,"H5Z\_FILTER\_SHUFFLE\_F")'})
00126   \textcolor{keywordflow}{ELSE} \textcolor{keywordflow}{IF}(filter\_id.EQ.h5z\_filter\_fletcher32\_f)\textcolor{keywordflow}{THEN}
00127      \textcolor{keyword}{WRITE}(*,\textcolor{stringliteral}{'(T2,"H5Z\_FILTER\_FLETCHER32\_F")'})
00128   \textcolor{keywordflow}{ELSE} \textcolor{keywordflow}{IF}(filter\_id.EQ.h5z\_filter\_szip\_f)\textcolor{keywordflow}{THEN}
00129      \textcolor{keyword}{WRITE}(*,\textcolor{stringliteral}{'(T2,"H5Z\_FILTER\_SZIP\_F")'})
00130 \textcolor{keywordflow}{  ENDIF}
00131   \textcolor{comment}{!}
00132   \textcolor{comment}{! Read the data using the default properties.}
00133   \textcolor{comment}{!}
00134   \textcolor{keyword}{CALL }h5dread\_f(dset, h5t\_native\_integer, rdata, dims, hdferr)
00135   \textcolor{comment}{!}
00136   \textcolor{comment}{! Check if the read was successful.  Normally we do not perform}
00137   \textcolor{comment}{! error checking in these examples for the sake of clarity, but in}
00138   \textcolor{comment}{! this case we will make an exception because this is how the}
00139   \textcolor{comment}{! fletcher32 checksum filter reports data errors.}
00140   \textcolor{comment}{!}
00141   \textcolor{keywordflow}{IF} (hdferr.LT.0)\textcolor{keywordflow}{THEN}
00142      \textcolor{keyword}{WRITE}(*,\textcolor{stringliteral}{'("Dataset read failed!")'})
00143      \textcolor{keyword}{CALL }h5pclose\_f(dcpl , hdferr)
00144      \textcolor{keyword}{CALL }h5dclose\_f(dset , hdferr)
00145      \textcolor{keyword}{CALL }h5fclose\_f(\hyperlink{structfile}{file} , hdferr)
00146      stop
00147 \textcolor{keywordflow}{  ENDIF}
00148   \textcolor{comment}{!}
00149   \textcolor{comment}{! Find the maximum value in the dataset, to verify that it was}
00150   \textcolor{comment}{! read correctly.}
00151   \textcolor{comment}{!}
00152   max = maxval(rdata)
00153   \textcolor{comment}{!}
00154   \textcolor{comment}{! Print the maximum value.}
00155   \textcolor{comment}{!}
00156   \textcolor{keyword}{WRITE}(*,\textcolor{stringliteral}{'("Maximum value in ",A," is: ",i10)'}) dataset, max
00157   \textcolor{comment}{!}
00158   \textcolor{comment}{! Close and release resources.}
00159   \textcolor{comment}{!}
00160   \textcolor{keyword}{CALL }h5pclose\_f(dcpl , hdferr)
00161   \textcolor{keyword}{CALL }h5dclose\_f(dset , hdferr)
00162   \textcolor{keyword}{CALL }h5fclose\_f(\hyperlink{structfile}{file} , hdferr)
00163 
00164 \textcolor{keyword}{END PROGRAM }main
\end{DoxyCode}
