\hypertarget{visual__studio_2_h_d_f5_21_810_81_2_h_d_f5_examples_2_c_2_h5_g_2h5ex__g__intermediate_8c_source}{}\section{visual\+\_\+studio/\+H\+D\+F5/1.10.1/\+H\+D\+F5\+Examples/\+C/\+H5\+G/h5ex\+\_\+g\+\_\+intermediate.c}
\label{visual__studio_2_h_d_f5_21_810_81_2_h_d_f5_examples_2_c_2_h5_g_2h5ex__g__intermediate_8c_source}\index{h5ex\+\_\+g\+\_\+intermediate.\+c@{h5ex\+\_\+g\+\_\+intermediate.\+c}}

\begin{DoxyCode}
00001 \textcolor{comment}{/************************************************************}
00002 \textcolor{comment}{}
00003 \textcolor{comment}{  This example shows how to create intermediate groups with}
00004 \textcolor{comment}{  a single call to H5Gcreate.}
00005 \textcolor{comment}{}
00006 \textcolor{comment}{  This file is intended for use with HDF5 Library version 1.8}
00007 \textcolor{comment}{}
00008 \textcolor{comment}{ ************************************************************/}
00009 
00010 \textcolor{preprocessor}{#include "hdf5.h"}
00011 
00012 \textcolor{preprocessor}{#define FILE            "h5ex\_g\_intermediate.h5"}
00013 
00014 \textcolor{comment}{/*}
00015 \textcolor{comment}{ * Operator function to be called by H5Ovisit.}
00016 \textcolor{comment}{ */}
00017 herr\_t op\_func (hid\_t loc\_id, \textcolor{keyword}{const} \textcolor{keywordtype}{char} *name, \textcolor{keyword}{const} \hyperlink{struct_h5_o__info__t}{H5O\_info\_t} *info,
00018             \textcolor{keywordtype}{void} *operator\_data);
00019 
00020 \textcolor{keywordtype}{int}
00021 main(\textcolor{keywordtype}{void})
00022 \{
00023     hid\_t       \hyperlink{structfile}{file}, group, gcpl;      \textcolor{comment}{/* Handles */}
00024     herr\_t      status;
00025 
00026     \textcolor{comment}{/*}
00027 \textcolor{comment}{     * Create a new file using the default properties.}
00028 \textcolor{comment}{     */}
00029     file = H5Fcreate (FILE, H5F\_ACC\_TRUNC, H5P\_DEFAULT, H5P\_DEFAULT);
00030 
00031     \textcolor{comment}{/*}
00032 \textcolor{comment}{     * Create group creation property list and set it to allow creation}
00033 \textcolor{comment}{     * of intermediate groups.}
00034 \textcolor{comment}{     */}
00035     gcpl = H5Pcreate (H5P\_LINK\_CREATE);
00036     status = H5Pset\_create\_intermediate\_group (gcpl, 1);
00037 
00038     \textcolor{comment}{/*}
00039 \textcolor{comment}{     * Create the group /G1/G2/G3.  Note that /G1 and /G1/G2 do not}
00040 \textcolor{comment}{     * exist yet.  This call would cause an error if we did not use the}
00041 \textcolor{comment}{     * previously created property list.}
00042 \textcolor{comment}{     */}
00043     group = H5Gcreate (file, \textcolor{stringliteral}{"/G1/G2/G3"}, gcpl, H5P\_DEFAULT, H5P\_DEFAULT);
00044 
00045     \textcolor{comment}{/*}
00046 \textcolor{comment}{     * Print all the objects in the files to show that intermediate}
00047 \textcolor{comment}{     * groups have been created.  See h5ex\_g\_visit for more information}
00048 \textcolor{comment}{     * on how to use H5Ovisit.}
00049 \textcolor{comment}{     */}
00050     printf (\textcolor{stringliteral}{"Objects in the file:\(\backslash\)n"});
00051     status = H5Ovisit (file, H5\_INDEX\_NAME, H5\_ITER\_NATIVE, op\_func, NULL);
00052 
00053     \textcolor{comment}{/*}
00054 \textcolor{comment}{     * Close and release resources.}
00055 \textcolor{comment}{     */}
00056     status = H5Pclose (gcpl);
00057     status = H5Gclose (group);
00058     status = H5Fclose (file);
00059 
00060     \textcolor{keywordflow}{return} 0;
00061 \}
00062 
00063 
00064 \textcolor{comment}{/************************************************************}
00065 \textcolor{comment}{}
00066 \textcolor{comment}{  Operator function for H5Ovisit.  This function prints the}
00067 \textcolor{comment}{  name and type of the object passed to it.}
00068 \textcolor{comment}{}
00069 \textcolor{comment}{ ************************************************************/}
00070 herr\_t op\_func (hid\_t loc\_id, \textcolor{keyword}{const} \textcolor{keywordtype}{char} *name, \textcolor{keyword}{const} \hyperlink{struct_h5_o__info__t}{H5O\_info\_t} *info,
00071             \textcolor{keywordtype}{void} *operator\_data)
00072 \{
00073     printf (\textcolor{stringliteral}{"/"});               \textcolor{comment}{/* Print root group in object path */}
00074 
00075     \textcolor{comment}{/*}
00076 \textcolor{comment}{     * Check if the current object is the root group, and if not print}
00077 \textcolor{comment}{     * the full path name and type.}
00078 \textcolor{comment}{     */}
00079     \textcolor{keywordflow}{if} (name[0] == \textcolor{charliteral}{'.'})         \textcolor{comment}{/* Root group, do not print '.' */}
00080         printf (\textcolor{stringliteral}{"  (Group)\(\backslash\)n"});
00081     \textcolor{keywordflow}{else}
00082         \textcolor{keywordflow}{switch} (info->type) \{
00083             \textcolor{keywordflow}{case} H5O\_TYPE\_GROUP:
00084                 printf (\textcolor{stringliteral}{"%s  (Group)\(\backslash\)n"}, name);
00085                 \textcolor{keywordflow}{break};
00086             \textcolor{keywordflow}{case} H5O\_TYPE\_DATASET:
00087                 printf (\textcolor{stringliteral}{"%s  (Dataset)\(\backslash\)n"}, name);
00088                 \textcolor{keywordflow}{break};
00089             \textcolor{keywordflow}{case} H5O\_TYPE\_NAMED\_DATATYPE:
00090                 printf (\textcolor{stringliteral}{"%s  (Datatype)\(\backslash\)n"}, name);
00091                 \textcolor{keywordflow}{break};
00092             \textcolor{keywordflow}{default}:
00093                 printf (\textcolor{stringliteral}{"%s  (Unknown)\(\backslash\)n"}, name);
00094         \}
00095 
00096     \textcolor{keywordflow}{return} 0;
00097 \}
\end{DoxyCode}
