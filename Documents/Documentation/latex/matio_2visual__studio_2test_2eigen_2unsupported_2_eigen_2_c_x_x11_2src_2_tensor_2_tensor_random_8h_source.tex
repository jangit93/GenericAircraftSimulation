\hypertarget{matio_2visual__studio_2test_2eigen_2unsupported_2_eigen_2_c_x_x11_2src_2_tensor_2_tensor_random_8h_source}{}\section{matio/visual\+\_\+studio/test/eigen/unsupported/\+Eigen/\+C\+X\+X11/src/\+Tensor/\+Tensor\+Random.h}
\label{matio_2visual__studio_2test_2eigen_2unsupported_2_eigen_2_c_x_x11_2src_2_tensor_2_tensor_random_8h_source}\index{Tensor\+Random.\+h@{Tensor\+Random.\+h}}

\begin{DoxyCode}
00001 \textcolor{comment}{// This file is part of Eigen, a lightweight C++ template library}
00002 \textcolor{comment}{// for linear algebra.}
00003 \textcolor{comment}{//}
00004 \textcolor{comment}{// Copyright (C) 2016 Benoit Steiner <benoit.steiner.goog@gmail.com>}
00005 \textcolor{comment}{//}
00006 \textcolor{comment}{// This Source Code Form is subject to the terms of the Mozilla}
00007 \textcolor{comment}{// Public License v. 2.0. If a copy of the MPL was not distributed}
00008 \textcolor{comment}{// with this file, You can obtain one at http://mozilla.org/MPL/2.0/.}
00009 
00010 \textcolor{preprocessor}{#ifndef EIGEN\_CXX11\_TENSOR\_TENSOR\_RANDOM\_H}
00011 \textcolor{preprocessor}{#define EIGEN\_CXX11\_TENSOR\_TENSOR\_RANDOM\_H}
00012 
00013 \textcolor{keyword}{namespace }\hyperlink{namespace_eigen}{Eigen} \{
00014 \textcolor{keyword}{namespace }\hyperlink{namespaceinternal}{internal} \{
00015 
00016 \textcolor{keyword}{namespace }\{
00017 
00018 EIGEN\_DEVICE\_FUNC uint64\_t get\_random\_seed() \{
00019 \textcolor{preprocessor}{#ifdef \_\_CUDA\_ARCH\_\_}
00020   \textcolor{comment}{// We don't support 3d kernels since we currently only use 1 and}
00021   \textcolor{comment}{// 2d kernels.}
00022   assert(threadIdx.z == 0);
00023   \textcolor{keywordflow}{return} clock64() +
00024       blockIdx.x * blockDim.x + threadIdx.x +
00025       gridDim.x * blockDim.x * (blockIdx.y * blockDim.y + threadIdx.y);
00026 
00027 \textcolor{preprocessor}{#elif defined \_WIN32}
00028   \textcolor{comment}{// Use the current time as a baseline.}
00029   SYSTEMTIME st;
00030   GetSystemTime(&st);
00031   \textcolor{keywordtype}{int} time = st.wSecond + 1000 * st.wMilliseconds;
00032   \textcolor{comment}{// Mix in a random number to make sure that we get different seeds if}
00033   \textcolor{comment}{// we try to generate seeds faster than the clock resolution.}
00034   \textcolor{comment}{// We need 2 random values since the generator only generate 16 bits at}
00035   \textcolor{comment}{// a time (https://msdn.microsoft.com/en-us/library/398ax69y.aspx)}
00036   \textcolor{keywordtype}{int} rnd1 = ::rand();
00037   \textcolor{keywordtype}{int} rnd2 = ::rand();
00038   uint64\_t rnd = (rnd1 | rnd2 << 16) ^ time;
00039   \textcolor{keywordflow}{return} rnd;
00040 
00041 \textcolor{preprocessor}{#elif defined \_\_APPLE\_\_}
00042   \textcolor{comment}{// Same approach as for win32, except that the random number generator}
00043   \textcolor{comment}{// is better (//
       https://developer.apple.com/legacy/library/documentation/Darwin/Reference/ManPages/man3/random.3.html#//apple\_ref/doc/man/3/random).}
00044   uint64\_t rnd = ::random() ^ mach\_absolute\_time();
00045   \textcolor{keywordflow}{return} rnd;
00046 
00047 \textcolor{preprocessor}{#else}
00048   \textcolor{comment}{// Augment the current time with pseudo random number generation}
00049   \textcolor{comment}{// to ensure that we get different seeds if we try to generate seeds}
00050   \textcolor{comment}{// faster than the clock resolution.}
00051   timespec ts;
00052   clock\_gettime(CLOCK\_REALTIME, &ts);
00053   uint64\_t rnd = ::random() ^ ts.tv\_nsec;
00054   \textcolor{keywordflow}{return} rnd;
00055 \textcolor{preprocessor}{#endif}
00056 \}
00057 
00058 \textcolor{keyword}{static} EIGEN\_DEVICE\_FUNC EIGEN\_STRONG\_INLINE \textcolor{keywordtype}{unsigned} PCG\_XSH\_RS\_generator(uint64\_t* 
      \hyperlink{structstate}{state}) \{
00059   \textcolor{comment}{// TODO: Unify with the implementation in the non blocking thread pool.}
00060   uint64\_t current = *state;
00061   \textcolor{comment}{// Update the internal state}
00062   *state = current * 6364136223846793005ULL + 0xda3e39cb94b95bdbULL;
00063   \textcolor{comment}{// Generate the random output (using the PCG-XSH-RS scheme)}
00064   \textcolor{keywordflow}{return} \textcolor{keyword}{static\_cast<}\textcolor{keywordtype}{unsigned}\textcolor{keyword}{>}((current ^ (current >> 22)) >> (22 + (current >> 61)));
00065 \}
00066 
00067 \textcolor{keyword}{static} EIGEN\_DEVICE\_FUNC EIGEN\_STRONG\_INLINE uint64\_t PCG\_XSH\_RS\_state(uint64\_t seed) \{
00068   seed = seed ? seed : get\_random\_seed();
00069   \textcolor{keywordflow}{return} seed * 6364136223846793005ULL + 0xda3e39cb94b95bdbULL;
00070 \}
00071 
00072 \}  \textcolor{comment}{// namespace}
00073 
00074 
00075 \textcolor{keyword}{template} <\textcolor{keyword}{typename} T> EIGEN\_DEVICE\_FUNC EIGEN\_STRONG\_INLINE
00076 \hyperlink{group___sparse_core___module_class_eigen_1_1_triplet}{T} RandomToTypeUniform(uint64\_t* \hyperlink{structstate}{state}) \{
00077   \textcolor{keywordtype}{unsigned} rnd = PCG\_XSH\_RS\_generator(state);
00078   \textcolor{keywordflow}{return} \textcolor{keyword}{static\_cast<}\hyperlink{group___sparse_core___module_class_eigen_1_1_triplet}{T}\textcolor{keyword}{>}(rnd);
00079 \}
00080 
00081 
00082 \textcolor{keyword}{template} <> EIGEN\_DEVICE\_FUNC EIGEN\_STRONG\_INLINE
00083 \hyperlink{struct_eigen_1_1half}{Eigen::half} RandomToTypeUniform<Eigen::half>(uint64\_t* \hyperlink{structstate}{state}) \{
00084   \hyperlink{struct_eigen_1_1half}{Eigen::half} result;
00085   \textcolor{comment}{// Generate 10 random bits for the mantissa}
00086   \textcolor{keywordtype}{unsigned} rnd = PCG\_XSH\_RS\_generator(\hyperlink{structstate}{state});
00087   result.x = \textcolor{keyword}{static\_cast<}uint16\_t\textcolor{keyword}{>}(rnd & 0x3ffu);
00088   \textcolor{comment}{// Set the exponent}
00089   result.x |= (\textcolor{keyword}{static\_cast<}uint16\_t\textcolor{keyword}{>}(15) << 10);
00090   \textcolor{comment}{// Return the final result}
00091   \textcolor{keywordflow}{return} result - \hyperlink{struct_eigen_1_1half}{Eigen::half}(1.0f);
00092 \}
00093 
00094 
00095 \textcolor{keyword}{template} <> EIGEN\_DEVICE\_FUNC EIGEN\_STRONG\_INLINE
00096 \textcolor{keywordtype}{float} RandomToTypeUniform<float>(uint64\_t* \hyperlink{structstate}{state}) \{
00097   \textcolor{keyword}{typedef} \textcolor{keyword}{union }\{
00098     uint32\_t raw;
00099     \textcolor{keywordtype}{float} fp;
00100   \} \textcolor{keyword}{internal};
00101   \textcolor{keyword}{internal} result;
00102   \textcolor{comment}{// Generate 23 random bits for the mantissa mantissa}
00103   \textcolor{keyword}{const} \textcolor{keywordtype}{unsigned} rnd = PCG\_XSH\_RS\_generator(\hyperlink{structstate}{state});
00104   result.raw = rnd & 0x7fffffu;
00105   \textcolor{comment}{// Set the exponent}
00106   result.raw |= (\textcolor{keyword}{static\_cast<}uint32\_t\textcolor{keyword}{>}(127) << 23);
00107   \textcolor{comment}{// Return the final result}
00108   \textcolor{keywordflow}{return} result.fp - 1.0f;
00109 \}
00110 
00111 \textcolor{keyword}{template} <> EIGEN\_DEVICE\_FUNC EIGEN\_STRONG\_INLINE
00112 \textcolor{keywordtype}{double} RandomToTypeUniform<double>(uint64\_t* \hyperlink{structstate}{state}) \{
00113   \textcolor{keyword}{typedef} \textcolor{keyword}{union }\{
00114     uint64\_t raw;
00115     \textcolor{keywordtype}{double} dp;
00116   \} \textcolor{keyword}{internal};
00117   \textcolor{keyword}{internal} result;
00118   result.raw = 0;
00119   \textcolor{comment}{// Generate 52 random bits for the mantissa}
00120   \textcolor{comment}{// First generate the upper 20 bits}
00121   \textcolor{keywordtype}{unsigned} rnd1 = PCG\_XSH\_RS\_generator(\hyperlink{structstate}{state}) & 0xfffffu;
00122   \textcolor{comment}{// The generate the lower 32 bits}
00123   \textcolor{keywordtype}{unsigned} rnd2 = PCG\_XSH\_RS\_generator(\hyperlink{structstate}{state});
00124   result.raw = (\textcolor{keyword}{static\_cast<}uint64\_t\textcolor{keyword}{>}(rnd1) << 32) | rnd2;
00125   \textcolor{comment}{// Set the exponent}
00126   result.raw |= (\textcolor{keyword}{static\_cast<}uint64\_t\textcolor{keyword}{>}(1023) << 52);
00127   \textcolor{comment}{// Return the final result}
00128   \textcolor{keywordflow}{return} result.dp - 1.0;
00129 \}
00130 
00131 \textcolor{keyword}{template} <> EIGEN\_DEVICE\_FUNC EIGEN\_STRONG\_INLINE
00132 std::complex<float> RandomToTypeUniform<std::complex<float> >(uint64\_t* \hyperlink{structstate}{state}) \{
00133   \textcolor{keywordflow}{return} std::complex<float>(RandomToTypeUniform<float>(\hyperlink{structstate}{state}),
00134                              RandomToTypeUniform<float>(\hyperlink{structstate}{state}));
00135 \}
00136 \textcolor{keyword}{template} <> EIGEN\_DEVICE\_FUNC EIGEN\_STRONG\_INLINE
00137 std::complex<double> RandomToTypeUniform<std::complex<double> >(uint64\_t* \hyperlink{structstate}{state}) \{
00138   \textcolor{keywordflow}{return} std::complex<double>(RandomToTypeUniform<double>(\hyperlink{structstate}{state}),
00139                               RandomToTypeUniform<double>(\hyperlink{structstate}{state}));
00140 \}
00141 
00142 \textcolor{keyword}{template} <\textcolor{keyword}{typename} T> \textcolor{keyword}{class }UniformRandomGenerator \{
00143  \textcolor{keyword}{public}:
00144   \textcolor{keyword}{static} \textcolor{keyword}{const} \textcolor{keywordtype}{bool} PacketAccess = \textcolor{keyword}{true};
00145 
00146   \textcolor{comment}{// Uses the given "seed" if non-zero, otherwise uses a random seed.}
00147   EIGEN\_DEVICE\_FUNC EIGEN\_STRONG\_INLINE UniformRandomGenerator(
00148       uint64\_t seed = 0) \{
00149     m\_state = PCG\_XSH\_RS\_state(seed);
00150   \}
00151   EIGEN\_DEVICE\_FUNC EIGEN\_STRONG\_INLINE UniformRandomGenerator(
00152       \textcolor{keyword}{const} UniformRandomGenerator& other) \{
00153     m\_state = other.m\_state;
00154   \}
00155 
00156   \textcolor{keyword}{template}<\textcolor{keyword}{typename} Index> EIGEN\_DEVICE\_FUNC EIGEN\_STRONG\_INLINE
00157   \hyperlink{group___sparse_core___module_class_eigen_1_1_triplet}{T} operator()(\hyperlink{namespace_eigen_a62e77e0933482dafde8fe197d9a2cfde}{Index} i)\textcolor{keyword}{ const }\{
00158     uint64\_t local\_state = m\_state + i;
00159     \hyperlink{group___sparse_core___module_class_eigen_1_1_triplet}{T} result = RandomToTypeUniform<T>(&local\_state);
00160     m\_state = local\_state;
00161     \textcolor{keywordflow}{return} result;
00162   \}
00163 
00164   \textcolor{keyword}{template}<\textcolor{keyword}{typename} Packet, \textcolor{keyword}{typename} Index> EIGEN\_DEVICE\_FUNC EIGEN\_STRONG\_INLINE
00165   Packet packetOp(\hyperlink{namespace_eigen_a62e77e0933482dafde8fe197d9a2cfde}{Index} i)\textcolor{keyword}{ const }\{
00166     \textcolor{keyword}{const} \textcolor{keywordtype}{int} packetSize = internal::unpacket\_traits<Packet>::size;
00167     EIGEN\_ALIGN\_MAX \hyperlink{group___sparse_core___module_class_eigen_1_1_triplet}{T} values[packetSize];
00168     uint64\_t local\_state = m\_state + i;
00169     \textcolor{keywordflow}{for} (\textcolor{keywordtype}{int} j = 0; j < packetSize; ++j) \{
00170       values[j] = RandomToTypeUniform<T>(&local\_state);
00171     \}
00172     m\_state = local\_state;
00173     \textcolor{keywordflow}{return} internal::pload<Packet>(values);
00174   \}
00175 
00176  \textcolor{keyword}{private}:
00177   \textcolor{keyword}{mutable} uint64\_t m\_state;
00178 \};
00179 
00180 \textcolor{keyword}{template} <\textcolor{keyword}{typename} Scalar>
00181 \textcolor{keyword}{struct }functor\_traits<UniformRandomGenerator<Scalar> > \{
00182   \textcolor{keyword}{enum} \{
00183     \textcolor{comment}{// Rough estimate for floating point, multiplied by ceil(sizeof(T) / sizeof(float)).}
00184     Cost = 12 * NumTraits<Scalar>::AddCost *
00185            ((\textcolor{keyword}{sizeof}(Scalar) + \textcolor{keyword}{sizeof}(\textcolor{keywordtype}{float}) - 1) / \textcolor{keyword}{sizeof}(\textcolor{keywordtype}{float})),
00186     PacketAccess = UniformRandomGenerator<Scalar>::PacketAccess
00187   \};
00188 \};
00189 
00190 
00191 
00192 \textcolor{keyword}{template} <\textcolor{keyword}{typename} T> EIGEN\_DEVICE\_FUNC EIGEN\_STRONG\_INLINE
00193 \hyperlink{group___sparse_core___module_class_eigen_1_1_triplet}{T} RandomToTypeNormal(uint64\_t* \hyperlink{structstate}{state}) \{
00194   \textcolor{comment}{// Use the ratio of uniform method to generate numbers following a normal}
00195   \textcolor{comment}{// distribution. See for example Numerical Recipes chapter 7.3.9 for the}
00196   \textcolor{comment}{// details.}
00197   \hyperlink{group___sparse_core___module_class_eigen_1_1_triplet}{T} u, v, q;
00198   \textcolor{keywordflow}{do} \{
00199     u = RandomToTypeUniform<T>(state);
00200     v = \hyperlink{group___sparse_core___module_class_eigen_1_1_triplet}{T}(1.7156) * (RandomToTypeUniform<T>(state) - \hyperlink{group___sparse_core___module_class_eigen_1_1_triplet}{T}(0.5));
00201     \textcolor{keyword}{const} \hyperlink{group___sparse_core___module_class_eigen_1_1_triplet}{T} x = u - \hyperlink{group___sparse_core___module_class_eigen_1_1_triplet}{T}(0.449871);
00202     \textcolor{keyword}{const} \hyperlink{group___sparse_core___module_class_eigen_1_1_triplet}{T} y = numext::abs(v) + \hyperlink{group___sparse_core___module_class_eigen_1_1_triplet}{T}(0.386595);
00203     q = x*x + y * (\hyperlink{group___sparse_core___module_class_eigen_1_1_triplet}{T}(0.196)*y - \hyperlink{group___sparse_core___module_class_eigen_1_1_triplet}{T}(0.25472)*x);
00204   \} \textcolor{keywordflow}{while} (q > \hyperlink{group___sparse_core___module_class_eigen_1_1_triplet}{T}(0.27597) &&
00205            (q > \hyperlink{group___sparse_core___module_class_eigen_1_1_triplet}{T}(0.27846) || v*v > \hyperlink{group___sparse_core___module_class_eigen_1_1_triplet}{T}(-4) * numext::log(u) * u*u));
00206 
00207   \textcolor{keywordflow}{return} v/u;
00208 \}
00209 
00210 \textcolor{keyword}{template} <> EIGEN\_DEVICE\_FUNC EIGEN\_STRONG\_INLINE
00211 std::complex<float> RandomToTypeNormal<std::complex<float> >(uint64\_t* state) \{
00212   \textcolor{keywordflow}{return} std::complex<float>(RandomToTypeNormal<float>(state),
00213                              RandomToTypeNormal<float>(state));
00214 \}
00215 \textcolor{keyword}{template} <> EIGEN\_DEVICE\_FUNC EIGEN\_STRONG\_INLINE
00216 std::complex<double> RandomToTypeNormal<std::complex<double> >(uint64\_t* state) \{
00217   \textcolor{keywordflow}{return} std::complex<double>(RandomToTypeNormal<double>(state),
00218                               RandomToTypeNormal<double>(state));
00219 \}
00220 
00221 
00222 \textcolor{keyword}{template} <\textcolor{keyword}{typename} T> \textcolor{keyword}{class }NormalRandomGenerator \{
00223  \textcolor{keyword}{public}:
00224   \textcolor{keyword}{static} \textcolor{keyword}{const} \textcolor{keywordtype}{bool} PacketAccess = \textcolor{keyword}{true};
00225 
00226   \textcolor{comment}{// Uses the given "seed" if non-zero, otherwise uses a random seed.}
00227   EIGEN\_DEVICE\_FUNC EIGEN\_STRONG\_INLINE NormalRandomGenerator(uint64\_t seed = 0) \{
00228     m\_state = PCG\_XSH\_RS\_state(seed);
00229   \}
00230   EIGEN\_DEVICE\_FUNC EIGEN\_STRONG\_INLINE NormalRandomGenerator(
00231       \textcolor{keyword}{const} NormalRandomGenerator& other) \{
00232     m\_state = other.m\_state;
00233   \}
00234 
00235  \textcolor{keyword}{template}<\textcolor{keyword}{typename} Index> EIGEN\_DEVICE\_FUNC EIGEN\_STRONG\_INLINE
00236   \hyperlink{group___sparse_core___module_class_eigen_1_1_triplet}{T} operator()(\hyperlink{namespace_eigen_a62e77e0933482dafde8fe197d9a2cfde}{Index} i)\textcolor{keyword}{ const }\{
00237     uint64\_t local\_state = m\_state + i;
00238     \hyperlink{group___sparse_core___module_class_eigen_1_1_triplet}{T} result = RandomToTypeNormal<T>(&local\_state);
00239     m\_state = local\_state;
00240     \textcolor{keywordflow}{return} result;
00241   \}
00242 
00243   \textcolor{keyword}{template}<\textcolor{keyword}{typename} Packet, \textcolor{keyword}{typename} Index> EIGEN\_DEVICE\_FUNC EIGEN\_STRONG\_INLINE
00244   Packet packetOp(\hyperlink{namespace_eigen_a62e77e0933482dafde8fe197d9a2cfde}{Index} i)\textcolor{keyword}{ const }\{
00245     \textcolor{keyword}{const} \textcolor{keywordtype}{int} packetSize = internal::unpacket\_traits<Packet>::size;
00246     EIGEN\_ALIGN\_MAX \hyperlink{group___sparse_core___module_class_eigen_1_1_triplet}{T} values[packetSize];
00247     uint64\_t local\_state = m\_state + i;
00248     \textcolor{keywordflow}{for} (\textcolor{keywordtype}{int} j = 0; j < packetSize; ++j) \{
00249       values[j] = RandomToTypeNormal<T>(&local\_state);
00250     \}
00251     m\_state = local\_state;
00252     \textcolor{keywordflow}{return} internal::pload<Packet>(values);
00253   \}
00254 
00255  \textcolor{keyword}{private}:
00256   \textcolor{keyword}{mutable} uint64\_t m\_state;
00257 \};
00258 
00259 
00260 \textcolor{keyword}{template} <\textcolor{keyword}{typename} Scalar>
00261 \textcolor{keyword}{struct }functor\_traits<NormalRandomGenerator<Scalar> > \{
00262   \textcolor{keyword}{enum} \{
00263     \textcolor{comment}{// On average, we need to generate about 3 random numbers}
00264     \textcolor{comment}{// 15 mul, 8 add, 1.5 logs}
00265     Cost = 3 * functor\_traits<UniformRandomGenerator<Scalar> >::Cost +
00266            15 * NumTraits<Scalar>::AddCost + 8 * NumTraits<Scalar>::AddCost +
00267            3 * functor\_traits<scalar\_log\_op<Scalar> >::Cost / 2,
00268     PacketAccess = NormalRandomGenerator<Scalar>::PacketAccess
00269   \};
00270 \};
00271 
00272 
00273 \} \textcolor{comment}{// end namespace internal}
00274 \} \textcolor{comment}{// end namespace Eigen}
00275 
00276 \textcolor{preprocessor}{#endif // EIGEN\_CXX11\_TENSOR\_TENSOR\_RANDOM\_H}
\end{DoxyCode}
