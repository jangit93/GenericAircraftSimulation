\hypertarget{eigen_2lapack_2sladiv_8f_source}{}\section{eigen/lapack/sladiv.f}
\label{eigen_2lapack_2sladiv_8f_source}\index{sladiv.\+f@{sladiv.\+f}}

\begin{DoxyCode}
00001 \textcolor{comment}{*> \(\backslash\)brief \(\backslash\)b SLADIV}
00002 \textcolor{comment}{*}
00003 \textcolor{comment}{*  =========== DOCUMENTATION ===========}
00004 \textcolor{comment}{*}
00005 \textcolor{comment}{* Online html documentation available at }
00006 \textcolor{comment}{*            http://www.netlib.org/lapack/explore-html/ }
00007 \textcolor{comment}{*}
00008 \textcolor{comment}{*> \(\backslash\)htmlonly}
00009 \textcolor{comment}{*> Download SLADIV + dependencies }
00010 \textcolor{comment}{*> <a
       href="http://www.netlib.org/cgi-bin/netlibfiles.tgz?format=tgz&filename=/lapack/lapack\_routine/sladiv.f"> }
00011 \textcolor{comment}{*> [TGZ]</a> }
00012 \textcolor{comment}{*> <a
       href="http://www.netlib.org/cgi-bin/netlibfiles.zip?format=zip&filename=/lapack/lapack\_routine/sladiv.f"> }
00013 \textcolor{comment}{*> [ZIP]</a> }
00014 \textcolor{comment}{*> <a
       href="http://www.netlib.org/cgi-bin/netlibfiles.txt?format=txt&filename=/lapack/lapack\_routine/sladiv.f"> }
00015 \textcolor{comment}{*> [TXT]</a>}
00016 \textcolor{comment}{*> \(\backslash\)endhtmlonly }
00017 \textcolor{comment}{*}
00018 \textcolor{comment}{*  Definition:}
00019 \textcolor{comment}{*  ===========}
00020 \textcolor{comment}{*}
00021 \textcolor{comment}{*       SUBROUTINE SLADIV( A, B, C, D, P, Q )}
00022 \textcolor{comment}{* }
00023 \textcolor{comment}{*       .. Scalar Arguments ..}
00024 \textcolor{comment}{*       REAL               A, B, C, D, P, Q}
00025 \textcolor{comment}{*       ..}
00026 \textcolor{comment}{*  }
00027 \textcolor{comment}{*}
00028 \textcolor{comment}{*> \(\backslash\)par Purpose:}
00029 \textcolor{comment}{*  =============}
00030 \textcolor{comment}{*>}
00031 \textcolor{comment}{*> \(\backslash\)verbatim}
00032 \textcolor{comment}{*>}
00033 \textcolor{comment}{*> SLADIV performs complex division in  real arithmetic}
00034 \textcolor{comment}{*>}
00035 \textcolor{comment}{*>                       a + i*b}
00036 \textcolor{comment}{*>            p + i*q = ---------}
00037 \textcolor{comment}{*>                       c + i*d}
00038 \textcolor{comment}{*>}
00039 \textcolor{comment}{*> The algorithm is due to Robert L. Smith and can be found}
00040 \textcolor{comment}{*> in D. Knuth, The art of Computer Programming, Vol.2, p.195}
00041 \textcolor{comment}{*> \(\backslash\)endverbatim}
00042 \textcolor{comment}{*}
00043 \textcolor{comment}{*  Arguments:}
00044 \textcolor{comment}{*  ==========}
00045 \textcolor{comment}{*}
00046 \textcolor{comment}{*> \(\backslash\)param[in] A}
00047 \textcolor{comment}{*> \(\backslash\)verbatim}
00048 \textcolor{comment}{*>          A is REAL}
00049 \textcolor{comment}{*> \(\backslash\)endverbatim}
00050 \textcolor{comment}{*>}
00051 \textcolor{comment}{*> \(\backslash\)param[in] B}
00052 \textcolor{comment}{*> \(\backslash\)verbatim}
00053 \textcolor{comment}{*>          B is REAL}
00054 \textcolor{comment}{*> \(\backslash\)endverbatim}
00055 \textcolor{comment}{*>}
00056 \textcolor{comment}{*> \(\backslash\)param[in] C}
00057 \textcolor{comment}{*> \(\backslash\)verbatim}
00058 \textcolor{comment}{*>          C is REAL}
00059 \textcolor{comment}{*> \(\backslash\)endverbatim}
00060 \textcolor{comment}{*>}
00061 \textcolor{comment}{*> \(\backslash\)param[in] D}
00062 \textcolor{comment}{*> \(\backslash\)verbatim}
00063 \textcolor{comment}{*>          D is REAL}
00064 \textcolor{comment}{*>          The scalars a, b, c, and d in the above expression.}
00065 \textcolor{comment}{*> \(\backslash\)endverbatim}
00066 \textcolor{comment}{*>}
00067 \textcolor{comment}{*> \(\backslash\)param[out] P}
00068 \textcolor{comment}{*> \(\backslash\)verbatim}
00069 \textcolor{comment}{*>          P is REAL}
00070 \textcolor{comment}{*> \(\backslash\)endverbatim}
00071 \textcolor{comment}{*>}
00072 \textcolor{comment}{*> \(\backslash\)param[out] Q}
00073 \textcolor{comment}{*> \(\backslash\)verbatim}
00074 \textcolor{comment}{*>          Q is REAL}
00075 \textcolor{comment}{*>          The scalars p and q in the above expression.}
00076 \textcolor{comment}{*> \(\backslash\)endverbatim}
00077 \textcolor{comment}{*}
00078 \textcolor{comment}{*  Authors:}
00079 \textcolor{comment}{*  ========}
00080 \textcolor{comment}{*}
00081 \textcolor{comment}{*> \(\backslash\)author Univ. of Tennessee }
00082 \textcolor{comment}{*> \(\backslash\)author Univ. of California Berkeley }
00083 \textcolor{comment}{*> \(\backslash\)author Univ. of Colorado Denver }
00084 \textcolor{comment}{*> \(\backslash\)author NAG Ltd. }
00085 \textcolor{comment}{*}
00086 \textcolor{comment}{*> \(\backslash\)date November 2011}
00087 \textcolor{comment}{*}
00088 \textcolor{comment}{*> \(\backslash\)ingroup auxOTHERauxiliary}
00089 \textcolor{comment}{*}
00090 \textcolor{comment}{*  =====================================================================}
00091 \textcolor{keyword}{      SUBROUTINE }sladiv( A, B, C, D, P, Q )
00092 \textcolor{comment}{*}
00093 \textcolor{comment}{*  -- LAPACK auxiliary routine (version 3.4.0) --}
00094 \textcolor{comment}{*  -- LAPACK is a software package provided by Univ. of Tennessee,    --}
00095 \textcolor{comment}{*  -- Univ. of California Berkeley, Univ. of Colorado Denver and NAG Ltd..--}
00096 \textcolor{comment}{*     November 2011}
00097 \textcolor{comment}{*}
00098 \textcolor{comment}{*     .. Scalar Arguments ..}
00099       \textcolor{keywordtype}{REAL}               a, b, c, d, p, q
00100 \textcolor{comment}{*     ..}
00101 \textcolor{comment}{*}
00102 \textcolor{comment}{*  =====================================================================}
00103 \textcolor{comment}{*}
00104 \textcolor{comment}{*     .. Local Scalars ..}
00105       \textcolor{keywordtype}{REAL}               e, f
00106 \textcolor{comment}{*     ..}
00107 \textcolor{comment}{*     .. Intrinsic Functions ..}
00108       \textcolor{keywordtype}{INTRINSIC}          abs
00109 \textcolor{comment}{*     ..}
00110 \textcolor{comment}{*     .. Executable Statements ..}
00111 \textcolor{comment}{*}
00112       \textcolor{keywordflow}{IF}( abs( d ).LT.abs( c ) ) \textcolor{keywordflow}{THEN}
00113          e = d / c
00114          f = c + d*e
00115          p = ( a+b*e ) / f
00116          q = ( b-a*e ) / f
00117       \textcolor{keywordflow}{ELSE}
00118          e = c / d
00119          f = d + c*e
00120          p = ( b+a*e ) / f
00121          q = ( -a+b*e ) / f
00122 \textcolor{keywordflow}{      END IF}
00123 \textcolor{comment}{*}
00124       \textcolor{keywordflow}{RETURN}
00125 \textcolor{comment}{*}
00126 \textcolor{comment}{*     End of SLADIV}
00127 \textcolor{comment}{*}
00128 \textcolor{keyword}{      END}
\end{DoxyCode}
