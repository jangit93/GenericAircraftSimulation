\hypertarget{eigen_2lapack_2ilaclc_8f_source}{}\section{eigen/lapack/ilaclc.f}
\label{eigen_2lapack_2ilaclc_8f_source}\index{ilaclc.\+f@{ilaclc.\+f}}

\begin{DoxyCode}
00001 \textcolor{comment}{*> \(\backslash\)brief \(\backslash\)b ILACLC}
00002 \textcolor{comment}{*}
00003 \textcolor{comment}{*  =========== DOCUMENTATION ===========}
00004 \textcolor{comment}{*}
00005 \textcolor{comment}{* Online html documentation available at }
00006 \textcolor{comment}{*            http://www.netlib.org/lapack/explore-html/ }
00007 \textcolor{comment}{*}
00008 \textcolor{comment}{*> \(\backslash\)htmlonly}
00009 \textcolor{comment}{*> Download ILACLC + dependencies }
00010 \textcolor{comment}{*> <a
       href="http://www.netlib.org/cgi-bin/netlibfiles.tgz?format=tgz&filename=/lapack/lapack\_routine/ilaclc.f"> }
00011 \textcolor{comment}{*> [TGZ]</a> }
00012 \textcolor{comment}{*> <a
       href="http://www.netlib.org/cgi-bin/netlibfiles.zip?format=zip&filename=/lapack/lapack\_routine/ilaclc.f"> }
00013 \textcolor{comment}{*> [ZIP]</a> }
00014 \textcolor{comment}{*> <a
       href="http://www.netlib.org/cgi-bin/netlibfiles.txt?format=txt&filename=/lapack/lapack\_routine/ilaclc.f"> }
00015 \textcolor{comment}{*> [TXT]</a>}
00016 \textcolor{comment}{*> \(\backslash\)endhtmlonly }
00017 \textcolor{comment}{*}
00018 \textcolor{comment}{*  Definition:}
00019 \textcolor{comment}{*  ===========}
00020 \textcolor{comment}{*}
00021 \textcolor{comment}{*       INTEGER FUNCTION ILACLC( M, N, A, LDA )}
00022 \textcolor{comment}{* }
00023 \textcolor{comment}{*       .. Scalar Arguments ..}
00024 \textcolor{comment}{*       INTEGER            M, N, LDA}
00025 \textcolor{comment}{*       ..}
00026 \textcolor{comment}{*       .. Array Arguments ..}
00027 \textcolor{comment}{*       COMPLEX            A( LDA, * )}
00028 \textcolor{comment}{*       ..}
00029 \textcolor{comment}{*  }
00030 \textcolor{comment}{*}
00031 \textcolor{comment}{*> \(\backslash\)par Purpose:}
00032 \textcolor{comment}{*  =============}
00033 \textcolor{comment}{*>}
00034 \textcolor{comment}{*> \(\backslash\)verbatim}
00035 \textcolor{comment}{*>}
00036 \textcolor{comment}{*> ILACLC scans A for its last non-zero column.}
00037 \textcolor{comment}{*> \(\backslash\)endverbatim}
00038 \textcolor{comment}{*}
00039 \textcolor{comment}{*  Arguments:}
00040 \textcolor{comment}{*  ==========}
00041 \textcolor{comment}{*}
00042 \textcolor{comment}{*> \(\backslash\)param[in] M}
00043 \textcolor{comment}{*> \(\backslash\)verbatim}
00044 \textcolor{comment}{*>          M is INTEGER}
00045 \textcolor{comment}{*>          The number of rows of the matrix A.}
00046 \textcolor{comment}{*> \(\backslash\)endverbatim}
00047 \textcolor{comment}{*>}
00048 \textcolor{comment}{*> \(\backslash\)param[in] N}
00049 \textcolor{comment}{*> \(\backslash\)verbatim}
00050 \textcolor{comment}{*>          N is INTEGER}
00051 \textcolor{comment}{*>          The number of columns of the matrix A.}
00052 \textcolor{comment}{*> \(\backslash\)endverbatim}
00053 \textcolor{comment}{*>}
00054 \textcolor{comment}{*> \(\backslash\)param[in] A}
00055 \textcolor{comment}{*> \(\backslash\)verbatim}
00056 \textcolor{comment}{*>          A is COMPLEX array, dimension (LDA,N)}
00057 \textcolor{comment}{*>          The m by n matrix A.}
00058 \textcolor{comment}{*> \(\backslash\)endverbatim}
00059 \textcolor{comment}{*>}
00060 \textcolor{comment}{*> \(\backslash\)param[in] LDA}
00061 \textcolor{comment}{*> \(\backslash\)verbatim}
00062 \textcolor{comment}{*>          LDA is INTEGER}
00063 \textcolor{comment}{*>          The leading dimension of the array A. LDA >= max(1,M).}
00064 \textcolor{comment}{*> \(\backslash\)endverbatim}
00065 \textcolor{comment}{*}
00066 \textcolor{comment}{*  Authors:}
00067 \textcolor{comment}{*  ========}
00068 \textcolor{comment}{*}
00069 \textcolor{comment}{*> \(\backslash\)author Univ. of Tennessee }
00070 \textcolor{comment}{*> \(\backslash\)author Univ. of California Berkeley }
00071 \textcolor{comment}{*> \(\backslash\)author Univ. of Colorado Denver }
00072 \textcolor{comment}{*> \(\backslash\)author NAG Ltd. }
00073 \textcolor{comment}{*}
00074 \textcolor{comment}{*> \(\backslash\)date November 2011}
00075 \textcolor{comment}{*}
00076 \textcolor{comment}{*> \(\backslash\)ingroup complexOTHERauxiliary}
00077 \textcolor{comment}{*}
00078 \textcolor{comment}{*  =====================================================================}
00079 \textcolor{keyword}{      INTEGER }\textcolor{keyword}{FUNCTION }ilaclc( M, N, A, LDA )
00080 \textcolor{comment}{*}
00081 \textcolor{comment}{*  -- LAPACK auxiliary routine (version 3.4.0) --}
00082 \textcolor{comment}{*  -- LAPACK is a software package provided by Univ. of Tennessee,    --}
00083 \textcolor{comment}{*  -- Univ. of California Berkeley, Univ. of Colorado Denver and NAG Ltd..--}
00084 \textcolor{comment}{*     November 2011}
00085 \textcolor{comment}{*}
00086 \textcolor{comment}{*     .. Scalar Arguments ..}
00087       \textcolor{keywordtype}{INTEGER}            m, n, lda
00088 \textcolor{comment}{*     ..}
00089 \textcolor{comment}{*     .. Array Arguments ..}
00090       \textcolor{keywordtype}{COMPLEX}            a( lda, * )
00091 \textcolor{comment}{*     ..}
00092 \textcolor{comment}{*}
00093 \textcolor{comment}{*  =====================================================================}
00094 \textcolor{comment}{*}
00095 \textcolor{comment}{*     .. Parameters ..}
00096       \textcolor{keywordtype}{COMPLEX}          zero
00097       parameter( zero = (0.0e+0, 0.0e+0) )
00098 \textcolor{comment}{*     ..}
00099 \textcolor{comment}{*     .. Local Scalars ..}
00100       \textcolor{keywordtype}{INTEGER} i
00101 \textcolor{comment}{*     ..}
00102 \textcolor{comment}{*     .. Executable Statements ..}
00103 \textcolor{comment}{*}
00104 \textcolor{comment}{*     Quick test for the common case where one corner is non-zero.}
00105       \textcolor{keywordflow}{IF}( n.EQ.0 ) \textcolor{keywordflow}{THEN}
00106          ilaclc = n
00107       \textcolor{keywordflow}{ELSE} \textcolor{keywordflow}{IF}( a(1, n).NE.zero .OR. a(m, n).NE.zero ) \textcolor{keywordflow}{THEN}
00108          ilaclc = n
00109       \textcolor{keywordflow}{ELSE}
00110 \textcolor{comment}{*     Now scan each column from the end, returning with the first non-zero.}
00111          \textcolor{keywordflow}{DO} ilaclc = n, 1, -1
00112             \textcolor{keywordflow}{DO} i = 1, m
00113                \textcolor{keywordflow}{IF}( a(i, ilaclc).NE.zero ) \textcolor{keywordflow}{RETURN}
00114 \textcolor{keywordflow}{            END DO}
00115 \textcolor{keywordflow}{         END DO}
00116 \textcolor{keywordflow}{      END IF}
00117       \textcolor{keywordflow}{RETURN}
00118 \textcolor{keyword}{      END}
\end{DoxyCode}
