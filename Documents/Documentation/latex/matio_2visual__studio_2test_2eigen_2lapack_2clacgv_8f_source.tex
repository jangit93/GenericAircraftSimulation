\hypertarget{matio_2visual__studio_2test_2eigen_2lapack_2clacgv_8f_source}{}\section{matio/visual\+\_\+studio/test/eigen/lapack/clacgv.f}
\label{matio_2visual__studio_2test_2eigen_2lapack_2clacgv_8f_source}\index{clacgv.\+f@{clacgv.\+f}}

\begin{DoxyCode}
00001 \textcolor{comment}{*> \(\backslash\)brief \(\backslash\)b CLACGV}
00002 \textcolor{comment}{*}
00003 \textcolor{comment}{*  =========== DOCUMENTATION ===========}
00004 \textcolor{comment}{*}
00005 \textcolor{comment}{* Online html documentation available at }
00006 \textcolor{comment}{*            http://www.netlib.org/lapack/explore-html/ }
00007 \textcolor{comment}{*}
00008 \textcolor{comment}{*> \(\backslash\)htmlonly}
00009 \textcolor{comment}{*> Download CLACGV + dependencies }
00010 \textcolor{comment}{*> <a
       href="http://www.netlib.org/cgi-bin/netlibfiles.tgz?format=tgz&filename=/lapack/lapack\_routine/clacgv.f"> }
00011 \textcolor{comment}{*> [TGZ]</a> }
00012 \textcolor{comment}{*> <a
       href="http://www.netlib.org/cgi-bin/netlibfiles.zip?format=zip&filename=/lapack/lapack\_routine/clacgv.f"> }
00013 \textcolor{comment}{*> [ZIP]</a> }
00014 \textcolor{comment}{*> <a
       href="http://www.netlib.org/cgi-bin/netlibfiles.txt?format=txt&filename=/lapack/lapack\_routine/clacgv.f"> }
00015 \textcolor{comment}{*> [TXT]</a>}
00016 \textcolor{comment}{*> \(\backslash\)endhtmlonly }
00017 \textcolor{comment}{*}
00018 \textcolor{comment}{*  Definition:}
00019 \textcolor{comment}{*  ===========}
00020 \textcolor{comment}{*}
00021 \textcolor{comment}{*       SUBROUTINE CLACGV( N, X, INCX )}
00022 \textcolor{comment}{* }
00023 \textcolor{comment}{*       .. Scalar Arguments ..}
00024 \textcolor{comment}{*       INTEGER            INCX, N}
00025 \textcolor{comment}{*       ..}
00026 \textcolor{comment}{*       .. Array Arguments ..}
00027 \textcolor{comment}{*       COMPLEX            X( * )}
00028 \textcolor{comment}{*       ..}
00029 \textcolor{comment}{*  }
00030 \textcolor{comment}{*}
00031 \textcolor{comment}{*> \(\backslash\)par Purpose:}
00032 \textcolor{comment}{*  =============}
00033 \textcolor{comment}{*>}
00034 \textcolor{comment}{*> \(\backslash\)verbatim}
00035 \textcolor{comment}{*>}
00036 \textcolor{comment}{*> CLACGV conjugates a complex vector of length N.}
00037 \textcolor{comment}{*> \(\backslash\)endverbatim}
00038 \textcolor{comment}{*}
00039 \textcolor{comment}{*  Arguments:}
00040 \textcolor{comment}{*  ==========}
00041 \textcolor{comment}{*}
00042 \textcolor{comment}{*> \(\backslash\)param[in] N}
00043 \textcolor{comment}{*> \(\backslash\)verbatim}
00044 \textcolor{comment}{*>          N is INTEGER}
00045 \textcolor{comment}{*>          The length of the vector X.  N >= 0.}
00046 \textcolor{comment}{*> \(\backslash\)endverbatim}
00047 \textcolor{comment}{*>}
00048 \textcolor{comment}{*> \(\backslash\)param[in,out] X}
00049 \textcolor{comment}{*> \(\backslash\)verbatim}
00050 \textcolor{comment}{*>          X is COMPLEX array, dimension}
00051 \textcolor{comment}{*>                         (1+(N-1)*abs(INCX))}
00052 \textcolor{comment}{*>          On entry, the vector of length N to be conjugated.}
00053 \textcolor{comment}{*>          On exit, X is overwritten with conjg(X).}
00054 \textcolor{comment}{*> \(\backslash\)endverbatim}
00055 \textcolor{comment}{*>}
00056 \textcolor{comment}{*> \(\backslash\)param[in] INCX}
00057 \textcolor{comment}{*> \(\backslash\)verbatim}
00058 \textcolor{comment}{*>          INCX is INTEGER}
00059 \textcolor{comment}{*>          The spacing between successive elements of X.}
00060 \textcolor{comment}{*> \(\backslash\)endverbatim}
00061 \textcolor{comment}{*}
00062 \textcolor{comment}{*  Authors:}
00063 \textcolor{comment}{*  ========}
00064 \textcolor{comment}{*}
00065 \textcolor{comment}{*> \(\backslash\)author Univ. of Tennessee }
00066 \textcolor{comment}{*> \(\backslash\)author Univ. of California Berkeley }
00067 \textcolor{comment}{*> \(\backslash\)author Univ. of Colorado Denver }
00068 \textcolor{comment}{*> \(\backslash\)author NAG Ltd. }
00069 \textcolor{comment}{*}
00070 \textcolor{comment}{*> \(\backslash\)date November 2011}
00071 \textcolor{comment}{*}
00072 \textcolor{comment}{*> \(\backslash\)ingroup complexOTHERauxiliary}
00073 \textcolor{comment}{*}
00074 \textcolor{comment}{*  =====================================================================}
00075 \textcolor{keyword}{      SUBROUTINE }clacgv( N, X, INCX )
00076 \textcolor{comment}{*}
00077 \textcolor{comment}{*  -- LAPACK auxiliary routine (version 3.4.0) --}
00078 \textcolor{comment}{*  -- LAPACK is a software package provided by Univ. of Tennessee,    --}
00079 \textcolor{comment}{*  -- Univ. of California Berkeley, Univ. of Colorado Denver and NAG Ltd..--}
00080 \textcolor{comment}{*     November 2011}
00081 \textcolor{comment}{*}
00082 \textcolor{comment}{*     .. Scalar Arguments ..}
00083       \textcolor{keywordtype}{INTEGER}            incx, n
00084 \textcolor{comment}{*     ..}
00085 \textcolor{comment}{*     .. Array Arguments ..}
00086       \textcolor{keywordtype}{COMPLEX}            x( * )
00087 \textcolor{comment}{*     ..}
00088 \textcolor{comment}{*}
00089 \textcolor{comment}{* =====================================================================}
00090 \textcolor{comment}{*}
00091 \textcolor{comment}{*     .. Local Scalars ..}
00092       \textcolor{keywordtype}{INTEGER}            i, ioff
00093 \textcolor{comment}{*     ..}
00094 \textcolor{comment}{*     .. Intrinsic Functions ..}
00095       \textcolor{keywordtype}{INTRINSIC}          conjg
00096 \textcolor{comment}{*     ..}
00097 \textcolor{comment}{*     .. Executable Statements ..}
00098 \textcolor{comment}{*}
00099       \textcolor{keywordflow}{IF}( incx.EQ.1 ) \textcolor{keywordflow}{THEN}
00100          \textcolor{keywordflow}{DO} 10 i = 1, n
00101             x( i ) = conjg( x( i ) )
00102    10    \textcolor{keywordflow}{CONTINUE}
00103       \textcolor{keywordflow}{ELSE}
00104          ioff = 1
00105          \textcolor{keywordflow}{IF}( incx.LT.0 )
00106      $      ioff = 1 - ( n-1 )*incx
00107          \textcolor{keywordflow}{DO} 20 i = 1, n
00108             x( ioff ) = conjg( x( ioff ) )
00109             ioff = ioff + incx
00110    20    \textcolor{keywordflow}{CONTINUE}
00111 \textcolor{keywordflow}{      END IF}
00112       \textcolor{keywordflow}{RETURN}
00113 \textcolor{comment}{*}
00114 \textcolor{comment}{*     End of CLACGV}
00115 \textcolor{comment}{*}
00116 \textcolor{keyword}{      END}
\end{DoxyCode}
