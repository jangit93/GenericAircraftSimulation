\hypertarget{visual__studio_2_h_d_f5_21_810_81_2_h_d_f5_examples_2_c_2_h5_g_2h5ex__g__phase_8c_source}{}\section{visual\+\_\+studio/\+H\+D\+F5/1.10.1/\+H\+D\+F5\+Examples/\+C/\+H5\+G/h5ex\+\_\+g\+\_\+phase.c}
\label{visual__studio_2_h_d_f5_21_810_81_2_h_d_f5_examples_2_c_2_h5_g_2h5ex__g__phase_8c_source}\index{h5ex\+\_\+g\+\_\+phase.\+c@{h5ex\+\_\+g\+\_\+phase.\+c}}

\begin{DoxyCode}
00001 \textcolor{comment}{/************************************************************}
00002 \textcolor{comment}{}
00003 \textcolor{comment}{  This example shows how to set the conditions for}
00004 \textcolor{comment}{  conversion between compact and dense (indexed) groups.}
00005 \textcolor{comment}{}
00006 \textcolor{comment}{  This file is intended for use with HDF5 Library version 1.8}
00007 \textcolor{comment}{}
00008 \textcolor{comment}{ ************************************************************/}
00009 
00010 \textcolor{preprocessor}{#include "hdf5.h"}
00011 \textcolor{preprocessor}{#include <stdio.h>}
00012 
00013 \textcolor{preprocessor}{#define FILE        "h5ex\_g\_phase.h5"}
00014 \textcolor{preprocessor}{#define MAX\_GROUPS  7}
00015 \textcolor{preprocessor}{#define MAX\_COMPACT 5}
00016 \textcolor{preprocessor}{#define MIN\_DENSE   3}
00017 
00018 \textcolor{keywordtype}{int}
00019 main (\textcolor{keywordtype}{void})
00020 \{
00021     hid\_t       \hyperlink{structfile}{file}, group, subgroup, fapl, gcpl;      \textcolor{comment}{/* Handles */}
00022     herr\_t      status;
00023     \hyperlink{struct_h5_g__info__t}{H5G\_info\_t}  ginfo;
00024     \textcolor{keywordtype}{char}        name[3]=\textcolor{stringliteral}{"G0"};                  \textcolor{comment}{/* Name of subgroup */}
00025     \textcolor{keywordtype}{unsigned}    i;
00026 
00027     \textcolor{comment}{/*}
00028 \textcolor{comment}{     * Set file access property list to allow the latest file format.}
00029 \textcolor{comment}{     * This will allow the library to create new format groups.}
00030 \textcolor{comment}{     */}
00031     fapl = H5Pcreate (H5P\_FILE\_ACCESS);
00032     status = H5Pset\_libver\_bounds (fapl, H5F\_LIBVER\_LATEST, H5F\_LIBVER\_LATEST);
00033 
00034     \textcolor{comment}{/*}
00035 \textcolor{comment}{     * Create group access property list and set the phase change}
00036 \textcolor{comment}{     * conditions.  In this example we lowered the conversion threshold}
00037 \textcolor{comment}{     * to simplify the output, though this may not be optimal.}
00038 \textcolor{comment}{     */}
00039     gcpl = H5Pcreate (H5P\_GROUP\_CREATE);
00040     status = H5Pset\_link\_phase\_change (gcpl, MAX\_COMPACT, MIN\_DENSE);
00041 
00042     \textcolor{comment}{/*}
00043 \textcolor{comment}{     * Create a new file using the default properties.}
00044 \textcolor{comment}{     */}
00045     file = H5Fcreate (FILE, H5F\_ACC\_TRUNC, H5P\_DEFAULT, fapl);
00046 
00047     \textcolor{comment}{/*}
00048 \textcolor{comment}{     * Create primary group.}
00049 \textcolor{comment}{     */}
00050     group = H5Gcreate (file, name, H5P\_DEFAULT, gcpl, H5P\_DEFAULT);
00051 
00052     \textcolor{comment}{/*}
00053 \textcolor{comment}{     * Add subgroups to "group" one at a time, print the storage type}
00054 \textcolor{comment}{     * for "group" after each subgroup is created.}
00055 \textcolor{comment}{     */}
00056     \textcolor{keywordflow}{for} (i=1; i<=MAX\_GROUPS; i++) \{
00057 
00058         \textcolor{comment}{/*}
00059 \textcolor{comment}{         * Define the subgroup name and create the subgroup.}
00060 \textcolor{comment}{         */}
00061         name[1] = ((char) i) + \textcolor{charliteral}{'0'};     \textcolor{comment}{/* G1, G2, G3 etc. */}
00062         subgroup = H5Gcreate (group, name, H5P\_DEFAULT, H5P\_DEFAULT,
00063                     H5P\_DEFAULT);
00064         status = H5Gclose (subgroup);
00065 
00066         \textcolor{comment}{/*}
00067 \textcolor{comment}{         * Obtain the group info and print the group storage type}
00068 \textcolor{comment}{         */}
00069         status = H5Gget\_info (group, &ginfo);
00070         printf (\textcolor{stringliteral}{"%d Group%s: Storage type is "}, (\textcolor{keywordtype}{int}) ginfo.nlinks,
00071                     ginfo.nlinks == 1 ? \textcolor{stringliteral}{" "} : \textcolor{stringliteral}{"s"});
00072         \textcolor{keywordflow}{switch} (ginfo.storage\_type) \{
00073             \textcolor{keywordflow}{case} H5G\_STORAGE\_TYPE\_COMPACT:
00074                 printf (\textcolor{stringliteral}{"H5G\_STORAGE\_TYPE\_COMPACT\(\backslash\)n"}); \textcolor{comment}{/* New compact format */}
00075                 \textcolor{keywordflow}{break};
00076             \textcolor{keywordflow}{case} H5G\_STORAGE\_TYPE\_DENSE:
00077                 printf (\textcolor{stringliteral}{"H5G\_STORAGE\_TYPE\_DENSE\(\backslash\)n"}); \textcolor{comment}{/* New dense (indexed) format */}
00078                 \textcolor{keywordflow}{break};
00079             \textcolor{keywordflow}{case} H5G\_STORAGE\_TYPE\_SYMBOL\_TABLE:
00080                 printf (\textcolor{stringliteral}{"H5G\_STORAGE\_TYPE\_SYMBOL\_TABLE\(\backslash\)n"}); \textcolor{comment}{/* Original format */}
00081                 \textcolor{keywordflow}{break};
00082             \textcolor{keywordflow}{case} H5G\_STORAGE\_TYPE\_UNKNOWN:
00083                 printf (\textcolor{stringliteral}{"H5G\_STORAGE\_TYPE\_UNKNOWN\(\backslash\)n"}); \textcolor{comment}{/* Unknown format */}
00084         \}
00085     \}
00086 
00087     printf(\textcolor{stringliteral}{"\(\backslash\)n"});
00088 
00089     \textcolor{comment}{/*}
00090 \textcolor{comment}{     * Delete subgroups one at a time, print the storage type for}
00091 \textcolor{comment}{     * "group" after each subgroup is deleted.}
00092 \textcolor{comment}{     */}
00093     \textcolor{keywordflow}{for} (i=MAX\_GROUPS; i>=1; i--) \{
00094 
00095         \textcolor{comment}{/*}
00096 \textcolor{comment}{         * Define the subgroup name and delete the subgroup.}
00097 \textcolor{comment}{         */}
00098         name[1] = ((char) i) + \textcolor{charliteral}{'0'};     \textcolor{comment}{/* G1, G2, G3 etc. */}
00099         status = H5Ldelete (group, name, H5P\_DEFAULT);
00100 
00101         \textcolor{comment}{/*}
00102 \textcolor{comment}{         * Obtain the group info and print the group storage type}
00103 \textcolor{comment}{         */}
00104         status = H5Gget\_info (group, &ginfo);
00105         printf (\textcolor{stringliteral}{"%d Group%s: Storage type is "}, (\textcolor{keywordtype}{int}) ginfo.nlinks,
00106                     ginfo.nlinks == 1 ? \textcolor{stringliteral}{" "} : \textcolor{stringliteral}{"s"});
00107         \textcolor{keywordflow}{switch} (ginfo.storage\_type) \{
00108             \textcolor{keywordflow}{case} H5G\_STORAGE\_TYPE\_COMPACT:
00109                 printf (\textcolor{stringliteral}{"H5G\_STORAGE\_TYPE\_COMPACT\(\backslash\)n"}); \textcolor{comment}{/* New compact format */}
00110                 \textcolor{keywordflow}{break};
00111             \textcolor{keywordflow}{case} H5G\_STORAGE\_TYPE\_DENSE:
00112                 printf (\textcolor{stringliteral}{"H5G\_STORAGE\_TYPE\_DENSE\(\backslash\)n"}); \textcolor{comment}{/* New dense (indexed) format */}
00113                 \textcolor{keywordflow}{break};
00114             \textcolor{keywordflow}{case} H5G\_STORAGE\_TYPE\_SYMBOL\_TABLE:
00115                 printf (\textcolor{stringliteral}{"H5G\_STORAGE\_TYPE\_SYMBOL\_TABLE\(\backslash\)n"}); \textcolor{comment}{/* Original format */}
00116                 \textcolor{keywordflow}{break};
00117             \textcolor{keywordflow}{case} H5G\_STORAGE\_TYPE\_UNKNOWN:
00118                 printf (\textcolor{stringliteral}{"H5G\_STORAGE\_TYPE\_UNKNOWN\(\backslash\)n"}); \textcolor{comment}{/* Unknown format */}
00119         \}
00120     \}
00121 
00122     \textcolor{comment}{/*}
00123 \textcolor{comment}{     * Close and release resources.}
00124 \textcolor{comment}{     */}
00125     status = H5Pclose (fapl);
00126     status = H5Pclose (gcpl);
00127     status = H5Gclose (group);
00128     status = H5Fclose (file);
00129 
00130     \textcolor{keywordflow}{return} 0;
00131 \}
\end{DoxyCode}
