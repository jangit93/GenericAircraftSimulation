\hypertarget{visual__studio_2zlib_2contrib_2blast_2blast_8h_source}{}\section{visual\+\_\+studio/zlib/contrib/blast/blast.h}
\label{visual__studio_2zlib_2contrib_2blast_2blast_8h_source}\index{blast.\+h@{blast.\+h}}

\begin{DoxyCode}
00001 \textcolor{comment}{/* blast.h -- interface for blast.c}
00002 \textcolor{comment}{  Copyright (C) 2003, 2012, 2013 Mark Adler}
00003 \textcolor{comment}{  version 1.3, 24 Aug 2013}
00004 \textcolor{comment}{}
00005 \textcolor{comment}{  This software is provided 'as-is', without any express or implied}
00006 \textcolor{comment}{  warranty.  In no event will the author be held liable for any damages}
00007 \textcolor{comment}{  arising from the use of this software.}
00008 \textcolor{comment}{}
00009 \textcolor{comment}{  Permission is granted to anyone to use this software for any purpose,}
00010 \textcolor{comment}{  including commercial applications, and to alter it and redistribute it}
00011 \textcolor{comment}{  freely, subject to the following restrictions:}
00012 \textcolor{comment}{}
00013 \textcolor{comment}{  1. The origin of this software must not be misrepresented; you must not}
00014 \textcolor{comment}{     claim that you wrote the original software. If you use this software}
00015 \textcolor{comment}{     in a product, an acknowledgment in the product documentation would be}
00016 \textcolor{comment}{     appreciated but is not required.}
00017 \textcolor{comment}{  2. Altered source versions must be plainly marked as such, and must not be}
00018 \textcolor{comment}{     misrepresented as being the original software.}
00019 \textcolor{comment}{  3. This notice may not be removed or altered from any source distribution.}
00020 \textcolor{comment}{}
00021 \textcolor{comment}{  Mark Adler    madler@alumni.caltech.edu}
00022 \textcolor{comment}{ */}
00023 
00024 
00025 \textcolor{comment}{/*}
00026 \textcolor{comment}{ * blast() decompresses the PKWare Data Compression Library (DCL) compressed}
00027 \textcolor{comment}{ * format.  It provides the same functionality as the explode() function in}
00028 \textcolor{comment}{ * that library.  (Note: PKWare overused the "implode" verb, and the format}
00029 \textcolor{comment}{ * used by their library implode() function is completely different and}
00030 \textcolor{comment}{ * incompatible with the implode compression method supported by PKZIP.)}
00031 \textcolor{comment}{ *}
00032 \textcolor{comment}{ * The binary mode for stdio functions should be used to assure that the}
00033 \textcolor{comment}{ * compressed data is not corrupted when read or written.  For example:}
00034 \textcolor{comment}{ * fopen(..., "rb") and fopen(..., "wb").}
00035 \textcolor{comment}{ */}
00036 
00037 
00038 \textcolor{keyword}{typedef} unsigned (*blast\_in)(\textcolor{keywordtype}{void} *how, \textcolor{keywordtype}{unsigned} \textcolor{keywordtype}{char} **buf);
00039 \textcolor{keyword}{typedef} int (*blast\_out)(\textcolor{keywordtype}{void} *how, \textcolor{keywordtype}{unsigned} \textcolor{keywordtype}{char} *buf, \textcolor{keywordtype}{unsigned} len);
00040 \textcolor{comment}{/* Definitions for input/output functions passed to blast().  See below for}
00041 \textcolor{comment}{ * what the provided functions need to do.}
00042 \textcolor{comment}{ */}
00043 
00044 
00045 \textcolor{keywordtype}{int} blast(blast\_in infun, \textcolor{keywordtype}{void} *inhow, blast\_out outfun, \textcolor{keywordtype}{void} *outhow,
00046           \textcolor{keywordtype}{unsigned} *left, \textcolor{keywordtype}{unsigned} \textcolor{keywordtype}{char} **in);
00047 \textcolor{comment}{/* Decompress input to output using the provided infun() and outfun() calls.}
00048 \textcolor{comment}{ * On success, the return value of blast() is zero.  If there is an error in}
00049 \textcolor{comment}{ * the source data, i.e. it is not in the proper format, then a negative value}
00050 \textcolor{comment}{ * is returned.  If there is not enough input available or there is not enough}
00051 \textcolor{comment}{ * output space, then a positive error is returned.}
00052 \textcolor{comment}{ *}
00053 \textcolor{comment}{ * The input function is invoked: len = infun(how, &buf), where buf is set by}
00054 \textcolor{comment}{ * infun() to point to the input buffer, and infun() returns the number of}
00055 \textcolor{comment}{ * available bytes there.  If infun() returns zero, then blast() returns with}
00056 \textcolor{comment}{ * an input error.  (blast() only asks for input if it needs it.)  inhow is for}
00057 \textcolor{comment}{ * use by the application to pass an input descriptor to infun(), if desired.}
00058 \textcolor{comment}{ *}
00059 \textcolor{comment}{ * If left and in are not NULL and *left is not zero when blast() is called,}
00060 \textcolor{comment}{ * then the *left bytes are *in are consumed for input before infun() is used.}
00061 \textcolor{comment}{ *}
00062 \textcolor{comment}{ * The output function is invoked: err = outfun(how, buf, len), where the bytes}
00063 \textcolor{comment}{ * to be written are buf[0..len-1].  If err is not zero, then blast() returns}
00064 \textcolor{comment}{ * with an output error.  outfun() is always called with len <= 4096.  outhow}
00065 \textcolor{comment}{ * is for use by the application to pass an output descriptor to outfun(), if}
00066 \textcolor{comment}{ * desired.}
00067 \textcolor{comment}{ *}
00068 \textcolor{comment}{ * If there is any unused input, *left is set to the number of bytes that were}
00069 \textcolor{comment}{ * read and *in points to them.  Otherwise *left is set to zero and *in is set}
00070 \textcolor{comment}{ * to NULL.  If left or in are NULL, then they are not set.}
00071 \textcolor{comment}{ *}
00072 \textcolor{comment}{ * The return codes are:}
00073 \textcolor{comment}{ *}
00074 \textcolor{comment}{ *   2:  ran out of input before completing decompression}
00075 \textcolor{comment}{ *   1:  output error before completing decompression}
00076 \textcolor{comment}{ *   0:  successful decompression}
00077 \textcolor{comment}{ *  -1:  literal flag not zero or one}
00078 \textcolor{comment}{ *  -2:  dictionary size not in 4..6}
00079 \textcolor{comment}{ *  -3:  distance is too far back}
00080 \textcolor{comment}{ *}
00081 \textcolor{comment}{ * At the bottom of blast.c is an example program that uses blast() that can be}
00082 \textcolor{comment}{ * compiled to produce a command-line decompression filter by defining TEST.}
00083 \textcolor{comment}{ */}
\end{DoxyCode}
