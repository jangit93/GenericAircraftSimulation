\hypertarget{visual__studio_2zlib_2contrib_2infback9_2infback9_8c_source}{}\section{visual\+\_\+studio/zlib/contrib/infback9/infback9.c}
\label{visual__studio_2zlib_2contrib_2infback9_2infback9_8c_source}\index{infback9.\+c@{infback9.\+c}}

\begin{DoxyCode}
00001 \textcolor{comment}{/* infback9.c -- inflate deflate64 data using a call-back interface}
00002 \textcolor{comment}{ * Copyright (C) 1995-2008 Mark Adler}
00003 \textcolor{comment}{ * For conditions of distribution and use, see copyright notice in zlib.h}
00004 \textcolor{comment}{ */}
00005 
00006 \textcolor{preprocessor}{#include "zutil.h"}
00007 \textcolor{preprocessor}{#include "infback9.h"}
00008 \textcolor{preprocessor}{#include "inftree9.h"}
00009 \textcolor{preprocessor}{#include "inflate9.h"}
00010 
00011 \textcolor{preprocessor}{#define WSIZE 65536UL}
00012 
00013 \textcolor{comment}{/*}
00014 \textcolor{comment}{   strm provides memory allocation functions in zalloc and zfree, or}
00015 \textcolor{comment}{   Z\_NULL to use the library memory allocation functions.}
00016 \textcolor{comment}{}
00017 \textcolor{comment}{   window is a user-supplied window and output buffer that is 64K bytes.}
00018 \textcolor{comment}{ */}
00019 \textcolor{keywordtype}{int} ZEXPORT inflateBack9Init\_(strm, window, version, stream\_size)
00020 \hyperlink{structz__stream__s}{z\_stream} FAR *strm;
00021 \textcolor{keywordtype}{unsigned} \textcolor{keywordtype}{char} FAR *window;
00022 \textcolor{keyword}{const} \textcolor{keywordtype}{char} *version;
00023 \textcolor{keywordtype}{int} stream\_size;
00024 \{
00025     \textcolor{keyword}{struct }\hyperlink{structinflate__state}{inflate\_state} FAR *\hyperlink{structstate}{state};
00026 
00027     \textcolor{keywordflow}{if} (version == Z\_NULL || version[0] != ZLIB\_VERSION[0] ||
00028         stream\_size != (\textcolor{keywordtype}{int})(\textcolor{keyword}{sizeof}(\hyperlink{structz__stream__s}{z\_stream})))
00029         \textcolor{keywordflow}{return} Z\_VERSION\_ERROR;
00030     \textcolor{keywordflow}{if} (strm == Z\_NULL || window == Z\_NULL)
00031         \textcolor{keywordflow}{return} Z\_STREAM\_ERROR;
00032     strm->msg = Z\_NULL;                 \textcolor{comment}{/* in case we return an error */}
00033     \textcolor{keywordflow}{if} (strm->zalloc == (alloc\_func)0) \{
00034         strm->zalloc = zcalloc;
00035         strm->opaque = (voidpf)0;
00036     \}
00037     \textcolor{keywordflow}{if} (strm->zfree == (free\_func)0) strm->zfree = zcfree;
00038     state = (\textcolor{keyword}{struct} \hyperlink{structinflate__state}{inflate\_state} FAR *)ZALLOC(strm, 1,
00039                                                \textcolor{keyword}{sizeof}(\textcolor{keyword}{struct} \hyperlink{structinflate__state}{inflate\_state}));
00040     \textcolor{keywordflow}{if} (state == Z\_NULL) \textcolor{keywordflow}{return} Z\_MEM\_ERROR;
00041     Tracev((stderr, \textcolor{stringliteral}{"inflate: allocated\(\backslash\)n"}));
00042     strm->state = (voidpf)state;
00043     state->window = window;
00044     \textcolor{keywordflow}{return} Z\_OK;
00045 \}
00046 
00047 \textcolor{comment}{/*}
00048 \textcolor{comment}{   Build and output length and distance decoding tables for fixed code}
00049 \textcolor{comment}{   decoding.}
00050 \textcolor{comment}{ */}
00051 \textcolor{preprocessor}{#ifdef MAKEFIXED}
00052 \textcolor{preprocessor}{#include <stdio.h>}
00053 
00054 \textcolor{keywordtype}{void} makefixed9(\textcolor{keywordtype}{void})
00055 \{
00056     \textcolor{keywordtype}{unsigned} sym, bits, low, size;
00057     \hyperlink{structcode}{code} *next, *lenfix, *distfix;
00058     \textcolor{keyword}{struct }\hyperlink{structinflate__state}{inflate\_state} state;
00059     \hyperlink{structcode}{code} fixed[544];
00060 
00061     \textcolor{comment}{/* literal/length table */}
00062     sym = 0;
00063     \textcolor{keywordflow}{while} (sym < 144) state.lens[sym++] = 8;
00064     \textcolor{keywordflow}{while} (sym < 256) state.lens[sym++] = 9;
00065     \textcolor{keywordflow}{while} (sym < 280) state.lens[sym++] = 7;
00066     \textcolor{keywordflow}{while} (sym < 288) state.lens[sym++] = 8;
00067     next = fixed;
00068     lenfix = next;
00069     bits = 9;
00070     inflate\_table9(LENS, state.lens, 288, &(next), &(bits), state.work);
00071 
00072     \textcolor{comment}{/* distance table */}
00073     sym = 0;
00074     \textcolor{keywordflow}{while} (sym < 32) state.lens[sym++] = 5;
00075     distfix = next;
00076     bits = 5;
00077     inflate\_table9(DISTS, state.lens, 32, &(next), &(bits), state.work);
00078 
00079     \textcolor{comment}{/* write tables */}
00080     puts(\textcolor{stringliteral}{"    /* inffix9.h -- table for decoding deflate64 fixed codes"});
00081     puts(\textcolor{stringliteral}{"     * Generated automatically by makefixed9()."});
00082     puts(\textcolor{stringliteral}{"     */"});
00083     puts(\textcolor{stringliteral}{""});
00084     puts(\textcolor{stringliteral}{"    /* WARNING: this file should *not* be used by applications."});
00085     puts(\textcolor{stringliteral}{"       It is part of the implementation of this library and is"});
00086     puts(\textcolor{stringliteral}{"       subject to change. Applications should only use zlib.h."});
00087     puts(\textcolor{stringliteral}{"     */"});
00088     puts(\textcolor{stringliteral}{""});
00089     size = 1U << 9;
00090     printf(\textcolor{stringliteral}{"    static const code lenfix[%u] = \{"}, size);
00091     low = 0;
00092     \textcolor{keywordflow}{for} (;;) \{
00093         \textcolor{keywordflow}{if} ((low % 6) == 0) printf(\textcolor{stringliteral}{"\(\backslash\)n        "});
00094         printf(\textcolor{stringliteral}{"\{%u,%u,%d\}"}, lenfix[low].op, lenfix[low].bits,
00095                lenfix[low].val);
00096         \textcolor{keywordflow}{if} (++low == size) \textcolor{keywordflow}{break};
00097         putchar(\textcolor{charliteral}{','});
00098     \}
00099     puts(\textcolor{stringliteral}{"\(\backslash\)n    \};"});
00100     size = 1U << 5;
00101     printf(\textcolor{stringliteral}{"\(\backslash\)n    static const code distfix[%u] = \{"}, size);
00102     low = 0;
00103     \textcolor{keywordflow}{for} (;;) \{
00104         \textcolor{keywordflow}{if} ((low % 5) == 0) printf(\textcolor{stringliteral}{"\(\backslash\)n        "});
00105         printf(\textcolor{stringliteral}{"\{%u,%u,%d\}"}, distfix[low].op, distfix[low].bits,
00106                distfix[low].val);
00107         \textcolor{keywordflow}{if} (++low == size) \textcolor{keywordflow}{break};
00108         putchar(\textcolor{charliteral}{','});
00109     \}
00110     puts(\textcolor{stringliteral}{"\(\backslash\)n    \};"});
00111 \}
00112 \textcolor{preprocessor}{#endif }\textcolor{comment}{/* MAKEFIXED */}\textcolor{preprocessor}{}
00113 
00114 \textcolor{comment}{/* Macros for inflateBack(): */}
00115 
00116 \textcolor{comment}{/* Clear the input bit accumulator */}
00117 \textcolor{preprocessor}{#define INITBITS() \(\backslash\)}
00118 \textcolor{preprocessor}{    do \{ \(\backslash\)}
00119 \textcolor{preprocessor}{        hold = 0; \(\backslash\)}
00120 \textcolor{preprocessor}{        bits = 0; \(\backslash\)}
00121 \textcolor{preprocessor}{    \} while (0)}
00122 
00123 \textcolor{comment}{/* Assure that some input is available.  If input is requested, but denied,}
00124 \textcolor{comment}{   then return a Z\_BUF\_ERROR from inflateBack(). */}
00125 \textcolor{preprocessor}{#define PULL() \(\backslash\)}
00126 \textcolor{preprocessor}{    do \{ \(\backslash\)}
00127 \textcolor{preprocessor}{        if (have == 0) \{ \(\backslash\)}
00128 \textcolor{preprocessor}{            have = in(in\_desc, &next); \(\backslash\)}
00129 \textcolor{preprocessor}{            if (have == 0) \{ \(\backslash\)}
00130 \textcolor{preprocessor}{                next = Z\_NULL; \(\backslash\)}
00131 \textcolor{preprocessor}{                ret = Z\_BUF\_ERROR; \(\backslash\)}
00132 \textcolor{preprocessor}{                goto inf\_leave; \(\backslash\)}
00133 \textcolor{preprocessor}{            \} \(\backslash\)}
00134 \textcolor{preprocessor}{        \} \(\backslash\)}
00135 \textcolor{preprocessor}{    \} while (0)}
00136 
00137 \textcolor{comment}{/* Get a byte of input into the bit accumulator, or return from inflateBack()}
00138 \textcolor{comment}{   with an error if there is no input available. */}
00139 \textcolor{preprocessor}{#define PULLBYTE() \(\backslash\)}
00140 \textcolor{preprocessor}{    do \{ \(\backslash\)}
00141 \textcolor{preprocessor}{        PULL(); \(\backslash\)}
00142 \textcolor{preprocessor}{        have--; \(\backslash\)}
00143 \textcolor{preprocessor}{        hold += (unsigned long)(*next++) << bits; \(\backslash\)}
00144 \textcolor{preprocessor}{        bits += 8; \(\backslash\)}
00145 \textcolor{preprocessor}{    \} while (0)}
00146 
00147 \textcolor{comment}{/* Assure that there are at least n bits in the bit accumulator.  If there is}
00148 \textcolor{comment}{   not enough available input to do that, then return from inflateBack() with}
00149 \textcolor{comment}{   an error. */}
00150 \textcolor{preprocessor}{#define NEEDBITS(n) \(\backslash\)}
00151 \textcolor{preprocessor}{    do \{ \(\backslash\)}
00152 \textcolor{preprocessor}{        while (bits < (unsigned)(n)) \(\backslash\)}
00153 \textcolor{preprocessor}{            PULLBYTE(); \(\backslash\)}
00154 \textcolor{preprocessor}{    \} while (0)}
00155 
00156 \textcolor{comment}{/* Return the low n bits of the bit accumulator (n <= 16) */}
00157 \textcolor{preprocessor}{#define BITS(n) \(\backslash\)}
00158 \textcolor{preprocessor}{    ((unsigned)hold & ((1U << (n)) - 1))}
00159 
00160 \textcolor{comment}{/* Remove n bits from the bit accumulator */}
00161 \textcolor{preprocessor}{#define DROPBITS(n) \(\backslash\)}
00162 \textcolor{preprocessor}{    do \{ \(\backslash\)}
00163 \textcolor{preprocessor}{        hold >>= (n); \(\backslash\)}
00164 \textcolor{preprocessor}{        bits -= (unsigned)(n); \(\backslash\)}
00165 \textcolor{preprocessor}{    \} while (0)}
00166 
00167 \textcolor{comment}{/* Remove zero to seven bits as needed to go to a byte boundary */}
00168 \textcolor{preprocessor}{#define BYTEBITS() \(\backslash\)}
00169 \textcolor{preprocessor}{    do \{ \(\backslash\)}
00170 \textcolor{preprocessor}{        hold >>= bits & 7; \(\backslash\)}
00171 \textcolor{preprocessor}{        bits -= bits & 7; \(\backslash\)}
00172 \textcolor{preprocessor}{    \} while (0)}
00173 
00174 \textcolor{comment}{/* Assure that some output space is available, by writing out the window}
00175 \textcolor{comment}{   if it's full.  If the write fails, return from inflateBack() with a}
00176 \textcolor{comment}{   Z\_BUF\_ERROR. */}
00177 \textcolor{preprocessor}{#define ROOM() \(\backslash\)}
00178 \textcolor{preprocessor}{    do \{ \(\backslash\)}
00179 \textcolor{preprocessor}{        if (left == 0) \{ \(\backslash\)}
00180 \textcolor{preprocessor}{            put = window; \(\backslash\)}
00181 \textcolor{preprocessor}{            left = WSIZE; \(\backslash\)}
00182 \textcolor{preprocessor}{            wrap = 1; \(\backslash\)}
00183 \textcolor{preprocessor}{            if (out(out\_desc, put, (unsigned)left)) \{ \(\backslash\)}
00184 \textcolor{preprocessor}{                ret = Z\_BUF\_ERROR; \(\backslash\)}
00185 \textcolor{preprocessor}{                goto inf\_leave; \(\backslash\)}
00186 \textcolor{preprocessor}{            \} \(\backslash\)}
00187 \textcolor{preprocessor}{        \} \(\backslash\)}
00188 \textcolor{preprocessor}{    \} while (0)}
00189 
00190 \textcolor{comment}{/*}
00191 \textcolor{comment}{   strm provides the memory allocation functions and window buffer on input,}
00192 \textcolor{comment}{   and provides information on the unused input on return.  For Z\_DATA\_ERROR}
00193 \textcolor{comment}{   returns, strm will also provide an error message.}
00194 \textcolor{comment}{}
00195 \textcolor{comment}{   in() and out() are the call-back input and output functions.  When}
00196 \textcolor{comment}{   inflateBack() needs more input, it calls in().  When inflateBack() has}
00197 \textcolor{comment}{   filled the window with output, or when it completes with data in the}
00198 \textcolor{comment}{   window, it calls out() to write out the data.  The application must not}
00199 \textcolor{comment}{   change the provided input until in() is called again or inflateBack()}
00200 \textcolor{comment}{   returns.  The application must not change the window/output buffer until}
00201 \textcolor{comment}{   inflateBack() returns.}
00202 \textcolor{comment}{}
00203 \textcolor{comment}{   in() and out() are called with a descriptor parameter provided in the}
00204 \textcolor{comment}{   inflateBack() call.  This parameter can be a structure that provides the}
00205 \textcolor{comment}{   information required to do the read or write, as well as accumulated}
00206 \textcolor{comment}{   information on the input and output such as totals and check values.}
00207 \textcolor{comment}{}
00208 \textcolor{comment}{   in() should return zero on failure.  out() should return non-zero on}
00209 \textcolor{comment}{   failure.  If either in() or out() fails, than inflateBack() returns a}
00210 \textcolor{comment}{   Z\_BUF\_ERROR.  strm->next\_in can be checked for Z\_NULL to see whether it}
00211 \textcolor{comment}{   was in() or out() that caused in the error.  Otherwise,  inflateBack()}
00212 \textcolor{comment}{   returns Z\_STREAM\_END on success, Z\_DATA\_ERROR for an deflate format}
00213 \textcolor{comment}{   error, or Z\_MEM\_ERROR if it could not allocate memory for the state.}
00214 \textcolor{comment}{   inflateBack() can also return Z\_STREAM\_ERROR if the input parameters}
00215 \textcolor{comment}{   are not correct, i.e. strm is Z\_NULL or the state was not initialized.}
00216 \textcolor{comment}{ */}
00217 \textcolor{keywordtype}{int} ZEXPORT inflateBack9(strm, in, in\_desc, out, out\_desc)
00218 \hyperlink{structz__stream__s}{z\_stream} FAR *strm;
00219 in\_func in;
00220 \textcolor{keywordtype}{void} FAR *in\_desc;
00221 out\_func out;
00222 \textcolor{keywordtype}{void} FAR *out\_desc;
00223 \{
00224     \textcolor{keyword}{struct }\hyperlink{structinflate__state}{inflate\_state} FAR *state;
00225     z\_const \textcolor{keywordtype}{unsigned} \textcolor{keywordtype}{char} FAR *next;    \textcolor{comment}{/* next input */}
00226     \textcolor{keywordtype}{unsigned} \textcolor{keywordtype}{char} FAR *put;     \textcolor{comment}{/* next output */}
00227     \textcolor{keywordtype}{unsigned} have;              \textcolor{comment}{/* available input */}
00228     \textcolor{keywordtype}{unsigned} \textcolor{keywordtype}{long} left;         \textcolor{comment}{/* available output */}
00229     inflate\_mode mode;          \textcolor{comment}{/* current inflate mode */}
00230     \textcolor{keywordtype}{int} lastblock;              \textcolor{comment}{/* true if processing last block */}
00231     \textcolor{keywordtype}{int} wrap;                   \textcolor{comment}{/* true if the window has wrapped */}
00232     \textcolor{keywordtype}{unsigned} \textcolor{keywordtype}{char} FAR *window;  \textcolor{comment}{/* allocated sliding window, if needed */}
00233     \textcolor{keywordtype}{unsigned} \textcolor{keywordtype}{long} hold;         \textcolor{comment}{/* bit buffer */}
00234     \textcolor{keywordtype}{unsigned} bits;              \textcolor{comment}{/* bits in bit buffer */}
00235     \textcolor{keywordtype}{unsigned} extra;             \textcolor{comment}{/* extra bits needed */}
00236     \textcolor{keywordtype}{unsigned} \textcolor{keywordtype}{long} length;       \textcolor{comment}{/* literal or length of data to copy */}
00237     \textcolor{keywordtype}{unsigned} \textcolor{keywordtype}{long} offset;       \textcolor{comment}{/* distance back to copy string from */}
00238     \textcolor{keywordtype}{unsigned} \textcolor{keywordtype}{long} copy;         \textcolor{comment}{/* number of stored or match bytes to copy */}
00239     \textcolor{keywordtype}{unsigned} \textcolor{keywordtype}{char} FAR *from;    \textcolor{comment}{/* where to copy match bytes from */}
00240     \hyperlink{structcode}{code} \textcolor{keyword}{const} FAR *lencode;    \textcolor{comment}{/* starting table for length/literal codes */}
00241     \hyperlink{structcode}{code} \textcolor{keyword}{const} FAR *distcode;   \textcolor{comment}{/* starting table for distance codes */}
00242     \textcolor{keywordtype}{unsigned} lenbits;           \textcolor{comment}{/* index bits for lencode */}
00243     \textcolor{keywordtype}{unsigned} distbits;          \textcolor{comment}{/* index bits for distcode */}
00244     \hyperlink{structcode}{code} here;                  \textcolor{comment}{/* current decoding table entry */}
00245     \hyperlink{structcode}{code} last;                  \textcolor{comment}{/* parent table entry */}
00246     \textcolor{keywordtype}{unsigned} len;               \textcolor{comment}{/* length to copy for repeats, bits to drop */}
00247     \textcolor{keywordtype}{int} ret;                    \textcolor{comment}{/* return code */}
00248     \textcolor{keyword}{static} \textcolor{keyword}{const} \textcolor{keywordtype}{unsigned} \textcolor{keywordtype}{short} order[19] = \textcolor{comment}{/* permutation of code lengths */}
00249         \{16, 17, 18, 0, 8, 7, 9, 6, 10, 5, 11, 4, 12, 3, 13, 2, 14, 1, 15\};
00250 \textcolor{preprocessor}{#include "inffix9.h"}
00251 
00252     \textcolor{comment}{/* Check that the strm exists and that the state was initialized */}
00253     \textcolor{keywordflow}{if} (strm == Z\_NULL || strm->state == Z\_NULL)
00254         \textcolor{keywordflow}{return} Z\_STREAM\_ERROR;
00255     state = (\textcolor{keyword}{struct }\hyperlink{structinflate__state}{inflate\_state} FAR *)strm->state;
00256 
00257     \textcolor{comment}{/* Reset the state */}
00258     strm->msg = Z\_NULL;
00259     mode = TYPE;
00260     lastblock = 0;
00261     wrap = 0;
00262     window = state->window;
00263     next = strm->next\_in;
00264     have = next != Z\_NULL ? strm->avail\_in : 0;
00265     hold = 0;
00266     bits = 0;
00267     put = window;
00268     left = WSIZE;
00269     lencode = Z\_NULL;
00270     distcode = Z\_NULL;
00271 
00272     \textcolor{comment}{/* Inflate until end of block marked as last */}
00273     for (;;)
00274         \textcolor{keywordflow}{switch} (mode) \{
00275         \textcolor{keywordflow}{case} TYPE:
00276             \textcolor{comment}{/* determine and dispatch block type */}
00277             \textcolor{keywordflow}{if} (lastblock) \{
00278                 BYTEBITS();
00279                 mode = DONE;
00280                 \textcolor{keywordflow}{break};
00281             \}
00282             NEEDBITS(3);
00283             lastblock = BITS(1);
00284             DROPBITS(1);
00285             \textcolor{keywordflow}{switch} (BITS(2)) \{
00286             \textcolor{keywordflow}{case} 0:                             \textcolor{comment}{/* stored block */}
00287                 Tracev((stderr, \textcolor{stringliteral}{"inflate:     stored block%s\(\backslash\)n"},
00288                         lastblock ? \textcolor{stringliteral}{" (last)"} : \textcolor{stringliteral}{""}));
00289                 mode = STORED;
00290                 \textcolor{keywordflow}{break};
00291             \textcolor{keywordflow}{case} 1:                             \textcolor{comment}{/* fixed block */}
00292                 lencode = lenfix;
00293                 lenbits = 9;
00294                 distcode = distfix;
00295                 distbits = 5;
00296                 Tracev((stderr, \textcolor{stringliteral}{"inflate:     fixed codes block%s\(\backslash\)n"},
00297                         lastblock ? \textcolor{stringliteral}{" (last)"} : \textcolor{stringliteral}{""}));
00298                 mode = LEN;                     \textcolor{comment}{/* decode codes */}
00299                 \textcolor{keywordflow}{break};
00300             \textcolor{keywordflow}{case} 2:                             \textcolor{comment}{/* dynamic block */}
00301                 Tracev((stderr, \textcolor{stringliteral}{"inflate:     dynamic codes block%s\(\backslash\)n"},
00302                         lastblock ? \textcolor{stringliteral}{" (last)"} : \textcolor{stringliteral}{""}));
00303                 mode = TABLE;
00304                 \textcolor{keywordflow}{break};
00305             \textcolor{keywordflow}{case} 3:
00306                 strm->msg = (\textcolor{keywordtype}{char} *)\textcolor{stringliteral}{"invalid block type"};
00307                 mode = BAD;
00308             \}
00309             DROPBITS(2);
00310             \textcolor{keywordflow}{break};
00311 
00312         \textcolor{keywordflow}{case} STORED:
00313             \textcolor{comment}{/* get and verify stored block length */}
00314             BYTEBITS();                         \textcolor{comment}{/* go to byte boundary */}
00315             NEEDBITS(32);
00316             \textcolor{keywordflow}{if} ((hold & 0xffff) != ((hold >> 16) ^ 0xffff)) \{
00317                 strm->msg = (\textcolor{keywordtype}{char} *)\textcolor{stringliteral}{"invalid stored block lengths"};
00318                 mode = BAD;
00319                 \textcolor{keywordflow}{break};
00320             \}
00321             length = (unsigned)hold & 0xffff;
00322             Tracev((stderr, \textcolor{stringliteral}{"inflate:       stored length %lu\(\backslash\)n"},
00323                     length));
00324             INITBITS();
00325 
00326             \textcolor{comment}{/* copy stored block from input to output */}
00327             \textcolor{keywordflow}{while} (length != 0) \{
00328                 copy = length;
00329                 PULL();
00330                 ROOM();
00331                 \textcolor{keywordflow}{if} (copy > have) copy = have;
00332                 \textcolor{keywordflow}{if} (copy > left) copy = left;
00333                 zmemcpy(put, next, copy);
00334                 have -= copy;
00335                 next += copy;
00336                 left -= copy;
00337                 put += copy;
00338                 length -= copy;
00339             \}
00340             Tracev((stderr, \textcolor{stringliteral}{"inflate:       stored end\(\backslash\)n"}));
00341             mode = TYPE;
00342             \textcolor{keywordflow}{break};
00343 
00344         \textcolor{keywordflow}{case} TABLE:
00345             \textcolor{comment}{/* get dynamic table entries descriptor */}
00346             NEEDBITS(14);
00347             state->nlen = BITS(5) + 257;
00348             DROPBITS(5);
00349             state->ndist = BITS(5) + 1;
00350             DROPBITS(5);
00351             state->ncode = BITS(4) + 4;
00352             DROPBITS(4);
00353             \textcolor{keywordflow}{if} (state->nlen > 286) \{
00354                 strm->msg = (\textcolor{keywordtype}{char} *)\textcolor{stringliteral}{"too many length symbols"};
00355                 mode = BAD;
00356                 \textcolor{keywordflow}{break};
00357             \}
00358             Tracev((stderr, \textcolor{stringliteral}{"inflate:       table sizes ok\(\backslash\)n"}));
00359 
00360             \textcolor{comment}{/* get code length code lengths (not a typo) */}
00361             state->have = 0;
00362             \textcolor{keywordflow}{while} (state->have < state->ncode) \{
00363                 NEEDBITS(3);
00364                 state->lens[order[state->have++]] = (\textcolor{keywordtype}{unsigned} short)BITS(3);
00365                 DROPBITS(3);
00366             \}
00367             \textcolor{keywordflow}{while} (state->have < 19)
00368                 state->lens[order[state->have++]] = 0;
00369             state->next = state->codes;
00370             lencode = (\hyperlink{structcode}{code} \textcolor{keyword}{const} FAR *)(state->next);
00371             lenbits = 7;
00372             ret = inflate\_table9(CODES, state->lens, 19, &(state->next),
00373                                 &(lenbits), state->work);
00374             \textcolor{keywordflow}{if} (ret) \{
00375                 strm->msg = (\textcolor{keywordtype}{char} *)\textcolor{stringliteral}{"invalid code lengths set"};
00376                 mode = BAD;
00377                 \textcolor{keywordflow}{break};
00378             \}
00379             Tracev((stderr, \textcolor{stringliteral}{"inflate:       code lengths ok\(\backslash\)n"}));
00380 
00381             \textcolor{comment}{/* get length and distance code code lengths */}
00382             state->have = 0;
00383             \textcolor{keywordflow}{while} (state->have < state->nlen + state->ndist) \{
00384                 \textcolor{keywordflow}{for} (;;) \{
00385                     here = lencode[BITS(lenbits)];
00386                     \textcolor{keywordflow}{if} ((\textcolor{keywordtype}{unsigned})(here.bits) <= bits) \textcolor{keywordflow}{break};
00387                     PULLBYTE();
00388                 \}
00389                 \textcolor{keywordflow}{if} (here.val < 16) \{
00390                     NEEDBITS(here.bits);
00391                     DROPBITS(here.bits);
00392                     state->lens[state->have++] = here.val;
00393                 \}
00394                 \textcolor{keywordflow}{else} \{
00395                     \textcolor{keywordflow}{if} (here.val == 16) \{
00396                         NEEDBITS(here.bits + 2);
00397                         DROPBITS(here.bits);
00398                         \textcolor{keywordflow}{if} (state->have == 0) \{
00399                             strm->msg = (\textcolor{keywordtype}{char} *)\textcolor{stringliteral}{"invalid bit length repeat"};
00400                             mode = BAD;
00401                             \textcolor{keywordflow}{break};
00402                         \}
00403                         len = (unsigned)(state->lens[state->have - 1]);
00404                         copy = 3 + BITS(2);
00405                         DROPBITS(2);
00406                     \}
00407                     \textcolor{keywordflow}{else} \textcolor{keywordflow}{if} (here.val == 17) \{
00408                         NEEDBITS(here.bits + 3);
00409                         DROPBITS(here.bits);
00410                         len = 0;
00411                         copy = 3 + BITS(3);
00412                         DROPBITS(3);
00413                     \}
00414                     \textcolor{keywordflow}{else} \{
00415                         NEEDBITS(here.bits + 7);
00416                         DROPBITS(here.bits);
00417                         len = 0;
00418                         copy = 11 + BITS(7);
00419                         DROPBITS(7);
00420                     \}
00421                     \textcolor{keywordflow}{if} (state->have + copy > state->nlen + state->ndist) \{
00422                         strm->msg = (\textcolor{keywordtype}{char} *)\textcolor{stringliteral}{"invalid bit length repeat"};
00423                         mode = BAD;
00424                         \textcolor{keywordflow}{break};
00425                     \}
00426                     \textcolor{keywordflow}{while} (copy--)
00427                         state->lens[state->have++] = (\textcolor{keywordtype}{unsigned} short)len;
00428                 \}
00429             \}
00430 
00431             \textcolor{comment}{/* handle error breaks in while */}
00432             \textcolor{keywordflow}{if} (mode == BAD) \textcolor{keywordflow}{break};
00433 
00434             \textcolor{comment}{/* check for end-of-block code (better have one) */}
00435             \textcolor{keywordflow}{if} (state->lens[256] == 0) \{
00436                 strm->msg = (\textcolor{keywordtype}{char} *)\textcolor{stringliteral}{"invalid code -- missing end-of-block"};
00437                 mode = BAD;
00438                 \textcolor{keywordflow}{break};
00439             \}
00440 
00441             \textcolor{comment}{/* build code tables -- note: do not change the lenbits or distbits}
00442 \textcolor{comment}{               values here (9 and 6) without reading the comments in inftree9.h}
00443 \textcolor{comment}{               concerning the ENOUGH constants, which depend on those values */}
00444             state->next = state->codes;
00445             lencode = (\hyperlink{structcode}{code} \textcolor{keyword}{const} FAR *)(state->next);
00446             lenbits = 9;
00447             ret = inflate\_table9(LENS, state->lens, state->nlen,
00448                             &(state->next), &(lenbits), state->work);
00449             \textcolor{keywordflow}{if} (ret) \{
00450                 strm->msg = (\textcolor{keywordtype}{char} *)\textcolor{stringliteral}{"invalid literal/lengths set"};
00451                 mode = BAD;
00452                 \textcolor{keywordflow}{break};
00453             \}
00454             distcode = (\hyperlink{structcode}{code} \textcolor{keyword}{const} FAR *)(state->next);
00455             distbits = 6;
00456             ret = inflate\_table9(DISTS, state->lens + state->nlen,
00457                             state->ndist, &(state->next), &(distbits),
00458                             state->work);
00459             \textcolor{keywordflow}{if} (ret) \{
00460                 strm->msg = (\textcolor{keywordtype}{char} *)\textcolor{stringliteral}{"invalid distances set"};
00461                 mode = BAD;
00462                 \textcolor{keywordflow}{break};
00463             \}
00464             Tracev((stderr, \textcolor{stringliteral}{"inflate:       codes ok\(\backslash\)n"}));
00465             mode = LEN;
00466 
00467         \textcolor{keywordflow}{case} LEN:
00468             \textcolor{comment}{/* get a literal, length, or end-of-block code */}
00469             \textcolor{keywordflow}{for} (;;) \{
00470                 here = lencode[BITS(lenbits)];
00471                 \textcolor{keywordflow}{if} ((\textcolor{keywordtype}{unsigned})(here.bits) <= bits) \textcolor{keywordflow}{break};
00472                 PULLBYTE();
00473             \}
00474             \textcolor{keywordflow}{if} (here.op && (here.op & 0xf0) == 0) \{
00475                 last = here;
00476                 \textcolor{keywordflow}{for} (;;) \{
00477                     here = lencode[last.val +
00478                             (BITS(last.bits + last.op) >> last.bits)];
00479                     \textcolor{keywordflow}{if} ((\textcolor{keywordtype}{unsigned})(last.bits + here.bits) <= bits) \textcolor{keywordflow}{break};
00480                     PULLBYTE();
00481                 \}
00482                 DROPBITS(last.bits);
00483             \}
00484             DROPBITS(here.bits);
00485             length = (unsigned)here.val;
00486 
00487             \textcolor{comment}{/* process literal */}
00488             if (here.op == 0) \{
00489                 Tracevv((stderr, here.val >= 0x20 && here.val < 0x7f ?
00490                         \textcolor{stringliteral}{"inflate:         literal '%c'\(\backslash\)n"} :
00491                         \textcolor{stringliteral}{"inflate:         literal 0x%02x\(\backslash\)n"}, here.val));
00492                 ROOM();
00493                 *put++ = (\textcolor{keywordtype}{unsigned} char)(length);
00494                 left--;
00495                 mode = LEN;
00496                 \textcolor{keywordflow}{break};
00497             \}
00498 
00499             \textcolor{comment}{/* process end of block */}
00500             \textcolor{keywordflow}{if} (here.op & 32) \{
00501                 Tracevv((stderr, \textcolor{stringliteral}{"inflate:         end of block\(\backslash\)n"}));
00502                 mode = TYPE;
00503                 \textcolor{keywordflow}{break};
00504             \}
00505 
00506             \textcolor{comment}{/* invalid code */}
00507             \textcolor{keywordflow}{if} (here.op & 64) \{
00508                 strm->msg = (\textcolor{keywordtype}{char} *)\textcolor{stringliteral}{"invalid literal/length code"};
00509                 mode = BAD;
00510                 \textcolor{keywordflow}{break};
00511             \}
00512 
00513             \textcolor{comment}{/* length code -- get extra bits, if any */}
00514             extra = (unsigned)(here.op) & 31;
00515             \textcolor{keywordflow}{if} (extra != 0) \{
00516                 NEEDBITS(extra);
00517                 length += BITS(extra);
00518                 DROPBITS(extra);
00519             \}
00520             Tracevv((stderr, \textcolor{stringliteral}{"inflate:         length %lu\(\backslash\)n"}, length));
00521 
00522             \textcolor{comment}{/* get distance code */}
00523             \textcolor{keywordflow}{for} (;;) \{
00524                 here = distcode[BITS(distbits)];
00525                 \textcolor{keywordflow}{if} ((\textcolor{keywordtype}{unsigned})(here.bits) <= bits) \textcolor{keywordflow}{break};
00526                 PULLBYTE();
00527             \}
00528             \textcolor{keywordflow}{if} ((here.op & 0xf0) == 0) \{
00529                 last = here;
00530                 \textcolor{keywordflow}{for} (;;) \{
00531                     here = distcode[last.val +
00532                             (BITS(last.bits + last.op) >> last.bits)];
00533                     \textcolor{keywordflow}{if} ((\textcolor{keywordtype}{unsigned})(last.bits + here.bits) <= bits) \textcolor{keywordflow}{break};
00534                     PULLBYTE();
00535                 \}
00536                 DROPBITS(last.bits);
00537             \}
00538             DROPBITS(here.bits);
00539             \textcolor{keywordflow}{if} (here.op & 64) \{
00540                 strm->msg = (\textcolor{keywordtype}{char} *)\textcolor{stringliteral}{"invalid distance code"};
00541                 mode = BAD;
00542                 \textcolor{keywordflow}{break};
00543             \}
00544             offset = (unsigned)here.val;
00545 
00546             \textcolor{comment}{/* get distance extra bits, if any */}
00547             extra = (\textcolor{keywordtype}{unsigned})(here.op) & 15;
00548             \textcolor{keywordflow}{if} (extra != 0) \{
00549                 NEEDBITS(extra);
00550                 offset += BITS(extra);
00551                 DROPBITS(extra);
00552             \}
00553             \textcolor{keywordflow}{if} (offset > WSIZE - (wrap ? 0: left)) \{
00554                 strm->msg = (\textcolor{keywordtype}{char} *)\textcolor{stringliteral}{"invalid distance too far back"};
00555                 mode = BAD;
00556                 \textcolor{keywordflow}{break};
00557             \}
00558             Tracevv((stderr, \textcolor{stringliteral}{"inflate:         distance %lu\(\backslash\)n"}, offset));
00559 
00560             \textcolor{comment}{/* copy match from window to output */}
00561             \textcolor{keywordflow}{do} \{
00562                 ROOM();
00563                 copy = WSIZE - offset;
00564                 \textcolor{keywordflow}{if} (copy < left) \{
00565                     from = put + copy;
00566                     copy = left - copy;
00567                 \}
00568                 \textcolor{keywordflow}{else} \{
00569                     from = put - offset;
00570                     copy = left;
00571                 \}
00572                 \textcolor{keywordflow}{if} (copy > length) copy = length;
00573                 length -= copy;
00574                 left -= copy;
00575                 \textcolor{keywordflow}{do} \{
00576                     *put++ = *from++;
00577                 \} \textcolor{keywordflow}{while} (--copy);
00578             \} \textcolor{keywordflow}{while} (length != 0);
00579             \textcolor{keywordflow}{break};
00580 
00581         \textcolor{keywordflow}{case} DONE:
00582             \textcolor{comment}{/* inflate stream terminated properly -- write leftover output */}
00583             ret = Z\_STREAM\_END;
00584             \textcolor{keywordflow}{if} (left < WSIZE) \{
00585                 \textcolor{keywordflow}{if} (out(out\_desc, window, (\textcolor{keywordtype}{unsigned})(WSIZE - left)))
00586                     ret = Z\_BUF\_ERROR;
00587             \}
00588             \textcolor{keywordflow}{goto} inf\_leave;
00589 
00590         \textcolor{keywordflow}{case} BAD:
00591             ret = Z\_DATA\_ERROR;
00592             \textcolor{keywordflow}{goto} inf\_leave;
00593 
00594         \textcolor{keywordflow}{default}:                \textcolor{comment}{/* can't happen, but makes compilers happy */}
00595             ret = Z\_STREAM\_ERROR;
00596             \textcolor{keywordflow}{goto} inf\_leave;
00597         \}
00598 
00599     \textcolor{comment}{/* Return unused input */}
00600   inf\_leave:
00601     strm->next\_in = next;
00602     strm->avail\_in = have;
00603     \textcolor{keywordflow}{return} ret;
00604 \}
00605 
00606 \textcolor{keywordtype}{int} ZEXPORT inflateBack9End(strm)
00607 \hyperlink{structz__stream__s}{z\_stream} FAR *strm;
00608 \{
00609     \textcolor{keywordflow}{if} (strm == Z\_NULL || strm->state == Z\_NULL || strm->zfree == (free\_func)0)
00610         \textcolor{keywordflow}{return} Z\_STREAM\_ERROR;
00611     ZFREE(strm, strm->state);
00612     strm->state = Z\_NULL;
00613     Tracev((stderr, \textcolor{stringliteral}{"inflate: end\(\backslash\)n"}));
00614     \textcolor{keywordflow}{return} Z\_OK;
00615 \}
\end{DoxyCode}
