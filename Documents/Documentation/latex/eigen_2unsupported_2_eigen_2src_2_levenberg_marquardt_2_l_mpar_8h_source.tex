\hypertarget{eigen_2unsupported_2_eigen_2src_2_levenberg_marquardt_2_l_mpar_8h_source}{}\section{eigen/unsupported/\+Eigen/src/\+Levenberg\+Marquardt/\+L\+Mpar.h}
\label{eigen_2unsupported_2_eigen_2src_2_levenberg_marquardt_2_l_mpar_8h_source}\index{L\+Mpar.\+h@{L\+Mpar.\+h}}

\begin{DoxyCode}
00001 \textcolor{comment}{// This file is part of Eigen, a lightweight C++ template library}
00002 \textcolor{comment}{// for linear algebra.}
00003 \textcolor{comment}{//}
00004 \textcolor{comment}{// This code initially comes from MINPACK whose original authors are:}
00005 \textcolor{comment}{// Copyright Jorge More - Argonne National Laboratory}
00006 \textcolor{comment}{// Copyright Burt Garbow - Argonne National Laboratory}
00007 \textcolor{comment}{// Copyright Ken Hillstrom - Argonne National Laboratory}
00008 \textcolor{comment}{//}
00009 \textcolor{comment}{// This Source Code Form is subject to the terms of the Minpack license}
00010 \textcolor{comment}{// (a BSD-like license) described in the campaigned CopyrightMINPACK.txt file.}
00011 
00012 \textcolor{preprocessor}{#ifndef EIGEN\_LMPAR\_H}
00013 \textcolor{preprocessor}{#define EIGEN\_LMPAR\_H}
00014 
00015 \textcolor{keyword}{namespace }\hyperlink{namespace_eigen}{Eigen} \{
00016 
00017 \textcolor{keyword}{namespace }\hyperlink{namespaceinternal}{internal} \{
00018   
00019   \textcolor{keyword}{template} <\textcolor{keyword}{typename} QRSolver, \textcolor{keyword}{typename} VectorType>
00020     \textcolor{keywordtype}{void} lmpar2(
00021     \textcolor{keyword}{const} QRSolver &qr,
00022     \textcolor{keyword}{const} \hyperlink{struct_vector_type}{VectorType}  &diag,
00023     \textcolor{keyword}{const} \hyperlink{struct_vector_type}{VectorType}  &qtb,
00024     \textcolor{keyword}{typename} VectorType::Scalar m\_delta,
00025     \textcolor{keyword}{typename} VectorType::Scalar &par,
00026     \hyperlink{struct_vector_type}{VectorType}  &x)
00027 
00028   \{
00029     \textcolor{keyword}{using} std::sqrt;
00030     \textcolor{keyword}{using} std::abs;
00031     \textcolor{keyword}{typedef} \textcolor{keyword}{typename} QRSolver::MatrixType MatrixType;
00032     \textcolor{keyword}{typedef} \textcolor{keyword}{typename} QRSolver::Scalar Scalar;
00033 \textcolor{comment}{//    typedef typename QRSolver::StorageIndex StorageIndex;}
00034 
00035     \textcolor{comment}{/* Local variables */}
00036     \hyperlink{namespace_eigen_a62e77e0933482dafde8fe197d9a2cfde}{Index} j;
00037     Scalar fp;
00038     Scalar parc, parl;
00039     \hyperlink{namespace_eigen_a62e77e0933482dafde8fe197d9a2cfde}{Index} iter;
00040     Scalar temp, paru;
00041     Scalar gnorm;
00042     Scalar dxnorm;
00043     
00044     \textcolor{comment}{// Make a copy of the triangular factor. }
00045     \textcolor{comment}{// This copy is modified during call the qrsolv}
00046     MatrixType s;
00047     s = qr.matrixR();
00048 
00049     \textcolor{comment}{/* Function Body */}
00050     \textcolor{keyword}{const} Scalar dwarf = (std::numeric\_limits<Scalar>::min)();
00051     \textcolor{keyword}{const} \hyperlink{namespace_eigen_a62e77e0933482dafde8fe197d9a2cfde}{Index} n = qr.matrixR().cols();
00052     eigen\_assert(n==diag.size());
00053     eigen\_assert(n==qtb.size());
00054 
00055     \hyperlink{struct_vector_type}{VectorType}  wa1, wa2;
00056 
00057     \textcolor{comment}{/* compute and store in x the gauss-newton direction. if the */}
00058     \textcolor{comment}{/* jacobian is rank-deficient, obtain a least squares solution. */}
00059 
00060     \textcolor{comment}{//    const Index rank = qr.nonzeroPivots(); // exactly double(0.)}
00061     \textcolor{keyword}{const} \hyperlink{namespace_eigen_a62e77e0933482dafde8fe197d9a2cfde}{Index} rank = qr.rank(); \textcolor{comment}{// use a threshold}
00062     wa1 = qtb;
00063     wa1.tail(n-rank).setZero();
00064     \textcolor{comment}{//FIXME There is no solve in place for sparse triangularView}
00065     wa1.head(rank) = s.topLeftCorner(rank,rank).template triangularView<Upper>().solve(qtb.head(rank));
00066 
00067     x = qr.colsPermutation()*wa1;
00068 
00069     \textcolor{comment}{/* initialize the iteration counter. */}
00070     \textcolor{comment}{/* evaluate the function at the origin, and test */}
00071     \textcolor{comment}{/* for acceptance of the gauss-newton direction. */}
00072     iter = 0;
00073     wa2 = diag.cwiseProduct(x);
00074     dxnorm = wa2.blueNorm();
00075     fp = dxnorm - m\_delta;
00076     \textcolor{keywordflow}{if} (fp <= Scalar(0.1) * m\_delta) \{
00077       par = 0;
00078       \textcolor{keywordflow}{return};
00079     \}
00080 
00081     \textcolor{comment}{/* if the jacobian is not rank deficient, the newton */}
00082     \textcolor{comment}{/* step provides a lower bound, parl, for the zero of */}
00083     \textcolor{comment}{/* the function. otherwise set this bound to zero. */}
00084     parl = 0.;
00085     \textcolor{keywordflow}{if} (rank==n) \{
00086       wa1 = qr.colsPermutation().inverse() *  diag.cwiseProduct(wa2)/dxnorm;
00087       s.topLeftCorner(n,n).transpose().template triangularView<Lower>().solveInPlace(wa1);
00088       temp = wa1.blueNorm();
00089       parl = fp / m\_delta / temp / temp;
00090     \}
00091 
00092     \textcolor{comment}{/* calculate an upper bound, paru, for the zero of the function. */}
00093     \textcolor{keywordflow}{for} (j = 0; j < n; ++j)
00094       wa1[j] = s.col(j).head(j+1).dot(qtb.head(j+1)) / diag[qr.colsPermutation().indices()(j)];
00095 
00096     gnorm = wa1.stableNorm();
00097     paru = gnorm / m\_delta;
00098     \textcolor{keywordflow}{if} (paru == 0.)
00099       paru = dwarf / (std::min)(m\_delta,Scalar(0.1));
00100 
00101     \textcolor{comment}{/* if the input par lies outside of the interval (parl,paru), */}
00102     \textcolor{comment}{/* set par to the closer endpoint. */}
00103     par = (std::max)(par,parl);
00104     par = (std::min)(par,paru);
00105     \textcolor{keywordflow}{if} (par == 0.)
00106       par = gnorm / dxnorm;
00107 
00108     \textcolor{comment}{/* beginning of an iteration. */}
00109     \textcolor{keywordflow}{while} (\textcolor{keyword}{true}) \{
00110       ++iter;
00111 
00112       \textcolor{comment}{/* evaluate the function at the current value of par. */}
00113       \textcolor{keywordflow}{if} (par == 0.)
00114         par = (std::max)(dwarf,Scalar(.001) * paru); \textcolor{comment}{/* Computing MAX */}
00115       wa1 = sqrt(par)* diag;
00116 
00117       \hyperlink{struct_vector_type}{VectorType} sdiag(n);
00118       lmqrsolv(s, qr.colsPermutation(), wa1, qtb, x, sdiag);
00119 
00120       wa2 = diag.cwiseProduct(x);
00121       dxnorm = wa2.blueNorm();
00122       temp = fp;
00123       fp = dxnorm - m\_delta;
00124 
00125       \textcolor{comment}{/* if the function is small enough, accept the current value */}
00126       \textcolor{comment}{/* of par. also test for the exceptional cases where parl */}
00127       \textcolor{comment}{/* is zero or the number of iterations has reached 10. */}
00128       \textcolor{keywordflow}{if} (abs(fp) <= Scalar(0.1) * m\_delta || (parl == 0. && fp <= temp && temp < 0.) || iter == 10)
00129         \textcolor{keywordflow}{break};
00130 
00131       \textcolor{comment}{/* compute the newton correction. */}
00132       wa1 = qr.colsPermutation().inverse() * diag.cwiseProduct(wa2/dxnorm);
00133       \textcolor{comment}{// we could almost use this here, but the diagonal is outside qr, in sdiag[]}
00134       \textcolor{keywordflow}{for} (j = 0; j < n; ++j) \{
00135         wa1[j] /= sdiag[j];
00136         temp = wa1[j];
00137         \textcolor{keywordflow}{for} (\hyperlink{namespace_eigen_a62e77e0933482dafde8fe197d9a2cfde}{Index} i = j+1; i < n; ++i)
00138           wa1[i] -= s.coeff(i,j) * temp;
00139       \}
00140       temp = wa1.blueNorm();
00141       parc = fp / m\_delta / temp / temp;
00142 
00143       \textcolor{comment}{/* depending on the sign of the function, update parl or paru. */}
00144       \textcolor{keywordflow}{if} (fp > 0.)
00145         parl = (std::max)(parl,par);
00146       \textcolor{keywordflow}{if} (fp < 0.)
00147         paru = (std::min)(paru,par);
00148 
00149       \textcolor{comment}{/* compute an improved estimate for par. */}
00150       par = (std::max)(parl,par+parc);
00151     \}
00152     \textcolor{keywordflow}{if} (iter == 0)
00153       par = 0.;
00154     \textcolor{keywordflow}{return};
00155   \}
00156 \} \textcolor{comment}{// end namespace internal}
00157 
00158 \} \textcolor{comment}{// end namespace Eigen}
00159 
00160 \textcolor{preprocessor}{#endif // EIGEN\_LMPAR\_H}
\end{DoxyCode}
