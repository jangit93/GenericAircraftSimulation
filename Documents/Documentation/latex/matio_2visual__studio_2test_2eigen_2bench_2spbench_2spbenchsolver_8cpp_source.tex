\hypertarget{matio_2visual__studio_2test_2eigen_2bench_2spbench_2spbenchsolver_8cpp_source}{}\section{matio/visual\+\_\+studio/test/eigen/bench/spbench/spbenchsolver.cpp}
\label{matio_2visual__studio_2test_2eigen_2bench_2spbench_2spbenchsolver_8cpp_source}\index{spbenchsolver.\+cpp@{spbenchsolver.\+cpp}}

\begin{DoxyCode}
00001 \textcolor{preprocessor}{#include <bench/spbench/spbenchsolver.h>}
00002 
00003 \textcolor{keywordtype}{void} bench\_printhelp()
00004 \{
00005     cout<< \textcolor{stringliteral}{" \(\backslash\)nbenchsolver : performs a benchmark of all the solvers available in Eigen \(\backslash\)n\(\backslash\)n"};
00006     cout<< \textcolor{stringliteral}{" MATRIX FOLDER : \(\backslash\)n"};
00007     cout<< \textcolor{stringliteral}{" The matrices for the benchmark should be collected in a folder specified with an environment
       variable EIGEN\_MATRIXDIR \(\backslash\)n"};
00008     cout<< \textcolor{stringliteral}{" The matrices are stored using the matrix market coordinate format \(\backslash\)n"};
00009     cout<< \textcolor{stringliteral}{" The matrix and associated right-hand side (rhs) files are named respectively \(\backslash\)n"};
00010     cout<< \textcolor{stringliteral}{" as MatrixName.mtx and MatrixName\_b.mtx. If the rhs does not exist, a random one is generated. 
      \(\backslash\)n"};
00011     cout<< \textcolor{stringliteral}{" If a matrix is SPD, the matrix should be named as MatrixName\_SPD.mtx \(\backslash\)n"};
00012     cout<< \textcolor{stringliteral}{" If a true solution exists, it should be named as MatrixName\_x.mtx; \(\backslash\)n"}     ;
00013     cout<< \textcolor{stringliteral}{" it will be used to compute the norm of the error relative to the computed solutions\(\backslash\)n\(\backslash\)n"};
00014     cout<< \textcolor{stringliteral}{" OPTIONS : \(\backslash\)n"}; 
00015     cout<< \textcolor{stringliteral}{" -h or --help \(\backslash\)n    print this help and return\(\backslash\)n\(\backslash\)n"};
00016     cout<< \textcolor{stringliteral}{" -d matrixdir \(\backslash\)n    Use matrixdir as the matrix folder instead of the one specified in the
       environment variable EIGEN\_MATRIXDIR\(\backslash\)n\(\backslash\)n"}; 
00017     cout<< \textcolor{stringliteral}{" -o outputfile.xml \(\backslash\)n    Output the statistics to a xml file \(\backslash\)n\(\backslash\)n"};
00018     cout<< \textcolor{stringliteral}{" --eps <RelErr> Sets the relative tolerance for iterative solvers (default 1e-08) \(\backslash\)n\(\backslash\)n"};
00019     cout<< \textcolor{stringliteral}{" --maxits <MaxIts> Sets the maximum number of iterations (default 1000) \(\backslash\)n\(\backslash\)n"};
00020     
00021 \}
00022 \textcolor{keywordtype}{int} main(\textcolor{keywordtype}{int} argc, \textcolor{keywordtype}{char} ** args)
00023 \{
00024   
00025   \textcolor{keywordtype}{bool} help = ( get\_options(argc, args, \textcolor{stringliteral}{"-h"}) || get\_options(argc, args, \textcolor{stringliteral}{"--help"}) );
00026   \textcolor{keywordflow}{if}(help) \{
00027     bench\_printhelp();
00028     \textcolor{keywordflow}{return} 0;
00029   \}
00030 
00031   \textcolor{comment}{// Get the location of the test matrices}
00032   \textcolor{keywordtype}{string} matrix\_dir;
00033   \textcolor{keywordflow}{if} (!get\_options(argc, args, \textcolor{stringliteral}{"-d"}, &matrix\_dir))
00034   \{
00035     \textcolor{keywordflow}{if}(getenv(\textcolor{stringliteral}{"EIGEN\_MATRIXDIR"}) == NULL)\{
00036       std::cerr << \textcolor{stringliteral}{"Please, specify the location of the matrices with -d mat\_folder or the environment
       variable EIGEN\_MATRIXDIR \(\backslash\)n"};
00037       std::cerr << \textcolor{stringliteral}{" Run with --help to see the list of all the available options \(\backslash\)n"};
00038       \textcolor{keywordflow}{return} -1;
00039     \}
00040     matrix\_dir = getenv(\textcolor{stringliteral}{"EIGEN\_MATRIXDIR"});
00041   \}
00042      
00043   std::ofstream statbuf;
00044   \textcolor{keywordtype}{string} statFile ;
00045   
00046   \textcolor{comment}{// Get the file to write the statistics}
00047   \textcolor{keywordtype}{bool} statFileExists = get\_options(argc, args, \textcolor{stringliteral}{"-o"}, &statFile);
00048   \textcolor{keywordflow}{if}(statFileExists)
00049   \{
00050     statbuf.open(statFile.c\_str(), std::ios::out);
00051     \textcolor{keywordflow}{if}(statbuf.good())\{
00052       statFileExists = \textcolor{keyword}{true}; 
00053       printStatheader(statbuf);
00054       statbuf.close();
00055     \}
00056     \textcolor{keywordflow}{else}
00057       std::cerr << \textcolor{stringliteral}{"Unable to open the provided file for writting... \(\backslash\)n"};
00058   \}       
00059   
00060   \textcolor{comment}{// Get the maximum number of iterations and the tolerance}
00061   \textcolor{keywordtype}{int} maxiters = 1000; 
00062   \textcolor{keywordtype}{double} tol = 1e-08; 
00063   \textcolor{keywordtype}{string} inval; 
00064   \textcolor{keywordflow}{if} (get\_options(argc, args, \textcolor{stringliteral}{"--eps"}, &inval))
00065     tol = atof(inval.c\_str()); 
00066   \textcolor{keywordflow}{if}(get\_options(argc, args, \textcolor{stringliteral}{"--maxits"}, &inval))
00067     maxiters = atoi(inval.c\_str()); 
00068   
00069   \textcolor{keywordtype}{string} current\_dir; 
00070   \textcolor{comment}{// Test the real-arithmetics matrices}
00071   Browse\_Matrices<double>(matrix\_dir, statFileExists, statFile,maxiters, tol);
00072   
00073   \textcolor{comment}{// Test the complex-arithmetics matrices}
00074   Browse\_Matrices<std::complex<double> >(matrix\_dir, statFileExists, statFile, maxiters, tol); 
00075   
00076   \textcolor{keywordflow}{if}(statFileExists)
00077   \{
00078     statbuf.open(statFile.c\_str(), std::ios::app); 
00079     statbuf << \textcolor{stringliteral}{"</BENCH> \(\backslash\)n"};
00080     cout << \textcolor{stringliteral}{"\(\backslash\)n Output written in "} << statFile << \textcolor{stringliteral}{" ...\(\backslash\)n"};
00081     statbuf.close();
00082   \}
00083 
00084   \textcolor{keywordflow}{return} 0;
00085 \}
00086 
00087       
\end{DoxyCode}
