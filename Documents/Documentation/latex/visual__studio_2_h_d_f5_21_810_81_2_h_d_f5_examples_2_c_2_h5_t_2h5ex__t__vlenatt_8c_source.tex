\hypertarget{visual__studio_2_h_d_f5_21_810_81_2_h_d_f5_examples_2_c_2_h5_t_2h5ex__t__vlenatt_8c_source}{}\section{visual\+\_\+studio/\+H\+D\+F5/1.10.1/\+H\+D\+F5\+Examples/\+C/\+H5\+T/h5ex\+\_\+t\+\_\+vlenatt.c}
\label{visual__studio_2_h_d_f5_21_810_81_2_h_d_f5_examples_2_c_2_h5_t_2h5ex__t__vlenatt_8c_source}\index{h5ex\+\_\+t\+\_\+vlenatt.\+c@{h5ex\+\_\+t\+\_\+vlenatt.\+c}}

\begin{DoxyCode}
00001 \textcolor{comment}{/************************************************************}
00002 \textcolor{comment}{}
00003 \textcolor{comment}{  This example shows how to read and write variable-length}
00004 \textcolor{comment}{  datatypes to an attribute.  The program first writes two}
00005 \textcolor{comment}{  variable-length integer arrays to the attribute then}
00006 \textcolor{comment}{  closes the file.  Next, it reopens the file, reads back}
00007 \textcolor{comment}{  the data, and outputs it to the screen.}
00008 \textcolor{comment}{}
00009 \textcolor{comment}{  This file is intended for use with HDF5 Library version 1.8}
00010 \textcolor{comment}{}
00011 \textcolor{comment}{ ************************************************************/}
00012 
00013 \textcolor{preprocessor}{#include "hdf5.h"}
00014 \textcolor{preprocessor}{#include <stdio.h>}
00015 \textcolor{preprocessor}{#include <stdlib.h>}
00016 
00017 \textcolor{preprocessor}{#define FILE            "h5ex\_t\_vlenatt.h5"}
00018 \textcolor{preprocessor}{#define DATASET         "DS1"}
00019 \textcolor{preprocessor}{#define ATTRIBUTE       "A1"}
00020 \textcolor{preprocessor}{#define LEN0            3}
00021 \textcolor{preprocessor}{#define LEN1            12}
00022 
00023 \textcolor{keywordtype}{int}
00024 main (\textcolor{keywordtype}{void})
00025 \{
00026     hid\_t       \hyperlink{structfile}{file}, filetype, memtype, space, dset, attr;
00027                                     \textcolor{comment}{/* Handles */}
00028     herr\_t      status;
00029     \hyperlink{structhvl__t}{hvl\_t}       wdata[2],           \textcolor{comment}{/* Array of vlen structures */}
00030                 *rdata;             \textcolor{comment}{/* Pointer to vlen structures */}
00031     hsize\_t     dims[1] = \{2\};
00032     \textcolor{keywordtype}{int}         *ptr,
00033                 ndims,
00034                 i, j;
00035 
00036     \textcolor{comment}{/*}
00037 \textcolor{comment}{     * Initialize variable-length data.  wdata[0] is a countdown of}
00038 \textcolor{comment}{     * length LEN0, wdata[1] is a Fibonacci sequence of length LEN1.}
00039 \textcolor{comment}{     */}
00040     wdata[0].len = LEN0;
00041     ptr = (\textcolor{keywordtype}{int} *) malloc (wdata[0].len * \textcolor{keyword}{sizeof} (\textcolor{keywordtype}{int}));
00042     \textcolor{keywordflow}{for} (i=0; i<wdata[0].len; i++)
00043         ptr[i] = wdata[0].len - (\textcolor{keywordtype}{size\_t})i;       \textcolor{comment}{/* 3 2 1 */}
00044     wdata[0].p = (\textcolor{keywordtype}{void} *) ptr;
00045 
00046     wdata[1].len = LEN1;
00047     ptr = (\textcolor{keywordtype}{int} *) malloc (wdata[1].len * \textcolor{keyword}{sizeof} (\textcolor{keywordtype}{int}));
00048     ptr[0] = 1;
00049     ptr[1] = 1;
00050     \textcolor{keywordflow}{for} (i=2; i<wdata[1].len; i++)
00051         ptr[i] = ptr[i-1] + ptr[i-2];   \textcolor{comment}{/* 1 1 2 3 5 8 etc. */}
00052     wdata[1].p = (\textcolor{keywordtype}{void} *) ptr;
00053 
00054     \textcolor{comment}{/*}
00055 \textcolor{comment}{     * Create a new file using the default properties.}
00056 \textcolor{comment}{     */}
00057     file = H5Fcreate (FILE, H5F\_ACC\_TRUNC, H5P\_DEFAULT, H5P\_DEFAULT);
00058 
00059     \textcolor{comment}{/*}
00060 \textcolor{comment}{     * Create variable-length datatype for file and memory.}
00061 \textcolor{comment}{     */}
00062     filetype = H5Tvlen\_create (H5T\_STD\_I32LE);
00063     memtype = H5Tvlen\_create (H5T\_NATIVE\_INT);
00064 
00065     \textcolor{comment}{/*}
00066 \textcolor{comment}{     * Create dataset with a null dataspace.}
00067 \textcolor{comment}{     */}
00068     space = H5Screate (H5S\_NULL);
00069     dset = H5Dcreate (file, DATASET, H5T\_STD\_I32LE, space, H5P\_DEFAULT,
00070                 H5P\_DEFAULT, H5P\_DEFAULT);
00071     status = H5Sclose (space);
00072 
00073     \textcolor{comment}{/*}
00074 \textcolor{comment}{     * Create dataspace.  Setting maximum size to NULL sets the maximum}
00075 \textcolor{comment}{     * size to be the current size.}
00076 \textcolor{comment}{     */}
00077     space = H5Screate\_simple (1, dims, NULL);
00078 
00079     \textcolor{comment}{/*}
00080 \textcolor{comment}{     * Create the attribute and write the variable-length data to it}
00081 \textcolor{comment}{     */}
00082     attr = H5Acreate (dset, ATTRIBUTE, filetype, space, H5P\_DEFAULT,
00083                 H5P\_DEFAULT);
00084     status = H5Awrite (attr, memtype, wdata);
00085 
00086     \textcolor{comment}{/*}
00087 \textcolor{comment}{     * Close and release resources.  Note the use of H5Dvlen\_reclaim}
00088 \textcolor{comment}{     * removes the need to manually free() the previously malloc'ed}
00089 \textcolor{comment}{     * data.}
00090 \textcolor{comment}{     */}
00091     status = H5Dvlen\_reclaim (memtype, space, H5P\_DEFAULT, wdata);
00092     status = H5Aclose (attr);
00093     status = H5Dclose (dset);
00094     status = H5Sclose (space);
00095     status = H5Tclose (filetype);
00096     status = H5Tclose (memtype);
00097     status = H5Fclose (file);
00098 
00099 
00100     \textcolor{comment}{/*}
00101 \textcolor{comment}{     * Now we begin the read section of this example.  Here we assume}
00102 \textcolor{comment}{     * the attribute has the same name and rank, but can have any size.}
00103 \textcolor{comment}{     * Therefore we must allocate a new array to read in data using}
00104 \textcolor{comment}{     * malloc().}
00105 \textcolor{comment}{     */}
00106 
00107     \textcolor{comment}{/*}
00108 \textcolor{comment}{     * Open file, dataset, and attribute.}
00109 \textcolor{comment}{     */}
00110     file = H5Fopen (FILE, H5F\_ACC\_RDONLY, H5P\_DEFAULT);
00111     dset = H5Dopen (file, DATASET, H5P\_DEFAULT);
00112     attr = H5Aopen (dset, ATTRIBUTE, H5P\_DEFAULT);
00113 
00114     \textcolor{comment}{/*}
00115 \textcolor{comment}{     * Get dataspace and allocate memory for array of vlen structures.}
00116 \textcolor{comment}{     * This does not actually allocate memory for the vlen data, that}
00117 \textcolor{comment}{     * will be done by the library.}
00118 \textcolor{comment}{     */}
00119     space = H5Aget\_space (attr);
00120     ndims = H5Sget\_simple\_extent\_dims (space, dims, NULL);
00121     rdata = (\hyperlink{structhvl__t}{hvl\_t} *) malloc (dims[0] * \textcolor{keyword}{sizeof} (\hyperlink{structhvl__t}{hvl\_t}));
00122 
00123     \textcolor{comment}{/*}
00124 \textcolor{comment}{     * Create the memory datatype.}
00125 \textcolor{comment}{     */}
00126     memtype = H5Tvlen\_create (H5T\_NATIVE\_INT);
00127 
00128     \textcolor{comment}{/*}
00129 \textcolor{comment}{     * Read the data.}
00130 \textcolor{comment}{     */}
00131     status = H5Aread (attr, memtype, rdata);
00132 
00133     \textcolor{comment}{/*}
00134 \textcolor{comment}{     * Output the variable-length data to the screen.}
00135 \textcolor{comment}{     */}
00136     \textcolor{keywordflow}{for} (i=0; i<dims[0]; i++) \{
00137         printf (\textcolor{stringliteral}{"%s[%u]:\(\backslash\)n  \{"},ATTRIBUTE,i);
00138         ptr = rdata[i].p;
00139         \textcolor{keywordflow}{for} (j=0; j<rdata[i].len; j++) \{
00140             printf (\textcolor{stringliteral}{" %d"}, ptr[j]);
00141             \textcolor{keywordflow}{if} ( (j+1) < rdata[i].len )
00142                 printf (\textcolor{stringliteral}{","});
00143         \}
00144         printf (\textcolor{stringliteral}{" \}\(\backslash\)n"});
00145     \}
00146 
00147     \textcolor{comment}{/*}
00148 \textcolor{comment}{     * Close and release resources.  Note we must still free the}
00149 \textcolor{comment}{     * top-level pointer "rdata", as H5Dvlen\_reclaim only frees the}
00150 \textcolor{comment}{     * actual variable-length data, and not the structures themselves.}
00151 \textcolor{comment}{     */}
00152     status = H5Dvlen\_reclaim (memtype, space, H5P\_DEFAULT, rdata);
00153     free (rdata);
00154     status = H5Aclose (attr);
00155     status = H5Dclose (dset);
00156     status = H5Sclose (space);
00157     status = H5Tclose (memtype);
00158     status = H5Fclose (file);
00159 
00160     \textcolor{keywordflow}{return} 0;
00161 \}
\end{DoxyCode}
