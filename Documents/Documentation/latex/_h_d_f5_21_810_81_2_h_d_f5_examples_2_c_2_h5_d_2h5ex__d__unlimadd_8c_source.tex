\hypertarget{_h_d_f5_21_810_81_2_h_d_f5_examples_2_c_2_h5_d_2h5ex__d__unlimadd_8c_source}{}\section{H\+D\+F5/1.10.1/\+H\+D\+F5\+Examples/\+C/\+H5\+D/h5ex\+\_\+d\+\_\+unlimadd.c}
\label{_h_d_f5_21_810_81_2_h_d_f5_examples_2_c_2_h5_d_2h5ex__d__unlimadd_8c_source}\index{h5ex\+\_\+d\+\_\+unlimadd.\+c@{h5ex\+\_\+d\+\_\+unlimadd.\+c}}

\begin{DoxyCode}
00001 \textcolor{comment}{/************************************************************}
00002 \textcolor{comment}{}
00003 \textcolor{comment}{  This example shows how to create and extend an unlimited}
00004 \textcolor{comment}{  dataset.  The program first writes integers to a dataset}
00005 \textcolor{comment}{  with dataspace dimensions of DIM0xDIM1, then closes the}
00006 \textcolor{comment}{  file.  Next, it reopens the file, reads back the data,}
00007 \textcolor{comment}{  outputs it to the screen, extends the dataset, and writes}
00008 \textcolor{comment}{  new data to the extended portions of the dataset.  Finally}
00009 \textcolor{comment}{  it reopens the file again, reads back the data, and}
00010 \textcolor{comment}{  outputs it to the screen.}
00011 \textcolor{comment}{}
00012 \textcolor{comment}{  This file is intended for use with HDF5 Library version 1.8}
00013 \textcolor{comment}{}
00014 \textcolor{comment}{ ************************************************************/}
00015 
00016 \textcolor{preprocessor}{#include "hdf5.h"}
00017 \textcolor{preprocessor}{#include <stdio.h>}
00018 \textcolor{preprocessor}{#include <stdlib.h>}
00019 
00020 \textcolor{preprocessor}{#define FILE            "h5ex\_d\_unlimadd.h5"}
00021 \textcolor{preprocessor}{#define DATASET         "DS1"}
00022 \textcolor{preprocessor}{#define DIM0            4}
00023 \textcolor{preprocessor}{#define DIM1            7}
00024 \textcolor{preprocessor}{#define EDIM0           6}
00025 \textcolor{preprocessor}{#define EDIM1           10}
00026 \textcolor{preprocessor}{#define CHUNK0          4}
00027 \textcolor{preprocessor}{#define CHUNK1          4}
00028 
00029 \textcolor{keywordtype}{int}
00030 main (\textcolor{keywordtype}{void})
00031 \{
00032     hid\_t           \hyperlink{structfile}{file}, space, dset, dcpl;    \textcolor{comment}{/* Handles */}
00033     herr\_t          status;
00034     hsize\_t         dims[2] = \{DIM0, DIM1\},
00035                     extdims[2] = \{EDIM0, EDIM1\},
00036                     maxdims[2],
00037                     chunk[2] = \{CHUNK0, CHUNK1\},
00038                     start[2],
00039                     count[2];
00040     \textcolor{keywordtype}{int}             wdata[DIM0][DIM1],          \textcolor{comment}{/* Write buffer */}
00041                     wdata2[EDIM0][EDIM1],       \textcolor{comment}{/* Write buffer for}
00042 \textcolor{comment}{                                                   extension */}
00043                     **rdata,                    \textcolor{comment}{/* Read buffer */}
00044                     ndims,
00045                     i, j;
00046 
00047     \textcolor{comment}{/*}
00048 \textcolor{comment}{     * Initialize data.}
00049 \textcolor{comment}{     */}
00050     \textcolor{keywordflow}{for} (i=0; i<DIM0; i++)
00051         \textcolor{keywordflow}{for} (j=0; j<DIM1; j++)
00052             wdata[i][j] = i * j - j;
00053 
00054     \textcolor{comment}{/*}
00055 \textcolor{comment}{     * Create a new file using the default properties.}
00056 \textcolor{comment}{     */}
00057     file = H5Fcreate (FILE, H5F\_ACC\_TRUNC, H5P\_DEFAULT, H5P\_DEFAULT);
00058 
00059     \textcolor{comment}{/*}
00060 \textcolor{comment}{     * Create dataspace with unlimited dimensions.}
00061 \textcolor{comment}{     */}
00062     maxdims[0] = H5S\_UNLIMITED;
00063     maxdims[1] = H5S\_UNLIMITED;
00064     space = H5Screate\_simple (2, dims, maxdims);
00065 
00066     \textcolor{comment}{/*}
00067 \textcolor{comment}{     * Create the dataset creation property list, and set the chunk}
00068 \textcolor{comment}{     * size.}
00069 \textcolor{comment}{     */}
00070     dcpl = H5Pcreate (H5P\_DATASET\_CREATE);
00071     status = H5Pset\_chunk (dcpl, 2, chunk);
00072 
00073     \textcolor{comment}{/*}
00074 \textcolor{comment}{     * Create the unlimited dataset.}
00075 \textcolor{comment}{     */}
00076     dset = H5Dcreate (file, DATASET, H5T\_STD\_I32LE, space, H5P\_DEFAULT, dcpl,
00077                 H5P\_DEFAULT);
00078 
00079     \textcolor{comment}{/*}
00080 \textcolor{comment}{     * Write the data to the dataset.}
00081 \textcolor{comment}{     */}
00082     status = H5Dwrite (dset, H5T\_NATIVE\_INT, H5S\_ALL, H5S\_ALL, H5P\_DEFAULT,
00083                 wdata[0]);
00084 
00085     \textcolor{comment}{/*}
00086 \textcolor{comment}{     * Close and release resources.}
00087 \textcolor{comment}{     */}
00088     status = H5Pclose (dcpl);
00089     status = H5Dclose (dset);
00090     status = H5Sclose (space);
00091     status = H5Fclose (file);
00092 
00093 
00094     \textcolor{comment}{/*}
00095 \textcolor{comment}{     * In this next section we read back the data, extend the dataset,}
00096 \textcolor{comment}{     * and write new data to the extended portions.}
00097 \textcolor{comment}{     */}
00098 
00099     \textcolor{comment}{/*}
00100 \textcolor{comment}{     * Open file and dataset using the default properties.}
00101 \textcolor{comment}{     */}
00102     file = H5Fopen (FILE, H5F\_ACC\_RDWR, H5P\_DEFAULT);
00103     dset = H5Dopen (file, DATASET, H5P\_DEFAULT);
00104 
00105     \textcolor{comment}{/*}
00106 \textcolor{comment}{     * Get dataspace and allocate memory for read buffer.  This is a}
00107 \textcolor{comment}{     * two dimensional dataset so the dynamic allocation must be done}
00108 \textcolor{comment}{     * in steps.}
00109 \textcolor{comment}{     */}
00110     space = H5Dget\_space (dset);
00111     ndims = H5Sget\_simple\_extent\_dims (space, dims, NULL);
00112 
00113     \textcolor{comment}{/*}
00114 \textcolor{comment}{     * Allocate array of pointers to rows.}
00115 \textcolor{comment}{     */}
00116     rdata = (\textcolor{keywordtype}{int} **) malloc (dims[0] * \textcolor{keyword}{sizeof} (\textcolor{keywordtype}{int} *));
00117 
00118     \textcolor{comment}{/*}
00119 \textcolor{comment}{     * Allocate space for integer data.}
00120 \textcolor{comment}{     */}
00121     rdata[0] = (\textcolor{keywordtype}{int} *) malloc (dims[0] * dims[1] * \textcolor{keyword}{sizeof} (\textcolor{keywordtype}{int}));
00122 
00123     \textcolor{comment}{/*}
00124 \textcolor{comment}{     * Set the rest of the pointers to rows to the correct addresses.}
00125 \textcolor{comment}{     */}
00126     \textcolor{keywordflow}{for} (i=1; i<dims[0]; i++)
00127         rdata[i] = rdata[0] + i * dims[1];
00128 
00129     \textcolor{comment}{/*}
00130 \textcolor{comment}{     * Read the data using the default properties.}
00131 \textcolor{comment}{     */}
00132     status = H5Dread (dset, H5T\_NATIVE\_INT, H5S\_ALL, H5S\_ALL, H5P\_DEFAULT,
00133                 rdata[0]);
00134 
00135     \textcolor{comment}{/*}
00136 \textcolor{comment}{     * Output the data to the screen.}
00137 \textcolor{comment}{     */}
00138     printf (\textcolor{stringliteral}{"Dataset before extension:\(\backslash\)n"});
00139     \textcolor{keywordflow}{for} (i=0; i<dims[0]; i++) \{
00140         printf (\textcolor{stringliteral}{" ["});
00141         \textcolor{keywordflow}{for} (j=0; j<dims[1]; j++)
00142             printf (\textcolor{stringliteral}{" %3d"}, rdata[i][j]);
00143         printf (\textcolor{stringliteral}{"]\(\backslash\)n"});
00144     \}
00145 
00146     status = H5Sclose (space);
00147 
00148     \textcolor{comment}{/*}
00149 \textcolor{comment}{     * Extend the dataset.}
00150 \textcolor{comment}{     */}
00151     status = H5Dset\_extent (dset, extdims);
00152 
00153     \textcolor{comment}{/*}
00154 \textcolor{comment}{     * Retrieve the dataspace for the newly extended dataset.}
00155 \textcolor{comment}{     */}
00156     space = H5Dget\_space (dset);
00157 
00158     \textcolor{comment}{/*}
00159 \textcolor{comment}{     * Initialize data for writing to the extended dataset.}
00160 \textcolor{comment}{     */}
00161     \textcolor{keywordflow}{for} (i=0; i<EDIM0; i++)
00162         \textcolor{keywordflow}{for} (j=0; j<EDIM1; j++)
00163             wdata2[i][j] = j;
00164 
00165     \textcolor{comment}{/*}
00166 \textcolor{comment}{     * Select the entire dataspace.}
00167 \textcolor{comment}{     */}
00168     status = H5Sselect\_all (space);
00169 
00170     \textcolor{comment}{/*}
00171 \textcolor{comment}{     * Subtract a hyperslab reflecting the original dimensions from the}
00172 \textcolor{comment}{     * selection.  The selection now contains only the newly extended}
00173 \textcolor{comment}{     * portions of the dataset.}
00174 \textcolor{comment}{     */}
00175     start[0] = 0;
00176     start[1] = 0;
00177     count[0] = dims[0];
00178     count[1] = dims[1];
00179     status = H5Sselect\_hyperslab (space, H5S\_SELECT\_NOTB, start, NULL, count,
00180                 NULL);
00181 
00182     \textcolor{comment}{/*}
00183 \textcolor{comment}{     * Write the data to the selected portion of the dataset.}
00184 \textcolor{comment}{     */}
00185     status = H5Dwrite (dset, H5T\_NATIVE\_INT, H5S\_ALL, space, H5P\_DEFAULT,
00186                 wdata2[0]);
00187 
00188     \textcolor{comment}{/*}
00189 \textcolor{comment}{     * Close and release resources.}
00190 \textcolor{comment}{     */}
00191     free (rdata[0]);
00192     free(rdata);
00193     status = H5Dclose (dset);
00194     status = H5Sclose (space);
00195     status = H5Fclose (file);
00196 
00197 
00198     \textcolor{comment}{/*}
00199 \textcolor{comment}{     * Now we simply read back the data and output it to the screen.}
00200 \textcolor{comment}{     */}
00201 
00202     \textcolor{comment}{/*}
00203 \textcolor{comment}{     * Open file and dataset using the default properties.}
00204 \textcolor{comment}{     */}
00205     file = H5Fopen (FILE, H5F\_ACC\_RDONLY, H5P\_DEFAULT);
00206     dset = H5Dopen (file, DATASET, H5P\_DEFAULT);
00207 
00208     \textcolor{comment}{/*}
00209 \textcolor{comment}{     * Get dataspace and allocate memory for the read buffer as before.}
00210 \textcolor{comment}{     */}
00211     space = H5Dget\_space (dset);
00212     ndims = H5Sget\_simple\_extent\_dims (space, dims, NULL);
00213     rdata = (\textcolor{keywordtype}{int} **) malloc (dims[0] * \textcolor{keyword}{sizeof} (\textcolor{keywordtype}{int} *));
00214     rdata[0] = (\textcolor{keywordtype}{int} *) malloc (dims[0] * dims[1] * \textcolor{keyword}{sizeof} (\textcolor{keywordtype}{int}));
00215     \textcolor{keywordflow}{for} (i=1; i<dims[0]; i++)
00216         rdata[i] = rdata[0] + i * dims[1];
00217 
00218     \textcolor{comment}{/*}
00219 \textcolor{comment}{     * Read the data using the default properties.}
00220 \textcolor{comment}{     */}
00221     status = H5Dread (dset, H5T\_NATIVE\_INT, H5S\_ALL, H5S\_ALL, H5P\_DEFAULT,
00222                 rdata[0]);
00223 
00224     \textcolor{comment}{/*}
00225 \textcolor{comment}{     * Output the data to the screen.}
00226 \textcolor{comment}{     */}
00227     printf (\textcolor{stringliteral}{"\(\backslash\)nDataset after extension:\(\backslash\)n"});
00228     \textcolor{keywordflow}{for} (i=0; i<dims[0]; i++) \{
00229         printf (\textcolor{stringliteral}{" ["});
00230         \textcolor{keywordflow}{for} (j=0; j<dims[1]; j++)
00231             printf (\textcolor{stringliteral}{" %3d"}, rdata[i][j]);
00232         printf (\textcolor{stringliteral}{"]\(\backslash\)n"});
00233     \}
00234 
00235     \textcolor{comment}{/*}
00236 \textcolor{comment}{     * Close and release resources.}
00237 \textcolor{comment}{     */}
00238     free (rdata[0]);
00239     free(rdata);
00240     status = H5Dclose (dset);
00241     status = H5Sclose (space);
00242     status = H5Fclose (file);
00243 
00244     \textcolor{keywordflow}{return} 0;
00245 \}
\end{DoxyCode}
