\hypertarget{eigen_2_eigen_2src_2_core_2arch_2_z_vector_2_math_functions_8h_source}{}\section{eigen/\+Eigen/src/\+Core/arch/\+Z\+Vector/\+Math\+Functions.h}
\label{eigen_2_eigen_2src_2_core_2arch_2_z_vector_2_math_functions_8h_source}\index{Math\+Functions.\+h@{Math\+Functions.\+h}}

\begin{DoxyCode}
00001 \textcolor{comment}{// This file is part of Eigen, a lightweight C++ template library}
00002 \textcolor{comment}{// for linear algebra.}
00003 \textcolor{comment}{//}
00004 \textcolor{comment}{// Copyright (C) 2007 Julien Pommier}
00005 \textcolor{comment}{// Copyright (C) 2009 Gael Guennebaud <gael.guennebaud@inria.fr>}
00006 \textcolor{comment}{// Copyright (C) 2016 Konstantinos Margaritis <markos@freevec.org>}
00007 \textcolor{comment}{//}
00008 \textcolor{comment}{// This Source Code Form is subject to the terms of the Mozilla}
00009 \textcolor{comment}{// Public License v. 2.0. If a copy of the MPL was not distributed}
00010 \textcolor{comment}{// with this file, You can obtain one at http://mozilla.org/MPL/2.0/.}
00011 
00012 \textcolor{comment}{/* The sin, cos, exp, and log functions of this file come from}
00013 \textcolor{comment}{ * Julien Pommier's sse math library: http://gruntthepeon.free.fr/ssemath/}
00014 \textcolor{comment}{ */}
00015 
00016 \textcolor{preprocessor}{#ifndef EIGEN\_MATH\_FUNCTIONS\_ALTIVEC\_H}
00017 \textcolor{preprocessor}{#define EIGEN\_MATH\_FUNCTIONS\_ALTIVEC\_H}
00018 
00019 \textcolor{keyword}{namespace }\hyperlink{namespace_eigen}{Eigen} \{
00020 
00021 \textcolor{keyword}{namespace }\hyperlink{namespaceinternal}{internal} \{
00022 
00023 \textcolor{keyword}{static} \_EIGEN\_DECLARE\_CONST\_Packet2d(1 , 1.0);
00024 \textcolor{keyword}{static} \_EIGEN\_DECLARE\_CONST\_Packet2d(2 , 2.0);
00025 \textcolor{keyword}{static} \_EIGEN\_DECLARE\_CONST\_Packet2d(half, 0.5);
00026 
00027 \textcolor{keyword}{static} \_EIGEN\_DECLARE\_CONST\_Packet2d(exp\_hi,  709.437);
00028 \textcolor{keyword}{static} \_EIGEN\_DECLARE\_CONST\_Packet2d(exp\_lo, -709.436139303);
00029 
00030 \textcolor{keyword}{static} \_EIGEN\_DECLARE\_CONST\_Packet2d(cephes\_LOG2EF, 1.4426950408889634073599);
00031 
00032 \textcolor{keyword}{static} \_EIGEN\_DECLARE\_CONST\_Packet2d(cephes\_exp\_p0, 1.26177193074810590878e-4);
00033 \textcolor{keyword}{static} \_EIGEN\_DECLARE\_CONST\_Packet2d(cephes\_exp\_p1, 3.02994407707441961300e-2);
00034 \textcolor{keyword}{static} \_EIGEN\_DECLARE\_CONST\_Packet2d(cephes\_exp\_p2, 9.99999999999999999910e-1);
00035 
00036 \textcolor{keyword}{static} \_EIGEN\_DECLARE\_CONST\_Packet2d(cephes\_exp\_q0, 3.00198505138664455042e-6);
00037 \textcolor{keyword}{static} \_EIGEN\_DECLARE\_CONST\_Packet2d(cephes\_exp\_q1, 2.52448340349684104192e-3);
00038 \textcolor{keyword}{static} \_EIGEN\_DECLARE\_CONST\_Packet2d(cephes\_exp\_q2, 2.27265548208155028766e-1);
00039 \textcolor{keyword}{static} \_EIGEN\_DECLARE\_CONST\_Packet2d(cephes\_exp\_q3, 2.00000000000000000009e0);
00040 
00041 \textcolor{keyword}{static} \_EIGEN\_DECLARE\_CONST\_Packet2d(cephes\_exp\_C1, 0.693145751953125);
00042 \textcolor{keyword}{static} \_EIGEN\_DECLARE\_CONST\_Packet2d(cephes\_exp\_C2, 1.42860682030941723212e-6);
00043 
00044 \textcolor{keyword}{template}<> EIGEN\_DEFINE\_FUNCTION\_ALLOWING\_MULTIPLE\_DEFINITIONS EIGEN\_UNUSED
00045 Packet2d pexp<Packet2d>(\textcolor{keyword}{const} Packet2d& \_x)
00046 \{
00047   Packet2d x = \_x;
00048 
00049   Packet2d tmp, fx;
00050   Packet2l emm0;
00051 
00052   \textcolor{comment}{// clamp x}
00053   x = pmax(pmin(x, p2d\_exp\_hi), p2d\_exp\_lo);
00054   \textcolor{comment}{/* express exp(x) as exp(g + n*log(2)) */}
00055   fx = pmadd(p2d\_cephes\_LOG2EF, x, p2d\_half);
00056 
00057   fx = vec\_floor(fx);
00058 
00059   tmp = pmul(fx, p2d\_cephes\_exp\_C1);
00060   Packet2d z = pmul(fx, p2d\_cephes\_exp\_C2);
00061   x = psub(x, tmp);
00062   x = psub(x, z);
00063 
00064   Packet2d x2 = pmul(x,x);
00065 
00066   Packet2d px = p2d\_cephes\_exp\_p0;
00067   px = pmadd(px, x2, p2d\_cephes\_exp\_p1);
00068   px = pmadd(px, x2, p2d\_cephes\_exp\_p2);
00069   px = pmul (px, x);
00070 
00071   Packet2d qx = p2d\_cephes\_exp\_q0;
00072   qx = pmadd(qx, x2, p2d\_cephes\_exp\_q1);
00073   qx = pmadd(qx, x2, p2d\_cephes\_exp\_q2);
00074   qx = pmadd(qx, x2, p2d\_cephes\_exp\_q3);
00075 
00076   x = pdiv(px,psub(qx,px));
00077   x = pmadd(p2d\_2,x,p2d\_1);
00078 
00079   \textcolor{comment}{// build 2^n}
00080   emm0 = vec\_ctsl(fx, 0);
00081 
00082   \textcolor{keyword}{static} \textcolor{keyword}{const} Packet2l p2l\_1023 = \{ 1023, 1023 \};
00083   \textcolor{keyword}{static} \textcolor{keyword}{const} Packet2ul p2ul\_52 = \{ 52, 52 \};
00084 
00085   emm0 = emm0 + p2l\_1023;
00086   emm0 = emm0 << reinterpret\_cast<Packet2l>(p2ul\_52);
00087 
00088   \textcolor{comment}{// Altivec's max & min operators just drop silent NaNs. Check NaNs in }
00089   \textcolor{comment}{// inputs and return them unmodified.}
00090   Packet2ul isnumber\_mask = \textcolor{keyword}{reinterpret\_cast<}Packet2ul\textcolor{keyword}{>}(vec\_cmpeq(\_x, \_x));
00091   \textcolor{keywordflow}{return} vec\_sel(\_x, pmax(pmul(x, reinterpret\_cast<Packet2d>(emm0)), \_x),
00092                  isnumber\_mask);
00093 \}
00094 
00095 \textcolor{keyword}{template}<> EIGEN\_DEFINE\_FUNCTION\_ALLOWING\_MULTIPLE\_DEFINITIONS EIGEN\_UNUSED
00096 Packet4f pexp<Packet4f>(\textcolor{keyword}{const} Packet4f& x)
00097 \{
00098   Packet4f res;
00099   res.v4f[0] = pexp<Packet2d>(x.v4f[0]);
00100   res.v4f[1] = pexp<Packet2d>(x.v4f[1]);
00101   \textcolor{keywordflow}{return} res;
00102 \}
00103 
00104 \textcolor{keyword}{template}<> EIGEN\_DEFINE\_FUNCTION\_ALLOWING\_MULTIPLE\_DEFINITIONS EIGEN\_UNUSED
00105 Packet2d psqrt<Packet2d>(\textcolor{keyword}{const} Packet2d& x)
00106 \{
00107   \textcolor{keywordflow}{return}  \_\_builtin\_s390\_vfsqdb(x);
00108 \}
00109 
00110 \textcolor{keyword}{template}<> EIGEN\_DEFINE\_FUNCTION\_ALLOWING\_MULTIPLE\_DEFINITIONS EIGEN\_UNUSED
00111 Packet4f psqrt<Packet4f>(\textcolor{keyword}{const} Packet4f& x)
00112 \{
00113   Packet4f res;
00114   res.v4f[0] = psqrt<Packet2d>(x.v4f[0]);
00115   res.v4f[1] = psqrt<Packet2d>(x.v4f[1]);
00116   \textcolor{keywordflow}{return} res;
00117 \}
00118 
00119 \textcolor{keyword}{template}<> EIGEN\_DEFINE\_FUNCTION\_ALLOWING\_MULTIPLE\_DEFINITIONS EIGEN\_UNUSED
00120 Packet2d prsqrt<Packet2d>(\textcolor{keyword}{const} Packet2d& x) \{
00121   \textcolor{comment}{// Unfortunately we can't use the much faster mm\_rqsrt\_pd since it only provides an approximation.}
00122   \textcolor{keywordflow}{return} pset1<Packet2d>(1.0) / psqrt<Packet2d>(x);
00123 \}
00124 
00125 \textcolor{keyword}{template}<> EIGEN\_DEFINE\_FUNCTION\_ALLOWING\_MULTIPLE\_DEFINITIONS EIGEN\_UNUSED
00126 Packet4f prsqrt<Packet4f>(\textcolor{keyword}{const} Packet4f& x) \{
00127   Packet4f res;
00128   res.v4f[0] = prsqrt<Packet2d>(x.v4f[0]);
00129   res.v4f[1] = prsqrt<Packet2d>(x.v4f[1]);
00130   \textcolor{keywordflow}{return} res;
00131 \}
00132 
00133 \}  \textcolor{comment}{// end namespace internal}
00134 
00135 \}  \textcolor{comment}{// end namespace Eigen}
00136 
00137 \textcolor{preprocessor}{#endif  // EIGEN\_MATH\_FUNCTIONS\_ALTIVEC\_H}
\end{DoxyCode}
