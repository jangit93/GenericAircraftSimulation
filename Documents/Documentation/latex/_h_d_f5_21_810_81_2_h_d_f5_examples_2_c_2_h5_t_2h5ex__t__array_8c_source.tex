\hypertarget{_h_d_f5_21_810_81_2_h_d_f5_examples_2_c_2_h5_t_2h5ex__t__array_8c_source}{}\section{H\+D\+F5/1.10.1/\+H\+D\+F5\+Examples/\+C/\+H5\+T/h5ex\+\_\+t\+\_\+array.c}
\label{_h_d_f5_21_810_81_2_h_d_f5_examples_2_c_2_h5_t_2h5ex__t__array_8c_source}\index{h5ex\+\_\+t\+\_\+array.\+c@{h5ex\+\_\+t\+\_\+array.\+c}}

\begin{DoxyCode}
00001 \textcolor{comment}{/************************************************************}
00002 \textcolor{comment}{}
00003 \textcolor{comment}{  This example shows how to read and write array datatypes}
00004 \textcolor{comment}{  to a dataset.  The program first writes integers arrays of}
00005 \textcolor{comment}{  dimension ADIM0xADIM1 to a dataset with a dataspace of}
00006 \textcolor{comment}{  DIM0, then closes the  file.  Next, it reopens the file,}
00007 \textcolor{comment}{  reads back the data, and outputs it to the screen.}
00008 \textcolor{comment}{}
00009 \textcolor{comment}{  This file is intended for use with HDF5 Library version 1.8}
00010 \textcolor{comment}{}
00011 \textcolor{comment}{ ************************************************************/}
00012 
00013 \textcolor{preprocessor}{#include "hdf5.h"}
00014 \textcolor{preprocessor}{#include <stdio.h>}
00015 \textcolor{preprocessor}{#include <stdlib.h>}
00016 
00017 \textcolor{preprocessor}{#define FILE            "h5ex\_t\_array.h5"}
00018 \textcolor{preprocessor}{#define DATASET         "DS1"}
00019 \textcolor{preprocessor}{#define DIM0            4}
00020 \textcolor{preprocessor}{#define ADIM0           3}
00021 \textcolor{preprocessor}{#define ADIM1           5}
00022 
00023 \textcolor{keywordtype}{int}
00024 main (\textcolor{keywordtype}{void})
00025 \{
00026     hid\_t       \hyperlink{structfile}{file}, filetype, memtype, space, dset;
00027                                                 \textcolor{comment}{/* Handles */}
00028     herr\_t      status;
00029     hsize\_t     dims[1] = \{DIM0\},
00030                 adims[2] = \{ADIM0, ADIM1\};
00031     \textcolor{keywordtype}{int}         wdata[DIM0][ADIM0][ADIM1],      \textcolor{comment}{/* Write buffer */}
00032                 ***rdata,                       \textcolor{comment}{/* Read buffer */}
00033                 ndims,
00034                 i, j, k;
00035 
00036     \textcolor{comment}{/*}
00037 \textcolor{comment}{     * Initialize data.  i is the element in the dataspace, j and k the}
00038 \textcolor{comment}{     * elements within the array datatype.}
00039 \textcolor{comment}{     */}
00040     \textcolor{keywordflow}{for} (i=0; i<DIM0; i++)
00041         \textcolor{keywordflow}{for} (j=0; j<ADIM0; j++)
00042             \textcolor{keywordflow}{for} (k=0; k<ADIM1; k++)
00043                 wdata[i][j][k] = i * j - j * k + i * k;
00044 
00045     \textcolor{comment}{/*}
00046 \textcolor{comment}{     * Create a new file using the default properties.}
00047 \textcolor{comment}{     */}
00048     file = H5Fcreate (FILE, H5F\_ACC\_TRUNC, H5P\_DEFAULT, H5P\_DEFAULT);
00049 
00050     \textcolor{comment}{/*}
00051 \textcolor{comment}{     * Create array datatypes for file and memory.}
00052 \textcolor{comment}{     */}
00053     filetype = H5Tarray\_create (H5T\_STD\_I64LE, 2, adims);
00054     memtype = H5Tarray\_create (H5T\_NATIVE\_INT, 2, adims);
00055 
00056     \textcolor{comment}{/*}
00057 \textcolor{comment}{     * Create dataspace.  Setting maximum size to NULL sets the maximum}
00058 \textcolor{comment}{     * size to be the current size.}
00059 \textcolor{comment}{     */}
00060     space = H5Screate\_simple (1, dims, NULL);
00061 
00062     \textcolor{comment}{/*}
00063 \textcolor{comment}{     * Create the dataset and write the array data to it.}
00064 \textcolor{comment}{     */}
00065     dset = H5Dcreate (file, DATASET, filetype, space, H5P\_DEFAULT, H5P\_DEFAULT,
00066                 H5P\_DEFAULT);
00067     status = H5Dwrite (dset, memtype, H5S\_ALL, H5S\_ALL, H5P\_DEFAULT,
00068                 wdata[0][0]);
00069 
00070     \textcolor{comment}{/*}
00071 \textcolor{comment}{     * Close and release resources.}
00072 \textcolor{comment}{     */}
00073     status = H5Dclose (dset);
00074     status = H5Sclose (space);
00075     status = H5Tclose (filetype);
00076     status = H5Tclose (memtype);
00077     status = H5Fclose (file);
00078 
00079 
00080     \textcolor{comment}{/*}
00081 \textcolor{comment}{     * Now we begin the read section of this example.  Here we assume}
00082 \textcolor{comment}{     * the dataset and array have the same name and rank, but can have}
00083 \textcolor{comment}{     * any size.  Therefore we must allocate a new array to read in}
00084 \textcolor{comment}{     * data using malloc().}
00085 \textcolor{comment}{     */}
00086 
00087     \textcolor{comment}{/*}
00088 \textcolor{comment}{     * Open file and dataset.}
00089 \textcolor{comment}{     */}
00090     file = H5Fopen (FILE, H5F\_ACC\_RDONLY, H5P\_DEFAULT);
00091     dset = H5Dopen (file, DATASET, H5P\_DEFAULT);
00092 
00093     \textcolor{comment}{/*}
00094 \textcolor{comment}{     * Get the datatype and its dimensions.}
00095 \textcolor{comment}{     */}
00096     filetype = H5Dget\_type (dset);
00097     ndims = H5Tget\_array\_dims (filetype, adims);
00098 
00099     \textcolor{comment}{/*}
00100 \textcolor{comment}{     * Get dataspace and allocate memory for read buffer.  This is a}
00101 \textcolor{comment}{     * three dimensional dataset when the array datatype is included so}
00102 \textcolor{comment}{     * the dynamic allocation must be done in steps.}
00103 \textcolor{comment}{     */}
00104     space = H5Dget\_space (dset);
00105     ndims = H5Sget\_simple\_extent\_dims (space, dims, NULL);
00106 
00107     \textcolor{comment}{/*}
00108 \textcolor{comment}{     * Allocate array of pointers to two-dimensional arrays (the}
00109 \textcolor{comment}{     * elements of the dataset.}
00110 \textcolor{comment}{     */}
00111     rdata = (\textcolor{keywordtype}{int} ***) malloc (dims[0] * \textcolor{keyword}{sizeof} (\textcolor{keywordtype}{int} **));
00112 
00113     \textcolor{comment}{/*}
00114 \textcolor{comment}{     * Allocate two dimensional array of pointers to rows in the data}
00115 \textcolor{comment}{     * elements.}
00116 \textcolor{comment}{     */}
00117     rdata[0] = (\textcolor{keywordtype}{int} **) malloc (dims[0] * adims[0] * \textcolor{keyword}{sizeof} (\textcolor{keywordtype}{int} *));
00118 
00119     \textcolor{comment}{/*}
00120 \textcolor{comment}{     * Allocate space for integer data.}
00121 \textcolor{comment}{     */}
00122     rdata[0][0] = (\textcolor{keywordtype}{int} *) malloc (dims[0] * adims[0] * adims[1] * \textcolor{keyword}{sizeof} (\textcolor{keywordtype}{int}));
00123 
00124     \textcolor{comment}{/*}
00125 \textcolor{comment}{     * Set the members of the pointer arrays allocated above to point}
00126 \textcolor{comment}{     * to the correct locations in their respective arrays.}
00127 \textcolor{comment}{     */}
00128     \textcolor{keywordflow}{for} (i=0; i<dims[0]; i++) \{
00129         rdata[i] = rdata[0] + i * adims[0];
00130         \textcolor{keywordflow}{for} (j=0; j<adims[0]; j++)
00131             rdata[i][j] = rdata[0][0] + (adims[0] * adims[1] * i) +
00132                         (adims[1] * j);
00133     \}
00134 
00135     \textcolor{comment}{/*}
00136 \textcolor{comment}{     * Create the memory datatype.}
00137 \textcolor{comment}{     */}
00138     memtype = H5Tarray\_create (H5T\_NATIVE\_INT, 2, adims);
00139 
00140     \textcolor{comment}{/*}
00141 \textcolor{comment}{     * Read the data.}
00142 \textcolor{comment}{     */}
00143     status = H5Dread (dset, memtype, H5S\_ALL, H5S\_ALL, H5P\_DEFAULT,
00144                 rdata[0][0]);
00145 
00146     \textcolor{comment}{/*}
00147 \textcolor{comment}{     * Output the data to the screen.}
00148 \textcolor{comment}{     */}
00149     \textcolor{keywordflow}{for} (i=0; i<dims[0]; i++) \{
00150         printf (\textcolor{stringliteral}{"%s[%d]:\(\backslash\)n"}, DATASET, i);
00151         \textcolor{keywordflow}{for} (j=0; j<adims[0]; j++) \{
00152             printf (\textcolor{stringliteral}{" ["});
00153             \textcolor{keywordflow}{for} (k=0; k<adims[1]; k++)
00154                 printf (\textcolor{stringliteral}{" %3d"}, rdata[i][j][k]);
00155             printf (\textcolor{stringliteral}{"]\(\backslash\)n"});
00156         \}
00157         printf(\textcolor{stringliteral}{"\(\backslash\)n"});
00158     \}
00159 
00160     \textcolor{comment}{/*}
00161 \textcolor{comment}{     * Close and release resources.}
00162 \textcolor{comment}{     */}
00163     free (rdata[0][0]);
00164     free (rdata[0]);
00165     free (rdata);
00166     status = H5Dclose (dset);
00167     status = H5Sclose (space);
00168     status = H5Tclose (filetype);
00169     status = H5Tclose (memtype);
00170     status = H5Fclose (file);
00171 
00172     \textcolor{keywordflow}{return} 0;
00173 \}
\end{DoxyCode}
