\hypertarget{visual__studio_2_h_d_f5_21_810_81_2_h_d_f5_examples_2_c_2_h5_d_2h5ex__d__rdwr_8c_source}{}\section{visual\+\_\+studio/\+H\+D\+F5/1.10.1/\+H\+D\+F5\+Examples/\+C/\+H5\+D/h5ex\+\_\+d\+\_\+rdwr.c}
\label{visual__studio_2_h_d_f5_21_810_81_2_h_d_f5_examples_2_c_2_h5_d_2h5ex__d__rdwr_8c_source}\index{h5ex\+\_\+d\+\_\+rdwr.\+c@{h5ex\+\_\+d\+\_\+rdwr.\+c}}

\begin{DoxyCode}
00001 \textcolor{comment}{/************************************************************}
00002 \textcolor{comment}{}
00003 \textcolor{comment}{  This example shows how to read and write data to a}
00004 \textcolor{comment}{  dataset.  The program first writes integers to a dataset}
00005 \textcolor{comment}{  with dataspace dimensions of DIM0xDIM1, then closes the}
00006 \textcolor{comment}{  file.  Next, it reopens the file, reads back the data, and}
00007 \textcolor{comment}{  outputs it to the screen.}
00008 \textcolor{comment}{}
00009 \textcolor{comment}{  This file is intended for use with HDF5 Library version 1.8}
00010 \textcolor{comment}{}
00011 \textcolor{comment}{ ************************************************************/}
00012 
00013 \textcolor{preprocessor}{#include "hdf5.h"}
00014 \textcolor{preprocessor}{#include <stdio.h>}
00015 \textcolor{preprocessor}{#include <stdlib.h>}
00016 
00017 \textcolor{preprocessor}{#define FILE            "h5ex\_d\_rdwr.h5"}
00018 \textcolor{preprocessor}{#define DATASET         "DS1"}
00019 \textcolor{preprocessor}{#define DIM0            4}
00020 \textcolor{preprocessor}{#define DIM1            7}
00021 
00022 \textcolor{keywordtype}{int}
00023 main (\textcolor{keywordtype}{void})
00024 \{
00025     hid\_t       \hyperlink{structfile}{file}, space, dset;          \textcolor{comment}{/* Handles */}
00026     herr\_t      status;
00027     hsize\_t     dims[2] = \{DIM0, DIM1\};
00028     \textcolor{keywordtype}{int}         wdata[DIM0][DIM1],          \textcolor{comment}{/* Write buffer */}
00029                 rdata[DIM0][DIM1],          \textcolor{comment}{/* Read buffer */}
00030                 i, j;
00031 
00032     \textcolor{comment}{/*}
00033 \textcolor{comment}{     * Initialize data.}
00034 \textcolor{comment}{     */}
00035     \textcolor{keywordflow}{for} (i=0; i<DIM0; i++)
00036         \textcolor{keywordflow}{for} (j=0; j<DIM1; j++)
00037             wdata[i][j] = i * j - j;
00038 
00039     \textcolor{comment}{/*}
00040 \textcolor{comment}{     * Create a new file using the default properties.}
00041 \textcolor{comment}{     */}
00042     file = H5Fcreate (FILE, H5F\_ACC\_TRUNC, H5P\_DEFAULT, H5P\_DEFAULT);
00043 
00044     \textcolor{comment}{/*}
00045 \textcolor{comment}{     * Create dataspace.  Setting maximum size to NULL sets the maximum}
00046 \textcolor{comment}{     * size to be the current size.}
00047 \textcolor{comment}{     */}
00048     space = H5Screate\_simple (2, dims, NULL);
00049 
00050     \textcolor{comment}{/*}
00051 \textcolor{comment}{     * Create the dataset.  We will use all default properties for this}
00052 \textcolor{comment}{     * example.}
00053 \textcolor{comment}{     */}
00054     dset = H5Dcreate (file, DATASET, H5T\_STD\_I32LE, space, H5P\_DEFAULT,
00055                 H5P\_DEFAULT, H5P\_DEFAULT);
00056 
00057     \textcolor{comment}{/*}
00058 \textcolor{comment}{     * Write the data to the dataset.}
00059 \textcolor{comment}{     */}
00060     status = H5Dwrite (dset, H5T\_NATIVE\_INT, H5S\_ALL, H5S\_ALL, H5P\_DEFAULT,
00061                 wdata[0]);
00062 
00063     \textcolor{comment}{/*}
00064 \textcolor{comment}{     * Close and release resources.}
00065 \textcolor{comment}{     */}
00066     status = H5Dclose (dset);
00067     status = H5Sclose (space);
00068     status = H5Fclose (file);
00069 
00070 
00071     \textcolor{comment}{/*}
00072 \textcolor{comment}{     * Now we begin the read section of this example.}
00073 \textcolor{comment}{     */}
00074 
00075     \textcolor{comment}{/*}
00076 \textcolor{comment}{     * Open file and dataset using the default properties.}
00077 \textcolor{comment}{     */}
00078     file = H5Fopen (FILE, H5F\_ACC\_RDONLY, H5P\_DEFAULT);
00079     dset = H5Dopen (file, DATASET, H5P\_DEFAULT);
00080 
00081     \textcolor{comment}{/*}
00082 \textcolor{comment}{     * Read the data using the default properties.}
00083 \textcolor{comment}{     */}
00084     status = H5Dread (dset, H5T\_NATIVE\_INT, H5S\_ALL, H5S\_ALL, H5P\_DEFAULT,
00085                 rdata[0]);
00086 
00087     \textcolor{comment}{/*}
00088 \textcolor{comment}{     * Output the data to the screen.}
00089 \textcolor{comment}{     */}
00090     printf (\textcolor{stringliteral}{"%s:\(\backslash\)n"}, DATASET);
00091     \textcolor{keywordflow}{for} (i=0; i<DIM0; i++) \{
00092         printf (\textcolor{stringliteral}{" ["});
00093         \textcolor{keywordflow}{for} (j=0; j<DIM1; j++)
00094             printf (\textcolor{stringliteral}{" %3d"}, rdata[i][j]);
00095         printf (\textcolor{stringliteral}{"]\(\backslash\)n"});
00096     \}
00097 
00098     \textcolor{comment}{/*}
00099 \textcolor{comment}{     * Close and release resources.}
00100 \textcolor{comment}{     */}
00101     status = H5Dclose (dset);
00102     status = H5Fclose (file);
00103 
00104     \textcolor{keywordflow}{return} 0;
00105 \}
\end{DoxyCode}
