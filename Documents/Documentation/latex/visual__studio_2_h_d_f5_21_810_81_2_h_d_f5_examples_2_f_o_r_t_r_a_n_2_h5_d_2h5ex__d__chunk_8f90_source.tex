\hypertarget{visual__studio_2_h_d_f5_21_810_81_2_h_d_f5_examples_2_f_o_r_t_r_a_n_2_h5_d_2h5ex__d__chunk_8f90_source}{}\section{visual\+\_\+studio/\+H\+D\+F5/1.10.1/\+H\+D\+F5\+Examples/\+F\+O\+R\+T\+R\+A\+N/\+H5\+D/h5ex\+\_\+d\+\_\+chunk.f90}
\label{visual__studio_2_h_d_f5_21_810_81_2_h_d_f5_examples_2_f_o_r_t_r_a_n_2_h5_d_2h5ex__d__chunk_8f90_source}\index{h5ex\+\_\+d\+\_\+chunk.\+f90@{h5ex\+\_\+d\+\_\+chunk.\+f90}}

\begin{DoxyCode}
00001 \textcolor{comment}{! ************************************************************}
00002 \textcolor{comment}{!}
00003 \textcolor{comment}{!  This example shows how to create a chunked dataset.  The}
00004 \textcolor{comment}{!  program first writes integers in a hyperslab selection to}
00005 \textcolor{comment}{!  a chunked dataset with dataspace dimensions of DIM0xDIM1}
00006 \textcolor{comment}{!  and chunk size of CHUNK0xCHUNK1, then closes the file.}
00007 \textcolor{comment}{!  Next, it reopens the file, reads back the data, and}
00008 \textcolor{comment}{!  outputs it to the screen.  Finally it reads the data again}
00009 \textcolor{comment}{!  using a different hyperslab selection, and outputs}
00010 \textcolor{comment}{!  the result to the screen.}
00011 \textcolor{comment}{!}
00012 \textcolor{comment}{!  This file is intended for use with HDF5 Library verion 1.8}
00013 \textcolor{comment}{!}
00014 \textcolor{comment}{! ************************************************************}
00015 
00016 \textcolor{keyword}{PROGRAM} main
00017 
00018   \textcolor{keywordtype}{USE }hdf5
00019 
00020   \textcolor{keywordtype}{IMPLICIT NONE}
00021 
00022   \textcolor{keywordtype}{CHARACTER(LEN=18)}, \textcolor{keywordtype}{PARAMETER} :: filename = \textcolor{stringliteral}{"h5ex\_d\_chunk.h5"}
00023   \textcolor{keywordtype}{CHARACTER(LEN=3)} , \textcolor{keywordtype}{PARAMETER} :: dataset  = \textcolor{stringliteral}{"DS1"}
00024   \textcolor{keywordtype}{INTEGER}          , \textcolor{keywordtype}{PARAMETER} :: dim0     = 6
00025   \textcolor{keywordtype}{INTEGER}          , \textcolor{keywordtype}{PARAMETER} :: dim1     = 8
00026   \textcolor{keywordtype}{INTEGER}          , \textcolor{keywordtype}{PARAMETER} :: chunk0   = 4
00027   \textcolor{keywordtype}{INTEGER}          , \textcolor{keywordtype}{PARAMETER} :: chunk1   = 4
00028 
00029   \textcolor{keywordtype}{INTEGER} :: hdferr
00030   \textcolor{keywordtype}{INTEGER} :: layout
00031   \textcolor{keywordtype}{INTEGER(HID\_T)}  :: \hyperlink{structfile}{file}, space, dset, dcpl \textcolor{comment}{! Handles}
00032   \textcolor{keywordtype}{INTEGER(HSIZE\_T)}, \textcolor{keywordtype}{DIMENSION(1:2)}   :: dims = (/dim0, dim1/), chunk = (/chunk0,chunk1/)
00033   \textcolor{keywordtype}{INTEGER(HSIZE\_T)}, \textcolor{keywordtype}{DIMENSION(1:2)}   :: start, stride, count, block
00034 
00035   \textcolor{keywordtype}{INTEGER}, \textcolor{keywordtype}{DIMENSION(1:dim0, 1:dim1)} :: wdata, & \textcolor{comment}{! Write buffer}
00036                                         rdata    \textcolor{comment}{! Read buffer}
00037   \textcolor{keywordtype}{INTEGER} :: i, j
00038   \textcolor{comment}{!}
00039   \textcolor{comment}{! Initialize FORTRAN interface.}
00040   \textcolor{comment}{!}
00041   \textcolor{keyword}{CALL }h5open\_f(hdferr)
00042   \textcolor{comment}{! Initialize data to "1", to make it easier to see the selections.}
00043   \textcolor{comment}{!}
00044   wdata = 1
00045   \textcolor{comment}{!}
00046   \textcolor{comment}{! Print the data to the screen.}
00047   \textcolor{comment}{!}
00048   \textcolor{keyword}{WRITE}(*, \textcolor{stringliteral}{'(/,"Original Data:")'})
00049   \textcolor{keywordflow}{DO} i=1, dim0
00050      \textcolor{keyword}{WRITE}(*,\textcolor{stringliteral}{'(" [")'}, advance=\textcolor{stringliteral}{'NO'})
00051      \textcolor{keyword}{WRITE}(*,\textcolor{stringliteral}{'(80i3)'}, advance=\textcolor{stringliteral}{'NO'}) wdata(i,:)
00052      \textcolor{keyword}{WRITE}(*,\textcolor{stringliteral}{'(" ]")'})
00053 \textcolor{keywordflow}{  ENDDO}
00054   \textcolor{comment}{!}
00055   \textcolor{comment}{! Create a new file using the default properties.}
00056   \textcolor{comment}{!}
00057   \textcolor{keyword}{CALL }h5fcreate\_f(filename, h5f\_acc\_trunc\_f, \hyperlink{structfile}{file}, hdferr)
00058   \textcolor{comment}{!}
00059   \textcolor{comment}{! Create dataspace.  Setting maximum size to be the current size.}
00060   \textcolor{comment}{!}
00061   \textcolor{keyword}{CALL }h5screate\_simple\_f(2, dims, space, hdferr)
00062   \textcolor{comment}{!}
00063   \textcolor{comment}{! Create the dataset creation property list, and set the chunk}
00064   \textcolor{comment}{! size.}
00065   \textcolor{comment}{!}
00066   \textcolor{keyword}{CALL }h5pcreate\_f(h5p\_dataset\_create\_f, dcpl, hdferr)
00067   \textcolor{keyword}{CALL }h5pset\_chunk\_f(dcpl, 2, chunk, hdferr)
00068   \textcolor{comment}{!}
00069   \textcolor{comment}{! Create the chunked dataset.}
00070   \textcolor{comment}{!}
00071   \textcolor{keyword}{CALL }h5dcreate\_f(\hyperlink{structfile}{file}, dataset, h5t\_std\_i32le, space, dset, hdferr, dcpl)
00072   \textcolor{comment}{!}
00073   \textcolor{comment}{! Define and select the first part of the hyperslab selection.}
00074   \textcolor{comment}{!}
00075   start = 0
00076   stride = 3
00077   count(1:2) = (/2,3/)
00078   block = 2
00079   \textcolor{keyword}{CALL }h5sselect\_hyperslab\_f (space, h5s\_select\_set\_f, start, count, &
00080        hdferr, stride, block)
00081   \textcolor{comment}{!}
00082   \textcolor{comment}{! Define and select the second part of the hyperslab selection,}
00083   \textcolor{comment}{! which is subtracted from the first selection by the use of}
00084   \textcolor{comment}{! H5S\_SELECT\_NOTB}
00085   \textcolor{comment}{!}
00086   block = 1
00087   \textcolor{keyword}{CALL }h5sselect\_hyperslab\_f (space, h5s\_select\_notb\_f, start, count, &
00088        hdferr, stride, block)
00089   \textcolor{comment}{!}
00090   \textcolor{comment}{! Write the data to the dataset.}
00091   \textcolor{comment}{!}
00092   \textcolor{keyword}{CALL }h5dwrite\_f(dset, h5t\_native\_integer, wdata, dims, hdferr, file\_space\_id=space)
00093   \textcolor{comment}{!}
00094   \textcolor{comment}{! Close and release resources.}
00095   \textcolor{comment}{!}
00096   \textcolor{keyword}{CALL }h5pclose\_f(dcpl , hdferr)
00097   \textcolor{keyword}{CALL }h5dclose\_f(dset , hdferr)
00098   \textcolor{keyword}{CALL }h5sclose\_f(space, hdferr)
00099   \textcolor{keyword}{CALL }h5fclose\_f(\hyperlink{structfile}{file} , hdferr)
00100   \textcolor{comment}{!}
00101   \textcolor{comment}{! Now we begin the read section of this example.}
00102   \textcolor{comment}{!}
00103   \textcolor{comment}{!}
00104   \textcolor{comment}{! Open file and dataset using the default properties.}
00105   \textcolor{comment}{!}
00106   \textcolor{keyword}{CALL }h5fopen\_f(filename, h5f\_acc\_rdonly\_f, \hyperlink{structfile}{file}, hdferr)
00107   \textcolor{keyword}{CALL }h5dopen\_f (\hyperlink{structfile}{file}, dataset, dset, hdferr)
00108   \textcolor{comment}{!}
00109   \textcolor{comment}{! Retrieve the dataset creation property list, and print the}
00110   \textcolor{comment}{! storage layout.}
00111   \textcolor{comment}{!}
00112   \textcolor{keyword}{CALL }h5dget\_create\_plist\_f(dset, dcpl, hdferr)
00113   \textcolor{keyword}{CALL }h5pget\_layout\_f(dcpl, layout, hdferr)
00114   \textcolor{keyword}{WRITE}(*,\textcolor{stringliteral}{'(/,"Storage layout for ", A," is: ")'}, advance=\textcolor{stringliteral}{'NO'}) dataset
00115   \textcolor{keywordflow}{IF}(layout.EQ.h5d\_compact\_f)\textcolor{keywordflow}{THEN}
00116      \textcolor{keyword}{WRITE}(*,\textcolor{stringliteral}{'("H5D\_COMPACT\_F",/)'})
00117   \textcolor{keywordflow}{ELSE} \textcolor{keywordflow}{IF} (layout.EQ.h5d\_contiguous\_f)\textcolor{keywordflow}{THEN}
00118      \textcolor{keyword}{WRITE}(*,\textcolor{stringliteral}{'("H5D\_CONTIGUOUS\_F",/)'})
00119   \textcolor{keywordflow}{ELSE} \textcolor{keywordflow}{IF} (layout.EQ.h5d\_chunked\_f)\textcolor{keywordflow}{THEN}
00120      \textcolor{keyword}{WRITE}(*,\textcolor{stringliteral}{'("H5D\_CHUNKED\_F",/)'})
00121   \textcolor{keywordflow}{ELSE} \textcolor{keywordflow}{IF} (layout.EQ.h5d\_virtual\_f)\textcolor{keywordflow}{THEN}
00122      \textcolor{keyword}{WRITE}(*,\textcolor{stringliteral}{'("H5D\_VIRTUAL\_F",/)'})
00123   \textcolor{keywordflow}{ELSE}
00124      \textcolor{keyword}{WRITE}(*,\textcolor{stringliteral}{'("Layout Error",/)'})
00125 \textcolor{keywordflow}{  ENDIF}
00126   \textcolor{comment}{!}
00127   \textcolor{comment}{! Read the data using the default properties.}
00128   \textcolor{comment}{!}
00129   \textcolor{keyword}{CALL }h5dread\_f(dset, h5t\_native\_integer, rdata, dims, hdferr)
00130   \textcolor{comment}{!}
00131   \textcolor{comment}{! Output the data to the screen.}
00132   \textcolor{comment}{!}
00133   \textcolor{keyword}{WRITE}(*, \textcolor{stringliteral}{'("Data as written to disk by hyberslabs:")'})
00134   \textcolor{keywordflow}{DO} i=1, dim0
00135      \textcolor{keyword}{WRITE}(*,\textcolor{stringliteral}{'(" [")'}, advance=\textcolor{stringliteral}{'NO'})
00136      \textcolor{keyword}{WRITE}(*,\textcolor{stringliteral}{'(80i3)'}, advance=\textcolor{stringliteral}{'NO'}) rdata(i,:)
00137      \textcolor{keyword}{WRITE}(*,\textcolor{stringliteral}{'(" ]")'})
00138 \textcolor{keywordflow}{  ENDDO}
00139   \textcolor{comment}{!}
00140   \textcolor{comment}{! Initialize the read array.}
00141   \textcolor{comment}{!}
00142   rdata = 0
00143   \textcolor{comment}{!}
00144   \textcolor{comment}{! Define and select the hyperslab to use for reading.}
00145   \textcolor{comment}{!}
00146   \textcolor{keyword}{CALL }h5dget\_space\_f(dset, space, hdferr)
00147   start(1:2) = (/0,1/)
00148   stride = 4
00149   count = 2
00150   block(1:2) = (/2,3/)
00151 
00152   \textcolor{keyword}{CALL }h5sselect\_hyperslab\_f (space, h5s\_select\_set\_f, start, count, &
00153        hdferr, stride, block)
00154   \textcolor{comment}{!}
00155   \textcolor{comment}{! Read the data using the previously defined hyperslab.}
00156   \textcolor{comment}{!}
00157   \textcolor{keyword}{CALL }h5dread\_f(dset, h5t\_native\_integer, rdata, dims, hdferr, file\_space\_id=space)
00158   \textcolor{comment}{!}
00159   \textcolor{comment}{! Output the data to the screen.}
00160   \textcolor{comment}{!}
00161   \textcolor{keyword}{WRITE}(*, \textcolor{stringliteral}{'(/,"Data as read from disk by hyperslab:")'})
00162   \textcolor{keywordflow}{DO} i=1, dim0
00163      \textcolor{keyword}{WRITE}(*,\textcolor{stringliteral}{'(" [")'}, advance=\textcolor{stringliteral}{'NO'})
00164      \textcolor{keyword}{WRITE}(*,\textcolor{stringliteral}{'(80i3)'}, advance=\textcolor{stringliteral}{'NO'}) rdata(i,:)
00165      \textcolor{keyword}{WRITE}(*,\textcolor{stringliteral}{'(" ]")'})
00166 \textcolor{keywordflow}{  ENDDO}
00167   \textcolor{comment}{!}
00168   \textcolor{comment}{! Close and release resources.}
00169   \textcolor{comment}{!}
00170   \textcolor{keyword}{CALL }h5pclose\_f(dcpl , hdferr)
00171   \textcolor{keyword}{CALL }h5dclose\_f(dset , hdferr)
00172   \textcolor{keyword}{CALL }h5sclose\_f(space, hdferr)
00173   \textcolor{keyword}{CALL }h5fclose\_f(\hyperlink{structfile}{file} , hdferr)
00174 
00175 \textcolor{keyword}{END PROGRAM }main
\end{DoxyCode}
