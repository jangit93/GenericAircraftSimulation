\hypertarget{_h_d_f5_21_810_81_2_h_d_f5_examples_2_f_o_r_t_r_a_n_2_h5_d_2h5ex__d__hyper_8f90_source}{}\section{H\+D\+F5/1.10.1/\+H\+D\+F5\+Examples/\+F\+O\+R\+T\+R\+A\+N/\+H5\+D/h5ex\+\_\+d\+\_\+hyper.f90}
\label{_h_d_f5_21_810_81_2_h_d_f5_examples_2_f_o_r_t_r_a_n_2_h5_d_2h5ex__d__hyper_8f90_source}\index{h5ex\+\_\+d\+\_\+hyper.\+f90@{h5ex\+\_\+d\+\_\+hyper.\+f90}}

\begin{DoxyCode}
00001 \textcolor{comment}{!************************************************************}
00002 \textcolor{comment}{!}
00003 \textcolor{comment}{!  This example shows how to read and write data to a}
00004 \textcolor{comment}{!  dataset by hyberslabs.  The program first writes integers}
00005 \textcolor{comment}{!  in a hyperslab selection to a dataset with dataspace}
00006 \textcolor{comment}{!  dimensions of DIM0xDIM1, then closes the file.  Next, it}
00007 \textcolor{comment}{!  reopens the file, reads back the data, and outputs it to}
00008 \textcolor{comment}{!  the screen.  Finally it reads the data again using a}
00009 \textcolor{comment}{!  different hyperslab selection, and outputs the result to}
00010 \textcolor{comment}{!  the screen.}
00011 \textcolor{comment}{!}
00012 \textcolor{comment}{!  This file is intended for use with HDF5 Library verion 1.8}
00013 \textcolor{comment}{!}
00014 \textcolor{comment}{! ************************************************************/}
00015 \textcolor{keyword}{PROGRAM} main
00016 
00017   \textcolor{keywordtype}{USE }hdf5
00018 
00019   \textcolor{keywordtype}{IMPLICIT NONE}
00020 
00021   \textcolor{keywordtype}{CHARACTER(LEN=15)}, \textcolor{keywordtype}{PARAMETER} :: filename = \textcolor{stringliteral}{"h5ex\_d\_hyper.h5"}
00022   \textcolor{keywordtype}{CHARACTER(LEN=3)} , \textcolor{keywordtype}{PARAMETER} :: dataset  = \textcolor{stringliteral}{"DS1"}
00023   \textcolor{keywordtype}{INTEGER}          , \textcolor{keywordtype}{PARAMETER} :: dim0     = 6
00024   \textcolor{keywordtype}{INTEGER}          , \textcolor{keywordtype}{PARAMETER} :: dim1     = 8
00025 
00026   \textcolor{keywordtype}{INTEGER(HID\_T)}  :: \hyperlink{structfile}{file}, space, dset \textcolor{comment}{! Handles}
00027   \textcolor{keywordtype}{INTEGER}         :: hdferr
00028   \textcolor{keywordtype}{INTEGER(HSIZE\_T)}, \textcolor{keywordtype}{DIMENSION(1:2)} :: dims = (/dim0, dim1/)
00029   \textcolor{keywordtype}{INTEGER(HSIZE\_T)}, \textcolor{keywordtype}{DIMENSION(1:2)}   :: start, stride, count, block
00030 
00031   \textcolor{keywordtype}{INTEGER}, \textcolor{keywordtype}{DIMENSION(1:dim0, 1:dim1)} :: wdata, & \textcolor{comment}{! Write buffer }
00032                                         rdata    \textcolor{comment}{! Read buffer}
00033   \textcolor{keywordtype}{INTEGER} :: i, j
00034 
00035   \textcolor{comment}{!}
00036   \textcolor{comment}{! Initialize FORTRAN interface.}
00037   \textcolor{comment}{!}
00038   \textcolor{keyword}{CALL }h5open\_f(hdferr)
00039   \textcolor{comment}{! Initialize data to "1", to make it easier to see the selections.}
00040   \textcolor{comment}{!}
00041   wdata = 1
00042   \textcolor{comment}{!}
00043   \textcolor{comment}{! Print the data to the screen.}
00044   \textcolor{comment}{!}
00045   \textcolor{keyword}{WRITE}(*, \textcolor{stringliteral}{'(/,"Original Data:")'})
00046   \textcolor{keywordflow}{DO} i=1, dim0
00047      \textcolor{keyword}{WRITE}(*,\textcolor{stringliteral}{'(" [")'}, advance=\textcolor{stringliteral}{'NO'})
00048      \textcolor{keyword}{WRITE}(*,\textcolor{stringliteral}{'(80i3)'}, advance=\textcolor{stringliteral}{'NO'}) wdata(i,:)
00049      \textcolor{keyword}{WRITE}(*,\textcolor{stringliteral}{'(" ]")'})
00050 \textcolor{keywordflow}{  ENDDO}
00051   \textcolor{comment}{!}
00052   \textcolor{comment}{! Create a new file using the default properties.}
00053   \textcolor{comment}{!}
00054   \textcolor{keyword}{CALL }h5fcreate\_f(filename, h5f\_acc\_trunc\_f, \hyperlink{structfile}{file}, hdferr)
00055   \textcolor{comment}{!}
00056   \textcolor{comment}{! Create dataspace.  Setting maximum size to be the current size.}
00057   \textcolor{comment}{!}
00058   \textcolor{keyword}{CALL }h5screate\_simple\_f(2, dims, space, hdferr)
00059   \textcolor{comment}{! Create the dataset.  We will use all default properties for this}
00060   \textcolor{comment}{! example.}
00061   \textcolor{comment}{!}
00062   \textcolor{keyword}{CALL }h5dcreate\_f(\hyperlink{structfile}{file}, dataset, h5t\_std\_i32le, space, dset, hdferr)
00063   \textcolor{comment}{!}
00064   \textcolor{comment}{! Define and select the first part of the hyperslab selection.}
00065   \textcolor{comment}{!  }
00066   start = 0
00067   stride = 3
00068   count(1:2) = (/2,3/)
00069   block = 2
00070   \textcolor{keyword}{CALL }h5sselect\_hyperslab\_f (space, h5s\_select\_set\_f, start, count, &
00071        hdferr, stride, block)
00072   \textcolor{comment}{!}
00073   \textcolor{comment}{! Define and select the second part of the hyperslab selection,}
00074   \textcolor{comment}{! which is subtracted from the first selection by the use of}
00075   \textcolor{comment}{! H5S\_SELECT\_NOTB}
00076   \textcolor{comment}{!}
00077   block = 1
00078   \textcolor{keyword}{CALL }h5sselect\_hyperslab\_f (space, h5s\_select\_notb\_f, start, count, &
00079        hdferr, stride, block)
00080   \textcolor{comment}{!}
00081   \textcolor{comment}{! Write the data to the dataset.}
00082   \textcolor{comment}{!}
00083   \textcolor{keyword}{CALL }h5dwrite\_f(dset, h5t\_native\_integer, wdata, dims, hdferr, file\_space\_id=space)
00084   \textcolor{comment}{!}
00085   \textcolor{comment}{! Close and release resources.}
00086   \textcolor{comment}{!}
00087   \textcolor{keyword}{CALL }h5dclose\_f(dset , hdferr)
00088   \textcolor{keyword}{CALL }h5sclose\_f(space, hdferr)
00089   \textcolor{keyword}{CALL }h5fclose\_f(\hyperlink{structfile}{file} , hdferr)
00090   \textcolor{comment}{!}
00091   \textcolor{comment}{! Now we begin the read section of this example.}
00092   \textcolor{comment}{!}
00093   \textcolor{comment}{!}
00094   \textcolor{comment}{! Open file and dataset using the default properties.}
00095   \textcolor{comment}{!}
00096   \textcolor{keyword}{CALL }h5fopen\_f(filename, h5f\_acc\_rdonly\_f, \hyperlink{structfile}{file}, hdferr)
00097   \textcolor{keyword}{CALL }h5dopen\_f (\hyperlink{structfile}{file}, dataset, dset, hdferr)
00098   \textcolor{comment}{!}
00099   \textcolor{comment}{! Read the data using the default properties.}
00100   \textcolor{comment}{!}
00101   \textcolor{keyword}{CALL }h5dread\_f(dset, h5t\_native\_integer, rdata, dims, hdferr)
00102   \textcolor{comment}{!}
00103   \textcolor{comment}{! Output the data to the screen.}
00104   \textcolor{comment}{!}
00105   \textcolor{keyword}{WRITE}(*, \textcolor{stringliteral}{'(/,"Data as written to disk by hyberslabs:")'})
00106   \textcolor{keywordflow}{DO} i=1, dim0
00107      \textcolor{keyword}{WRITE}(*,\textcolor{stringliteral}{'(" [")'}, advance=\textcolor{stringliteral}{'NO'})
00108      \textcolor{keyword}{WRITE}(*,\textcolor{stringliteral}{'(80i3)'}, advance=\textcolor{stringliteral}{'NO'}) rdata(i,:)
00109      \textcolor{keyword}{WRITE}(*,\textcolor{stringliteral}{'(" ]")'})
00110 \textcolor{keywordflow}{  ENDDO}
00111   \textcolor{comment}{!}
00112   \textcolor{comment}{! Initialize the read array.}
00113   \textcolor{comment}{!}
00114   rdata = 0
00115   \textcolor{comment}{!}
00116   \textcolor{comment}{! Define and select the hyperslab to use for reading.}
00117   \textcolor{comment}{!}
00118   \textcolor{keyword}{CALL }h5dget\_space\_f(dset, space, hdferr)
00119 
00120   start(1:2)=(/0,1/)
00121   stride = 4
00122   count = 2
00123   block(1:2)=(/2,3/)
00124 
00125   \textcolor{keyword}{CALL }h5sselect\_hyperslab\_f (space, h5s\_select\_set\_f, start, count, &
00126        hdferr, stride, block)
00127   \textcolor{comment}{!}
00128   \textcolor{comment}{! Read the data using the previously defined hyperslab.}
00129   \textcolor{comment}{!}
00130   \textcolor{keyword}{CALL }h5dread\_f(dset, h5t\_native\_integer, rdata, dims, hdferr, file\_space\_id=space)
00131   \textcolor{comment}{!}
00132   \textcolor{comment}{! Output the DATA to the screen.}
00133   \textcolor{comment}{!}
00134   \textcolor{keyword}{WRITE}(*, \textcolor{stringliteral}{'(/,"Data as read from disk by hyberslabs:")'})
00135   \textcolor{keywordflow}{DO} i=1, dim0
00136      \textcolor{keyword}{WRITE}(*,\textcolor{stringliteral}{'(" [")'}, advance=\textcolor{stringliteral}{'NO'})
00137      \textcolor{keyword}{WRITE}(*,\textcolor{stringliteral}{'(80i3)'}, advance=\textcolor{stringliteral}{'NO'}) rdata(i,:)
00138      \textcolor{keyword}{WRITE}(*,\textcolor{stringliteral}{'(" ]")'})
00139 \textcolor{keywordflow}{  ENDDO}
00140   \textcolor{keyword}{WRITE}(*,\textcolor{stringliteral}{'(/)'})
00141   \textcolor{comment}{!}
00142   \textcolor{comment}{! Close and release resources.}
00143   \textcolor{comment}{!}
00144   \textcolor{keyword}{CALL }h5dclose\_f(dset , hdferr)
00145   \textcolor{keyword}{CALL }h5sclose\_f(space, hdferr)
00146   \textcolor{keyword}{CALL }h5fclose\_f(\hyperlink{structfile}{file} , hdferr)
00147   
00148 \textcolor{keyword}{END PROGRAM }main
\end{DoxyCode}
