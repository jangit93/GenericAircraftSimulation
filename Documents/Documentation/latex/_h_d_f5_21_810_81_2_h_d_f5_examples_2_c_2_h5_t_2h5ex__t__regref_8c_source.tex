\hypertarget{_h_d_f5_21_810_81_2_h_d_f5_examples_2_c_2_h5_t_2h5ex__t__regref_8c_source}{}\section{H\+D\+F5/1.10.1/\+H\+D\+F5\+Examples/\+C/\+H5\+T/h5ex\+\_\+t\+\_\+regref.c}
\label{_h_d_f5_21_810_81_2_h_d_f5_examples_2_c_2_h5_t_2h5ex__t__regref_8c_source}\index{h5ex\+\_\+t\+\_\+regref.\+c@{h5ex\+\_\+t\+\_\+regref.\+c}}

\begin{DoxyCode}
00001 \textcolor{comment}{/************************************************************}
00002 \textcolor{comment}{}
00003 \textcolor{comment}{  This example shows how to read and write region references}
00004 \textcolor{comment}{  to a dataset.  The program first creates a dataset}
00005 \textcolor{comment}{  containing characters and writes references to region of}
00006 \textcolor{comment}{  the dataset to a new dataset with a dataspace of DIM0,}
00007 \textcolor{comment}{  then closes the file.  Next, it reopens the file,}
00008 \textcolor{comment}{  dereferences the references, and outputs the referenced}
00009 \textcolor{comment}{  regions to the screen.}
00010 \textcolor{comment}{}
00011 \textcolor{comment}{  This file is intended for use with HDF5 Library version 1.8}
00012 \textcolor{comment}{}
00013 \textcolor{comment}{ ************************************************************/}
00014 
00015 \textcolor{preprocessor}{#include "hdf5.h"}
00016 \textcolor{preprocessor}{#include <stdio.h>}
00017 \textcolor{preprocessor}{#include <stdlib.h>}
00018 
00019 \textcolor{preprocessor}{#define FILE            "h5ex\_t\_regref.h5"}
00020 \textcolor{preprocessor}{#define DATASET         "DS1"}
00021 \textcolor{preprocessor}{#define DATASET2        "DS2"}
00022 \textcolor{preprocessor}{#define DIM0            2}
00023 \textcolor{preprocessor}{#define DS2DIM0         3}
00024 \textcolor{preprocessor}{#define DS2DIM1         16}
00025 
00026 \textcolor{keywordtype}{int}
00027 main (\textcolor{keywordtype}{void})
00028 \{
00029     hid\_t               \hyperlink{structfile}{file}, space, memspace, dset, dset2;
00030                                                     \textcolor{comment}{/* Handles */}
00031     herr\_t              status;
00032     hsize\_t             dims[1] = \{DIM0\},
00033                         dims2[2] = \{DS2DIM0, DS2DIM1\},
00034                         coords[4][2] = \{ \{0,  1\},
00035                                          \{2, 11\},
00036                                          \{1,  0\},
00037                                          \{2,  4\} \},
00038                         start[2] = \{0, 0\},
00039                         stride[2] = \{2, 11\},
00040                         count[2] = \{2, 2\},
00041                         block[2] = \{1, 3\};
00042     hssize\_t            npoints;
00043     hdset\_reg\_ref\_t     wdata[DIM0],                \textcolor{comment}{/* Write buffer */}
00044                         *rdata;                     \textcolor{comment}{/* Read buffer */}
00045     ssize\_t             size;
00046     \textcolor{keywordtype}{char}                wdata2[DS2DIM0][DS2DIM1] = \{\textcolor{stringliteral}{"The quick brown"},
00047                                                     \textcolor{stringliteral}{"fox jumps over "},
00048                                                     \textcolor{stringliteral}{"the 5 lazy dogs"}\},
00049                         *rdata2,
00050                         *name;
00051     \textcolor{keywordtype}{int}                 ndims,
00052                         i;
00053 
00054     \textcolor{comment}{/*}
00055 \textcolor{comment}{     * Create a new file using the default properties.}
00056 \textcolor{comment}{     */}
00057     file = H5Fcreate (FILE, H5F\_ACC\_TRUNC, H5P\_DEFAULT, H5P\_DEFAULT);
00058 
00059     \textcolor{comment}{/*}
00060 \textcolor{comment}{     * Create a dataset with character data.}
00061 \textcolor{comment}{     */}
00062     space = H5Screate\_simple (2, dims2, NULL);
00063     dset2 = H5Dcreate (file, DATASET2, H5T\_STD\_I8LE, space, H5P\_DEFAULT,
00064                 H5P\_DEFAULT, H5P\_DEFAULT);
00065     status = H5Dwrite (dset2, H5T\_NATIVE\_CHAR, H5S\_ALL, H5S\_ALL, H5P\_DEFAULT,
00066                 wdata2);
00067 
00068     \textcolor{comment}{/*}
00069 \textcolor{comment}{     * Create reference to a list of elements in dset2.}
00070 \textcolor{comment}{     */}
00071     status = H5Sselect\_elements (space, H5S\_SELECT\_SET, 4, coords[0]);
00072     status = H5Rcreate (&wdata[0], file, DATASET2, H5R\_DATASET\_REGION, space);
00073 
00074     \textcolor{comment}{/*}
00075 \textcolor{comment}{     * Create reference to a hyperslab in dset2, close dataspace.}
00076 \textcolor{comment}{     */}
00077     status = H5Sselect\_hyperslab (space, H5S\_SELECT\_SET, start, stride, count,
00078                 block);
00079     status = H5Rcreate (&wdata[1], file, DATASET2, H5R\_DATASET\_REGION, space);
00080     status = H5Sclose (space);
00081 
00082     \textcolor{comment}{/*}
00083 \textcolor{comment}{     * Create dataspace.  Setting maximum size to NULL sets the maximum}
00084 \textcolor{comment}{     * size to be the current size.}
00085 \textcolor{comment}{     */}
00086     space = H5Screate\_simple (1, dims, NULL);
00087 
00088     \textcolor{comment}{/*}
00089 \textcolor{comment}{     * Create the dataset and write the region references to it.}
00090 \textcolor{comment}{     */}
00091     dset = H5Dcreate (file, DATASET, H5T\_STD\_REF\_DSETREG, space, H5P\_DEFAULT,
00092                 H5P\_DEFAULT, H5P\_DEFAULT);
00093     status = H5Dwrite (dset, H5T\_STD\_REF\_DSETREG, H5S\_ALL, H5S\_ALL, H5P\_DEFAULT,
00094                 wdata);
00095 
00096     \textcolor{comment}{/*}
00097 \textcolor{comment}{     * Close and release resources.}
00098 \textcolor{comment}{     */}
00099     status = H5Dclose (dset);
00100     status = H5Dclose (dset2);
00101     status = H5Sclose (space);
00102     status = H5Fclose (file);
00103 
00104 
00105     \textcolor{comment}{/*}
00106 \textcolor{comment}{     * Now we begin the read section of this example.  Here we assume}
00107 \textcolor{comment}{     * the dataset has the same name and rank, but can have any size.}
00108 \textcolor{comment}{     * Therefore we must allocate a new array to read in data using}
00109 \textcolor{comment}{     * malloc().}
00110 \textcolor{comment}{     */}
00111 
00112     \textcolor{comment}{/*}
00113 \textcolor{comment}{     * Open file and dataset.}
00114 \textcolor{comment}{     */}
00115     file = H5Fopen (FILE, H5F\_ACC\_RDONLY, H5P\_DEFAULT);
00116     dset = H5Dopen (file, DATASET, H5P\_DEFAULT);
00117 
00118     \textcolor{comment}{/*}
00119 \textcolor{comment}{     * Get dataspace and allocate memory for read buffer.}
00120 \textcolor{comment}{     */}
00121     space = H5Dget\_space (dset);
00122     ndims = H5Sget\_simple\_extent\_dims (space, dims, NULL);
00123     rdata = (hdset\_reg\_ref\_t *) malloc (dims[0] * \textcolor{keyword}{sizeof} (hdset\_reg\_ref\_t));
00124     status = H5Sclose (space);
00125 
00126     \textcolor{comment}{/*}
00127 \textcolor{comment}{     * Read the data.}
00128 \textcolor{comment}{     */}
00129     status = H5Dread (dset, H5T\_STD\_REF\_DSETREG, H5S\_ALL, H5S\_ALL, H5P\_DEFAULT,
00130                 rdata);
00131 
00132     \textcolor{comment}{/*}
00133 \textcolor{comment}{     * Output the data to the screen.}
00134 \textcolor{comment}{     */}
00135     \textcolor{keywordflow}{for} (i=0; i<dims[0]; i++) \{
00136         printf (\textcolor{stringliteral}{"%s[%d]:\(\backslash\)n  ->"}, DATASET, i);
00137 
00138         \textcolor{comment}{/*}
00139 \textcolor{comment}{         * Open the referenced object, retrieve its region as a}
00140 \textcolor{comment}{         * dataspace selection.}
00141 \textcolor{comment}{         */}
00142         dset2 = H5Rdereference (dset, H5P\_DEFAULT, H5R\_DATASET\_REGION, &rdata[i]);
00143         space = H5Rget\_region (dset, H5R\_DATASET\_REGION, &rdata[i]);
00144 
00145         \textcolor{comment}{/*}
00146 \textcolor{comment}{         * Get the length of the object's name, allocate space, then}
00147 \textcolor{comment}{         * retrieve the name.}
00148 \textcolor{comment}{         */}
00149         size = 1 + H5Iget\_name (dset2, NULL, 0);
00150         name = (\textcolor{keywordtype}{char} *) malloc (size);
00151         size = H5Iget\_name (dset2, name, size);
00152 
00153         \textcolor{comment}{/*}
00154 \textcolor{comment}{         * Allocate space for the read buffer.  We will only allocate}
00155 \textcolor{comment}{         * enough space for the selection, plus a null terminator.  The}
00156 \textcolor{comment}{         * read buffer will be 1-dimensional.}
00157 \textcolor{comment}{         */}
00158         npoints = H5Sget\_select\_npoints (space);
00159         rdata2 = (\textcolor{keywordtype}{char} *) malloc (npoints + 1);
00160 
00161         \textcolor{comment}{/*}
00162 \textcolor{comment}{         * Read the dataset region, and add a null terminator so we can}
00163 \textcolor{comment}{         * print it as a string.}
00164 \textcolor{comment}{         */}
00165         memspace = H5Screate\_simple (1, (hsize\_t *) &npoints, NULL);
00166         status = H5Dread (dset2, H5T\_NATIVE\_CHAR, memspace, space, H5P\_DEFAULT,
00167                     rdata2);
00168         rdata2[npoints] = \textcolor{charliteral}{'\(\backslash\)0'};
00169 
00170         \textcolor{comment}{/*}
00171 \textcolor{comment}{         * Print the name and region data, close and release resources.}
00172 \textcolor{comment}{         */}
00173         printf (\textcolor{stringliteral}{" %s: %s\(\backslash\)n"}, name, rdata2);
00174         free (rdata2);
00175         free (name);
00176         status = H5Sclose (space);
00177         status = H5Sclose (memspace);
00178         status = H5Dclose (dset2);
00179     \}
00180 
00181     \textcolor{comment}{/*}
00182 \textcolor{comment}{     * Close and release resources.}
00183 \textcolor{comment}{     */}
00184     free (rdata);
00185     status = H5Dclose (dset);
00186     status = H5Fclose (file);
00187 
00188     \textcolor{keywordflow}{return} 0;
00189 \}
\end{DoxyCode}
