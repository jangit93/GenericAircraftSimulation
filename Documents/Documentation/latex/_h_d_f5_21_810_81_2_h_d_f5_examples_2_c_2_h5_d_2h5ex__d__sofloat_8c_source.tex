\hypertarget{_h_d_f5_21_810_81_2_h_d_f5_examples_2_c_2_h5_d_2h5ex__d__sofloat_8c_source}{}\section{H\+D\+F5/1.10.1/\+H\+D\+F5\+Examples/\+C/\+H5\+D/h5ex\+\_\+d\+\_\+sofloat.c}
\label{_h_d_f5_21_810_81_2_h_d_f5_examples_2_c_2_h5_d_2h5ex__d__sofloat_8c_source}\index{h5ex\+\_\+d\+\_\+sofloat.\+c@{h5ex\+\_\+d\+\_\+sofloat.\+c}}

\begin{DoxyCode}
00001 \textcolor{comment}{/************************************************************}
00002 \textcolor{comment}{}
00003 \textcolor{comment}{  This example shows how to read and write data to a dataset}
00004 \textcolor{comment}{  using the Scale-Offset filter.  The program first checks}
00005 \textcolor{comment}{  if the Scale-Offset filter is available, then if it is it}
00006 \textcolor{comment}{  writes floating point numbers to a dataset using}
00007 \textcolor{comment}{  Scale-Offset, then closes the file Next, it reopens the}
00008 \textcolor{comment}{  file, reads back the data, and outputs the type of filter}
00009 \textcolor{comment}{  and the maximum value in the dataset to the screen.}
00010 \textcolor{comment}{}
00011 \textcolor{comment}{  This file is intended for use with HDF5 Library version 1.8}
00012 \textcolor{comment}{}
00013 \textcolor{comment}{ ************************************************************/}
00014 
00015 \textcolor{preprocessor}{#include "hdf5.h"}
00016 \textcolor{preprocessor}{#include <stdio.h>}
00017 \textcolor{preprocessor}{#include <stdlib.h>}
00018 
00019 \textcolor{preprocessor}{#define FILE            "h5ex\_d\_sofloat.h5"}
00020 \textcolor{preprocessor}{#define DATASET         "DS1"}
00021 \textcolor{preprocessor}{#define DIM0            32}
00022 \textcolor{preprocessor}{#define DIM1            64}
00023 \textcolor{preprocessor}{#define CHUNK0          4}
00024 \textcolor{preprocessor}{#define CHUNK1          8}
00025 
00026 \textcolor{keywordtype}{int}
00027 main (\textcolor{keywordtype}{void})
00028 \{
00029     hid\_t           \hyperlink{structfile}{file}, space, dset, dcpl;
00030                                                 \textcolor{comment}{/* Handles */}
00031     herr\_t          status;
00032     htri\_t          avail;
00033     H5Z\_filter\_t    filter\_type;
00034     hsize\_t         dims[2] = \{DIM0, DIM1\},
00035                     chunk[2] = \{CHUNK0, CHUNK1\};
00036     \textcolor{keywordtype}{size\_t}          nelmts;
00037     \textcolor{keywordtype}{unsigned} \textcolor{keywordtype}{int}    flags,
00038                     filter\_info;
00039     \textcolor{keywordtype}{double}          wdata[DIM0][DIM1],          \textcolor{comment}{/* Write buffer */}
00040                     rdata[DIM0][DIM1],          \textcolor{comment}{/* Read buffer */}
00041                     max,
00042                     min;
00043     \textcolor{keywordtype}{int}             i, j;
00044 
00045     \textcolor{comment}{/*}
00046 \textcolor{comment}{     * Check if Scale-Offset compression is available and can be used}
00047 \textcolor{comment}{     * for both compression and decompression.  Normally we do not}
00048 \textcolor{comment}{     * perform error checking in these examples for the sake of}
00049 \textcolor{comment}{     * clarity, but in this case we will make an exception because this}
00050 \textcolor{comment}{     * filter is an optional part of the hdf5 library.}
00051 \textcolor{comment}{     */}
00052     avail = H5Zfilter\_avail(H5Z\_FILTER\_SCALEOFFSET);
00053     \textcolor{keywordflow}{if} (!avail) \{
00054         printf (\textcolor{stringliteral}{"Scale-Offset filter not available.\(\backslash\)n"});
00055         \textcolor{keywordflow}{return} 1;
00056     \}
00057     status = H5Zget\_filter\_info (H5Z\_FILTER\_SCALEOFFSET, &filter\_info);
00058     \textcolor{keywordflow}{if} ( !(filter\_info & H5Z\_FILTER\_CONFIG\_ENCODE\_ENABLED) ||
00059                 !(filter\_info & H5Z\_FILTER\_CONFIG\_DECODE\_ENABLED) ) \{
00060         printf (\textcolor{stringliteral}{"Scale-Offset filter not available for encoding and decoding.\(\backslash\)n"});
00061         \textcolor{keywordflow}{return} 1;
00062     \}
00063 
00064     \textcolor{comment}{/*}
00065 \textcolor{comment}{     * Initialize data.}
00066 \textcolor{comment}{     */}
00067     \textcolor{keywordflow}{for} (i=0; i<DIM0; i++)
00068         \textcolor{keywordflow}{for} (j=0; j<DIM1; j++)
00069             wdata[i][j] = (\textcolor{keywordtype}{double}) (i + 1) / (j + 0.3) + j;
00070 
00071     \textcolor{comment}{/*}
00072 \textcolor{comment}{     * Find the maximum value in the dataset, to verify that it was}
00073 \textcolor{comment}{     * read correctly.}
00074 \textcolor{comment}{     */}
00075     max = wdata[0][0];
00076     min = wdata[0][0];
00077     \textcolor{keywordflow}{for} (i=0; i<DIM0; i++)
00078         \textcolor{keywordflow}{for} (j=0; j<DIM1; j++) \{
00079             \textcolor{keywordflow}{if} (max < wdata[i][j])
00080                 max = wdata[i][j];
00081             \textcolor{keywordflow}{if} (min > wdata[i][j])
00082                 min = wdata[i][j];
00083         \}
00084 
00085     \textcolor{comment}{/*}
00086 \textcolor{comment}{     * Print the maximum value.}
00087 \textcolor{comment}{     */}
00088     printf (\textcolor{stringliteral}{"Maximum value in write buffer is: %f\(\backslash\)n"}, max);
00089     printf (\textcolor{stringliteral}{"Minimum value in write buffer is: %f\(\backslash\)n"}, min);
00090 
00091     \textcolor{comment}{/*}
00092 \textcolor{comment}{     * Create a new file using the default properties.}
00093 \textcolor{comment}{     */}
00094     file = H5Fcreate (FILE, H5F\_ACC\_TRUNC, H5P\_DEFAULT, H5P\_DEFAULT);
00095 
00096     \textcolor{comment}{/*}
00097 \textcolor{comment}{     * Create dataspace.  Setting maximum size to NULL sets the maximum}
00098 \textcolor{comment}{     * size to be the current size.}
00099 \textcolor{comment}{     */}
00100     space = H5Screate\_simple (2, dims, NULL);
00101 
00102     \textcolor{comment}{/*}
00103 \textcolor{comment}{     * Create the dataset creation property list, add the Scale-Offset}
00104 \textcolor{comment}{     * filter and set the chunk size.}
00105 \textcolor{comment}{     */}
00106     dcpl = H5Pcreate (H5P\_DATASET\_CREATE);
00107     status = H5Pset\_scaleoffset (dcpl, H5Z\_SO\_FLOAT\_DSCALE, 2);
00108     status = H5Pset\_chunk (dcpl, 2, chunk);
00109 
00110     \textcolor{comment}{/*}
00111 \textcolor{comment}{     * Create the dataset.}
00112 \textcolor{comment}{     */}
00113     dset = H5Dcreate (file, DATASET, H5T\_IEEE\_F64LE, space, H5P\_DEFAULT, dcpl,
00114                 H5P\_DEFAULT);
00115 
00116     \textcolor{comment}{/*}
00117 \textcolor{comment}{     * Write the data to the dataset.}
00118 \textcolor{comment}{     */}
00119     status = H5Dwrite (dset, H5T\_NATIVE\_DOUBLE, H5S\_ALL, H5S\_ALL, H5P\_DEFAULT,
00120                 wdata[0]);
00121 
00122     \textcolor{comment}{/*}
00123 \textcolor{comment}{     * Close and release resources.}
00124 \textcolor{comment}{     */}
00125     status = H5Pclose (dcpl);
00126     status = H5Dclose (dset);
00127     status = H5Sclose (space);
00128     status = H5Fclose (file);
00129 
00130 
00131     \textcolor{comment}{/*}
00132 \textcolor{comment}{     * Now we begin the read section of this example.}
00133 \textcolor{comment}{     */}
00134 
00135     \textcolor{comment}{/*}
00136 \textcolor{comment}{     * Open file and dataset using the default properties.}
00137 \textcolor{comment}{     */}
00138     file = H5Fopen (FILE, H5F\_ACC\_RDONLY, H5P\_DEFAULT);
00139     dset = H5Dopen (file, DATASET, H5P\_DEFAULT);
00140 
00141     \textcolor{comment}{/*}
00142 \textcolor{comment}{     * Retrieve dataset creation property list.}
00143 \textcolor{comment}{     */}
00144     dcpl = H5Dget\_create\_plist (dset);
00145 
00146     \textcolor{comment}{/*}
00147 \textcolor{comment}{     * Retrieve and print the filter type.  Here we only retrieve the}
00148 \textcolor{comment}{     * first filter because we know that we only added one filter.}
00149 \textcolor{comment}{     */}
00150     nelmts = 0;
00151     filter\_type = H5Pget\_filter (dcpl, 0, &flags, &nelmts, NULL, 0, NULL, &filter\_info);
00152     printf (\textcolor{stringliteral}{"Filter type is: "});
00153     \textcolor{keywordflow}{switch} (filter\_type) \{
00154         \textcolor{keywordflow}{case} H5Z\_FILTER\_DEFLATE:
00155             printf (\textcolor{stringliteral}{"H5Z\_FILTER\_DEFLATE\(\backslash\)n"});
00156             \textcolor{keywordflow}{break};
00157         \textcolor{keywordflow}{case} H5Z\_FILTER\_SHUFFLE:
00158             printf (\textcolor{stringliteral}{"H5Z\_FILTER\_SHUFFLE\(\backslash\)n"});
00159             \textcolor{keywordflow}{break};
00160         \textcolor{keywordflow}{case} H5Z\_FILTER\_FLETCHER32:
00161             printf (\textcolor{stringliteral}{"H5Z\_FILTER\_FLETCHER32\(\backslash\)n"});
00162             \textcolor{keywordflow}{break};
00163         \textcolor{keywordflow}{case} H5Z\_FILTER\_SZIP:
00164             printf (\textcolor{stringliteral}{"H5Z\_FILTER\_SZIP\(\backslash\)n"});
00165             \textcolor{keywordflow}{break};
00166         \textcolor{keywordflow}{case} H5Z\_FILTER\_NBIT:
00167             printf (\textcolor{stringliteral}{"H5Z\_FILTER\_NBIT\(\backslash\)n"});
00168             \textcolor{keywordflow}{break};
00169         \textcolor{keywordflow}{case} H5Z\_FILTER\_SCALEOFFSET:
00170             printf (\textcolor{stringliteral}{"H5Z\_FILTER\_SCALEOFFSET\(\backslash\)n"});
00171     \}
00172 
00173     \textcolor{comment}{/*}
00174 \textcolor{comment}{     * Read the data using the default properties.}
00175 \textcolor{comment}{     */}
00176     status = H5Dread (dset, H5T\_NATIVE\_DOUBLE, H5S\_ALL, H5S\_ALL, H5P\_DEFAULT,
00177                 rdata[0]);
00178 
00179     \textcolor{comment}{/*}
00180 \textcolor{comment}{     * Find the maximum value in the dataset, to verify that it was}
00181 \textcolor{comment}{     * read correctly.}
00182 \textcolor{comment}{     */}
00183     max = rdata[0][0];
00184     min = rdata[0][0];
00185     \textcolor{keywordflow}{for} (i=0; i<DIM0; i++)
00186         \textcolor{keywordflow}{for} (j=0; j<DIM1; j++) \{
00187             \textcolor{keywordflow}{if} (max < rdata[i][j])
00188                 max = rdata[i][j];
00189             \textcolor{keywordflow}{if} (min > rdata[i][j])
00190                 min = rdata[i][j];
00191         \}
00192 
00193     \textcolor{comment}{/*}
00194 \textcolor{comment}{     * Print the maximum value.}
00195 \textcolor{comment}{     */}
00196     printf (\textcolor{stringliteral}{"Maximum value in %s is: %f\(\backslash\)n"}, DATASET, max);
00197     printf (\textcolor{stringliteral}{"Minimum value in %s is: %f\(\backslash\)n"}, DATASET, min);
00198 
00199     \textcolor{comment}{/*}
00200 \textcolor{comment}{     * Close and release resources.}
00201 \textcolor{comment}{     */}
00202     status = H5Pclose (dcpl);
00203     status = H5Dclose (dset);
00204     status = H5Fclose (file);
00205 
00206     \textcolor{keywordflow}{return} 0;
00207 \}
\end{DoxyCode}
