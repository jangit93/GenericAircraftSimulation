\hypertarget{_h_d_f5_21_810_81_2_h_d_f5_examples_2_f_o_r_t_r_a_n_2_h5_t_2h5ex__t__regref___f03_8f90_source}{}\section{H\+D\+F5/1.10.1/\+H\+D\+F5\+Examples/\+F\+O\+R\+T\+R\+A\+N/\+H5\+T/h5ex\+\_\+t\+\_\+regref\+\_\+\+F03.f90}
\label{_h_d_f5_21_810_81_2_h_d_f5_examples_2_f_o_r_t_r_a_n_2_h5_t_2h5ex__t__regref___f03_8f90_source}\index{h5ex\+\_\+t\+\_\+regref\+\_\+\+F03.\+f90@{h5ex\+\_\+t\+\_\+regref\+\_\+\+F03.\+f90}}

\begin{DoxyCode}
00001 \textcolor{comment}{!************************************************************}
00002 \textcolor{comment}{!}
00003 \textcolor{comment}{!  This example shows how to read and write region references}
00004 \textcolor{comment}{!  to a dataset.  The program first creates a dataset}
00005 \textcolor{comment}{!  containing characters and writes references to region of}
00006 \textcolor{comment}{!  the dataset to a new dataset with a dataspace of DIM0,}
00007 \textcolor{comment}{!  then closes the file.  Next, it reopens the file,}
00008 \textcolor{comment}{!  dereferences the references, and outputs the referenced}
00009 \textcolor{comment}{!  regions to the screen.}
00010 \textcolor{comment}{!}
00011 \textcolor{comment}{!  This file is intended for use with HDF5 Library verion 1.8}
00012 \textcolor{comment}{!  with --enable-fortran2003 }
00013 \textcolor{comment}{!}
00014 \textcolor{comment}{!************************************************************}
00015 \textcolor{keyword}{PROGRAM} main
00016 
00017   \textcolor{keywordtype}{USE }hdf5
00018   \textcolor{keywordtype}{use }iso\_c\_binding
00019 
00020   \textcolor{keywordtype}{IMPLICIT NONE}
00021 
00022   \textcolor{keywordtype}{CHARACTER(LEN=22)}, \textcolor{keywordtype}{PARAMETER} :: filename  = \textcolor{stringliteral}{"h5ex\_t\_regref\_F03.h5"}
00023   \textcolor{keywordtype}{CHARACTER(LEN=3)} , \textcolor{keywordtype}{PARAMETER} :: dataset   = \textcolor{stringliteral}{"DS1"}
00024   \textcolor{keywordtype}{CHARACTER(LEN=3)} , \textcolor{keywordtype}{PARAMETER} :: dataset2  = \textcolor{stringliteral}{"DS2"}
00025   \textcolor{keywordtype}{INTEGER}          , \textcolor{keywordtype}{PARAMETER} :: dim0      = 2
00026   \textcolor{keywordtype}{INTEGER}          , \textcolor{keywordtype}{PARAMETER} :: ds2dim0   = 16
00027   \textcolor{keywordtype}{INTEGER}          , \textcolor{keywordtype}{PARAMETER} :: ds2dim1   = 3
00028 
00029   \textcolor{keywordtype}{INTEGER(HID\_T)}  :: \hyperlink{structfile}{file}, memspace, space, dset, dset2 \textcolor{comment}{! Handles}
00030   \textcolor{keywordtype}{INTEGER} :: hdferr
00031 
00032   \textcolor{keywordtype}{INTEGER(HSIZE\_T)}, \textcolor{keywordtype}{DIMENSION(1:1)}   :: dims = (/dim0/)
00033   \textcolor{keywordtype}{INTEGER(HSIZE\_T)}, \textcolor{keywordtype}{DIMENSION(1:1)}   :: dims3 
00034   \textcolor{keywordtype}{INTEGER(HSIZE\_T)}, \textcolor{keywordtype}{DIMENSION(1:2)}   :: dims2 = (/ds2dim0,ds2dim1/)
00035 
00036   \textcolor{keywordtype}{INTEGER(HSIZE\_T)}, \textcolor{keywordtype}{DIMENSION(1:2,1:4)} :: coords = reshape((/2,1,12,3,1,2,5,3/),(/2,4/))
00037   
00038   \textcolor{keywordtype}{INTEGER(HSIZE\_T)}, \textcolor{keywordtype}{DIMENSION(1:2)} :: start=(/0,0/),stride=(/11,2/),count=(/2,2/), block=(/3,1/)
00039 
00040   \textcolor{keywordtype}{INTEGER(HSIZE\_T)}, \textcolor{keywordtype}{DIMENSION(1:1)} :: maxdims
00041   \textcolor{keywordtype}{INTEGER(hssize\_t)} :: npoints
00042   \textcolor{keywordtype}{TYPE}(hdset\_reg\_ref\_t\_f), \textcolor{keywordtype}{DIMENSION(1:dim0)}, \textcolor{keywordtype}{TARGET} :: wdata \textcolor{comment}{! Write buffer}
00043   \textcolor{keywordtype}{TYPE}(hdset\_reg\_ref\_t\_f), \textcolor{keywordtype}{DIMENSION(:)}, \textcolor{keywordtype}{ALLOCATABLE}, \textcolor{keywordtype}{TARGET} :: rdata \textcolor{comment}{! Read buffer}
00044 
00045   \textcolor{keywordtype}{INTEGER(size\_t)} :: size
00046   \textcolor{keywordtype}{CHARACTER(LEN=1)}, \textcolor{keywordtype}{DIMENSION(1:ds2dim0,1:ds2dim1)}, \textcolor{keywordtype}{TARGET} :: wdata2
00047 
00048   \textcolor{keywordtype}{CHARACTER(LEN=80)},\textcolor{keywordtype}{DIMENSION(1:1)}, \textcolor{keywordtype}{TARGET} :: rdata2
00049   \textcolor{keywordtype}{CHARACTER(LEN=80)} :: name
00050   \textcolor{keywordtype}{INTEGER} :: i
00051   \textcolor{keywordtype}{TYPE}(c\_ptr) :: f\_ptr
00052   \textcolor{keywordtype}{CHARACTER(LEN=ds2dim0)} :: chrvar
00053   \textcolor{comment}{!}
00054   \textcolor{comment}{! Initialize FORTRAN interface.}
00055   \textcolor{comment}{!}
00056   \textcolor{keyword}{CALL }h5open\_f(hdferr)
00057 
00058   chrvar = \textcolor{stringliteral}{"The quick brown "}
00059   \textcolor{keyword}{READ}(chrvar,\textcolor{stringliteral}{'(16A1)'}) wdata2(1:16,1)
00060   chrvar = \textcolor{stringliteral}{"fox jumps over  "}
00061   \textcolor{keyword}{READ}(chrvar,\textcolor{stringliteral}{'(16A1)'}) wdata2(1:16,2)
00062   chrvar = \textcolor{stringliteral}{"the 5 lazy dogs "}
00063   \textcolor{keyword}{READ}(chrvar,\textcolor{stringliteral}{'(16A1)'}) wdata2(1:16,3)
00064   \textcolor{comment}{!}
00065   \textcolor{comment}{! Create a new file using the default properties.}
00066   \textcolor{comment}{!}
00067   \textcolor{keyword}{CALL }h5fcreate\_f(filename, h5f\_acc\_trunc\_f, \hyperlink{structfile}{file}, hdferr)
00068   \textcolor{comment}{!}
00069   \textcolor{comment}{! Create a dataset with character data.}
00070   \textcolor{comment}{!}
00071   \textcolor{keyword}{CALL }h5screate\_simple\_f(2, dims2, space, hdferr)
00072   \textcolor{keyword}{CALL }h5dcreate\_f(\hyperlink{structfile}{file},dataset2, h5t\_std\_i8le, space, dset2, hdferr)
00073   f\_ptr = c\_loc(wdata2(1,1))
00074   \textcolor{keyword}{CALL }h5dwrite\_f(dset2, h5kind\_to\_type(kind(wdata2(1,1)), h5\_integer\_kind), f\_ptr, hdferr)
00075   \textcolor{comment}{!}
00076   \textcolor{comment}{! Create reference to a list of elements in dset2.}
00077   \textcolor{comment}{!}
00078   \textcolor{keyword}{CALL }h5sselect\_elements\_f(space, h5s\_select\_set\_f, 2, int(4,size\_t), coords, hdferr)
00079   f\_ptr = c\_loc(wdata(1))
00080   \textcolor{keyword}{CALL }h5rcreate\_f(\hyperlink{structfile}{file}, dataset2, h5r\_dataset\_region\_f, f\_ptr, hdferr, space)
00081   \textcolor{comment}{!}
00082   \textcolor{comment}{! Create reference to a hyperslab in dset2, close dataspace.}
00083   \textcolor{comment}{!}
00084   \textcolor{keyword}{CALL }h5sselect\_hyperslab\_f (space, h5s\_select\_set\_f, start, count, hdferr, stride, block)
00085   f\_ptr = c\_loc(wdata(2))
00086   \textcolor{keyword}{CALL }h5rcreate\_f(\hyperlink{structfile}{file}, dataset2, h5r\_dataset\_region\_f, f\_ptr, hdferr, space)
00087 
00088   \textcolor{keyword}{CALL }h5sclose\_f(space, hdferr)
00089   \textcolor{comment}{!}
00090   \textcolor{comment}{! Create dataspace.  Setting maximum size to the current size.}
00091   \textcolor{comment}{!}
00092   \textcolor{keyword}{CALL }h5screate\_simple\_f(1, dims, space, hdferr)
00093 
00094   \textcolor{comment}{!}
00095   \textcolor{comment}{! Create the dataset and write the region references to it.}
00096   \textcolor{comment}{!}
00097   \textcolor{keyword}{CALL }h5dcreate\_f(\hyperlink{structfile}{file}, dataset, h5t\_std\_ref\_dsetreg, space, dset, hdferr)
00098   f\_ptr = c\_loc(wdata(1))
00099   \textcolor{keyword}{CALL }h5dwrite\_f(dset, h5t\_std\_ref\_dsetreg, f\_ptr, hdferr)
00100   \textcolor{comment}{!}
00101   \textcolor{comment}{! Close and release resources.}
00102   \textcolor{comment}{!}
00103   \textcolor{keyword}{CALL }h5dclose\_f(dset , hdferr)
00104   \textcolor{keyword}{CALL }h5dclose\_f(dset2, hdferr)
00105   \textcolor{keyword}{CALL }h5sclose\_f(space, hdferr)
00106   \textcolor{keyword}{CALL }h5fclose\_f(\hyperlink{structfile}{file} , hdferr)
00107 
00108   \textcolor{comment}{!}
00109   \textcolor{comment}{! Now we begin the read section of this example.}
00110   \textcolor{comment}{!}
00111   \textcolor{comment}{!}
00112   \textcolor{comment}{! Open file and dataset.}
00113   \textcolor{comment}{!}
00114   \textcolor{keyword}{CALL }h5fopen\_f(filename, h5f\_acc\_rdonly\_f, \hyperlink{structfile}{file}, hdferr)
00115   \textcolor{keyword}{CALL }h5dopen\_f(\hyperlink{structfile}{file}, dataset, dset, hdferr)
00116   \textcolor{comment}{!}
00117   \textcolor{comment}{! Get dataspace and allocate memory for read buffer.}
00118   \textcolor{comment}{!}
00119   \textcolor{keyword}{CALL }h5dget\_space\_f(dset, space, hdferr)
00120   \textcolor{keyword}{CALL }h5sget\_simple\_extent\_dims\_f(space, dims, maxdims, hdferr)
00121   \textcolor{keyword}{ALLOCATE}(rdata(1:dims(1)))
00122   \textcolor{keyword}{CALL }h5sclose\_f(space, hdferr)
00123   \textcolor{comment}{!}
00124   \textcolor{comment}{! Read the data.}
00125   \textcolor{comment}{!}
00126   f\_ptr = c\_loc(rdata(1))
00127   \textcolor{keyword}{CALL }h5dread\_f( dset, h5t\_std\_ref\_dsetreg, f\_ptr, hdferr)
00128   \textcolor{comment}{!}
00129   \textcolor{comment}{! Output the data to the screen.}
00130   \textcolor{comment}{!}
00131   \textcolor{keywordflow}{DO} i = 1, dims(1)
00132      
00133      \textcolor{keyword}{WRITE}(*,\textcolor{stringliteral}{'(A,"[",i1,"]:",/,2X,"->")'}, advance=\textcolor{stringliteral}{'NO'}) dataset, i-1
00134      \textcolor{comment}{!}
00135      \textcolor{comment}{! Open the referenced object, retrieve its region as a}
00136      \textcolor{comment}{! dataspace selection.}
00137      \textcolor{comment}{!}
00138      \textcolor{keyword}{CALL }h5rdereference\_f(dset, rdata(i), dset2, hdferr)
00139      \textcolor{keyword}{CALL }h5rget\_region\_f(dset, rdata(i), space, hdferr)
00140   
00141      \textcolor{comment}{!}
00142      \textcolor{comment}{! Get the length of the object's name, allocate space, then}
00143      \textcolor{comment}{! retrieve the name.}
00144      \textcolor{comment}{!}
00145      \textcolor{keyword}{CALL }h5iget\_name\_f(dset2, name, 80\_size\_t, \textcolor{keyword}{size}, hdferr)
00146      \textcolor{comment}{!}
00147      \textcolor{comment}{! Allocate space for the read buffer.}
00148      \textcolor{comment}{!}
00149      \textcolor{keyword}{CALL }h5sget\_select\_npoints\_f(space, npoints, hdferr)
00150      dims3(1) = npoints
00151      \textcolor{comment}{!}
00152      \textcolor{comment}{! Read the dataset region.}
00153      \textcolor{comment}{!}
00154      \textcolor{keyword}{CALL }h5screate\_simple\_f(1, dims3, memspace, hdferr)
00155 
00156      f\_ptr = c\_loc(rdata2(1))
00157      \textcolor{keyword}{CALL }h5dread\_f( dset2, h5kind\_to\_type(kind(rdata2(1)), h5\_integer\_kind), f\_ptr, hdferr, memspace, 
      space)
00158      \textcolor{comment}{!}
00159      \textcolor{comment}{! Print the name and region data, close and release resources.}
00160      \textcolor{comment}{!}
00161      \textcolor{keyword}{WRITE}(*,\textcolor{stringliteral}{'(A,": ",A)'}) name(1:size),rdata2(1)(1:npoints) 
00162 
00163      \textcolor{keyword}{CALL }h5sclose\_f(space, hdferr)
00164      \textcolor{keyword}{CALL }h5sclose\_f(memspace, hdferr)
00165      \textcolor{keyword}{CALL }h5dclose\_f(dset2, hdferr)
00166 
00167 \textcolor{keywordflow}{  END DO}
00168   \textcolor{comment}{!}
00169   \textcolor{comment}{! Close and release resources.}
00170   \textcolor{comment}{!}
00171   \textcolor{keyword}{DEALLOCATE}(rdata)
00172   \textcolor{keyword}{CALL }h5dclose\_f(dset, hdferr)
00173   \textcolor{keyword}{CALL }h5fclose\_f(\hyperlink{structfile}{file}, hdferr)
00174 
00175 \textcolor{keyword}{END PROGRAM }main
\end{DoxyCode}
