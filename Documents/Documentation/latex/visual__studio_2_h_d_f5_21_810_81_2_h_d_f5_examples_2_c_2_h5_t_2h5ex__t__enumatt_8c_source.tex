\hypertarget{visual__studio_2_h_d_f5_21_810_81_2_h_d_f5_examples_2_c_2_h5_t_2h5ex__t__enumatt_8c_source}{}\section{visual\+\_\+studio/\+H\+D\+F5/1.10.1/\+H\+D\+F5\+Examples/\+C/\+H5\+T/h5ex\+\_\+t\+\_\+enumatt.c}
\label{visual__studio_2_h_d_f5_21_810_81_2_h_d_f5_examples_2_c_2_h5_t_2h5ex__t__enumatt_8c_source}\index{h5ex\+\_\+t\+\_\+enumatt.\+c@{h5ex\+\_\+t\+\_\+enumatt.\+c}}

\begin{DoxyCode}
00001 \textcolor{comment}{/************************************************************}
00002 \textcolor{comment}{}
00003 \textcolor{comment}{  This example shows how to read and write enumerated}
00004 \textcolor{comment}{  datatypes to an attribute.  The program first writes}
00005 \textcolor{comment}{  enumerated values to an attribute with a dataspace of}
00006 \textcolor{comment}{  DIM0xDIM1, then closes the file.  Next, it reopens the}
00007 \textcolor{comment}{  file, reads back the data, and outputs it to the screen.}
00008 \textcolor{comment}{}
00009 \textcolor{comment}{  This file is intended for use with HDF5 Library version 1.8}
00010 \textcolor{comment}{}
00011 \textcolor{comment}{ ************************************************************/}
00012 
00013 \textcolor{preprocessor}{#include "hdf5.h"}
00014 \textcolor{preprocessor}{#include <stdio.h>}
00015 \textcolor{preprocessor}{#include <stdlib.h>}
00016 
00017 \textcolor{preprocessor}{#define FILE            "h5ex\_t\_enumatt.h5"}
00018 \textcolor{preprocessor}{#define DATASET         "DS1"}
00019 \textcolor{preprocessor}{#define ATTRIBUTE       "A1"}
00020 \textcolor{preprocessor}{#define DIM0            4}
00021 \textcolor{preprocessor}{#define DIM1            7}
00022 \textcolor{preprocessor}{#define F\_BASET         H5T\_STD\_I16BE       }\textcolor{comment}{/* File base type */}\textcolor{preprocessor}{}
00023 \textcolor{preprocessor}{#define M\_BASET         H5T\_NATIVE\_INT      }\textcolor{comment}{/* Memory base type */}\textcolor{preprocessor}{}
00024 \textcolor{preprocessor}{#define NAME\_BUF\_SIZE   16}
00025 
00026 \textcolor{keyword}{typedef} \textcolor{keyword}{enum} \{
00027     SOLID,
00028     LIQUID,
00029     GAS,
00030     PLASMA
00031 \} phase\_t;                                  \textcolor{comment}{/* Enumerated type */}
00032 
00033 \textcolor{keywordtype}{int}
00034 main (\textcolor{keywordtype}{void})
00035 \{
00036     hid\_t       \hyperlink{structfile}{file}, filetype, memtype, space, dset, attr;
00037                                             \textcolor{comment}{/* Handles */}
00038     herr\_t      status;
00039     hsize\_t     dims[2] = \{DIM0, DIM1\};
00040     phase\_t     wdata[DIM0][DIM1],          \textcolor{comment}{/* Write buffer */}
00041                 **rdata,                    \textcolor{comment}{/* Read buffer */}
00042                 val;
00043     \textcolor{keywordtype}{char}        *names[4] = \{\textcolor{stringliteral}{"SOLID"}, \textcolor{stringliteral}{"LIQUID"}, \textcolor{stringliteral}{"GAS"}, \textcolor{stringliteral}{"PLASMA"}\},
00044                 name[NAME\_BUF\_SIZE];
00045     \textcolor{keywordtype}{int}         ndims,
00046                 i, j;
00047 
00048     \textcolor{comment}{/*}
00049 \textcolor{comment}{     * Initialize data.}
00050 \textcolor{comment}{     */}
00051     \textcolor{keywordflow}{for} (i=0; i<DIM0; i++)
00052         \textcolor{keywordflow}{for} (j=0; j<DIM1; j++)
00053             wdata[i][j] = (phase\_t) ( (i + 1) * j - j) % (int) (PLASMA + 1);
00054 
00055     \textcolor{comment}{/*}
00056 \textcolor{comment}{     * Create a new file using the default properties.}
00057 \textcolor{comment}{     */}
00058     file = H5Fcreate (FILE, H5F\_ACC\_TRUNC, H5P\_DEFAULT, H5P\_DEFAULT);
00059 
00060     \textcolor{comment}{/*}
00061 \textcolor{comment}{     * Create the enumerated datatypes for file and memory.  This}
00062 \textcolor{comment}{     * process is simplified if native types are used for the file,}
00063 \textcolor{comment}{     * as only one type must be defined.}
00064 \textcolor{comment}{     */}
00065     filetype = H5Tenum\_create (F\_BASET);
00066     memtype = H5Tenum\_create (M\_BASET);
00067 
00068     \textcolor{keywordflow}{for} (i = (\textcolor{keywordtype}{int}) SOLID; i <= (int) PLASMA; i++) \{
00069         \textcolor{comment}{/*}
00070 \textcolor{comment}{         * Insert enumerated value for memtype.}
00071 \textcolor{comment}{         */}
00072         val = (phase\_t) i;
00073         status = H5Tenum\_insert (memtype, names[i], &val);
00074         \textcolor{comment}{/*}
00075 \textcolor{comment}{         * Insert enumerated value for filetype.  We must first convert}
00076 \textcolor{comment}{         * the numerical value val to the base type of the destination.}
00077 \textcolor{comment}{         */}
00078         status = H5Tconvert (M\_BASET, F\_BASET, 1, &val, NULL, H5P\_DEFAULT);
00079         status = H5Tenum\_insert (filetype, names[i], &val);
00080     \}
00081 
00082     \textcolor{comment}{/*}
00083 \textcolor{comment}{     * Create dataset with a null dataspace.}
00084 \textcolor{comment}{     */}
00085     space = H5Screate (H5S\_NULL);
00086     dset = H5Dcreate (file, DATASET, H5T\_STD\_I32LE, space, H5P\_DEFAULT,
00087                 H5P\_DEFAULT, H5P\_DEFAULT);
00088     status = H5Sclose (space);
00089 
00090     \textcolor{comment}{/*}
00091 \textcolor{comment}{     * Create dataspace.  Setting maximum size to NULL sets the maximum}
00092 \textcolor{comment}{     * size to be the current size.}
00093 \textcolor{comment}{     */}
00094     space = H5Screate\_simple (2, dims, NULL);
00095 
00096     \textcolor{comment}{/*}
00097 \textcolor{comment}{     * Create the attribute and write the enumerated data to it.}
00098 \textcolor{comment}{     */}
00099     attr = H5Acreate (dset, ATTRIBUTE, filetype, space, H5P\_DEFAULT,
00100                 H5P\_DEFAULT);
00101     status = H5Awrite (attr, memtype, wdata[0]);
00102 
00103     \textcolor{comment}{/*}
00104 \textcolor{comment}{     * Close and release resources.}
00105 \textcolor{comment}{     */}
00106     status = H5Aclose (attr);
00107     status = H5Dclose (dset);
00108     status = H5Sclose (space);
00109     status = H5Tclose (filetype);
00110     status = H5Fclose (file);
00111 
00112 
00113     \textcolor{comment}{/*}
00114 \textcolor{comment}{     * Now we begin the read section of this example.  Here we assume}
00115 \textcolor{comment}{     * the attribute has the same name and rank, but can have any size.}
00116 \textcolor{comment}{     * Therefore we must allocate a new array to read in data using}
00117 \textcolor{comment}{     * malloc().  For simplicity, we do not rebuild memtype.}
00118 \textcolor{comment}{     */}
00119 
00120     \textcolor{comment}{/*}
00121 \textcolor{comment}{     * Open file, dataset, and attribute.}
00122 \textcolor{comment}{     */}
00123     file = H5Fopen (FILE, H5F\_ACC\_RDONLY, H5P\_DEFAULT);
00124     dset = H5Dopen (file, DATASET, H5P\_DEFAULT);
00125     attr = H5Aopen (dset, ATTRIBUTE, H5P\_DEFAULT);
00126 
00127     \textcolor{comment}{/*}
00128 \textcolor{comment}{     * Get dataspace and allocate memory for read buffer.  This is a}
00129 \textcolor{comment}{     * two dimensional attribute so the dynamic allocation must be done}
00130 \textcolor{comment}{     * in steps.}
00131 \textcolor{comment}{     */}
00132     space = H5Aget\_space (attr);
00133     ndims = H5Sget\_simple\_extent\_dims (space, dims, NULL);
00134 
00135     \textcolor{comment}{/*}
00136 \textcolor{comment}{     * Allocate array of pointers to rows.}
00137 \textcolor{comment}{     */}
00138     rdata = (phase\_t **) malloc (dims[0] * \textcolor{keyword}{sizeof} (phase\_t *));
00139 
00140     \textcolor{comment}{/*}
00141 \textcolor{comment}{     * Allocate space for enumerated data.}
00142 \textcolor{comment}{     */}
00143     rdata[0] = (phase\_t *) malloc (dims[0] * dims[1] * \textcolor{keyword}{sizeof} (phase\_t));
00144 
00145     \textcolor{comment}{/*}
00146 \textcolor{comment}{     * Set the rest of the pointers to rows to the correct addresses.}
00147 \textcolor{comment}{     */}
00148     \textcolor{keywordflow}{for} (i=1; i<dims[0]; i++)
00149         rdata[i] = rdata[0] + i * dims[1];
00150 
00151     \textcolor{comment}{/*}
00152 \textcolor{comment}{     * Read the data.}
00153 \textcolor{comment}{     */}
00154     status = H5Aread (attr, memtype, rdata[0]);
00155 
00156     \textcolor{comment}{/*}
00157 \textcolor{comment}{     * Output the data to the screen.}
00158 \textcolor{comment}{     */}
00159     printf (\textcolor{stringliteral}{"%s:\(\backslash\)n"}, ATTRIBUTE);
00160     \textcolor{keywordflow}{for} (i=0; i<dims[0]; i++) \{
00161         printf (\textcolor{stringliteral}{" ["});
00162         \textcolor{keywordflow}{for} (j=0; j<dims[1]; j++) \{
00163 
00164             \textcolor{comment}{/*}
00165 \textcolor{comment}{             * Get the name of the enumeration member.}
00166 \textcolor{comment}{             */}
00167             status = H5Tenum\_nameof (memtype, &rdata[i][j], name,
00168                         NAME\_BUF\_SIZE);
00169             printf (\textcolor{stringliteral}{" %-6s"}, name);
00170         \}
00171         printf (\textcolor{stringliteral}{"]\(\backslash\)n"});
00172     \}
00173 
00174     \textcolor{comment}{/*}
00175 \textcolor{comment}{     * Close and release resources.}
00176 \textcolor{comment}{     */}
00177     free (rdata[0]);
00178     free (rdata);
00179     status = H5Aclose (attr);
00180     status = H5Dclose (dset);
00181     status = H5Sclose (space);
00182     status = H5Tclose (memtype);
00183     status = H5Fclose (file);
00184 
00185     \textcolor{keywordflow}{return} 0;
00186 \}
\end{DoxyCode}
