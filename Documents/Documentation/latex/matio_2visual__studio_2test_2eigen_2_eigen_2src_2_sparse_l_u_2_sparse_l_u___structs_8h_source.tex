\hypertarget{matio_2visual__studio_2test_2eigen_2_eigen_2src_2_sparse_l_u_2_sparse_l_u___structs_8h_source}{}\section{matio/visual\+\_\+studio/test/eigen/\+Eigen/src/\+Sparse\+L\+U/\+Sparse\+L\+U\+\_\+\+Structs.h}
\label{matio_2visual__studio_2test_2eigen_2_eigen_2src_2_sparse_l_u_2_sparse_l_u___structs_8h_source}\index{Sparse\+L\+U\+\_\+\+Structs.\+h@{Sparse\+L\+U\+\_\+\+Structs.\+h}}

\begin{DoxyCode}
00001 \textcolor{comment}{// This file is part of Eigen, a lightweight C++ template library}
00002 \textcolor{comment}{// for linear algebra.}
00003 \textcolor{comment}{//}
00004 \textcolor{comment}{// Copyright (C) 2012 Désiré Nuentsa-Wakam <desire.nuentsa\_wakam@inria.fr>}
00005 \textcolor{comment}{//}
00006 \textcolor{comment}{// This Source Code Form is subject to the terms of the Mozilla}
00007 \textcolor{comment}{// Public License v. 2.0. If a copy of the MPL was not distributed}
00008 \textcolor{comment}{// with this file, You can obtain one at http://mozilla.org/MPL/2.0/.}
00009 
00010 \textcolor{comment}{/* }
00011 \textcolor{comment}{ * NOTE: This file comes from a partly modified version of files slu\_[s,d,c,z]defs.h}
00012 \textcolor{comment}{ * -- SuperLU routine (version 4.1) --}
00013 \textcolor{comment}{ * Univ. of California Berkeley, Xerox Palo Alto Research Center,}
00014 \textcolor{comment}{ * and Lawrence Berkeley National Lab.}
00015 \textcolor{comment}{ * November, 2010}
00016 \textcolor{comment}{ * }
00017 \textcolor{comment}{ * Global data structures used in LU factorization -}
00018 \textcolor{comment}{ * }
00019 \textcolor{comment}{ *   nsuper: #supernodes = nsuper + 1, numbered [0, nsuper].}
00020 \textcolor{comment}{ *   (xsup,supno): supno[i] is the supernode no to which i belongs;}
00021 \textcolor{comment}{ *  xsup(s) points to the beginning of the s-th supernode.}
00022 \textcolor{comment}{ *  e.g.   supno 0 1 2 2 3 3 3 4 4 4 4 4   (n=12)}
00023 \textcolor{comment}{ *          xsup 0 1 2 4 7 12}
00024 \textcolor{comment}{ *  Note: dfs will be performed on supernode rep. relative to the new }
00025 \textcolor{comment}{ *        row pivoting ordering}
00026 \textcolor{comment}{ *}
00027 \textcolor{comment}{ *   (xlsub,lsub): lsub[*] contains the compressed subscript of}
00028 \textcolor{comment}{ *  rectangular supernodes; xlsub[j] points to the starting}
00029 \textcolor{comment}{ *  location of the j-th column in lsub[*]. Note that xlsub }
00030 \textcolor{comment}{ *  is indexed by column.}
00031 \textcolor{comment}{ *  Storage: original row subscripts}
00032 \textcolor{comment}{ *}
00033 \textcolor{comment}{ *      During the course of sparse LU factorization, we also use}
00034 \textcolor{comment}{ *  (xlsub,lsub) for the purpose of symmetric pruning. For each}
00035 \textcolor{comment}{ *  supernode \{s,s+1,...,t=s+r\} with first column s and last}
00036 \textcolor{comment}{ *  column t, the subscript set}
00037 \textcolor{comment}{ *    lsub[j], j=xlsub[s], .., xlsub[s+1]-1}
00038 \textcolor{comment}{ *  is the structure of column s (i.e. structure of this supernode).}
00039 \textcolor{comment}{ *  It is used for the storage of numerical values.}
00040 \textcolor{comment}{ *  Furthermore,}
00041 \textcolor{comment}{ *    lsub[j], j=xlsub[t], .., xlsub[t+1]-1}
00042 \textcolor{comment}{ *  is the structure of the last column t of this supernode.}
00043 \textcolor{comment}{ *  It is for the purpose of symmetric pruning. Therefore, the}
00044 \textcolor{comment}{ *  structural subscripts can be rearranged without making physical}
00045 \textcolor{comment}{ *  interchanges among the numerical values.}
00046 \textcolor{comment}{ *}
00047 \textcolor{comment}{ *  However, if the supernode has only one column, then we}
00048 \textcolor{comment}{ *  only keep one set of subscripts. For any subscript interchange}
00049 \textcolor{comment}{ *  performed, similar interchange must be done on the numerical}
00050 \textcolor{comment}{ *  values.}
00051 \textcolor{comment}{ *}
00052 \textcolor{comment}{ *  The last column structures (for pruning) will be removed}
00053 \textcolor{comment}{ *  after the numercial LU factorization phase.}
00054 \textcolor{comment}{ *}
00055 \textcolor{comment}{ *   (xlusup,lusup): lusup[*] contains the numerical values of the}
00056 \textcolor{comment}{ *  rectangular supernodes; xlusup[j] points to the starting}
00057 \textcolor{comment}{ *  location of the j-th column in storage vector lusup[*]}
00058 \textcolor{comment}{ *  Note: xlusup is indexed by column.}
00059 \textcolor{comment}{ *  Each rectangular supernode is stored by column-major}
00060 \textcolor{comment}{ *  scheme, consistent with Fortran 2-dim array storage.}
00061 \textcolor{comment}{ *}
00062 \textcolor{comment}{ *   (xusub,ucol,usub): ucol[*] stores the numerical values of}
00063 \textcolor{comment}{ *  U-columns outside the rectangular supernodes. The row}
00064 \textcolor{comment}{ *  subscript of nonzero ucol[k] is stored in usub[k].}
00065 \textcolor{comment}{ *  xusub[i] points to the starting location of column i in ucol.}
00066 \textcolor{comment}{ *  Storage: new row subscripts; that is subscripts of PA.}
00067 \textcolor{comment}{ */}
00068 
00069 \textcolor{preprocessor}{#ifndef EIGEN\_LU\_STRUCTS}
00070 \textcolor{preprocessor}{#define EIGEN\_LU\_STRUCTS}
00071 \textcolor{keyword}{namespace }\hyperlink{namespace_eigen}{Eigen} \{
00072 \textcolor{keyword}{namespace }\hyperlink{namespaceinternal}{internal} \{
00073   
00074 \textcolor{keyword}{typedef} \textcolor{keyword}{enum} \{LUSUP, UCOL, LSUB, USUB, LLVL, ULVL\} MemType; 
00075 
00076 \textcolor{keyword}{template} <\textcolor{keyword}{typename} IndexVector, \textcolor{keyword}{typename} ScalarVector>
00077 \textcolor{keyword}{struct }LU\_GlobalLU\_t \{
00078   \textcolor{keyword}{typedef} \textcolor{keyword}{typename} IndexVector::Scalar StorageIndex; 
00079   IndexVector xsup; \textcolor{comment}{//First supernode column ... xsup(s) points to the beginning of the s-th supernode}
00080   IndexVector supno; \textcolor{comment}{// Supernode number corresponding to this column (column to supernode mapping)}
00081   ScalarVector  lusup; \textcolor{comment}{// nonzero values of L ordered by columns }
00082   IndexVector lsub; \textcolor{comment}{// Compressed row indices of L rectangular supernodes. }
00083   IndexVector xlusup; \textcolor{comment}{// pointers to the beginning of each column in lusup}
00084   IndexVector xlsub; \textcolor{comment}{// pointers to the beginning of each column in lsub}
00085   \hyperlink{namespace_eigen_a62e77e0933482dafde8fe197d9a2cfde}{Index}   nzlmax; \textcolor{comment}{// Current max size of lsub}
00086   \hyperlink{namespace_eigen_a62e77e0933482dafde8fe197d9a2cfde}{Index}   nzlumax; \textcolor{comment}{// Current max size of lusup}
00087   ScalarVector  ucol; \textcolor{comment}{// nonzero values of U ordered by columns }
00088   IndexVector usub; \textcolor{comment}{// row indices of U columns in ucol}
00089   IndexVector xusub; \textcolor{comment}{// Pointers to the beginning of each column of U in ucol }
00090   \hyperlink{namespace_eigen_a62e77e0933482dafde8fe197d9a2cfde}{Index}   nzumax; \textcolor{comment}{// Current max size of ucol}
00091   \hyperlink{namespace_eigen_a62e77e0933482dafde8fe197d9a2cfde}{Index}   n; \textcolor{comment}{// Number of columns in the matrix  }
00092   \hyperlink{namespace_eigen_a62e77e0933482dafde8fe197d9a2cfde}{Index}   num\_expansions; 
00093 \};
00094 
00095 \textcolor{comment}{// Values to set for performance}
00096 \textcolor{keyword}{struct }perfvalues \{
00097   \hyperlink{namespace_eigen_a62e77e0933482dafde8fe197d9a2cfde}{Index} panel\_size; \textcolor{comment}{// a panel consists of at most <panel\_size> consecutive columns}
00098   \hyperlink{namespace_eigen_a62e77e0933482dafde8fe197d9a2cfde}{Index} relax; \textcolor{comment}{// To control degree of relaxing supernodes. If the number of nodes (columns) }
00099                 \textcolor{comment}{// in a subtree of the elimination tree is less than relax, this subtree is considered }
00100                 \textcolor{comment}{// as one supernode regardless of the row structures of those columns}
00101   \hyperlink{namespace_eigen_a62e77e0933482dafde8fe197d9a2cfde}{Index} maxsuper; \textcolor{comment}{// The maximum size for a supernode in complete LU}
00102   \hyperlink{namespace_eigen_a62e77e0933482dafde8fe197d9a2cfde}{Index} rowblk; \textcolor{comment}{// The minimum row dimension for 2-D blocking to be used;}
00103   \hyperlink{namespace_eigen_a62e77e0933482dafde8fe197d9a2cfde}{Index} colblk; \textcolor{comment}{// The minimum column dimension for 2-D blocking to be used;}
00104   \hyperlink{namespace_eigen_a62e77e0933482dafde8fe197d9a2cfde}{Index} fillfactor; \textcolor{comment}{// The estimated fills factors for L and U, compared with A}
00105 \}; 
00106 
00107 \} \textcolor{comment}{// end namespace internal}
00108 
00109 \} \textcolor{comment}{// end namespace Eigen}
00110 \textcolor{preprocessor}{#endif // EIGEN\_LU\_STRUCTS}
\end{DoxyCode}
