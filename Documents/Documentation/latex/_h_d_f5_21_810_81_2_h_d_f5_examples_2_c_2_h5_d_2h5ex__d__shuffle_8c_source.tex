\hypertarget{_h_d_f5_21_810_81_2_h_d_f5_examples_2_c_2_h5_d_2h5ex__d__shuffle_8c_source}{}\section{H\+D\+F5/1.10.1/\+H\+D\+F5\+Examples/\+C/\+H5\+D/h5ex\+\_\+d\+\_\+shuffle.c}
\label{_h_d_f5_21_810_81_2_h_d_f5_examples_2_c_2_h5_d_2h5ex__d__shuffle_8c_source}\index{h5ex\+\_\+d\+\_\+shuffle.\+c@{h5ex\+\_\+d\+\_\+shuffle.\+c}}

\begin{DoxyCode}
00001 \textcolor{comment}{/************************************************************}
00002 \textcolor{comment}{}
00003 \textcolor{comment}{  This example shows how to read and write data to a dataset}
00004 \textcolor{comment}{  using the shuffle filter with gzip compression.  The}
00005 \textcolor{comment}{  program first checks if the shuffle and gzip filters are}
00006 \textcolor{comment}{  available, then if they are it writes integers to a}
00007 \textcolor{comment}{  dataset using shuffle+gzip, then closes the file.  Next,}
00008 \textcolor{comment}{  it reopens the file, reads back the data, and outputs the}
00009 \textcolor{comment}{  types of filters and the maximum value in the dataset to}
00010 \textcolor{comment}{  the screen.}
00011 \textcolor{comment}{}
00012 \textcolor{comment}{  This file is intended for use with HDF5 Library version 1.8}
00013 \textcolor{comment}{}
00014 \textcolor{comment}{ ************************************************************/}
00015 
00016 \textcolor{preprocessor}{#include "hdf5.h"}
00017 \textcolor{preprocessor}{#include <stdio.h>}
00018 \textcolor{preprocessor}{#include <stdlib.h>}
00019 
00020 \textcolor{preprocessor}{#define FILE            "h5ex\_d\_shuffle.h5"}
00021 \textcolor{preprocessor}{#define DATASET         "DS1"}
00022 \textcolor{preprocessor}{#define DIM0            32}
00023 \textcolor{preprocessor}{#define DIM1            64}
00024 \textcolor{preprocessor}{#define CHUNK0          4}
00025 \textcolor{preprocessor}{#define CHUNK1          8}
00026 
00027 \textcolor{keywordtype}{int}
00028 main (\textcolor{keywordtype}{void})
00029 \{
00030     hid\_t           \hyperlink{structfile}{file}, space, dset, dcpl;    \textcolor{comment}{/* Handles */}
00031     herr\_t          status;
00032     htri\_t          avail;
00033     H5Z\_filter\_t    filter\_type;
00034     hsize\_t         dims[2] = \{DIM0, DIM1\},
00035                     chunk[2] = \{CHUNK0, CHUNK1\};
00036     \textcolor{keywordtype}{size\_t}          nelmts;
00037     \textcolor{keywordtype}{unsigned} \textcolor{keywordtype}{int}    flags,
00038                     filter\_info;
00039     \textcolor{keywordtype}{int}             wdata[DIM0][DIM1],          \textcolor{comment}{/* Write buffer */}
00040                     rdata[DIM0][DIM1],          \textcolor{comment}{/* Read buffer */}
00041                     max,
00042                     nfilters,
00043                     i, j;
00044 
00045     \textcolor{comment}{/*}
00046 \textcolor{comment}{     * Check if gzip compression is available and can be used for both}
00047 \textcolor{comment}{     * compression and decompression.  Normally we do not perform error}
00048 \textcolor{comment}{     * checking in these examples for the sake of clarity, but in this}
00049 \textcolor{comment}{     * case we will make an exception because this filter is an}
00050 \textcolor{comment}{     * optional part of the hdf5 library.}
00051 \textcolor{comment}{     */}
00052     avail = H5Zfilter\_avail(H5Z\_FILTER\_DEFLATE);
00053     \textcolor{keywordflow}{if} (!avail) \{
00054         printf (\textcolor{stringliteral}{"gzip filter not available.\(\backslash\)n"});
00055         \textcolor{keywordflow}{return} 1;
00056     \}
00057     status = H5Zget\_filter\_info (H5Z\_FILTER\_DEFLATE, &filter\_info);
00058     \textcolor{keywordflow}{if} ( !(filter\_info & H5Z\_FILTER\_CONFIG\_ENCODE\_ENABLED) ||
00059                 !(filter\_info & H5Z\_FILTER\_CONFIG\_DECODE\_ENABLED) ) \{
00060         printf (\textcolor{stringliteral}{"gzip filter not available for encoding and decoding.\(\backslash\)n"});
00061         \textcolor{keywordflow}{return} 1;
00062     \}
00063 
00064     \textcolor{comment}{/*}
00065 \textcolor{comment}{     * Similarly, check for availability of the shuffle filter.}
00066 \textcolor{comment}{     */}
00067     avail = H5Zfilter\_avail(H5Z\_FILTER\_SHUFFLE);
00068     \textcolor{keywordflow}{if} (!avail) \{
00069         printf (\textcolor{stringliteral}{"Shuffle filter not available.\(\backslash\)n"});
00070         \textcolor{keywordflow}{return} 1;
00071     \}
00072     status = H5Zget\_filter\_info (H5Z\_FILTER\_SHUFFLE, &filter\_info);
00073     \textcolor{keywordflow}{if} ( !(filter\_info & H5Z\_FILTER\_CONFIG\_ENCODE\_ENABLED) ||
00074                 !(filter\_info & H5Z\_FILTER\_CONFIG\_DECODE\_ENABLED) ) \{
00075         printf (\textcolor{stringliteral}{"Shuffle filter not available for encoding and decoding.\(\backslash\)n"});
00076         \textcolor{keywordflow}{return} 1;
00077     \}
00078 
00079     \textcolor{comment}{/*}
00080 \textcolor{comment}{     * Initialize data.}
00081 \textcolor{comment}{     */}
00082     \textcolor{keywordflow}{for} (i=0; i<DIM0; i++)
00083         \textcolor{keywordflow}{for} (j=0; j<DIM1; j++)
00084             wdata[i][j] = i * j - j;
00085 
00086     \textcolor{comment}{/*}
00087 \textcolor{comment}{     * Create a new file using the default properties.}
00088 \textcolor{comment}{     */}
00089     file = H5Fcreate (FILE, H5F\_ACC\_TRUNC, H5P\_DEFAULT, H5P\_DEFAULT);
00090 
00091     \textcolor{comment}{/*}
00092 \textcolor{comment}{     * Create dataspace.  Setting maximum size to NULL sets the maximum}
00093 \textcolor{comment}{     * size to be the current size.}
00094 \textcolor{comment}{     */}
00095     space = H5Screate\_simple (2, dims, NULL);
00096 
00097     \textcolor{comment}{/*}
00098 \textcolor{comment}{     * Create the dataset creation property list, add the shuffle}
00099 \textcolor{comment}{     * filter and the gzip compression filter and set the chunk size.}
00100 \textcolor{comment}{     * The order in which the filters are added here is significant -}
00101 \textcolor{comment}{     * we will see much greater results when the shuffle is applied}
00102 \textcolor{comment}{     * first.  The order in which the filters are added to the property}
00103 \textcolor{comment}{     * list is the order in which they will be invoked when writing}
00104 \textcolor{comment}{     * data.}
00105 \textcolor{comment}{     */}
00106     dcpl = H5Pcreate (H5P\_DATASET\_CREATE);
00107     status = H5Pset\_shuffle (dcpl);
00108     status = H5Pset\_deflate (dcpl, 9);
00109     status = H5Pset\_chunk (dcpl, 2, chunk);
00110 
00111     \textcolor{comment}{/*}
00112 \textcolor{comment}{     * Create the dataset.}
00113 \textcolor{comment}{     */}
00114     dset = H5Dcreate (file, DATASET, H5T\_STD\_I32LE, space, H5P\_DEFAULT, dcpl,
00115                 H5P\_DEFAULT);
00116 
00117     \textcolor{comment}{/*}
00118 \textcolor{comment}{     * Write the data to the dataset.}
00119 \textcolor{comment}{     */}
00120     status = H5Dwrite (dset, H5T\_NATIVE\_INT, H5S\_ALL, H5S\_ALL, H5P\_DEFAULT,
00121                 wdata[0]);
00122 
00123     \textcolor{comment}{/*}
00124 \textcolor{comment}{     * Close and release resources.}
00125 \textcolor{comment}{     */}
00126     status = H5Pclose (dcpl);
00127     status = H5Dclose (dset);
00128     status = H5Sclose (space);
00129     status = H5Fclose (file);
00130 
00131 
00132     \textcolor{comment}{/*}
00133 \textcolor{comment}{     * Now we begin the read section of this example.}
00134 \textcolor{comment}{     */}
00135 
00136     \textcolor{comment}{/*}
00137 \textcolor{comment}{     * Open file and dataset using the default properties.}
00138 \textcolor{comment}{     */}
00139     file = H5Fopen (FILE, H5F\_ACC\_RDONLY, H5P\_DEFAULT);
00140     dset = H5Dopen (file, DATASET, H5P\_DEFAULT);
00141 
00142     \textcolor{comment}{/*}
00143 \textcolor{comment}{     * Retrieve dataset creation property list.}
00144 \textcolor{comment}{     */}
00145     dcpl = H5Dget\_create\_plist (dset);
00146 
00147     \textcolor{comment}{/*}
00148 \textcolor{comment}{     * Retrieve the number of filters, and retrieve and print the}
00149 \textcolor{comment}{     * type of each.}
00150 \textcolor{comment}{     */}
00151     nfilters = H5Pget\_nfilters (dcpl);
00152     \textcolor{keywordflow}{for} (i=0; i<nfilters; i++) \{
00153         nelmts = 0;
00154         filter\_type = H5Pget\_filter (dcpl, i, &flags, &nelmts, NULL, 0, NULL,
00155                     &filter\_info);
00156         printf (\textcolor{stringliteral}{"Filter %d: Type is: "}, i);
00157         \textcolor{keywordflow}{switch} (filter\_type) \{
00158             \textcolor{keywordflow}{case} H5Z\_FILTER\_DEFLATE:
00159                 printf (\textcolor{stringliteral}{"H5Z\_FILTER\_DEFLATE\(\backslash\)n"});
00160                 \textcolor{keywordflow}{break};
00161             \textcolor{keywordflow}{case} H5Z\_FILTER\_SHUFFLE:
00162                 printf (\textcolor{stringliteral}{"H5Z\_FILTER\_SHUFFLE\(\backslash\)n"});
00163                 \textcolor{keywordflow}{break};
00164             \textcolor{keywordflow}{case} H5Z\_FILTER\_FLETCHER32:
00165                 printf (\textcolor{stringliteral}{"H5Z\_FILTER\_FLETCHER32\(\backslash\)n"});
00166                 \textcolor{keywordflow}{break};
00167             \textcolor{keywordflow}{case} H5Z\_FILTER\_SZIP:
00168                 printf (\textcolor{stringliteral}{"H5Z\_FILTER\_SZIP\(\backslash\)n"});
00169                 \textcolor{keywordflow}{break};
00170             \textcolor{keywordflow}{case} H5Z\_FILTER\_NBIT:
00171                 printf (\textcolor{stringliteral}{"H5Z\_FILTER\_NBIT\(\backslash\)n"});
00172                 \textcolor{keywordflow}{break};
00173             \textcolor{keywordflow}{case} H5Z\_FILTER\_SCALEOFFSET:
00174                 printf (\textcolor{stringliteral}{"H5Z\_FILTER\_SCALEOFFSET\(\backslash\)n"});
00175         \}
00176     \}
00177 
00178     \textcolor{comment}{/*}
00179 \textcolor{comment}{     * Read the data using the default properties.}
00180 \textcolor{comment}{     */}
00181     status = H5Dread (dset, H5T\_NATIVE\_INT, H5S\_ALL, H5S\_ALL, H5P\_DEFAULT,
00182                 rdata[0]);
00183 
00184     \textcolor{comment}{/*}
00185 \textcolor{comment}{     * Find the maximum value in the dataset, to verify that it was}
00186 \textcolor{comment}{     * read correctly.}
00187 \textcolor{comment}{     */}
00188     max = rdata[0][0];
00189     \textcolor{keywordflow}{for} (i=0; i<DIM0; i++)
00190         \textcolor{keywordflow}{for} (j=0; j<DIM1; j++)
00191             \textcolor{keywordflow}{if} (max < rdata[i][j])
00192                 max = rdata[i][j];
00193 
00194     \textcolor{comment}{/*}
00195 \textcolor{comment}{     * Print the maximum value.}
00196 \textcolor{comment}{     */}
00197     printf (\textcolor{stringliteral}{"Maximum value in %s is: %d\(\backslash\)n"}, DATASET, max);
00198 
00199     \textcolor{comment}{/*}
00200 \textcolor{comment}{     * Close and release resources.}
00201 \textcolor{comment}{     */}
00202     status = H5Pclose (dcpl);
00203     status = H5Dclose (dset);
00204     status = H5Fclose (file);
00205 
00206     \textcolor{keywordflow}{return} 0;
00207 \}
\end{DoxyCode}
