\hypertarget{read_in_data_8cpp_source}{}\section{read\+In\+Data.\+cpp}
\label{read_in_data_8cpp_source}\index{read\+In\+Data.\+cpp@{read\+In\+Data.\+cpp}}

\begin{DoxyCode}
00001 \textcolor{preprocessor}{#include "readInData.h"}
00002 readInData::readInData()
00003 \{
00004     PathName2 = \textcolor{stringliteral}{"../Input/"};
00005 \}
00006 
00007 readInData::~readInData()
00008 \{
00009 \}
00010 
00011 Float64 readInData::readInParameter(std::string CodeWord, std::string FileName)
00012 \{
00013     
00014 
00015     \textcolor{comment}{// �ffne File}
00016     input.open(PathName2+FileName, std::ios::in);
00017 
00018     \textcolor{comment}{// Exit if file opening failed}
00019     \textcolor{keywordflow}{if} (!input.is\_open()) \{
00020         std::cerr << \textcolor{stringliteral}{"Daten von  "} << CodeWord << \textcolor{stringliteral}{" aus "} << FileName << \textcolor{stringliteral}{" konnten nicht gelesen werden"} <<
       std::endl;
00021         \textcolor{keywordflow}{return} 0;
00022         \textcolor{comment}{//exit(1);}
00023     \}
00024 
00025     \textcolor{comment}{// Durchlaufe File, solange Daten vorhanden}
00026     \textcolor{keywordflow}{while} (!input.eof())
00027     \{
00028         \textcolor{comment}{// Einlesen Zeile f�r Zeile}
00029         std::getline(input, currtenLine);
00030 
00031         \textcolor{comment}{// Suche nach Code Wort in der aktuellen Zeile}
00032         std::size\_t found = currtenLine.find(CodeWord, 0);
00033 
00034         \textcolor{comment}{// Wenn Zeile gefunden, dann Parameter extrahieren}
00035         \textcolor{keywordflow}{if} (found != std::string::npos)
00036         \{
00037 
00038             \textcolor{keywordtype}{size\_t} pos = 0;
00039             std::string delimiter = \textcolor{stringliteral}{"="};
00040             std::string token = currtenLine.substr(pos, currtenLine.find(delimiter));
00041 
00042             \textcolor{keywordflow}{while} ((currtenLine.find(delimiter)) != std::string::npos) \{
00043                 token = currtenLine.substr(0, pos);
00044                 currtenLine.erase(0, pos + delimiter.length());
00045             \}
00046 
00047             tempValue = std::stod(currtenLine.c\_str());
00048 
00049         \}
00050     \}
00051     input.close();
00052 
00053     Float64 Value = tempValue;
00054     currtenLine.clear();
00055 
00056     \textcolor{keywordflow}{return} Value;
00057 \}
00058 
00059 MatrixXd readInData::readInTable(std::string FileName)
00060 \{
00061     \textcolor{comment}{// initialisation of data}
00062     lines, columns = -1;
00063 
00064 
00065     std::ifstream input(PathName2 + FileName);
00066 
00067 
00068     \textcolor{comment}{// Exit if file opening failed}
00069     \textcolor{keywordflow}{if} (!input.is\_open()) \{
00070         std::cerr << \textcolor{stringliteral}{"Daten von  "} << FileName << \textcolor{stringliteral}{" konnten nicht gelesen werden"} << std::endl;
00071         \textcolor{keywordflow}{return} tempMatrix.setZero();
00072     \}
00073 
00074     \textcolor{comment}{//Ignore header}
00075     std::string dummyline;
00076     getline(input, dummyline);
00077 
00078     \textcolor{keywordflow}{while} (getline(input, line))
00079     \{
00080         columncounter = 0;
00081         std::stringstream lineparser(line);
00082 
00083         \textcolor{keywordflow}{while} (lineparser >> element)
00084         \{
00085 
00086             tempData.push\_back(element);
00087             ++columncounter;
00088         \}
00089         \textcolor{keywordflow}{if} (columns == -1) columns = columncounter;
00090         \textcolor{comment}{/*if (columns != columncounter)}
00091 \textcolor{comment}{        \{}
00092 \textcolor{comment}{        cerr << "Fehler: Unterschiedliche Spaltenzahl in mindestens zwei Zeilen" << endl;}
00093 \textcolor{comment}{        exit(1);}
00094 \textcolor{comment}{        \}*/}
00095     \}
00096     lines = tempData.size() / columns;
00097 
00098     \textcolor{comment}{//write data to matrix}
00099     tempMatrix.resize(lines, columns);
00100     tempMatrix.setZero(lines, columns);
00101 
00102     \textcolor{keywordflow}{if} (columns == 1) \{
00103 
00104         tempVec.resize(lines);
00105         tempVec.setZero(lines);
00106     \}
00107     \textcolor{keywordflow}{else} \{
00108         \textcolor{keywordflow}{for} (\textcolor{keywordtype}{int} i = 0; i < lines; ++i)
00109         \{
00110             \textcolor{keywordflow}{for} (\textcolor{keywordtype}{int} j = 0; j < columns; ++j)
00111             \{
00112                 tempMatrix(i, j) = tempData[i*columns + j];
00113             \}
00114         \}
00115 
00116         tempData.clear();
00117         \textcolor{keywordflow}{if} (columns == 1) \{
00118             \textcolor{keywordflow}{return} tempVec;
00119         \}
00120         \textcolor{keywordflow}{return} tempMatrix;
00121 
00122     \}
00123 \}
00124 
00125 VectorXd readInData::readInVector(std::string FileName)
00126 \{
00127     \textcolor{comment}{// initialisation of data}
00128     lines, columns = -1;
00129 
00130 
00131     std::ifstream input(PathName2 + FileName);
00132 
00133 
00134     \textcolor{comment}{// Exit if file opening failed}
00135     \textcolor{keywordflow}{if} (!input.is\_open()) \{
00136         std::cerr << \textcolor{stringliteral}{"Daten von ... konnten nicht gelesen werden"} << std::endl;
00137         exit(1);
00138     \}
00139 
00140     \textcolor{keywordflow}{while} (getline(input, line))
00141     \{
00142         columncounter = 0;
00143         std::stringstream lineparser(line);
00144 
00145         \textcolor{keywordflow}{while} (lineparser >> element)
00146         \{
00147 
00148             tempData.push\_back(element);
00149             ++columncounter;
00150         \}
00151         \textcolor{keywordflow}{if} (columns == -1) \{
00152             columns = columncounter;
00153         \}
00154 
00155 
00156     \}
00157     lines = tempData.size() / columns;
00158 
00159 
00160     tempVec.resize(lines);
00161     tempVec.setZero(lines);
00162     \textcolor{keywordflow}{for} (\textcolor{keywordtype}{int} i = 0; i < lines; ++i)
00163     \{
00164         tempVec(i) = tempData[i];
00165     \}
00166 
00167 
00168     tempData.clear();
00169 
00170     \textcolor{keywordflow}{return} tempVec;
00171 \}
00172 
00173 \textcolor{keywordtype}{void} readInData::setPath(std::string Pathname)
00174 \{
00175     PathName2 = Pathname;
00176 \}
\end{DoxyCode}
