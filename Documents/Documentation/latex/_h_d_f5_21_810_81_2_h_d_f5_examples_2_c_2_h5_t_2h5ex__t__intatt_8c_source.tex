\hypertarget{_h_d_f5_21_810_81_2_h_d_f5_examples_2_c_2_h5_t_2h5ex__t__intatt_8c_source}{}\section{H\+D\+F5/1.10.1/\+H\+D\+F5\+Examples/\+C/\+H5\+T/h5ex\+\_\+t\+\_\+intatt.c}
\label{_h_d_f5_21_810_81_2_h_d_f5_examples_2_c_2_h5_t_2h5ex__t__intatt_8c_source}\index{h5ex\+\_\+t\+\_\+intatt.\+c@{h5ex\+\_\+t\+\_\+intatt.\+c}}

\begin{DoxyCode}
00001 \textcolor{comment}{/************************************************************}
00002 \textcolor{comment}{}
00003 \textcolor{comment}{  This example shows how to read and write integer datatypes}
00004 \textcolor{comment}{  to an attribute.  The program first writes integers to an}
00005 \textcolor{comment}{  attribute with a dataspace of DIM0xDIM1, then closes the}
00006 \textcolor{comment}{  file.  Next, it reopens the file, reads back the data, and}
00007 \textcolor{comment}{  outputs it to the screen.}
00008 \textcolor{comment}{}
00009 \textcolor{comment}{  This file is intended for use with HDF5 Library version 1.8}
00010 \textcolor{comment}{}
00011 \textcolor{comment}{ ************************************************************/}
00012 
00013 \textcolor{preprocessor}{#include "hdf5.h"}
00014 \textcolor{preprocessor}{#include <stdio.h>}
00015 \textcolor{preprocessor}{#include <stdlib.h>}
00016 
00017 \textcolor{preprocessor}{#define FILE            "h5ex\_t\_intatt.h5"}
00018 \textcolor{preprocessor}{#define DATASET         "DS1"}
00019 \textcolor{preprocessor}{#define ATTRIBUTE       "A1"}
00020 \textcolor{preprocessor}{#define DIM0            4}
00021 \textcolor{preprocessor}{#define DIM1            7}
00022 
00023 \textcolor{keywordtype}{int}
00024 main (\textcolor{keywordtype}{void})
00025 \{
00026     hid\_t       \hyperlink{structfile}{file}, space, dset, attr;            \textcolor{comment}{/* Handles */}
00027     herr\_t      status;
00028     hsize\_t     dims[2] = \{DIM0, DIM1\};
00029     \textcolor{keywordtype}{int}         wdata[DIM0][DIM1],                  \textcolor{comment}{/* Write buffer */}
00030                 **rdata,                            \textcolor{comment}{/* Read buffer */}
00031                 ndims,
00032                 i, j;
00033 
00034     \textcolor{comment}{/*}
00035 \textcolor{comment}{     * Initialize data.}
00036 \textcolor{comment}{     */}
00037     \textcolor{keywordflow}{for} (i=0; i<DIM0; i++)
00038         \textcolor{keywordflow}{for} (j=0; j<DIM1; j++)
00039             wdata[i][j] = i * j - j;
00040 
00041     \textcolor{comment}{/*}
00042 \textcolor{comment}{     * Create a new file using the default properties.}
00043 \textcolor{comment}{     */}
00044     file = H5Fcreate (FILE, H5F\_ACC\_TRUNC, H5P\_DEFAULT, H5P\_DEFAULT);
00045 
00046     \textcolor{comment}{/*}
00047 \textcolor{comment}{     * Create dataset with a null dataspace.}
00048 \textcolor{comment}{     */}
00049     space = H5Screate (H5S\_NULL);
00050     dset = H5Dcreate (file, DATASET, H5T\_STD\_I32LE, space, H5P\_DEFAULT,
00051                 H5P\_DEFAULT, H5P\_DEFAULT);
00052     status = H5Sclose (space);
00053 
00054     \textcolor{comment}{/*}
00055 \textcolor{comment}{     * Create dataspace.  Setting maximum size to NULL sets the maximum}
00056 \textcolor{comment}{     * size to be the current size.}
00057 \textcolor{comment}{     */}
00058     space = H5Screate\_simple (2, dims, NULL);
00059 
00060     \textcolor{comment}{/*}
00061 \textcolor{comment}{     * Create the attribute and write the integer data to it.  In this}
00062 \textcolor{comment}{     * example we will save the data as 64 bit big endian integers,}
00063 \textcolor{comment}{     * regardless of the native integer type.  The HDF5 library}
00064 \textcolor{comment}{     * automatically converts between different integer types.}
00065 \textcolor{comment}{     */}
00066     attr = H5Acreate (dset, ATTRIBUTE, H5T\_STD\_I64BE, space, H5P\_DEFAULT,
00067                 H5P\_DEFAULT);
00068     status = H5Awrite (attr, H5T\_NATIVE\_INT, wdata[0]);
00069 
00070     \textcolor{comment}{/*}
00071 \textcolor{comment}{     * Close and release resources.}
00072 \textcolor{comment}{     */}
00073     status = H5Aclose (attr);
00074     status = H5Dclose (dset);
00075     status = H5Sclose (space);
00076     status = H5Fclose (file);
00077 
00078 
00079     \textcolor{comment}{/*}
00080 \textcolor{comment}{     * Now we begin the read section of this example.  Here we assume}
00081 \textcolor{comment}{     * the attribute has the same name and rank, but can have any size.}
00082 \textcolor{comment}{     * Therefore we must allocate a new array to read in data using}
00083 \textcolor{comment}{     * malloc().}
00084 \textcolor{comment}{     */}
00085 
00086     \textcolor{comment}{/*}
00087 \textcolor{comment}{     * Open file, dataset, and attribute.}
00088 \textcolor{comment}{     */}
00089     file = H5Fopen (FILE, H5F\_ACC\_RDONLY, H5P\_DEFAULT);
00090     dset = H5Dopen (file, DATASET, H5P\_DEFAULT);
00091     attr = H5Aopen (dset, ATTRIBUTE, H5P\_DEFAULT);
00092 
00093     \textcolor{comment}{/*}
00094 \textcolor{comment}{     * Get dataspace and allocate memory for read buffer.  This is a}
00095 \textcolor{comment}{     * two dimensional attribute so the dynamic allocation must be done}
00096 \textcolor{comment}{     * in steps.}
00097 \textcolor{comment}{     */}
00098     space = H5Aget\_space (attr);
00099     ndims = H5Sget\_simple\_extent\_dims (space, dims, NULL);
00100 
00101     \textcolor{comment}{/*}
00102 \textcolor{comment}{     * Allocate array of pointers to rows.}
00103 \textcolor{comment}{     */}
00104     rdata = (\textcolor{keywordtype}{int} **) malloc (dims[0] * \textcolor{keyword}{sizeof} (\textcolor{keywordtype}{int} *));
00105 
00106     \textcolor{comment}{/*}
00107 \textcolor{comment}{     * Allocate space for integer data.}
00108 \textcolor{comment}{     */}
00109     rdata[0] = (\textcolor{keywordtype}{int} *) malloc (dims[0] * dims[1] * \textcolor{keyword}{sizeof} (\textcolor{keywordtype}{int}));
00110 
00111     \textcolor{comment}{/*}
00112 \textcolor{comment}{     * Set the rest of the pointers to rows to the correct addresses.}
00113 \textcolor{comment}{     */}
00114     \textcolor{keywordflow}{for} (i=1; i<dims[0]; i++)
00115         rdata[i] = rdata[0] + i * dims[1];
00116 
00117     \textcolor{comment}{/*}
00118 \textcolor{comment}{     * Read the data.}
00119 \textcolor{comment}{     */}
00120     status = H5Aread (attr, H5T\_NATIVE\_INT, rdata[0]);
00121 
00122     \textcolor{comment}{/*}
00123 \textcolor{comment}{     * Output the data to the screen.}
00124 \textcolor{comment}{     */}
00125     printf (\textcolor{stringliteral}{"%s:\(\backslash\)n"}, ATTRIBUTE);
00126     \textcolor{keywordflow}{for} (i=0; i<dims[0]; i++) \{
00127         printf (\textcolor{stringliteral}{" ["});
00128         \textcolor{keywordflow}{for} (j=0; j<dims[1]; j++)
00129             printf (\textcolor{stringliteral}{" %3d"}, rdata[i][j]);
00130         printf (\textcolor{stringliteral}{"]\(\backslash\)n"});
00131     \}
00132 
00133     \textcolor{comment}{/*}
00134 \textcolor{comment}{     * Close and release resources.}
00135 \textcolor{comment}{     */}
00136     free (rdata[0]);
00137     free (rdata);
00138     status = H5Aclose (attr);
00139     status = H5Dclose (dset);
00140     status = H5Sclose (space);
00141     status = H5Fclose (file);
00142 
00143     \textcolor{keywordflow}{return} 0;
00144 \}
\end{DoxyCode}
