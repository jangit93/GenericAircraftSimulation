\hypertarget{matio_2visual__studio_2test_2eigen_2_eigen_2src_2_iterative_linear_solvers_2_incomplete_l_u_t_8h_source}{}\section{matio/visual\+\_\+studio/test/eigen/\+Eigen/src/\+Iterative\+Linear\+Solvers/\+Incomplete\+L\+UT.h}
\label{matio_2visual__studio_2test_2eigen_2_eigen_2src_2_iterative_linear_solvers_2_incomplete_l_u_t_8h_source}\index{Incomplete\+L\+U\+T.\+h@{Incomplete\+L\+U\+T.\+h}}

\begin{DoxyCode}
00001 \textcolor{comment}{// This file is part of Eigen, a lightweight C++ template library}
00002 \textcolor{comment}{// for linear algebra.}
00003 \textcolor{comment}{//}
00004 \textcolor{comment}{// Copyright (C) 2012 Désiré Nuentsa-Wakam <desire.nuentsa\_wakam@inria.fr>}
00005 \textcolor{comment}{// Copyright (C) 2014 Gael Guennebaud <gael.guennebaud@inria.fr>}
00006 \textcolor{comment}{//}
00007 \textcolor{comment}{// This Source Code Form is subject to the terms of the Mozilla}
00008 \textcolor{comment}{// Public License v. 2.0. If a copy of the MPL was not distributed}
00009 \textcolor{comment}{// with this file, You can obtain one at http://mozilla.org/MPL/2.0/.}
00010 
00011 \textcolor{preprocessor}{#ifndef EIGEN\_INCOMPLETE\_LUT\_H}
00012 \textcolor{preprocessor}{#define EIGEN\_INCOMPLETE\_LUT\_H}
00013 
00014 
00015 \textcolor{keyword}{namespace }\hyperlink{namespace_eigen}{Eigen} \{ 
00016 
00017 \textcolor{keyword}{namespace }\hyperlink{namespaceinternal}{internal} \{
00018     
00028 \textcolor{keyword}{template} <\textcolor{keyword}{typename} VectorV, \textcolor{keyword}{typename} VectorI>
00029 \hyperlink{namespace_eigen_a62e77e0933482dafde8fe197d9a2cfde}{Index} QuickSplit(VectorV &row, VectorI &\hyperlink{structind}{ind}, \hyperlink{namespace_eigen_a62e77e0933482dafde8fe197d9a2cfde}{Index} ncut)
00030 \{
00031   \textcolor{keyword}{typedef} \textcolor{keyword}{typename} VectorV::RealScalar RealScalar;
00032   \textcolor{keyword}{using} \hyperlink{endian_8c_a3ca5ecd34b04d6a243c054ac3a57f68d}{std::swap};
00033   \textcolor{keyword}{using} std::abs;
00034   \hyperlink{namespace_eigen_a62e77e0933482dafde8fe197d9a2cfde}{Index} mid;
00035   \hyperlink{namespace_eigen_a62e77e0933482dafde8fe197d9a2cfde}{Index} n = row.size(); \textcolor{comment}{/* length of the vector */}
00036   \hyperlink{namespace_eigen_a62e77e0933482dafde8fe197d9a2cfde}{Index} first, last ;
00037   
00038   ncut--; \textcolor{comment}{/* to fit the zero-based indices */}
00039   first = 0; 
00040   last = n-1; 
00041   \textcolor{keywordflow}{if} (ncut < first || ncut > last ) \textcolor{keywordflow}{return} 0;
00042   
00043   \textcolor{keywordflow}{do} \{
00044     mid = first; 
00045     RealScalar abskey = abs(row(mid)); 
00046     \textcolor{keywordflow}{for} (\hyperlink{namespace_eigen_a62e77e0933482dafde8fe197d9a2cfde}{Index} j = first + 1; j <= last; j++) \{
00047       \textcolor{keywordflow}{if} ( abs(row(j)) > abskey) \{
00048         ++mid;
00049         \hyperlink{endian_8c_a3ca5ecd34b04d6a243c054ac3a57f68d}{swap}(row(mid), row(j));
00050         \hyperlink{endian_8c_a3ca5ecd34b04d6a243c054ac3a57f68d}{swap}(ind(mid), ind(j));
00051       \}
00052     \}
00053     \textcolor{comment}{/* Interchange for the pivot element */}
00054     \hyperlink{endian_8c_a3ca5ecd34b04d6a243c054ac3a57f68d}{swap}(row(mid), row(first));
00055     \hyperlink{endian_8c_a3ca5ecd34b04d6a243c054ac3a57f68d}{swap}(ind(mid), ind(first));
00056     
00057     \textcolor{keywordflow}{if} (mid > ncut) last = mid - 1;
00058     \textcolor{keywordflow}{else} \textcolor{keywordflow}{if} (mid < ncut ) first = mid + 1; 
00059   \} \textcolor{keywordflow}{while} (mid != ncut );
00060   
00061   \textcolor{keywordflow}{return} 0; \textcolor{comment}{/* mid is equal to ncut */} 
00062 \}
00063 
00064 \}\textcolor{comment}{// end namespace internal}
00065 
00098 \textcolor{keyword}{template} <\textcolor{keyword}{typename} \_Scalar, \textcolor{keyword}{typename} \_StorageIndex = \textcolor{keywordtype}{int}>
00099 \textcolor{keyword}{class }IncompleteLUT : \textcolor{keyword}{public} SparseSolverBase<IncompleteLUT<\_Scalar, \_StorageIndex> >
00100 \{
00101   \textcolor{keyword}{protected}:
00102     \textcolor{keyword}{typedef} SparseSolverBase<IncompleteLUT> Base;
00103     \textcolor{keyword}{using} Base::m\_isInitialized;
00104   \textcolor{keyword}{public}:
00105     \textcolor{keyword}{typedef} \_Scalar Scalar;
00106     \textcolor{keyword}{typedef} \_StorageIndex StorageIndex;
00107     \textcolor{keyword}{typedef} \textcolor{keyword}{typename} NumTraits<Scalar>::Real RealScalar;
00108     \textcolor{keyword}{typedef} Matrix<Scalar,Dynamic,1> Vector;
00109     \textcolor{keyword}{typedef} Matrix<StorageIndex,Dynamic,1> VectorI;
00110     \textcolor{keyword}{typedef} SparseMatrix<Scalar,RowMajor,StorageIndex> FactorType;
00111 
00112     \textcolor{keyword}{enum} \{
00113       ColsAtCompileTime = \hyperlink{namespace_eigen_ad81fa7195215a0ce30017dfac309f0b2}{Dynamic},
00114       MaxColsAtCompileTime = \hyperlink{namespace_eigen_ad81fa7195215a0ce30017dfac309f0b2}{Dynamic}
00115     \};
00116 
00117   \textcolor{keyword}{public}:
00118     
00119     IncompleteLUT()
00120       : m\_droptol(NumTraits<Scalar>::dummy\_precision()), m\_fillfactor(10),
00121         m\_analysisIsOk(false), m\_factorizationIsOk(false)
00122     \{\}
00123     
00124     \textcolor{keyword}{template}<\textcolor{keyword}{typename} MatrixType>
00125     \textcolor{keyword}{explicit} IncompleteLUT(\textcolor{keyword}{const} MatrixType& mat, \textcolor{keyword}{const} RealScalar& droptol=
      NumTraits<Scalar>::dummy\_precision(), \textcolor{keywordtype}{int} fillfactor = 10)
00126       : m\_droptol(droptol),m\_fillfactor(fillfactor),
00127         m\_analysisIsOk(false),m\_factorizationIsOk(false)
00128     \{
00129       eigen\_assert(fillfactor != 0);
00130       compute(mat); 
00131     \}
00132     
00133     \hyperlink{namespace_eigen_a62e77e0933482dafde8fe197d9a2cfde}{Index} rows()\textcolor{keyword}{ const }\{ \textcolor{keywordflow}{return} m\_lu.rows(); \}
00134     
00135     \hyperlink{namespace_eigen_a62e77e0933482dafde8fe197d9a2cfde}{Index} cols()\textcolor{keyword}{ const }\{ \textcolor{keywordflow}{return} m\_lu.cols(); \}
00136 
\Hypertarget{matio_2visual__studio_2test_2eigen_2_eigen_2src_2_iterative_linear_solvers_2_incomplete_l_u_t_8h_source_l00142}\hyperlink{group___iterative_linear_solvers___module_a941c7d34f15d7bc287e780636be0ee2b}{00142}     \hyperlink{group__enums_ga85fad7b87587764e5cf6b513a9e0ee5e}{ComputationInfo} \hyperlink{group___iterative_linear_solvers___module_a941c7d34f15d7bc287e780636be0ee2b}{info}()\textcolor{keyword}{ const}
00143 \textcolor{keyword}{    }\{
00144       eigen\_assert(m\_isInitialized && \textcolor{stringliteral}{"IncompleteLUT is not initialized."});
00145       \textcolor{keywordflow}{return} m\_info;
00146     \}
00147     
00148     \textcolor{keyword}{template}<\textcolor{keyword}{typename} MatrixType>
00149     \textcolor{keywordtype}{void} analyzePattern(\textcolor{keyword}{const} MatrixType& amat);
00150     
00151     \textcolor{keyword}{template}<\textcolor{keyword}{typename} MatrixType>
00152     \textcolor{keywordtype}{void} factorize(\textcolor{keyword}{const} MatrixType& amat);
00153     
00159     \textcolor{keyword}{template}<\textcolor{keyword}{typename} MatrixType>
\Hypertarget{matio_2visual__studio_2test_2eigen_2_eigen_2src_2_iterative_linear_solvers_2_incomplete_l_u_t_8h_source_l00160}\hyperlink{group___iterative_linear_solvers___module_a488e37ab51d8ed37a297eeca521f1817}{00160}     \hyperlink{group___iterative_linear_solvers___module_class_eigen_1_1_incomplete_l_u_t}{IncompleteLUT}& \hyperlink{group___iterative_linear_solvers___module_a488e37ab51d8ed37a297eeca521f1817}{compute}(\textcolor{keyword}{const} MatrixType& amat)
00161     \{
00162       analyzePattern(amat); 
00163       factorize(amat);
00164       \textcolor{keywordflow}{return} *\textcolor{keyword}{this};
00165     \}
00166 
00167     \textcolor{keywordtype}{void} setDroptol(\textcolor{keyword}{const} RealScalar& droptol); 
00168     \textcolor{keywordtype}{void} setFillfactor(\textcolor{keywordtype}{int} fillfactor); 
00169     
00170     \textcolor{keyword}{template}<\textcolor{keyword}{typename} Rhs, \textcolor{keyword}{typename} Dest>
00171     \textcolor{keywordtype}{void} \_solve\_impl(\textcolor{keyword}{const} Rhs& b, Dest& x)\textcolor{keyword}{ const}
00172 \textcolor{keyword}{    }\{
00173       x = m\_Pinv * b;
00174       x = m\_lu.template triangularView<UnitLower>().solve(x);
00175       x = m\_lu.template triangularView<Upper>().solve(x);
00176       x = m\_P * x; 
00177     \}
00178 
00179 \textcolor{keyword}{protected}:
00180 
00182     \textcolor{keyword}{struct }\hyperlink{struct_eigen_1_1_incomplete_l_u_t_1_1keep__diag}{keep\_diag} \{
00183       \textcolor{keyword}{inline} \textcolor{keywordtype}{bool} operator() (\textcolor{keyword}{const} \hyperlink{namespace_eigen_a62e77e0933482dafde8fe197d9a2cfde}{Index}& row, \textcolor{keyword}{const} \hyperlink{namespace_eigen_a62e77e0933482dafde8fe197d9a2cfde}{Index}& col, \textcolor{keyword}{const} Scalar&)\textcolor{keyword}{ const}
00184 \textcolor{keyword}{      }\{
00185         \textcolor{keywordflow}{return} row!=col;
00186       \}
00187     \};
00188 
00189 \textcolor{keyword}{protected}:
00190 
00191     FactorType m\_lu;
00192     RealScalar m\_droptol;
00193     \textcolor{keywordtype}{int} m\_fillfactor;
00194     \textcolor{keywordtype}{bool} m\_analysisIsOk;
00195     \textcolor{keywordtype}{bool} m\_factorizationIsOk;
00196     \hyperlink{group__enums_ga85fad7b87587764e5cf6b513a9e0ee5e}{ComputationInfo} m\_info;
00197     \hyperlink{group___core___module}{PermutationMatrix<Dynamic,Dynamic,StorageIndex>} m\_P;    
       \textcolor{comment}{// Fill-reducing permutation}
00198     \hyperlink{group___core___module}{PermutationMatrix<Dynamic,Dynamic,StorageIndex>} m\_Pinv; 
       \textcolor{comment}{// Inverse permutation}
00199 \};
00200 
00205 \textcolor{keyword}{template}<\textcolor{keyword}{typename} Scalar, \textcolor{keyword}{typename} StorageIndex>
00206 \textcolor{keywordtype}{void} \hyperlink{group___iterative_linear_solvers___module_a9628c5a595e9e984c72d1f8e671a6925}{IncompleteLUT<Scalar,StorageIndex>::setDroptol}(\textcolor{keyword}{const} 
      RealScalar& droptol)
00207 \{
00208   this->m\_droptol = droptol;   
00209 \}
00210 
00215 \textcolor{keyword}{template}<\textcolor{keyword}{typename} Scalar, \textcolor{keyword}{typename} StorageIndex>
00216 \textcolor{keywordtype}{void} \hyperlink{group___iterative_linear_solvers___module_a327767d12b55ff8a023f12a030051e17}{IncompleteLUT<Scalar,StorageIndex>::setFillfactor}(\textcolor{keywordtype}{int}
       fillfactor)
00217 \{
00218   this->m\_fillfactor = fillfactor;   
00219 \}
00220 
00221 \textcolor{keyword}{template} <\textcolor{keyword}{typename} Scalar, \textcolor{keyword}{typename} StorageIndex>
00222 \textcolor{keyword}{template}<\textcolor{keyword}{typename} \_MatrixType>
00223 \textcolor{keywordtype}{void} \hyperlink{group___iterative_linear_solvers___module_class_eigen_1_1_incomplete_l_u_t}{IncompleteLUT<Scalar,StorageIndex>::analyzePattern}(\textcolor{keyword}{
      const} \_MatrixType& amat)
00224 \{
00225   \textcolor{comment}{// Compute the Fill-reducing permutation}
00226   \textcolor{comment}{// Since ILUT does not perform any numerical pivoting,}
00227   \textcolor{comment}{// it is highly preferable to keep the diagonal through symmetric permutations.}
00228 \textcolor{preprocessor}{#ifndef EIGEN\_MPL2\_ONLY}
00229   \textcolor{comment}{// To this end, let's symmetrize the pattern and perform AMD on it.}
00230   \hyperlink{group___sparse_core___module}{SparseMatrix<Scalar,ColMajor, StorageIndex>} mat1 = amat;
00231   \hyperlink{group___sparse_core___module}{SparseMatrix<Scalar,ColMajor, StorageIndex>} mat2 = amat.
      transpose();
00232   \textcolor{comment}{// FIXME for a matrix with nearly symmetric pattern, mat2+mat1 is the appropriate choice.}
00233   \textcolor{comment}{//       on the other hand for a really non-symmetric pattern, mat2*mat1 should be prefered...}
00234   \hyperlink{group___sparse_core___module}{SparseMatrix<Scalar,ColMajor, StorageIndex>} AtA = mat2 + mat1;
00235   \hyperlink{group___ordering_methods___module_class_eigen_1_1_a_m_d_ordering}{AMDOrdering<StorageIndex>} ordering;
00236   ordering(AtA,m\_P);
00237   m\_Pinv  = m\_P.inverse(); \textcolor{comment}{// cache the inverse permutation}
00238 \textcolor{preprocessor}{#else}
00239   \textcolor{comment}{// If AMD is not available, (MPL2-only), then let's use the slower COLAMD routine.}
00240   \hyperlink{group___sparse_core___module}{SparseMatrix<Scalar,ColMajor, StorageIndex>} mat1 = amat;
00241   \hyperlink{group___ordering_methods___module_class_eigen_1_1_c_o_l_a_m_d_ordering}{COLAMDOrdering<StorageIndex>} ordering;
00242   ordering(mat1,m\_Pinv);
00243   m\_P = m\_Pinv.inverse();
00244 \textcolor{preprocessor}{#endif}
00245 
00246   m\_analysisIsOk = \textcolor{keyword}{true};
00247   m\_factorizationIsOk = \textcolor{keyword}{false};
00248   m\_isInitialized = \textcolor{keyword}{true};
00249 \}
00250 
00251 \textcolor{keyword}{template} <\textcolor{keyword}{typename} Scalar, \textcolor{keyword}{typename} StorageIndex>
00252 \textcolor{keyword}{template}<\textcolor{keyword}{typename} \_MatrixType>
00253 \textcolor{keywordtype}{void} \hyperlink{group___iterative_linear_solvers___module_class_eigen_1_1_incomplete_l_u_t}{IncompleteLUT<Scalar,StorageIndex>::factorize}(\textcolor{keyword}{const} 
      \_MatrixType& amat)
00254 \{
00255   \textcolor{keyword}{using} std::sqrt;
00256   \textcolor{keyword}{using} \hyperlink{endian_8c_a3ca5ecd34b04d6a243c054ac3a57f68d}{std::swap};
00257   \textcolor{keyword}{using} std::abs;
00258   \textcolor{keyword}{using} internal::convert\_index;
00259 
00260   eigen\_assert((amat.rows() == amat.cols()) && \textcolor{stringliteral}{"The factorization should be done on a square matrix"});
00261   \hyperlink{namespace_eigen_a62e77e0933482dafde8fe197d9a2cfde}{Index} n = amat.cols();  \textcolor{comment}{// Size of the matrix}
00262   m\_lu.resize(n,n);
00263   \textcolor{comment}{// Declare Working vectors and variables}
00264   Vector u(n) ;     \textcolor{comment}{// real values of the row -- maximum size is n --}
00265   VectorI ju(n);   \textcolor{comment}{// column position of the values in u -- maximum size  is n}
00266   VectorI jr(n);   \textcolor{comment}{// Indicate the position of the nonzero elements in the vector u -- A zero location is
       indicated by -1}
00267 
00268   \textcolor{comment}{// Apply the fill-reducing permutation}
00269   eigen\_assert(m\_analysisIsOk && \textcolor{stringliteral}{"You must first call analyzePattern()"});
00270   \hyperlink{group___sparse_core___module}{SparseMatrix<Scalar,RowMajor, StorageIndex>} mat;
00271   mat = amat.\hyperlink{group___sparse_core___module_a51d4898bd6a57cc3ba543a39b102423e}{twistedBy}(m\_Pinv);
00272 
00273   \textcolor{comment}{// Initialization}
00274   jr.fill(-1);
00275   ju.fill(0);
00276   u.fill(0);
00277 
00278   \textcolor{comment}{// number of largest elements to keep in each row:}
00279   \hyperlink{namespace_eigen_a62e77e0933482dafde8fe197d9a2cfde}{Index} fill\_in = (amat.nonZeros()*m\_fillfactor)/n + 1;
00280   \textcolor{keywordflow}{if} (fill\_in > n) fill\_in = n;
00281 
00282   \textcolor{comment}{// number of largest nonzero elements to keep in the L and the U part of the current row:}
00283   \hyperlink{namespace_eigen_a62e77e0933482dafde8fe197d9a2cfde}{Index} nnzL = fill\_in/2;
00284   \hyperlink{namespace_eigen_a62e77e0933482dafde8fe197d9a2cfde}{Index} nnzU = nnzL;
00285   m\_lu.reserve(n * (nnzL + nnzU + 1));
00286 
00287   \textcolor{comment}{// global loop over the rows of the sparse matrix}
00288   \textcolor{keywordflow}{for} (\hyperlink{namespace_eigen_a62e77e0933482dafde8fe197d9a2cfde}{Index} ii = 0; ii < n; ii++)
00289   \{
00290     \textcolor{comment}{// 1 - copy the lower and the upper part of the row i of mat in the working vector u}
00291 
00292     \hyperlink{namespace_eigen_a62e77e0933482dafde8fe197d9a2cfde}{Index} sizeu = 1; \textcolor{comment}{// number of nonzero elements in the upper part of the current row}
00293     \hyperlink{namespace_eigen_a62e77e0933482dafde8fe197d9a2cfde}{Index} sizel = 0; \textcolor{comment}{// number of nonzero elements in the lower part of the current row}
00294     ju(ii)    = convert\_index<StorageIndex>(ii);
00295     u(ii)     = 0;
00296     jr(ii)    = convert\_index<StorageIndex>(ii);
00297     RealScalar rownorm = 0;
00298 
00299     \textcolor{keyword}{typename} FactorType::InnerIterator j\_it(mat, ii); \textcolor{comment}{// Iterate through the current row ii}
00300     \textcolor{keywordflow}{for} (; j\_it; ++j\_it)
00301     \{
00302       \hyperlink{namespace_eigen_a62e77e0933482dafde8fe197d9a2cfde}{Index} k = j\_it.index();
00303       \textcolor{keywordflow}{if} (k < ii)
00304       \{
00305         \textcolor{comment}{// copy the lower part}
00306         ju(sizel) = convert\_index<StorageIndex>(k);
00307         u(sizel) = j\_it.value();
00308         jr(k) = convert\_index<StorageIndex>(sizel);
00309         ++sizel;
00310       \}
00311       \textcolor{keywordflow}{else} \textcolor{keywordflow}{if} (k == ii)
00312       \{
00313         u(ii) = j\_it.value();
00314       \}
00315       \textcolor{keywordflow}{else}
00316       \{
00317         \textcolor{comment}{// copy the upper part}
00318         \hyperlink{namespace_eigen_a62e77e0933482dafde8fe197d9a2cfde}{Index} jpos = ii + sizeu;
00319         ju(jpos) = convert\_index<StorageIndex>(k);
00320         u(jpos) = j\_it.value();
00321         jr(k) = convert\_index<StorageIndex>(jpos);
00322         ++sizeu;
00323       \}
00324       rownorm += numext::abs2(j\_it.value());
00325     \}
00326 
00327     \textcolor{comment}{// 2 - detect possible zero row}
00328     \textcolor{keywordflow}{if}(rownorm==0)
00329     \{
00330       m\_info = \hyperlink{group__enums_gga85fad7b87587764e5cf6b513a9e0ee5eaaf9b736d310a664e7729d163a035cc5f}{NumericalIssue};
00331       \textcolor{keywordflow}{return};
00332     \}
00333     \textcolor{comment}{// Take the 2-norm of the current row as a relative tolerance}
00334     rownorm = sqrt(rownorm);
00335 
00336     \textcolor{comment}{// 3 - eliminate the previous nonzero rows}
00337     \hyperlink{namespace_eigen_a62e77e0933482dafde8fe197d9a2cfde}{Index} jj = 0;
00338     \hyperlink{namespace_eigen_a62e77e0933482dafde8fe197d9a2cfde}{Index} len = 0;
00339     \textcolor{keywordflow}{while} (jj < sizel)
00340     \{
00341       \textcolor{comment}{// In order to eliminate in the correct order,}
00342       \textcolor{comment}{// we must select first the smallest column index among  ju(jj:sizel)}
00343       \hyperlink{namespace_eigen_a62e77e0933482dafde8fe197d9a2cfde}{Index} k;
00344       \hyperlink{namespace_eigen_a62e77e0933482dafde8fe197d9a2cfde}{Index} minrow = ju.segment(jj,sizel-jj).minCoeff(&k); \textcolor{comment}{// k is relative to the segment}
00345       k += jj;
00346       \textcolor{keywordflow}{if} (minrow != ju(jj))
00347       \{
00348         \textcolor{comment}{// swap the two locations}
00349         \hyperlink{namespace_eigen_a62e77e0933482dafde8fe197d9a2cfde}{Index} j = ju(jj);
00350         \hyperlink{endian_8c_a3ca5ecd34b04d6a243c054ac3a57f68d}{swap}(ju(jj), ju(k));
00351         jr(minrow) = convert\_index<StorageIndex>(jj);
00352         jr(j) = convert\_index<StorageIndex>(k);
00353         \hyperlink{endian_8c_a3ca5ecd34b04d6a243c054ac3a57f68d}{swap}(u(jj), u(k));
00354       \}
00355       \textcolor{comment}{// Reset this location}
00356       jr(minrow) = -1;
00357 
00358       \textcolor{comment}{// Start elimination}
00359       \textcolor{keyword}{typename} FactorType::InnerIterator ki\_it(m\_lu, minrow);
00360       \textcolor{keywordflow}{while} (ki\_it && ki\_it.index() < minrow) ++ki\_it;
00361       eigen\_internal\_assert(ki\_it && ki\_it.col()==minrow);
00362       Scalar fact = u(jj) / ki\_it.value();
00363 
00364       \textcolor{comment}{// drop too small elements}
00365       \textcolor{keywordflow}{if}(abs(fact) <= m\_droptol)
00366       \{
00367         jj++;
00368         \textcolor{keywordflow}{continue};
00369       \}
00370 
00371       \textcolor{comment}{// linear combination of the current row ii and the row minrow}
00372       ++ki\_it;
00373       \textcolor{keywordflow}{for} (; ki\_it; ++ki\_it)
00374       \{
00375         Scalar prod = fact * ki\_it.value();
00376         \hyperlink{namespace_eigen_a62e77e0933482dafde8fe197d9a2cfde}{Index} j     = ki\_it.index();
00377         \hyperlink{namespace_eigen_a62e77e0933482dafde8fe197d9a2cfde}{Index} jpos  = jr(j);
00378         \textcolor{keywordflow}{if} (jpos == -1) \textcolor{comment}{// fill-in element}
00379         \{
00380           \hyperlink{namespace_eigen_a62e77e0933482dafde8fe197d9a2cfde}{Index} newpos;
00381           \textcolor{keywordflow}{if} (j >= ii) \textcolor{comment}{// dealing with the upper part}
00382           \{
00383             newpos = ii + sizeu;
00384             sizeu++;
00385             eigen\_internal\_assert(sizeu<=n);
00386           \}
00387           \textcolor{keywordflow}{else} \textcolor{comment}{// dealing with the lower part}
00388           \{
00389             newpos = sizel;
00390             sizel++;
00391             eigen\_internal\_assert(sizel<=ii);
00392           \}
00393           ju(newpos) = convert\_index<StorageIndex>(j);
00394           u(newpos) = -prod;
00395           jr(j) = convert\_index<StorageIndex>(newpos);
00396         \}
00397         \textcolor{keywordflow}{else}
00398           u(jpos) -= prod;
00399       \}
00400       \textcolor{comment}{// store the pivot element}
00401       u(len)  = fact;
00402       ju(len) = convert\_index<StorageIndex>(minrow);
00403       ++len;
00404 
00405       jj++;
00406     \} \textcolor{comment}{// end of the elimination on the row ii}
00407 
00408     \textcolor{comment}{// reset the upper part of the pointer jr to zero}
00409     \textcolor{keywordflow}{for}(\hyperlink{namespace_eigen_a62e77e0933482dafde8fe197d9a2cfde}{Index} k = 0; k <sizeu; k++) jr(ju(ii+k)) = -1;
00410 
00411     \textcolor{comment}{// 4 - partially sort and insert the elements in the m\_lu matrix}
00412 
00413     \textcolor{comment}{// sort the L-part of the row}
00414     sizel = len;
00415     len = (std::min)(sizel, nnzL);
00416     \textcolor{keyword}{typename} Vector::SegmentReturnType ul(u.segment(0, sizel));
00417     \textcolor{keyword}{typename} VectorI::SegmentReturnType jul(ju.segment(0, sizel));
00418     internal::QuickSplit(ul, jul, len);
00419 
00420     \textcolor{comment}{// store the largest m\_fill elements of the L part}
00421     m\_lu.startVec(ii);
00422     \textcolor{keywordflow}{for}(\hyperlink{namespace_eigen_a62e77e0933482dafde8fe197d9a2cfde}{Index} k = 0; k < len; k++)
00423       m\_lu.insertBackByOuterInnerUnordered(ii,ju(k)) = u(k);
00424 
00425     \textcolor{comment}{// store the diagonal element}
00426     \textcolor{comment}{// apply a shifting rule to avoid zero pivots (we are doing an incomplete factorization)}
00427     \textcolor{keywordflow}{if} (u(ii) == Scalar(0))
00428       u(ii) = sqrt(m\_droptol) * rownorm;
00429     m\_lu.insertBackByOuterInnerUnordered(ii, ii) = u(ii);
00430 
00431     \textcolor{comment}{// sort the U-part of the row}
00432     \textcolor{comment}{// apply the dropping rule first}
00433     len = 0;
00434     \textcolor{keywordflow}{for}(\hyperlink{namespace_eigen_a62e77e0933482dafde8fe197d9a2cfde}{Index} k = 1; k < sizeu; k++)
00435     \{
00436       \textcolor{keywordflow}{if}(abs(u(ii+k)) > m\_droptol * rownorm )
00437       \{
00438         ++len;
00439         u(ii + len)  = u(ii + k);
00440         ju(ii + len) = ju(ii + k);
00441       \}
00442     \}
00443     sizeu = len + 1; \textcolor{comment}{// +1 to take into account the diagonal element}
00444     len = (std::min)(sizeu, nnzU);
00445     \textcolor{keyword}{typename} Vector::SegmentReturnType uu(u.segment(ii+1, sizeu-1));
00446     \textcolor{keyword}{typename} VectorI::SegmentReturnType juu(ju.segment(ii+1, sizeu-1));
00447     internal::QuickSplit(uu, juu, len);
00448 
00449     \textcolor{comment}{// store the largest elements of the U part}
00450     \textcolor{keywordflow}{for}(\hyperlink{namespace_eigen_a62e77e0933482dafde8fe197d9a2cfde}{Index} k = ii + 1; k < ii + len; k++)
00451       m\_lu.insertBackByOuterInnerUnordered(ii,ju(k)) = u(k);
00452   \}
00453   m\_lu.finalize();
00454   m\_lu.makeCompressed();
00455 
00456   m\_factorizationIsOk = \textcolor{keyword}{true};
00457   m\_info = \hyperlink{group__enums_gga85fad7b87587764e5cf6b513a9e0ee5ea52581b035f4b59c203b8ff999ef5fcea}{Success};
00458 \}
00459 
00460 \} \textcolor{comment}{// end namespace Eigen}
00461 
00462 \textcolor{preprocessor}{#endif // EIGEN\_INCOMPLETE\_LUT\_H}
\end{DoxyCode}
