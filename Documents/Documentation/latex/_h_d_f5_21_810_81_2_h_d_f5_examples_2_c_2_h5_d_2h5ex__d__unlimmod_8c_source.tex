\hypertarget{_h_d_f5_21_810_81_2_h_d_f5_examples_2_c_2_h5_d_2h5ex__d__unlimmod_8c_source}{}\section{H\+D\+F5/1.10.1/\+H\+D\+F5\+Examples/\+C/\+H5\+D/h5ex\+\_\+d\+\_\+unlimmod.c}
\label{_h_d_f5_21_810_81_2_h_d_f5_examples_2_c_2_h5_d_2h5ex__d__unlimmod_8c_source}\index{h5ex\+\_\+d\+\_\+unlimmod.\+c@{h5ex\+\_\+d\+\_\+unlimmod.\+c}}

\begin{DoxyCode}
00001 \textcolor{comment}{/************************************************************}
00002 \textcolor{comment}{}
00003 \textcolor{comment}{  This example shows how to create and extend an unlimited}
00004 \textcolor{comment}{  dataset.  The program first writes integers to a dataset}
00005 \textcolor{comment}{  with dataspace dimensions of DIM0xDIM1, then closes the}
00006 \textcolor{comment}{  file.  Next, it reopens the file, reads back the data,}
00007 \textcolor{comment}{  outputs it to the screen, extends the dataset, and writes}
00008 \textcolor{comment}{  new data to the entire extended dataset.  Finally it}
00009 \textcolor{comment}{  reopens the file again, reads back the data, and utputs it}
00010 \textcolor{comment}{  to the screen.}
00011 \textcolor{comment}{}
00012 \textcolor{comment}{  This file is intended for use with HDF5 Library version 1.8}
00013 \textcolor{comment}{}
00014 \textcolor{comment}{ ************************************************************/}
00015 
00016 \textcolor{preprocessor}{#include "hdf5.h"}
00017 \textcolor{preprocessor}{#include <stdio.h>}
00018 \textcolor{preprocessor}{#include <stdlib.h>}
00019 
00020 \textcolor{preprocessor}{#define FILE            "h5ex\_d\_unlimmod.h5"}
00021 \textcolor{preprocessor}{#define DATASET         "DS1"}
00022 \textcolor{preprocessor}{#define DIM0            4}
00023 \textcolor{preprocessor}{#define DIM1            7}
00024 \textcolor{preprocessor}{#define EDIM0           6}
00025 \textcolor{preprocessor}{#define EDIM1           10}
00026 \textcolor{preprocessor}{#define CHUNK0          4}
00027 \textcolor{preprocessor}{#define CHUNK1          4}
00028 
00029 \textcolor{keywordtype}{int}
00030 main (\textcolor{keywordtype}{void})
00031 \{
00032     hid\_t           \hyperlink{structfile}{file}, space, dset, dcpl;    \textcolor{comment}{/* Handles */}
00033     herr\_t          status;
00034     hsize\_t         dims[2] = \{DIM0, DIM1\},
00035                     extdims[2] = \{EDIM0, EDIM1\},
00036                     maxdims[2],
00037                     chunk[2] = \{CHUNK0, CHUNK1\};
00038     \textcolor{keywordtype}{int}             wdata[DIM0][DIM1],          \textcolor{comment}{/* Write buffer */}
00039                     wdata2[EDIM0][EDIM1],       \textcolor{comment}{/* Write buffer for}
00040 \textcolor{comment}{                                                   extension */}
00041                     **rdata,                    \textcolor{comment}{/* Read buffer */}
00042                     ndims,
00043                     i, j;
00044 
00045     \textcolor{comment}{/*}
00046 \textcolor{comment}{     * Initialize data.}
00047 \textcolor{comment}{     */}
00048     \textcolor{keywordflow}{for} (i=0; i<DIM0; i++)
00049         \textcolor{keywordflow}{for} (j=0; j<DIM1; j++)
00050             wdata[i][j] = i * j - j;
00051 
00052     \textcolor{comment}{/*}
00053 \textcolor{comment}{     * Create a new file using the default properties.}
00054 \textcolor{comment}{     */}
00055     file = H5Fcreate (FILE, H5F\_ACC\_TRUNC, H5P\_DEFAULT, H5P\_DEFAULT);
00056 
00057     \textcolor{comment}{/*}
00058 \textcolor{comment}{     * Create dataspace with unlimited dimensions.}
00059 \textcolor{comment}{     */}
00060     maxdims[0] = H5S\_UNLIMITED;
00061     maxdims[1] = H5S\_UNLIMITED;
00062     space = H5Screate\_simple (2, dims, maxdims);
00063 
00064     \textcolor{comment}{/*}
00065 \textcolor{comment}{     * Create the dataset creation property list, and set the chunk}
00066 \textcolor{comment}{     * size.}
00067 \textcolor{comment}{     */}
00068     dcpl = H5Pcreate (H5P\_DATASET\_CREATE);
00069     status = H5Pset\_chunk (dcpl, 2, chunk);
00070 
00071     \textcolor{comment}{/*}
00072 \textcolor{comment}{     * Create the unlimited dataset.}
00073 \textcolor{comment}{     */}
00074     dset = H5Dcreate (file, DATASET, H5T\_STD\_I32LE, space, H5P\_DEFAULT, dcpl,
00075                 H5P\_DEFAULT);
00076 
00077     \textcolor{comment}{/*}
00078 \textcolor{comment}{     * Write the data to the dataset.}
00079 \textcolor{comment}{     */}
00080     status = H5Dwrite (dset, H5T\_NATIVE\_INT, H5S\_ALL, H5S\_ALL, H5P\_DEFAULT,
00081                 wdata[0]);
00082 
00083     \textcolor{comment}{/*}
00084 \textcolor{comment}{     * Close and release resources.}
00085 \textcolor{comment}{     */}
00086     status = H5Pclose (dcpl);
00087     status = H5Dclose (dset);
00088     status = H5Sclose (space);
00089     status = H5Fclose (file);
00090 
00091 
00092     \textcolor{comment}{/*}
00093 \textcolor{comment}{     * In this next section we read back the data, extend the dataset,}
00094 \textcolor{comment}{     * and write new data to the entire dataset.}
00095 \textcolor{comment}{     */}
00096 
00097     \textcolor{comment}{/*}
00098 \textcolor{comment}{     * Open file and dataset using the default properties.}
00099 \textcolor{comment}{     */}
00100     file = H5Fopen (FILE, H5F\_ACC\_RDWR, H5P\_DEFAULT);
00101     dset = H5Dopen (file, DATASET, H5P\_DEFAULT);
00102 
00103     \textcolor{comment}{/*}
00104 \textcolor{comment}{     * Get dataspace and allocate memory for read buffer.  This is a}
00105 \textcolor{comment}{     * two dimensional dataset so the dynamic allocation must be done}
00106 \textcolor{comment}{     * in steps.}
00107 \textcolor{comment}{     */}
00108     space = H5Dget\_space (dset);
00109     ndims = H5Sget\_simple\_extent\_dims (space, dims, NULL);
00110 
00111     \textcolor{comment}{/*}
00112 \textcolor{comment}{     * Allocate array of pointers to rows.}
00113 \textcolor{comment}{     */}
00114     rdata = (\textcolor{keywordtype}{int} **) malloc (dims[0] * \textcolor{keyword}{sizeof} (\textcolor{keywordtype}{int} *));
00115 
00116     \textcolor{comment}{/*}
00117 \textcolor{comment}{     * Allocate space for integer data.}
00118 \textcolor{comment}{     */}
00119     rdata[0] = (\textcolor{keywordtype}{int} *) malloc (dims[0] * dims[1] * \textcolor{keyword}{sizeof} (\textcolor{keywordtype}{int}));
00120 
00121     \textcolor{comment}{/*}
00122 \textcolor{comment}{     * Set the rest of the pointers to rows to the correct addresses.}
00123 \textcolor{comment}{     */}
00124     \textcolor{keywordflow}{for} (i=1; i<dims[0]; i++)
00125         rdata[i] = rdata[0] + i * dims[1];
00126 
00127     \textcolor{comment}{/*}
00128 \textcolor{comment}{     * Read the data using the default properties.}
00129 \textcolor{comment}{     */}
00130     status = H5Dread (dset, H5T\_NATIVE\_INT, H5S\_ALL, H5S\_ALL, H5P\_DEFAULT,
00131                 rdata[0]);
00132 
00133     \textcolor{comment}{/*}
00134 \textcolor{comment}{     * Output the data to the screen.}
00135 \textcolor{comment}{     */}
00136     printf (\textcolor{stringliteral}{"Dataset before extension:\(\backslash\)n"});
00137     \textcolor{keywordflow}{for} (i=0; i<dims[0]; i++) \{
00138         printf (\textcolor{stringliteral}{" ["});
00139         \textcolor{keywordflow}{for} (j=0; j<dims[1]; j++)
00140             printf (\textcolor{stringliteral}{" %3d"}, rdata[i][j]);
00141         printf (\textcolor{stringliteral}{"]\(\backslash\)n"});
00142     \}
00143 
00144     \textcolor{comment}{/*}
00145 \textcolor{comment}{     * Extend the dataset.}
00146 \textcolor{comment}{     */}
00147     status = H5Dset\_extent (dset, extdims);
00148 
00149     \textcolor{comment}{/*}
00150 \textcolor{comment}{     * Initialize data for writing to the extended dataset.}
00151 \textcolor{comment}{     */}
00152     \textcolor{keywordflow}{for} (i=0; i<EDIM0; i++)
00153         \textcolor{keywordflow}{for} (j=0; j<EDIM1; j++)
00154             wdata2[i][j] = j;
00155 
00156     \textcolor{comment}{/*}
00157 \textcolor{comment}{     * Write the data to the extended dataset.}
00158 \textcolor{comment}{     */}
00159     status = H5Dwrite (dset, H5T\_NATIVE\_INT, H5S\_ALL, H5S\_ALL, H5P\_DEFAULT,
00160                 wdata2[0]);
00161 
00162     \textcolor{comment}{/*}
00163 \textcolor{comment}{     * Close and release resources.}
00164 \textcolor{comment}{     */}
00165     free (rdata[0]);
00166     free(rdata);
00167     status = H5Dclose (dset);
00168     status = H5Sclose (space);
00169     status = H5Fclose (file);
00170 
00171 
00172     \textcolor{comment}{/*}
00173 \textcolor{comment}{     * Now we simply read back the data and output it to the screen.}
00174 \textcolor{comment}{     */}
00175 
00176     \textcolor{comment}{/*}
00177 \textcolor{comment}{     * Open file and dataset using the default properties.}
00178 \textcolor{comment}{     */}
00179     file = H5Fopen (FILE, H5F\_ACC\_RDONLY, H5P\_DEFAULT);
00180     dset = H5Dopen (file, DATASET, H5P\_DEFAULT);
00181 
00182     \textcolor{comment}{/*}
00183 \textcolor{comment}{     * Get dataspace and allocate memory for the read buffer as before.}
00184 \textcolor{comment}{     */}
00185     space = H5Dget\_space (dset);
00186     ndims = H5Sget\_simple\_extent\_dims (space, dims, NULL);
00187     rdata = (\textcolor{keywordtype}{int} **) malloc (dims[0] * \textcolor{keyword}{sizeof} (\textcolor{keywordtype}{int} *));
00188     rdata[0] = (\textcolor{keywordtype}{int} *) malloc (dims[0] * dims[1] * \textcolor{keyword}{sizeof} (\textcolor{keywordtype}{int}));
00189     \textcolor{keywordflow}{for} (i=1; i<dims[0]; i++)
00190         rdata[i] = rdata[0] + i * dims[1];
00191 
00192     \textcolor{comment}{/*}
00193 \textcolor{comment}{     * Read the data using the default properties.}
00194 \textcolor{comment}{     */}
00195     status = H5Dread (dset, H5T\_NATIVE\_INT, H5S\_ALL, H5S\_ALL, H5P\_DEFAULT,
00196                 rdata[0]);
00197 
00198     \textcolor{comment}{/*}
00199 \textcolor{comment}{     * Output the data to the screen.}
00200 \textcolor{comment}{     */}
00201     printf (\textcolor{stringliteral}{"\(\backslash\)nDataset after extension:\(\backslash\)n"});
00202     \textcolor{keywordflow}{for} (i=0; i<dims[0]; i++) \{
00203         printf (\textcolor{stringliteral}{" ["});
00204         \textcolor{keywordflow}{for} (j=0; j<dims[1]; j++)
00205             printf (\textcolor{stringliteral}{" %3d"}, rdata[i][j]);
00206         printf (\textcolor{stringliteral}{"]\(\backslash\)n"});
00207     \}
00208 
00209     \textcolor{comment}{/*}
00210 \textcolor{comment}{     * Close and release resources.}
00211 \textcolor{comment}{     */}
00212     free (rdata[0]);
00213     free(rdata);
00214     status = H5Dclose (dset);
00215     status = H5Sclose (space);
00216     status = H5Fclose (file);
00217 
00218     \textcolor{keywordflow}{return} 0;
00219 \}
\end{DoxyCode}
