\hypertarget{_h_d_f5_21_810_81_2_h_d_f5_examples_2_c_2_h5_t_2h5ex__t__opaqueatt_8c_source}{}\section{H\+D\+F5/1.10.1/\+H\+D\+F5\+Examples/\+C/\+H5\+T/h5ex\+\_\+t\+\_\+opaqueatt.c}
\label{_h_d_f5_21_810_81_2_h_d_f5_examples_2_c_2_h5_t_2h5ex__t__opaqueatt_8c_source}\index{h5ex\+\_\+t\+\_\+opaqueatt.\+c@{h5ex\+\_\+t\+\_\+opaqueatt.\+c}}

\begin{DoxyCode}
00001 \textcolor{comment}{/************************************************************}
00002 \textcolor{comment}{}
00003 \textcolor{comment}{  This example shows how to read and write opaque datatypes}
00004 \textcolor{comment}{  to an attribute.  The program first writes opaque data to}
00005 \textcolor{comment}{  an attribute with a dataspace of DIM0, then closes the}
00006 \textcolor{comment}{  file. Next, it reopens the file, reads back the data, and}
00007 \textcolor{comment}{  outputs it to the screen.}
00008 \textcolor{comment}{}
00009 \textcolor{comment}{  This file is intended for use with HDF5 Library version 1.8}
00010 \textcolor{comment}{}
00011 \textcolor{comment}{ ************************************************************/}
00012 
00013 \textcolor{preprocessor}{#include "hdf5.h"}
00014 \textcolor{preprocessor}{#include <stdio.h>}
00015 \textcolor{preprocessor}{#include <stdlib.h>}
00016 
00017 \textcolor{preprocessor}{#define FILE            "h5ex\_t\_opaqueatt.h5"}
00018 \textcolor{preprocessor}{#define DATASET         "DS1"}
00019 \textcolor{preprocessor}{#define ATTRIBUTE       "A1"}
00020 \textcolor{preprocessor}{#define DIM0            4}
00021 \textcolor{preprocessor}{#define LEN             7}
00022 
00023 \textcolor{keywordtype}{int}
00024 main (\textcolor{keywordtype}{void})
00025 \{
00026     hid\_t       \hyperlink{structfile}{file}, space, dtype, dset, attr;     \textcolor{comment}{/* Handles */}
00027     herr\_t      status;
00028     hsize\_t     dims[1] = \{DIM0\};
00029     \textcolor{keywordtype}{size\_t}      len;
00030     \textcolor{keywordtype}{char}        wdata[DIM0*LEN],                    \textcolor{comment}{/* Write buffer */}
00031                 *rdata,                             \textcolor{comment}{/* Read buffer */}
00032                 str[LEN] = \textcolor{stringliteral}{"OPAQUE"},
00033                 *tag;
00034     \textcolor{keywordtype}{int}         ndims,
00035                 i, j;
00036 
00037     \textcolor{comment}{/*}
00038 \textcolor{comment}{     * Initialize data.}
00039 \textcolor{comment}{     */}
00040     \textcolor{keywordflow}{for} (i=0; i<DIM0; i++) \{
00041         \textcolor{keywordflow}{for} (j=0; j<LEN-1; j++)
00042             wdata[j + i * LEN] = str[j];
00043         wdata[LEN - 1 + i * LEN] = (char) i + \textcolor{charliteral}{'0'};
00044     \}
00045 
00046     \textcolor{comment}{/*}
00047 \textcolor{comment}{     * Create a new file using the default properties.}
00048 \textcolor{comment}{     */}
00049     file = H5Fcreate (FILE, H5F\_ACC\_TRUNC, H5P\_DEFAULT, H5P\_DEFAULT);
00050 
00051     \textcolor{comment}{/*}
00052 \textcolor{comment}{     * Create dataset with a null dataspace.}
00053 \textcolor{comment}{     */}
00054     space = H5Screate (H5S\_NULL);
00055     dset = H5Dcreate (file, DATASET, H5T\_STD\_I32LE, space, H5P\_DEFAULT,
00056                 H5P\_DEFAULT, H5P\_DEFAULT);
00057     status = H5Sclose (space);
00058 
00059     \textcolor{comment}{/*}
00060 \textcolor{comment}{     * Create opaque datatype and set the tag to something appropriate.}
00061 \textcolor{comment}{     * For this example we will write and view the data as a character}
00062 \textcolor{comment}{     * array.}
00063 \textcolor{comment}{     */}
00064     dtype = H5Tcreate (H5T\_OPAQUE, LEN);
00065     status = H5Tset\_tag (dtype, \textcolor{stringliteral}{"Character array"});
00066 
00067     \textcolor{comment}{/*}
00068 \textcolor{comment}{     * Create dataspace.  Setting maximum size to NULL sets the maximum}
00069 \textcolor{comment}{     * size to be the current size.}
00070 \textcolor{comment}{     */}
00071     space = H5Screate\_simple (1, dims, NULL);
00072 
00073     \textcolor{comment}{/*}
00074 \textcolor{comment}{     * Create the attribute and write the opaque data to it.}
00075 \textcolor{comment}{     */}
00076     attr = H5Acreate (dset, ATTRIBUTE, dtype, space, H5P\_DEFAULT, H5P\_DEFAULT);
00077     status = H5Awrite (attr, dtype, wdata);
00078 
00079     \textcolor{comment}{/*}
00080 \textcolor{comment}{     * Close and release resources.}
00081 \textcolor{comment}{     */}
00082     status = H5Aclose (attr);
00083     status = H5Dclose (dset);
00084     status = H5Sclose (space);
00085     status = H5Tclose (dtype);
00086     status = H5Fclose (file);
00087 
00088 
00089     \textcolor{comment}{/*}
00090 \textcolor{comment}{     * Now we begin the read section of this example.  Here we assume}
00091 \textcolor{comment}{     * the attribute has the same name and rank, but can have any size.}
00092 \textcolor{comment}{     * Therefore we must allocate a new array to read in data using}
00093 \textcolor{comment}{     * malloc().}
00094 \textcolor{comment}{     */}
00095 
00096     \textcolor{comment}{/*}
00097 \textcolor{comment}{     * Open file, dataset, and attribute.}
00098 \textcolor{comment}{     */}
00099     file = H5Fopen (FILE, H5F\_ACC\_RDONLY, H5P\_DEFAULT);
00100     dset = H5Dopen (file, DATASET, H5P\_DEFAULT);
00101     attr = H5Aopen (dset, ATTRIBUTE, H5P\_DEFAULT);
00102 
00103     \textcolor{comment}{/*}
00104 \textcolor{comment}{     * Get datatype and properties for the datatype.  Note that H5Tget\_tag}
00105 \textcolor{comment}{     * allocates space for the string in tag, so we must remember to H5free\_memory() it}
00106 \textcolor{comment}{     * later.}
00107 \textcolor{comment}{     */}
00108     dtype = H5Aget\_type (attr);
00109     len = H5Tget\_size (dtype);
00110     tag = H5Tget\_tag (dtype);
00111 
00112     \textcolor{comment}{/*}
00113 \textcolor{comment}{     * Get dataspace and allocate memory for read buffer.}
00114 \textcolor{comment}{     */}
00115     space = H5Aget\_space (attr);
00116     ndims = H5Sget\_simple\_extent\_dims (space, dims, NULL);
00117     rdata = (\textcolor{keywordtype}{char} *) malloc (dims[0] * len);
00118 
00119     \textcolor{comment}{/*}
00120 \textcolor{comment}{     * Read the data.}
00121 \textcolor{comment}{     */}
00122     status = H5Aread (attr, dtype, rdata);
00123 
00124     \textcolor{comment}{/*}
00125 \textcolor{comment}{     * Output the data to the screen.}
00126 \textcolor{comment}{     */}
00127     printf (\textcolor{stringliteral}{"Datatype tag for %s is: \(\backslash\)"%s\(\backslash\)"\(\backslash\)n"}, ATTRIBUTE, tag);
00128     \textcolor{keywordflow}{for} (i=0; i<dims[0]; i++) \{
00129         printf (\textcolor{stringliteral}{"%s[%u]: "},ATTRIBUTE,i);
00130         \textcolor{keywordflow}{for} (j=0; j<len; j++)
00131             printf (\textcolor{stringliteral}{"%c"}, rdata[j + i * len]);
00132         printf (\textcolor{stringliteral}{"\(\backslash\)n"});
00133     \}
00134 
00135     \textcolor{comment}{/*}
00136 \textcolor{comment}{     * Close and release resources.}
00137 \textcolor{comment}{     */}
00138     free (rdata);
00139     H5free\_memory (tag);
00140     status = H5Aclose (attr);
00141     status = H5Dclose (dset);
00142     status = H5Sclose (space);
00143     status = H5Tclose (dtype);
00144     status = H5Fclose (file);
00145 
00146     \textcolor{keywordflow}{return} 0;
00147 \}
\end{DoxyCode}
