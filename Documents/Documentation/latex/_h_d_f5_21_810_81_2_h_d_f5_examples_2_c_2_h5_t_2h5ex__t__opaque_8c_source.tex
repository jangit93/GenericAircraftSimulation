\hypertarget{_h_d_f5_21_810_81_2_h_d_f5_examples_2_c_2_h5_t_2h5ex__t__opaque_8c_source}{}\section{H\+D\+F5/1.10.1/\+H\+D\+F5\+Examples/\+C/\+H5\+T/h5ex\+\_\+t\+\_\+opaque.c}
\label{_h_d_f5_21_810_81_2_h_d_f5_examples_2_c_2_h5_t_2h5ex__t__opaque_8c_source}\index{h5ex\+\_\+t\+\_\+opaque.\+c@{h5ex\+\_\+t\+\_\+opaque.\+c}}

\begin{DoxyCode}
00001 \textcolor{comment}{/************************************************************}
00002 \textcolor{comment}{}
00003 \textcolor{comment}{  This example shows how to read and write opaque datatypes}
00004 \textcolor{comment}{  to a dataset.  The program first writes opaque data to a}
00005 \textcolor{comment}{  dataset with a dataspace of DIM0, then closes the file.}
00006 \textcolor{comment}{  Next, it reopens the file, reads back the data, and}
00007 \textcolor{comment}{  outputs it to the screen.}
00008 \textcolor{comment}{}
00009 \textcolor{comment}{  This file is intended for use with HDF5 Library version 1.8}
00010 \textcolor{comment}{}
00011 \textcolor{comment}{ ************************************************************/}
00012 
00013 \textcolor{preprocessor}{#include "hdf5.h"}
00014 \textcolor{preprocessor}{#include <stdio.h>}
00015 \textcolor{preprocessor}{#include <stdlib.h>}
00016 
00017 \textcolor{preprocessor}{#define FILE            "h5ex\_t\_opaque.h5"}
00018 \textcolor{preprocessor}{#define DATASET         "DS1"}
00019 \textcolor{preprocessor}{#define DIM0            4}
00020 \textcolor{preprocessor}{#define LEN             7}
00021 
00022 \textcolor{keywordtype}{int}
00023 main (\textcolor{keywordtype}{void})
00024 \{
00025     hid\_t       \hyperlink{structfile}{file}, space, dtype, dset;   \textcolor{comment}{/* Handles */}
00026     herr\_t      status;
00027     hsize\_t     dims[1] = \{DIM0\};
00028     \textcolor{keywordtype}{size\_t}      len;
00029     \textcolor{keywordtype}{char}        wdata[DIM0*LEN],            \textcolor{comment}{/* Write buffer */}
00030                 *rdata,                     \textcolor{comment}{/* Read buffer */}
00031                 str[LEN] = \textcolor{stringliteral}{"OPAQUE"},
00032                 *tag;
00033     \textcolor{keywordtype}{int}         ndims,
00034                 i, j;
00035 
00036     \textcolor{comment}{/*}
00037 \textcolor{comment}{     * Initialize data.}
00038 \textcolor{comment}{     */}
00039     \textcolor{keywordflow}{for} (i=0; i<DIM0; i++) \{
00040         \textcolor{keywordflow}{for} (j=0; j<LEN-1; j++)
00041             wdata[j + i * LEN] = str[j];
00042         wdata[LEN - 1 + i * LEN] = (char) i + \textcolor{charliteral}{'0'};
00043     \}
00044 
00045     \textcolor{comment}{/*}
00046 \textcolor{comment}{     * Create a new file using the default properties.}
00047 \textcolor{comment}{     */}
00048     file = H5Fcreate (FILE, H5F\_ACC\_TRUNC, H5P\_DEFAULT, H5P\_DEFAULT);
00049 
00050     \textcolor{comment}{/*}
00051 \textcolor{comment}{     * Create opaque datatype and set the tag to something appropriate.}
00052 \textcolor{comment}{     * For this example we will write and view the data as a character}
00053 \textcolor{comment}{     * array.}
00054 \textcolor{comment}{     */}
00055     dtype = H5Tcreate (H5T\_OPAQUE, LEN);
00056     status = H5Tset\_tag (dtype, \textcolor{stringliteral}{"Character array"});
00057 
00058     \textcolor{comment}{/*}
00059 \textcolor{comment}{     * Create dataspace.  Setting maximum size to NULL sets the maximum}
00060 \textcolor{comment}{     * size to be the current size.}
00061 \textcolor{comment}{     */}
00062     space = H5Screate\_simple (1, dims, NULL);
00063 
00064     \textcolor{comment}{/*}
00065 \textcolor{comment}{     * Create the dataset and write the opaque data to it.}
00066 \textcolor{comment}{     */}
00067     dset = H5Dcreate (file, DATASET, dtype, space, H5P\_DEFAULT, H5P\_DEFAULT,
00068                 H5P\_DEFAULT);
00069     status = H5Dwrite (dset, dtype, H5S\_ALL, H5S\_ALL, H5P\_DEFAULT, wdata);
00070 
00071     \textcolor{comment}{/*}
00072 \textcolor{comment}{     * Close and release resources.}
00073 \textcolor{comment}{     */}
00074     status = H5Dclose (dset);
00075     status = H5Sclose (space);
00076     status = H5Tclose (dtype);
00077     status = H5Fclose (file);
00078 
00079 
00080     \textcolor{comment}{/*}
00081 \textcolor{comment}{     * Now we begin the read section of this example.  Here we assume}
00082 \textcolor{comment}{     * the dataset has the same name and rank, but can have any size.}
00083 \textcolor{comment}{     * Therefore we must allocate a new array to read in data using}
00084 \textcolor{comment}{     * malloc().}
00085 \textcolor{comment}{     */}
00086 
00087     \textcolor{comment}{/*}
00088 \textcolor{comment}{     * Open file and dataset.}
00089 \textcolor{comment}{     */}
00090     file = H5Fopen (FILE, H5F\_ACC\_RDONLY, H5P\_DEFAULT);
00091     dset = H5Dopen (file, DATASET, H5P\_DEFAULT);
00092 
00093     \textcolor{comment}{/*}
00094 \textcolor{comment}{     * Get datatype and properties for the datatype.  Note that H5Tget\_tag}
00095 \textcolor{comment}{     * allocates space for the string in tag, so we must remember to H5free\_memory() it}
00096 \textcolor{comment}{     * later.}
00097 \textcolor{comment}{     */}
00098     dtype = H5Dget\_type (dset);
00099     len = H5Tget\_size (dtype);
00100     tag = H5Tget\_tag (dtype);
00101 
00102     \textcolor{comment}{/*}
00103 \textcolor{comment}{     * Get dataspace and allocate memory for read buffer.}
00104 \textcolor{comment}{     */}
00105     space = H5Dget\_space (dset);
00106     ndims = H5Sget\_simple\_extent\_dims (space, dims, NULL);
00107     rdata = (\textcolor{keywordtype}{char} *) malloc (dims[0] * len);
00108 
00109     \textcolor{comment}{/*}
00110 \textcolor{comment}{     * Read the data.}
00111 \textcolor{comment}{     */}
00112     status = H5Dread (dset, dtype, H5S\_ALL, H5S\_ALL, H5P\_DEFAULT, rdata);
00113 
00114     \textcolor{comment}{/*}
00115 \textcolor{comment}{     * Output the data to the screen.}
00116 \textcolor{comment}{     */}
00117     printf (\textcolor{stringliteral}{"Datatype tag for %s is: \(\backslash\)"%s\(\backslash\)"\(\backslash\)n"}, DATASET, tag);
00118     \textcolor{keywordflow}{for} (i=0; i<dims[0]; i++) \{
00119         printf (\textcolor{stringliteral}{"%s[%u]: "}, DATASET, i);
00120         \textcolor{keywordflow}{for} (j=0; j<len; j++)
00121             printf (\textcolor{stringliteral}{"%c"}, rdata[j + i * len]);
00122         printf (\textcolor{stringliteral}{"\(\backslash\)n"});
00123     \}
00124 
00125     \textcolor{comment}{/*}
00126 \textcolor{comment}{     * Close and release resources.}
00127 \textcolor{comment}{     */}
00128     free (rdata);
00129     H5free\_memory (tag);
00130     status = H5Dclose (dset);
00131     status = H5Sclose (space);
00132     status = H5Tclose (dtype);
00133     status = H5Fclose (file);
00134 
00135     \textcolor{keywordflow}{return} 0;
00136 \}
\end{DoxyCode}
