\hypertarget{matio_2visual__studio_2test_2eigen_2lapack_2dlarf_8f_source}{}\section{matio/visual\+\_\+studio/test/eigen/lapack/dlarf.f}
\label{matio_2visual__studio_2test_2eigen_2lapack_2dlarf_8f_source}\index{dlarf.\+f@{dlarf.\+f}}

\begin{DoxyCode}
00001 \textcolor{comment}{*> \(\backslash\)brief \(\backslash\)b DLARF}
00002 \textcolor{comment}{*}
00003 \textcolor{comment}{*  =========== DOCUMENTATION ===========}
00004 \textcolor{comment}{*}
00005 \textcolor{comment}{* Online html documentation available at }
00006 \textcolor{comment}{*            http://www.netlib.org/lapack/explore-html/ }
00007 \textcolor{comment}{*}
00008 \textcolor{comment}{*> \(\backslash\)htmlonly}
00009 \textcolor{comment}{*> Download DLARF + dependencies }
00010 \textcolor{comment}{*> <a
       href="http://www.netlib.org/cgi-bin/netlibfiles.tgz?format=tgz&filename=/lapack/lapack\_routine/dlarf.f"> }
00011 \textcolor{comment}{*> [TGZ]</a> }
00012 \textcolor{comment}{*> <a
       href="http://www.netlib.org/cgi-bin/netlibfiles.zip?format=zip&filename=/lapack/lapack\_routine/dlarf.f"> }
00013 \textcolor{comment}{*> [ZIP]</a> }
00014 \textcolor{comment}{*> <a
       href="http://www.netlib.org/cgi-bin/netlibfiles.txt?format=txt&filename=/lapack/lapack\_routine/dlarf.f"> }
00015 \textcolor{comment}{*> [TXT]</a>}
00016 \textcolor{comment}{*> \(\backslash\)endhtmlonly }
00017 \textcolor{comment}{*}
00018 \textcolor{comment}{*  Definition:}
00019 \textcolor{comment}{*  ===========}
00020 \textcolor{comment}{*}
00021 \textcolor{comment}{*       SUBROUTINE DLARF( SIDE, M, N, V, INCV, TAU, C, LDC, WORK )}
00022 \textcolor{comment}{* }
00023 \textcolor{comment}{*       .. Scalar Arguments ..}
00024 \textcolor{comment}{*       CHARACTER          SIDE}
00025 \textcolor{comment}{*       INTEGER            INCV, LDC, M, N}
00026 \textcolor{comment}{*       DOUBLE PRECISION   TAU}
00027 \textcolor{comment}{*       ..}
00028 \textcolor{comment}{*       .. Array Arguments ..}
00029 \textcolor{comment}{*       DOUBLE PRECISION   C( LDC, * ), V( * ), WORK( * )}
00030 \textcolor{comment}{*       ..}
00031 \textcolor{comment}{*  }
00032 \textcolor{comment}{*}
00033 \textcolor{comment}{*> \(\backslash\)par Purpose:}
00034 \textcolor{comment}{*  =============}
00035 \textcolor{comment}{*>}
00036 \textcolor{comment}{*> \(\backslash\)verbatim}
00037 \textcolor{comment}{*>}
00038 \textcolor{comment}{*> DLARF applies a real elementary reflector H to a real m by n matrix}
00039 \textcolor{comment}{*> C, from either the left or the right. H is represented in the form}
00040 \textcolor{comment}{*>}
00041 \textcolor{comment}{*>       H = I - tau * v * v**T}
00042 \textcolor{comment}{*>}
00043 \textcolor{comment}{*> where tau is a real scalar and v is a real vector.}
00044 \textcolor{comment}{*>}
00045 \textcolor{comment}{*> If tau = 0, then H is taken to be the unit matrix.}
00046 \textcolor{comment}{*> \(\backslash\)endverbatim}
00047 \textcolor{comment}{*}
00048 \textcolor{comment}{*  Arguments:}
00049 \textcolor{comment}{*  ==========}
00050 \textcolor{comment}{*}
00051 \textcolor{comment}{*> \(\backslash\)param[in] SIDE}
00052 \textcolor{comment}{*> \(\backslash\)verbatim}
00053 \textcolor{comment}{*>          SIDE is CHARACTER*1}
00054 \textcolor{comment}{*>          = 'L': form  H * C}
00055 \textcolor{comment}{*>          = 'R': form  C * H}
00056 \textcolor{comment}{*> \(\backslash\)endverbatim}
00057 \textcolor{comment}{*>}
00058 \textcolor{comment}{*> \(\backslash\)param[in] M}
00059 \textcolor{comment}{*> \(\backslash\)verbatim}
00060 \textcolor{comment}{*>          M is INTEGER}
00061 \textcolor{comment}{*>          The number of rows of the matrix C.}
00062 \textcolor{comment}{*> \(\backslash\)endverbatim}
00063 \textcolor{comment}{*>}
00064 \textcolor{comment}{*> \(\backslash\)param[in] N}
00065 \textcolor{comment}{*> \(\backslash\)verbatim}
00066 \textcolor{comment}{*>          N is INTEGER}
00067 \textcolor{comment}{*>          The number of columns of the matrix C.}
00068 \textcolor{comment}{*> \(\backslash\)endverbatim}
00069 \textcolor{comment}{*>}
00070 \textcolor{comment}{*> \(\backslash\)param[in] V}
00071 \textcolor{comment}{*> \(\backslash\)verbatim}
00072 \textcolor{comment}{*>          V is DOUBLE PRECISION array, dimension}
00073 \textcolor{comment}{*>                     (1 + (M-1)*abs(INCV)) if SIDE = 'L'}
00074 \textcolor{comment}{*>                  or (1 + (N-1)*abs(INCV)) if SIDE = 'R'}
00075 \textcolor{comment}{*>          The vector v in the representation of H. V is not used if}
00076 \textcolor{comment}{*>          TAU = 0.}
00077 \textcolor{comment}{*> \(\backslash\)endverbatim}
00078 \textcolor{comment}{*>}
00079 \textcolor{comment}{*> \(\backslash\)param[in] INCV}
00080 \textcolor{comment}{*> \(\backslash\)verbatim}
00081 \textcolor{comment}{*>          INCV is INTEGER}
00082 \textcolor{comment}{*>          The increment between elements of v. INCV <> 0.}
00083 \textcolor{comment}{*> \(\backslash\)endverbatim}
00084 \textcolor{comment}{*>}
00085 \textcolor{comment}{*> \(\backslash\)param[in] TAU}
00086 \textcolor{comment}{*> \(\backslash\)verbatim}
00087 \textcolor{comment}{*>          TAU is DOUBLE PRECISION}
00088 \textcolor{comment}{*>          The value tau in the representation of H.}
00089 \textcolor{comment}{*> \(\backslash\)endverbatim}
00090 \textcolor{comment}{*>}
00091 \textcolor{comment}{*> \(\backslash\)param[in,out] C}
00092 \textcolor{comment}{*> \(\backslash\)verbatim}
00093 \textcolor{comment}{*>          C is DOUBLE PRECISION array, dimension (LDC,N)}
00094 \textcolor{comment}{*>          On entry, the m by n matrix C.}
00095 \textcolor{comment}{*>          On exit, C is overwritten by the matrix H * C if SIDE = 'L',}
00096 \textcolor{comment}{*>          or C * H if SIDE = 'R'.}
00097 \textcolor{comment}{*> \(\backslash\)endverbatim}
00098 \textcolor{comment}{*>}
00099 \textcolor{comment}{*> \(\backslash\)param[in] LDC}
00100 \textcolor{comment}{*> \(\backslash\)verbatim}
00101 \textcolor{comment}{*>          LDC is INTEGER}
00102 \textcolor{comment}{*>          The leading dimension of the array C. LDC >= max(1,M).}
00103 \textcolor{comment}{*> \(\backslash\)endverbatim}
00104 \textcolor{comment}{*>}
00105 \textcolor{comment}{*> \(\backslash\)param[out] WORK}
00106 \textcolor{comment}{*> \(\backslash\)verbatim}
00107 \textcolor{comment}{*>          WORK is DOUBLE PRECISION array, dimension}
00108 \textcolor{comment}{*>                         (N) if SIDE = 'L'}
00109 \textcolor{comment}{*>                      or (M) if SIDE = 'R'}
00110 \textcolor{comment}{*> \(\backslash\)endverbatim}
00111 \textcolor{comment}{*}
00112 \textcolor{comment}{*  Authors:}
00113 \textcolor{comment}{*  ========}
00114 \textcolor{comment}{*}
00115 \textcolor{comment}{*> \(\backslash\)author Univ. of Tennessee }
00116 \textcolor{comment}{*> \(\backslash\)author Univ. of California Berkeley }
00117 \textcolor{comment}{*> \(\backslash\)author Univ. of Colorado Denver }
00118 \textcolor{comment}{*> \(\backslash\)author NAG Ltd. }
00119 \textcolor{comment}{*}
00120 \textcolor{comment}{*> \(\backslash\)date November 2011}
00121 \textcolor{comment}{*}
00122 \textcolor{comment}{*> \(\backslash\)ingroup doubleOTHERauxiliary}
00123 \textcolor{comment}{*}
00124 \textcolor{comment}{*  =====================================================================}
00125 \textcolor{keyword}{      SUBROUTINE }dlarf( SIDE, M, N, V, INCV, TAU, C, LDC, WORK )
00126 \textcolor{comment}{*}
00127 \textcolor{comment}{*  -- LAPACK auxiliary routine (version 3.4.0) --}
00128 \textcolor{comment}{*  -- LAPACK is a software package provided by Univ. of Tennessee,    --}
00129 \textcolor{comment}{*  -- Univ. of California Berkeley, Univ. of Colorado Denver and NAG Ltd..--}
00130 \textcolor{comment}{*     November 2011}
00131 \textcolor{comment}{*}
00132 \textcolor{comment}{*     .. Scalar Arguments ..}
00133       \textcolor{keywordtype}{CHARACTER}          side
00134       \textcolor{keywordtype}{INTEGER}            incv, ldc, m, n
00135       \textcolor{keywordtype}{DOUBLE PRECISION}   tau
00136 \textcolor{comment}{*     ..}
00137 \textcolor{comment}{*     .. Array Arguments ..}
00138       \textcolor{keywordtype}{DOUBLE PRECISION}   c( ldc, * ), v( * ), work( * )
00139 \textcolor{comment}{*     ..}
00140 \textcolor{comment}{*}
00141 \textcolor{comment}{*  =====================================================================}
00142 \textcolor{comment}{*}
00143 \textcolor{comment}{*     .. Parameters ..}
00144       \textcolor{keywordtype}{DOUBLE PRECISION}   one, zero
00145       parameter( one = 1.0d+0, zero = 0.0d+0 )
00146 \textcolor{comment}{*     ..}
00147 \textcolor{comment}{*     .. Local Scalars ..}
00148       \textcolor{keywordtype}{LOGICAL}            applyleft
00149       \textcolor{keywordtype}{INTEGER}            i, lastv, lastc
00150 \textcolor{comment}{*     ..}
00151 \textcolor{comment}{*     .. External Subroutines ..}
00152       \textcolor{keywordtype}{EXTERNAL}           dgemv, dger
00153 \textcolor{comment}{*     ..}
00154 \textcolor{comment}{*     .. External Functions ..}
00155       \textcolor{keywordtype}{LOGICAL}            lsame
00156       \textcolor{keywordtype}{INTEGER}            iladlr, iladlc
00157       \textcolor{keywordtype}{EXTERNAL}           lsame, iladlr, iladlc
00158 \textcolor{comment}{*     ..}
00159 \textcolor{comment}{*     .. Executable Statements ..}
00160 \textcolor{comment}{*}
00161       applyleft = lsame( side, \textcolor{stringliteral}{'L'} )
00162       lastv = 0
00163       lastc = 0
00164       \textcolor{keywordflow}{IF}( tau.NE.zero ) \textcolor{keywordflow}{THEN}
00165 \textcolor{comment}{!     Set up variables for scanning V.  LASTV begins pointing to the end}
00166 \textcolor{comment}{!     of V.}
00167          \textcolor{keywordflow}{IF}( applyleft ) \textcolor{keywordflow}{THEN}
00168             lastv = m
00169          \textcolor{keywordflow}{ELSE}
00170             lastv = n
00171 \textcolor{keywordflow}{         END IF}
00172          \textcolor{keywordflow}{IF}( incv.GT.0 ) \textcolor{keywordflow}{THEN}
00173             i = 1 + (lastv-1) * incv
00174          \textcolor{keywordflow}{ELSE}
00175             i = 1
00176 \textcolor{keywordflow}{         END IF}
00177 \textcolor{comment}{!     Look for the last non-zero row in V.}
00178          \textcolor{keywordflow}{DO} \textcolor{keywordflow}{WHILE}( lastv.GT.0 .AND. v( i ).EQ.zero )
00179             lastv = lastv - 1
00180             i = i - incv
00181 \textcolor{keywordflow}{         END DO}
00182          \textcolor{keywordflow}{IF}( applyleft ) \textcolor{keywordflow}{THEN}
00183 \textcolor{comment}{!     Scan for the last non-zero column in C(1:lastv,:).}
00184             lastc = iladlc(lastv, n, c, ldc)
00185          \textcolor{keywordflow}{ELSE}
00186 \textcolor{comment}{!     Scan for the last non-zero row in C(:,1:lastv).}
00187             lastc = iladlr(m, lastv, c, ldc)
00188 \textcolor{keywordflow}{         END IF}
00189 \textcolor{keywordflow}{      END IF}
00190 \textcolor{comment}{!     Note that lastc.eq.0 renders the BLAS operations null; no special}
00191 \textcolor{comment}{!     case is needed at this level.}
00192       \textcolor{keywordflow}{IF}( applyleft ) \textcolor{keywordflow}{THEN}
00193 \textcolor{comment}{*}
00194 \textcolor{comment}{*        Form  H * C}
00195 \textcolor{comment}{*}
00196          \textcolor{keywordflow}{IF}( lastv.GT.0 ) \textcolor{keywordflow}{THEN}
00197 \textcolor{comment}{*}
00198 \textcolor{comment}{*           w(1:lastc,1) := C(1:lastv,1:lastc)**T * v(1:lastv,1)}
00199 \textcolor{comment}{*}
00200             \textcolor{keyword}{CALL }dgemv( \textcolor{stringliteral}{'Transpose'}, lastv, lastc, one, c, ldc, v, incv,
00201      $           zero, work, 1 )
00202 \textcolor{comment}{*}
00203 \textcolor{comment}{*           C(1:lastv,1:lastc) := C(...) - v(1:lastv,1) * w(1:lastc,1)**T}
00204 \textcolor{comment}{*}
00205             \textcolor{keyword}{CALL }dger( lastv, lastc, -tau, v, incv, work, 1, c, ldc )
00206 \textcolor{keywordflow}{         END IF}
00207       \textcolor{keywordflow}{ELSE}
00208 \textcolor{comment}{*}
00209 \textcolor{comment}{*        Form  C * H}
00210 \textcolor{comment}{*}
00211          \textcolor{keywordflow}{IF}( lastv.GT.0 ) \textcolor{keywordflow}{THEN}
00212 \textcolor{comment}{*}
00213 \textcolor{comment}{*           w(1:lastc,1) := C(1:lastc,1:lastv) * v(1:lastv,1)}
00214 \textcolor{comment}{*}
00215             \textcolor{keyword}{CALL }dgemv( \textcolor{stringliteral}{'No transpose'}, lastc, lastv, one, c, ldc,
00216      $           v, incv, zero, work, 1 )
00217 \textcolor{comment}{*}
00218 \textcolor{comment}{*           C(1:lastc,1:lastv) := C(...) - w(1:lastc,1) * v(1:lastv,1)**T}
00219 \textcolor{comment}{*}
00220             \textcolor{keyword}{CALL }dger( lastc, lastv, -tau, work, 1, v, incv, c, ldc )
00221 \textcolor{keywordflow}{         END IF}
00222 \textcolor{keywordflow}{      END IF}
00223       \textcolor{keywordflow}{RETURN}
00224 \textcolor{comment}{*}
00225 \textcolor{comment}{*     End of DLARF}
00226 \textcolor{comment}{*}
00227 \textcolor{keyword}{      END}
\end{DoxyCode}
