\hypertarget{zlib_2inftrees_8c_source}{}\section{zlib/inftrees.c}
\label{zlib_2inftrees_8c_source}\index{inftrees.\+c@{inftrees.\+c}}

\begin{DoxyCode}
00001 \textcolor{comment}{/* inftrees.c -- generate Huffman trees for efficient decoding}
00002 \textcolor{comment}{ * Copyright (C) 1995-2017 Mark Adler}
00003 \textcolor{comment}{ * For conditions of distribution and use, see copyright notice in zlib.h}
00004 \textcolor{comment}{ */}
00005 
00006 \textcolor{preprocessor}{#include "zutil.h"}
00007 \textcolor{preprocessor}{#include "inftrees.h"}
00008 
00009 \textcolor{preprocessor}{#define MAXBITS 15}
00010 
00011 \textcolor{keyword}{const} \textcolor{keywordtype}{char} inflate\_copyright[] =
00012    \textcolor{stringliteral}{" inflate 1.2.11 Copyright 1995-2017 Mark Adler "};
00013 \textcolor{comment}{/*}
00014 \textcolor{comment}{  If you use the zlib library in a product, an acknowledgment is welcome}
00015 \textcolor{comment}{  in the documentation of your product. If for some reason you cannot}
00016 \textcolor{comment}{  include such an acknowledgment, I would appreciate that you keep this}
00017 \textcolor{comment}{  copyright string in the executable of your product.}
00018 \textcolor{comment}{ */}
00019 
00020 \textcolor{comment}{/*}
00021 \textcolor{comment}{   Build a set of tables to decode the provided canonical Huffman code.}
00022 \textcolor{comment}{   The code lengths are lens[0..codes-1].  The result starts at *table,}
00023 \textcolor{comment}{   whose indices are 0..2^bits-1.  work is a writable array of at least}
00024 \textcolor{comment}{   lens shorts, which is used as a work area.  type is the type of code}
00025 \textcolor{comment}{   to be generated, CODES, LENS, or DISTS.  On return, zero is success,}
00026 \textcolor{comment}{   -1 is an invalid code, and +1 means that ENOUGH isn't enough.  table}
00027 \textcolor{comment}{   on return points to the next available entry's address.  bits is the}
00028 \textcolor{comment}{   requested root table index bits, and on return it is the actual root}
00029 \textcolor{comment}{   table index bits.  It will differ if the request is greater than the}
00030 \textcolor{comment}{   longest code or if it is less than the shortest code.}
00031 \textcolor{comment}{ */}
00032 \textcolor{keywordtype}{int} ZLIB\_INTERNAL inflate\_table(type, lens, codes, table, bits, work)
00033 codetype type;
00034 \textcolor{keywordtype}{unsigned} \textcolor{keywordtype}{short} FAR *lens;
00035 \textcolor{keywordtype}{unsigned} codes;
00036 \hyperlink{structcode}{code} FAR * FAR *table;
00037 \textcolor{keywordtype}{unsigned} FAR *bits;
00038 \textcolor{keywordtype}{unsigned} \textcolor{keywordtype}{short} FAR *work;
00039 \{
00040     \textcolor{keywordtype}{unsigned} len;               \textcolor{comment}{/* a code's length in bits */}
00041     \textcolor{keywordtype}{unsigned} sym;               \textcolor{comment}{/* index of code symbols */}
00042     \textcolor{keywordtype}{unsigned} min, max;          \textcolor{comment}{/* minimum and maximum code lengths */}
00043     \textcolor{keywordtype}{unsigned} root;              \textcolor{comment}{/* number of index bits for root table */}
00044     \textcolor{keywordtype}{unsigned} curr;              \textcolor{comment}{/* number of index bits for current table */}
00045     \textcolor{keywordtype}{unsigned} drop;              \textcolor{comment}{/* code bits to drop for sub-table */}
00046     \textcolor{keywordtype}{int} left;                   \textcolor{comment}{/* number of prefix codes available */}
00047     \textcolor{keywordtype}{unsigned} used;              \textcolor{comment}{/* code entries in table used */}
00048     \textcolor{keywordtype}{unsigned} huff;              \textcolor{comment}{/* Huffman code */}
00049     \textcolor{keywordtype}{unsigned} incr;              \textcolor{comment}{/* for incrementing code, index */}
00050     \textcolor{keywordtype}{unsigned} fill;              \textcolor{comment}{/* index for replicating entries */}
00051     \textcolor{keywordtype}{unsigned} low;               \textcolor{comment}{/* low bits for current root entry */}
00052     \textcolor{keywordtype}{unsigned} mask;              \textcolor{comment}{/* mask for low root bits */}
00053     \hyperlink{structcode}{code} here;                  \textcolor{comment}{/* table entry for duplication */}
00054     \hyperlink{structcode}{code} FAR *next;             \textcolor{comment}{/* next available space in table */}
00055     \textcolor{keyword}{const} \textcolor{keywordtype}{unsigned} \textcolor{keywordtype}{short} FAR *base;     \textcolor{comment}{/* base value table to use */}
00056     \textcolor{keyword}{const} \textcolor{keywordtype}{unsigned} \textcolor{keywordtype}{short} FAR *extra;    \textcolor{comment}{/* extra bits table to use */}
00057     \textcolor{keywordtype}{unsigned} match;             \textcolor{comment}{/* use base and extra for symbol >= match */}
00058     \textcolor{keywordtype}{unsigned} \textcolor{keywordtype}{short} count[MAXBITS+1];    \textcolor{comment}{/* number of codes of each length */}
00059     \textcolor{keywordtype}{unsigned} \textcolor{keywordtype}{short} offs[MAXBITS+1];     \textcolor{comment}{/* offsets in table for each length */}
00060     \textcolor{keyword}{static} \textcolor{keyword}{const} \textcolor{keywordtype}{unsigned} \textcolor{keywordtype}{short} lbase[31] = \{ \textcolor{comment}{/* Length codes 257..285 base */}
00061         3, 4, 5, 6, 7, 8, 9, 10, 11, 13, 15, 17, 19, 23, 27, 31,
00062         35, 43, 51, 59, 67, 83, 99, 115, 131, 163, 195, 227, 258, 0, 0\};
00063     \textcolor{keyword}{static} \textcolor{keyword}{const} \textcolor{keywordtype}{unsigned} \textcolor{keywordtype}{short} lext[31] = \{ \textcolor{comment}{/* Length codes 257..285 extra */}
00064         16, 16, 16, 16, 16, 16, 16, 16, 17, 17, 17, 17, 18, 18, 18, 18,
00065         19, 19, 19, 19, 20, 20, 20, 20, 21, 21, 21, 21, 16, 77, 202\};
00066     \textcolor{keyword}{static} \textcolor{keyword}{const} \textcolor{keywordtype}{unsigned} \textcolor{keywordtype}{short} dbase[32] = \{ \textcolor{comment}{/* Distance codes 0..29 base */}
00067         1, 2, 3, 4, 5, 7, 9, 13, 17, 25, 33, 49, 65, 97, 129, 193,
00068         257, 385, 513, 769, 1025, 1537, 2049, 3073, 4097, 6145,
00069         8193, 12289, 16385, 24577, 0, 0\};
00070     \textcolor{keyword}{static} \textcolor{keyword}{const} \textcolor{keywordtype}{unsigned} \textcolor{keywordtype}{short} dext[32] = \{ \textcolor{comment}{/* Distance codes 0..29 extra */}
00071         16, 16, 16, 16, 17, 17, 18, 18, 19, 19, 20, 20, 21, 21, 22, 22,
00072         23, 23, 24, 24, 25, 25, 26, 26, 27, 27,
00073         28, 28, 29, 29, 64, 64\};
00074 
00075     \textcolor{comment}{/*}
00076 \textcolor{comment}{       Process a set of code lengths to create a canonical Huffman code.  The}
00077 \textcolor{comment}{       code lengths are lens[0..codes-1].  Each length corresponds to the}
00078 \textcolor{comment}{       symbols 0..codes-1.  The Huffman code is generated by first sorting the}
00079 \textcolor{comment}{       symbols by length from short to long, and retaining the symbol order}
00080 \textcolor{comment}{       for codes with equal lengths.  Then the code starts with all zero bits}
00081 \textcolor{comment}{       for the first code of the shortest length, and the codes are integer}
00082 \textcolor{comment}{       increments for the same length, and zeros are appended as the length}
00083 \textcolor{comment}{       increases.  For the deflate format, these bits are stored backwards}
00084 \textcolor{comment}{       from their more natural integer increment ordering, and so when the}
00085 \textcolor{comment}{       decoding tables are built in the large loop below, the integer codes}
00086 \textcolor{comment}{       are incremented backwards.}
00087 \textcolor{comment}{}
00088 \textcolor{comment}{       This routine assumes, but does not check, that all of the entries in}
00089 \textcolor{comment}{       lens[] are in the range 0..MAXBITS.  The caller must assure this.}
00090 \textcolor{comment}{       1..MAXBITS is interpreted as that code length.  zero means that that}
00091 \textcolor{comment}{       symbol does not occur in this code.}
00092 \textcolor{comment}{}
00093 \textcolor{comment}{       The codes are sorted by computing a count of codes for each length,}
00094 \textcolor{comment}{       creating from that a table of starting indices for each length in the}
00095 \textcolor{comment}{       sorted table, and then entering the symbols in order in the sorted}
00096 \textcolor{comment}{       table.  The sorted table is work[], with that space being provided by}
00097 \textcolor{comment}{       the caller.}
00098 \textcolor{comment}{}
00099 \textcolor{comment}{       The length counts are used for other purposes as well, i.e. finding}
00100 \textcolor{comment}{       the minimum and maximum length codes, determining if there are any}
00101 \textcolor{comment}{       codes at all, checking for a valid set of lengths, and looking ahead}
00102 \textcolor{comment}{       at length counts to determine sub-table sizes when building the}
00103 \textcolor{comment}{       decoding tables.}
00104 \textcolor{comment}{     */}
00105 
00106     \textcolor{comment}{/* accumulate lengths for codes (assumes lens[] all in 0..MAXBITS) */}
00107     \textcolor{keywordflow}{for} (len = 0; len <= MAXBITS; len++)
00108         count[len] = 0;
00109     \textcolor{keywordflow}{for} (sym = 0; sym < codes; sym++)
00110         count[lens[sym]]++;
00111 
00112     \textcolor{comment}{/* bound code lengths, force root to be within code lengths */}
00113     root = *bits;
00114     \textcolor{keywordflow}{for} (max = MAXBITS; max >= 1; max--)
00115         \textcolor{keywordflow}{if} (count[max] != 0) \textcolor{keywordflow}{break};
00116     \textcolor{keywordflow}{if} (root > max) root = max;
00117     \textcolor{keywordflow}{if} (max == 0) \{                     \textcolor{comment}{/* no symbols to code at all */}
00118         here.op = (\textcolor{keywordtype}{unsigned} char)64;    \textcolor{comment}{/* invalid code marker */}
00119         here.bits = (\textcolor{keywordtype}{unsigned} char)1;
00120         here.val = (\textcolor{keywordtype}{unsigned} short)0;
00121         *(*table)++ = here;             \textcolor{comment}{/* make a table to force an error */}
00122         *(*table)++ = here;
00123         *bits = 1;
00124         \textcolor{keywordflow}{return} 0;     \textcolor{comment}{/* no symbols, but wait for decoding to report error */}
00125     \}
00126     \textcolor{keywordflow}{for} (min = 1; min < max; min++)
00127         \textcolor{keywordflow}{if} (count[min] != 0) \textcolor{keywordflow}{break};
00128     \textcolor{keywordflow}{if} (root < min) root = min;
00129 
00130     \textcolor{comment}{/* check for an over-subscribed or incomplete set of lengths */}
00131     left = 1;
00132     \textcolor{keywordflow}{for} (len = 1; len <= MAXBITS; len++) \{
00133         left <<= 1;
00134         left -= count[len];
00135         \textcolor{keywordflow}{if} (left < 0) \textcolor{keywordflow}{return} -1;        \textcolor{comment}{/* over-subscribed */}
00136     \}
00137     \textcolor{keywordflow}{if} (left > 0 && (type == CODES || max != 1))
00138         \textcolor{keywordflow}{return} -1;                      \textcolor{comment}{/* incomplete set */}
00139 
00140     \textcolor{comment}{/* generate offsets into symbol table for each length for sorting */}
00141     offs[1] = 0;
00142     \textcolor{keywordflow}{for} (len = 1; len < MAXBITS; len++)
00143         offs[len + 1] = offs[len] + count[len];
00144 
00145     \textcolor{comment}{/* sort symbols by length, by symbol order within each length */}
00146     \textcolor{keywordflow}{for} (sym = 0; sym < codes; sym++)
00147         \textcolor{keywordflow}{if} (lens[sym] != 0) work[offs[lens[sym]]++] = (\textcolor{keywordtype}{unsigned} short)sym;
00148 
00149     \textcolor{comment}{/*}
00150 \textcolor{comment}{       Create and fill in decoding tables.  In this loop, the table being}
00151 \textcolor{comment}{       filled is at next and has curr index bits.  The code being used is huff}
00152 \textcolor{comment}{       with length len.  That code is converted to an index by dropping drop}
00153 \textcolor{comment}{       bits off of the bottom.  For codes where len is less than drop + curr,}
00154 \textcolor{comment}{       those top drop + curr - len bits are incremented through all values to}
00155 \textcolor{comment}{       fill the table with replicated entries.}
00156 \textcolor{comment}{}
00157 \textcolor{comment}{       root is the number of index bits for the root table.  When len exceeds}
00158 \textcolor{comment}{       root, sub-tables are created pointed to by the root entry with an index}
00159 \textcolor{comment}{       of the low root bits of huff.  This is saved in low to check for when a}
00160 \textcolor{comment}{       new sub-table should be started.  drop is zero when the root table is}
00161 \textcolor{comment}{       being filled, and drop is root when sub-tables are being filled.}
00162 \textcolor{comment}{}
00163 \textcolor{comment}{       When a new sub-table is needed, it is necessary to look ahead in the}
00164 \textcolor{comment}{       code lengths to determine what size sub-table is needed.  The length}
00165 \textcolor{comment}{       counts are used for this, and so count[] is decremented as codes are}
00166 \textcolor{comment}{       entered in the tables.}
00167 \textcolor{comment}{}
00168 \textcolor{comment}{       used keeps track of how many table entries have been allocated from the}
00169 \textcolor{comment}{       provided *table space.  It is checked for LENS and DIST tables against}
00170 \textcolor{comment}{       the constants ENOUGH\_LENS and ENOUGH\_DISTS to guard against changes in}
00171 \textcolor{comment}{       the initial root table size constants.  See the comments in inftrees.h}
00172 \textcolor{comment}{       for more information.}
00173 \textcolor{comment}{}
00174 \textcolor{comment}{       sym increments through all symbols, and the loop terminates when}
00175 \textcolor{comment}{       all codes of length max, i.e. all codes, have been processed.  This}
00176 \textcolor{comment}{       routine permits incomplete codes, so another loop after this one fills}
00177 \textcolor{comment}{       in the rest of the decoding tables with invalid code markers.}
00178 \textcolor{comment}{     */}
00179 
00180     \textcolor{comment}{/* set up for code type */}
00181     \textcolor{keywordflow}{switch} (type) \{
00182     \textcolor{keywordflow}{case} CODES:
00183         base = extra = work;    \textcolor{comment}{/* dummy value--not used */}
00184         match = 20;
00185         \textcolor{keywordflow}{break};
00186     \textcolor{keywordflow}{case} LENS:
00187         base = lbase;
00188         extra = lext;
00189         match = 257;
00190         \textcolor{keywordflow}{break};
00191     \textcolor{keywordflow}{default}:    \textcolor{comment}{/* DISTS */}
00192         base = dbase;
00193         extra = dext;
00194         match = 0;
00195     \}
00196 
00197     \textcolor{comment}{/* initialize state for loop */}
00198     huff = 0;                   \textcolor{comment}{/* starting code */}
00199     sym = 0;                    \textcolor{comment}{/* starting code symbol */}
00200     len = min;                  \textcolor{comment}{/* starting code length */}
00201     next = *table;              \textcolor{comment}{/* current table to fill in */}
00202     curr = root;                \textcolor{comment}{/* current table index bits */}
00203     drop = 0;                   \textcolor{comment}{/* current bits to drop from code for index */}
00204     low = (unsigned)(-1);       \textcolor{comment}{/* trigger new sub-table when len > root */}
00205     used = 1U << root;          \textcolor{comment}{/* use root table entries */}
00206     mask = used - 1;            \textcolor{comment}{/* mask for comparing low */}
00207 
00208     \textcolor{comment}{/* check available table space */}
00209     \textcolor{keywordflow}{if} ((type == LENS && used > ENOUGH\_LENS) ||
00210         (type == DISTS && used > ENOUGH\_DISTS))
00211         \textcolor{keywordflow}{return} 1;
00212 
00213     \textcolor{comment}{/* process all codes and make table entries */}
00214     \textcolor{keywordflow}{for} (;;) \{
00215         \textcolor{comment}{/* create table entry */}
00216         here.bits = (\textcolor{keywordtype}{unsigned} char)(len - drop);
00217         \textcolor{keywordflow}{if} (work[sym] + 1U < match) \{
00218             here.op = (\textcolor{keywordtype}{unsigned} char)0;
00219             here.val = work[sym];
00220         \}
00221         \textcolor{keywordflow}{else} \textcolor{keywordflow}{if} (work[sym] >= match) \{
00222             here.op = (\textcolor{keywordtype}{unsigned} char)(extra[work[sym] - match]);
00223             here.val = base[work[sym] - match];
00224         \}
00225         \textcolor{keywordflow}{else} \{
00226             here.op = (\textcolor{keywordtype}{unsigned} char)(32 + 64);         \textcolor{comment}{/* end of block */}
00227             here.val = 0;
00228         \}
00229 
00230         \textcolor{comment}{/* replicate for those indices with low len bits equal to huff */}
00231         incr = 1U << (len - drop);
00232         fill = 1U << curr;
00233         min = fill;                 \textcolor{comment}{/* save offset to next table */}
00234         \textcolor{keywordflow}{do} \{
00235             fill -= incr;
00236             next[(huff >> drop) + fill] = here;
00237         \} \textcolor{keywordflow}{while} (fill != 0);
00238 
00239         \textcolor{comment}{/* backwards increment the len-bit code huff */}
00240         incr = 1U << (len - 1);
00241         \textcolor{keywordflow}{while} (huff & incr)
00242             incr >>= 1;
00243         \textcolor{keywordflow}{if} (incr != 0) \{
00244             huff &= incr - 1;
00245             huff += incr;
00246         \}
00247         \textcolor{keywordflow}{else}
00248             huff = 0;
00249 
00250         \textcolor{comment}{/* go to next symbol, update count, len */}
00251         sym++;
00252         \textcolor{keywordflow}{if} (--(count[len]) == 0) \{
00253             \textcolor{keywordflow}{if} (len == max) \textcolor{keywordflow}{break};
00254             len = lens[work[sym]];
00255         \}
00256 
00257         \textcolor{comment}{/* create new sub-table if needed */}
00258         \textcolor{keywordflow}{if} (len > root && (huff & mask) != low) \{
00259             \textcolor{comment}{/* if first time, transition to sub-tables */}
00260             \textcolor{keywordflow}{if} (drop == 0)
00261                 drop = root;
00262 
00263             \textcolor{comment}{/* increment past last table */}
00264             next += min;            \textcolor{comment}{/* here min is 1 << curr */}
00265 
00266             \textcolor{comment}{/* determine length of next table */}
00267             curr = len - drop;
00268             left = (int)(1 << curr);
00269             \textcolor{keywordflow}{while} (curr + drop < max) \{
00270                 left -= count[curr + drop];
00271                 \textcolor{keywordflow}{if} (left <= 0) \textcolor{keywordflow}{break};
00272                 curr++;
00273                 left <<= 1;
00274             \}
00275 
00276             \textcolor{comment}{/* check for enough space */}
00277             used += 1U << curr;
00278             \textcolor{keywordflow}{if} ((type == LENS && used > ENOUGH\_LENS) ||
00279                 (type == DISTS && used > ENOUGH\_DISTS))
00280                 \textcolor{keywordflow}{return} 1;
00281 
00282             \textcolor{comment}{/* point entry in root table to sub-table */}
00283             low = huff & mask;
00284             (*table)[low].op = (\textcolor{keywordtype}{unsigned} char)curr;
00285             (*table)[low].bits = (\textcolor{keywordtype}{unsigned} char)root;
00286             (*table)[low].val = (\textcolor{keywordtype}{unsigned} short)(next - *table);
00287         \}
00288     \}
00289 
00290     \textcolor{comment}{/* fill in remaining table entry if code is incomplete (guaranteed to have}
00291 \textcolor{comment}{       at most one remaining entry, since if the code is incomplete, the}
00292 \textcolor{comment}{       maximum code length that was allowed to get this far is one bit) */}
00293     \textcolor{keywordflow}{if} (huff != 0) \{
00294         here.op = (\textcolor{keywordtype}{unsigned} char)64;            \textcolor{comment}{/* invalid code marker */}
00295         here.bits = (\textcolor{keywordtype}{unsigned} char)(len - drop);
00296         here.val = (\textcolor{keywordtype}{unsigned} short)0;
00297         next[huff] = here;
00298     \}
00299 
00300     \textcolor{comment}{/* set return parameters */}
00301     *table += used;
00302     *bits = root;
00303     \textcolor{keywordflow}{return} 0;
00304 \}
\end{DoxyCode}
