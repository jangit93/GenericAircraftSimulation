\hypertarget{_h_d_f5_21_810_81_2_h_d_f5_examples_2_c_2_h5_t_2h5ex__t__enum_8c_source}{}\section{H\+D\+F5/1.10.1/\+H\+D\+F5\+Examples/\+C/\+H5\+T/h5ex\+\_\+t\+\_\+enum.c}
\label{_h_d_f5_21_810_81_2_h_d_f5_examples_2_c_2_h5_t_2h5ex__t__enum_8c_source}\index{h5ex\+\_\+t\+\_\+enum.\+c@{h5ex\+\_\+t\+\_\+enum.\+c}}

\begin{DoxyCode}
00001 \textcolor{comment}{/************************************************************}
00002 \textcolor{comment}{}
00003 \textcolor{comment}{  This example shows how to read and write enumerated}
00004 \textcolor{comment}{  datatypes to a dataset.  The program first writes}
00005 \textcolor{comment}{  enumerated values to a dataset with a dataspace of}
00006 \textcolor{comment}{  DIM0xDIM1, then closes the file.  Next, it reopens the}
00007 \textcolor{comment}{  file, reads back the data, and outputs it to the screen.}
00008 \textcolor{comment}{}
00009 \textcolor{comment}{  This file is intended for use with HDF5 Library version 1.8}
00010 \textcolor{comment}{}
00011 \textcolor{comment}{ ************************************************************/}
00012 
00013 \textcolor{preprocessor}{#include "hdf5.h"}
00014 \textcolor{preprocessor}{#include <stdio.h>}
00015 \textcolor{preprocessor}{#include <stdlib.h>}
00016 
00017 \textcolor{preprocessor}{#define FILE            "h5ex\_t\_enum.h5"}
00018 \textcolor{preprocessor}{#define DATASET         "DS1"}
00019 \textcolor{preprocessor}{#define DIM0            4}
00020 \textcolor{preprocessor}{#define DIM1            7}
00021 \textcolor{preprocessor}{#define F\_BASET         H5T\_STD\_I16BE       }\textcolor{comment}{/* File base type */}\textcolor{preprocessor}{}
00022 \textcolor{preprocessor}{#define M\_BASET         H5T\_NATIVE\_INT      }\textcolor{comment}{/* Memory base type */}\textcolor{preprocessor}{}
00023 \textcolor{preprocessor}{#define NAME\_BUF\_SIZE   16}
00024 
00025 \textcolor{keyword}{typedef} \textcolor{keyword}{enum} \{
00026     SOLID,
00027     LIQUID,
00028     GAS,
00029     PLASMA
00030 \} phase\_t;                                  \textcolor{comment}{/* Enumerated type */}
00031 
00032 \textcolor{keywordtype}{int}
00033 main (\textcolor{keywordtype}{void})
00034 \{
00035     hid\_t       \hyperlink{structfile}{file}, filetype, memtype, space, dset;
00036                                             \textcolor{comment}{/* Handles */}
00037     herr\_t      status;
00038     hsize\_t     dims[2] = \{DIM0, DIM1\};
00039     phase\_t     wdata[DIM0][DIM1],          \textcolor{comment}{/* Write buffer */}
00040                 **rdata,                    \textcolor{comment}{/* Read buffer */}
00041                 val;
00042     \textcolor{keywordtype}{char}        *names[4] = \{\textcolor{stringliteral}{"SOLID"}, \textcolor{stringliteral}{"LIQUID"}, \textcolor{stringliteral}{"GAS"}, \textcolor{stringliteral}{"PLASMA"}\},
00043                 name[NAME\_BUF\_SIZE];
00044     \textcolor{keywordtype}{int}         ndims,
00045                 i, j;
00046 
00047     \textcolor{comment}{/*}
00048 \textcolor{comment}{     * Initialize data.}
00049 \textcolor{comment}{     */}
00050     \textcolor{keywordflow}{for} (i=0; i<DIM0; i++)
00051         \textcolor{keywordflow}{for} (j=0; j<DIM1; j++)
00052             wdata[i][j] = (phase\_t) ( (i + 1) * j - j) % (int) (PLASMA + 1);
00053 
00054     \textcolor{comment}{/*}
00055 \textcolor{comment}{     * Create a new file using the default properties.}
00056 \textcolor{comment}{     */}
00057     file = H5Fcreate (FILE, H5F\_ACC\_TRUNC, H5P\_DEFAULT, H5P\_DEFAULT);
00058 
00059     \textcolor{comment}{/*}
00060 \textcolor{comment}{     * Create the enumerated datatypes for file and memory.  This}
00061 \textcolor{comment}{     * process is simplified if native types are used for the file,}
00062 \textcolor{comment}{     * as only one type must be defined.}
00063 \textcolor{comment}{     */}
00064     filetype = H5Tenum\_create (F\_BASET);
00065     memtype = H5Tenum\_create (M\_BASET);
00066 
00067     \textcolor{keywordflow}{for} (i = (\textcolor{keywordtype}{int}) SOLID; i <= (int) PLASMA; i++) \{
00068         \textcolor{comment}{/*}
00069 \textcolor{comment}{         * Insert enumerated value for memtype.}
00070 \textcolor{comment}{         */}
00071         val = (phase\_t) i;
00072         status = H5Tenum\_insert (memtype, names[i], &val);
00073         \textcolor{comment}{/*}
00074 \textcolor{comment}{         * Insert enumerated value for filetype.  We must first convert}
00075 \textcolor{comment}{         * the numerical value val to the base type of the destination.}
00076 \textcolor{comment}{         */}
00077         status = H5Tconvert (M\_BASET, F\_BASET, 1, &val, NULL, H5P\_DEFAULT);
00078         status = H5Tenum\_insert (filetype, names[i], &val);
00079     \}
00080 
00081     \textcolor{comment}{/*}
00082 \textcolor{comment}{     * Create dataspace.  Setting maximum size to NULL sets the maximum}
00083 \textcolor{comment}{     * size to be the current size.}
00084 \textcolor{comment}{     */}
00085     space = H5Screate\_simple (2, dims, NULL);
00086 
00087     \textcolor{comment}{/*}
00088 \textcolor{comment}{     * Create the dataset and write the enumerated data to it.}
00089 \textcolor{comment}{     */}
00090     dset = H5Dcreate (file, DATASET, filetype, space, H5P\_DEFAULT, H5P\_DEFAULT,
00091                 H5P\_DEFAULT);
00092     status = H5Dwrite (dset, memtype, H5S\_ALL, H5S\_ALL, H5P\_DEFAULT, wdata[0]);
00093 
00094     \textcolor{comment}{/*}
00095 \textcolor{comment}{     * Close and release resources.}
00096 \textcolor{comment}{     */}
00097     status = H5Dclose (dset);
00098     status = H5Sclose (space);
00099     status = H5Tclose (filetype);
00100     status = H5Fclose (file);
00101 
00102 
00103     \textcolor{comment}{/*}
00104 \textcolor{comment}{     * Now we begin the read section of this example.  Here we assume}
00105 \textcolor{comment}{     * the dataset has the same name and rank, but can have any size.}
00106 \textcolor{comment}{     * Therefore we must allocate a new array to read in data using}
00107 \textcolor{comment}{     * malloc().  For simplicity, we do not rebuild memtype.}
00108 \textcolor{comment}{     */}
00109 
00110     \textcolor{comment}{/*}
00111 \textcolor{comment}{     * Open file and dataset.}
00112 \textcolor{comment}{     */}
00113     file = H5Fopen (FILE, H5F\_ACC\_RDONLY, H5P\_DEFAULT);
00114     dset = H5Dopen (file, DATASET, H5P\_DEFAULT);
00115 
00116     \textcolor{comment}{/*}
00117 \textcolor{comment}{     * Get dataspace and allocate memory for read buffer.  This is a}
00118 \textcolor{comment}{     * two dimensional dataset so the dynamic allocation must be done}
00119 \textcolor{comment}{     * in steps.}
00120 \textcolor{comment}{     */}
00121     space = H5Dget\_space (dset);
00122     ndims = H5Sget\_simple\_extent\_dims (space, dims, NULL);
00123 
00124     \textcolor{comment}{/*}
00125 \textcolor{comment}{     * Allocate array of pointers to rows.}
00126 \textcolor{comment}{     */}
00127     rdata = (phase\_t **) malloc (dims[0] * \textcolor{keyword}{sizeof} (phase\_t *));
00128 
00129     \textcolor{comment}{/*}
00130 \textcolor{comment}{     * Allocate space for enumerated data.}
00131 \textcolor{comment}{     */}
00132     rdata[0] = (phase\_t *) malloc (dims[0] * dims[1] * \textcolor{keyword}{sizeof} (phase\_t));
00133 
00134     \textcolor{comment}{/*}
00135 \textcolor{comment}{     * Set the rest of the pointers to rows to the correct addresses.}
00136 \textcolor{comment}{     */}
00137     \textcolor{keywordflow}{for} (i=1; i<dims[0]; i++)
00138         rdata[i] = rdata[0] + i * dims[1];
00139 
00140     \textcolor{comment}{/*}
00141 \textcolor{comment}{     * Read the data.}
00142 \textcolor{comment}{     */}
00143     status = H5Dread (dset, memtype, H5S\_ALL, H5S\_ALL, H5P\_DEFAULT, rdata[0]);
00144 
00145     \textcolor{comment}{/*}
00146 \textcolor{comment}{     * Output the data to the screen.}
00147 \textcolor{comment}{     */}
00148     printf (\textcolor{stringliteral}{"%s:\(\backslash\)n"}, DATASET);
00149     \textcolor{keywordflow}{for} (i=0; i<dims[0]; i++) \{
00150         printf (\textcolor{stringliteral}{" ["});
00151         \textcolor{keywordflow}{for} (j=0; j<dims[1]; j++) \{
00152 
00153             \textcolor{comment}{/*}
00154 \textcolor{comment}{             * Get the name of the enumeration member.}
00155 \textcolor{comment}{             */}
00156             status = H5Tenum\_nameof (memtype, &rdata[i][j], name,
00157                         NAME\_BUF\_SIZE);
00158             printf (\textcolor{stringliteral}{" %-6s"}, name);
00159         \}
00160         printf (\textcolor{stringliteral}{"]\(\backslash\)n"});
00161     \}
00162 
00163     \textcolor{comment}{/*}
00164 \textcolor{comment}{     * Close and release resources.}
00165 \textcolor{comment}{     */}
00166     free (rdata[0]);
00167     free (rdata);
00168     status = H5Dclose (dset);
00169     status = H5Sclose (space);
00170     status = H5Tclose (memtype);
00171     status = H5Fclose (file);
00172 
00173     \textcolor{keywordflow}{return} 0;
00174 \}
\end{DoxyCode}
