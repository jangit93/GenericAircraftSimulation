\hypertarget{visual__studio_2_h_d_f5_21_810_81_2include_2_h5_a_cpublic_8h_source}{}\section{visual\+\_\+studio/\+H\+D\+F5/1.10.1/include/\+H5\+A\+Cpublic.h}
\label{visual__studio_2_h_d_f5_21_810_81_2include_2_h5_a_cpublic_8h_source}\index{H5\+A\+Cpublic.\+h@{H5\+A\+Cpublic.\+h}}

\begin{DoxyCode}
00001 \textcolor{comment}{/* * * * * * * * * * * * * * * * * * * * * * * * * * * * * * * * * * * * * * *}
00002 \textcolor{comment}{ * Copyright by The HDF Group.                                               *}
00003 \textcolor{comment}{ * Copyright by the Board of Trustees of the University of Illinois.         *}
00004 \textcolor{comment}{ * All rights reserved.                                                      *}
00005 \textcolor{comment}{ *                                                                           *}
00006 \textcolor{comment}{ * This file is part of HDF5.  The full HDF5 copyright notice, including     *}
00007 \textcolor{comment}{ * terms governing use, modification, and redistribution, is contained in    *}
00008 \textcolor{comment}{ * the COPYING file, which can be found at the root of the source code       *}
00009 \textcolor{comment}{ * distribution tree, or in https://support.hdfgroup.org/ftp/HDF5/releases.  *}
00010 \textcolor{comment}{ * If you do not have access to either file, you may request a copy from     *}
00011 \textcolor{comment}{ * help@hdfgroup.org.                                                        *}
00012 \textcolor{comment}{ * * * * * * * * * * * * * * * * * * * * * * * * * * * * * * * * * * * * * * */}
00013 
00014 \textcolor{comment}{/*-------------------------------------------------------------------------}
00015 \textcolor{comment}{ *}
00016 \textcolor{comment}{ * Created:             H5ACpublic.h}
00017 \textcolor{comment}{ *                      Jul 10 1997}
00018 \textcolor{comment}{ *                      Robb Matzke <matzke@llnl.gov>}
00019 \textcolor{comment}{ *}
00020 \textcolor{comment}{ * Purpose:             Public include file for cache functions.}
00021 \textcolor{comment}{ *}
00022 \textcolor{comment}{ * Modifications:}
00023 \textcolor{comment}{ *}
00024 \textcolor{comment}{ *-------------------------------------------------------------------------}
00025 \textcolor{comment}{ */}
00026 \textcolor{preprocessor}{#ifndef \_H5ACpublic\_H}
00027 \textcolor{preprocessor}{#define \_H5ACpublic\_H}
00028 
00029 \textcolor{comment}{/* Public headers needed by this file */}
00030 \textcolor{preprocessor}{#include "H5public.h"}
00031 \textcolor{preprocessor}{#include "H5Cpublic.h"}
00032 
00033 \textcolor{preprocessor}{#ifdef \_\_cplusplus}
00034 \textcolor{keyword}{extern} \textcolor{stringliteral}{"C"} \{
00035 \textcolor{preprocessor}{#endif}
00036 
00037 \textcolor{comment}{/****************************************************************************}
00038 \textcolor{comment}{ *}
00039 \textcolor{comment}{ * structure H5AC\_cache\_config\_t}
00040 \textcolor{comment}{ *}
00041 \textcolor{comment}{ * H5AC\_cache\_config\_t is a public structure intended for use in public APIs.}
00042 \textcolor{comment}{ * At least in its initial incarnation, it is basicaly a copy of struct}
00043 \textcolor{comment}{ * H5C\_auto\_size\_ctl\_t, minus the report\_fcn field, and plus the}
00044 \textcolor{comment}{ * dirty\_bytes\_threshold field.}
00045 \textcolor{comment}{ *}
00046 \textcolor{comment}{ * The report\_fcn field is omitted, as including it would require us to}
00047 \textcolor{comment}{ * make H5C\_t structure public.}
00048 \textcolor{comment}{ *}
00049 \textcolor{comment}{ * The dirty\_bytes\_threshold field does not appear in H5C\_auto\_size\_ctl\_t,}
00050 \textcolor{comment}{ * as synchronization between caches on different processes is handled at}
00051 \textcolor{comment}{ * the H5AC level, not at the level of H5C.  Note however that there is}
00052 \textcolor{comment}{ * considerable interaction between this value and the other fields in this}
00053 \textcolor{comment}{ * structure.}
00054 \textcolor{comment}{ *}
00055 \textcolor{comment}{ * Similarly, the open\_trace\_file, close\_trace\_file, and trace\_file\_name}
00056 \textcolor{comment}{ * fields do not appear in H5C\_auto\_size\_ctl\_t, as most trace file}
00057 \textcolor{comment}{ * issues are handled at the H5AC level.  The one exception is storage of}
00058 \textcolor{comment}{ * the pointer to the trace file, which is handled by H5C.}
00059 \textcolor{comment}{ *}
00060 \textcolor{comment}{ * The structure is in H5ACpublic.h as we may wish to allow different}
00061 \textcolor{comment}{ * configuration options for metadata and raw data caches.}
00062 \textcolor{comment}{ *}
00063 \textcolor{comment}{ * The fields of the structure are discussed individually below:}
00064 \textcolor{comment}{ *}
00065 \textcolor{comment}{ * version: Integer field containing the version number of this version}
00066 \textcolor{comment}{ *      of the H5AC\_cache\_config\_t structure.  Any instance of}
00067 \textcolor{comment}{ *      H5AC\_cache\_config\_t passed to the cache must have a known}
00068 \textcolor{comment}{ *      version number, or an error will be flagged.}
00069 \textcolor{comment}{ *}
00070 \textcolor{comment}{ * rpt\_fcn\_enabled: Boolean field used to enable and disable the default}
00071 \textcolor{comment}{ *  reporting function.  This function is invoked every time the}
00072 \textcolor{comment}{ *  automatic cache resize code is run, and reports on its activities.}
00073 \textcolor{comment}{ *}
00074 \textcolor{comment}{ *  This is a debugging function, and should normally be turned off.}
00075 \textcolor{comment}{ *}
00076 \textcolor{comment}{ * open\_trace\_file: Boolean field indicating whether the trace\_file\_name}
00077 \textcolor{comment}{ *  field should be used to open a trace file for the cache.}
00078 \textcolor{comment}{ *}
00079 \textcolor{comment}{ *      *** DEPRECATED *** Use H5Fstart/stop logging functions instead}
00080 \textcolor{comment}{ *}
00081 \textcolor{comment}{ *  The trace file is a debuging feature that allow the capture of}
00082 \textcolor{comment}{ *  top level metadata cache requests for purposes of debugging and/or}
00083 \textcolor{comment}{ *  optimization.  This field should normally be set to FALSE, as}
00084 \textcolor{comment}{ *  trace file collection imposes considerable overhead.}
00085 \textcolor{comment}{ *}
00086 \textcolor{comment}{ *  This field should only be set to TRUE when the trace\_file\_name}
00087 \textcolor{comment}{ *  contains the full path of the desired trace file, and either}
00088 \textcolor{comment}{ *  there is no open trace file on the cache, or the close\_trace\_file}
00089 \textcolor{comment}{ *  field is also TRUE.}
00090 \textcolor{comment}{ *}
00091 \textcolor{comment}{ * close\_trace\_file: Boolean field indicating whether the current trace}
00092 \textcolor{comment}{ *  file (if any) should be closed.}
00093 \textcolor{comment}{ *}
00094 \textcolor{comment}{ *      *** DEPRECATED *** Use H5Fstart/stop logging functions instead}
00095 \textcolor{comment}{ *}
00096 \textcolor{comment}{ *  See the above comments on the open\_trace\_file field.  This field}
00097 \textcolor{comment}{ *  should be set to FALSE unless there is an open trace file on the}
00098 \textcolor{comment}{ *  cache that you wish to close.}
00099 \textcolor{comment}{ *}
00100 \textcolor{comment}{ * trace\_file\_name: Full path of the trace file to be opened if the}
00101 \textcolor{comment}{ *  open\_trace\_file field is TRUE.}
00102 \textcolor{comment}{ *}
00103 \textcolor{comment}{ *      *** DEPRECATED *** Use H5Fstart/stop logging functions instead}
00104 \textcolor{comment}{ *}
00105 \textcolor{comment}{ *  In the parallel case, an ascii representation of the mpi rank of}
00106 \textcolor{comment}{ *  the process will be appended to the file name to yield a unique}
00107 \textcolor{comment}{ *  trace file name for each process.}
00108 \textcolor{comment}{ *}
00109 \textcolor{comment}{ *  The length of the path must not exceed H5AC\_\_MAX\_TRACE\_FILE\_NAME\_LEN}
00110 \textcolor{comment}{ *  characters.}
00111 \textcolor{comment}{ *}
00112 \textcolor{comment}{ * evictions\_enabled:  Boolean field used to either report the current}
00113 \textcolor{comment}{ *  evictions enabled status of the cache, or to set the cache's}
00114 \textcolor{comment}{ *  evictions enabled status.}
00115 \textcolor{comment}{ *}
00116 \textcolor{comment}{ *  In general, the metadata cache should always be allowed to}
00117 \textcolor{comment}{ *  evict entries.  However, in some cases it is advantageous to}
00118 \textcolor{comment}{ *  disable evictions briefly, and thereby postpone metadata}
00119 \textcolor{comment}{ *  writes.  However, this must be done with care, as the cache}
00120 \textcolor{comment}{ *  can grow quickly.  If you do this, re-enable evictions as}
00121 \textcolor{comment}{ *  soon as possible and monitor cache size.}
00122 \textcolor{comment}{ *}
00123 \textcolor{comment}{ *  At present, evictions can only be disabled if automatic}
00124 \textcolor{comment}{ *  cache resizing is also disabled (that is, ( incr\_mode ==}
00125 \textcolor{comment}{ *  H5C\_incr\_\_off ) && ( decr\_mode == H5C\_decr\_\_off )).  There}
00126 \textcolor{comment}{ *  is no logical reason why this should be so, but it simplifies}
00127 \textcolor{comment}{ *  implementation and testing, and I can't think of any reason}
00128 \textcolor{comment}{ *  why it would be desireable.  If you can think of one, I'll}
00129 \textcolor{comment}{ *  revisit the issue.}
00130 \textcolor{comment}{ *}
00131 \textcolor{comment}{ * set\_initial\_size: Boolean flag indicating whether the size of the}
00132 \textcolor{comment}{ *      initial size of the cache is to be set to the value given in}
00133 \textcolor{comment}{ *      the initial\_size field.  If set\_initial\_size is FALSE, the}
00134 \textcolor{comment}{ *      initial\_size field is ignored.}
00135 \textcolor{comment}{ *}
00136 \textcolor{comment}{ * initial\_size: If enabled, this field contain the size the cache is}
00137 \textcolor{comment}{ *      to be set to upon receipt of this structure.  Needless to say,}
00138 \textcolor{comment}{ *      initial\_size must lie in the closed interval [min\_size, max\_size].}
00139 \textcolor{comment}{ *}
00140 \textcolor{comment}{ * min\_clean\_fraction: double in the range 0 to 1 indicating the fraction}
00141 \textcolor{comment}{ *      of the cache that is to be kept clean.  This field is only used}
00142 \textcolor{comment}{ *      in parallel mode.  Typical values are 0.1 to 0.5.}
00143 \textcolor{comment}{ *}
00144 \textcolor{comment}{ * max\_size: Maximum size to which the cache can be adjusted.  The}
00145 \textcolor{comment}{ *      supplied value must fall in the closed interval}
00146 \textcolor{comment}{ *      [MIN\_MAX\_CACHE\_SIZE, MAX\_MAX\_CACHE\_SIZE].  Also, max\_size must}
00147 \textcolor{comment}{ *      be greater than or equal to min\_size.}
00148 \textcolor{comment}{ *}
00149 \textcolor{comment}{ * min\_size: Minimum size to which the cache can be adjusted.  The}
00150 \textcolor{comment}{ *      supplied value must fall in the closed interval}
00151 \textcolor{comment}{ *      [H5C\_\_MIN\_MAX\_CACHE\_SIZE, H5C\_\_MAX\_MAX\_CACHE\_SIZE].  Also, min\_size}
00152 \textcolor{comment}{ *      must be less than or equal to max\_size.}
00153 \textcolor{comment}{ *}
00154 \textcolor{comment}{ * epoch\_length: Number of accesses on the cache over which to collect}
00155 \textcolor{comment}{ *      hit rate stats before running the automatic cache resize code,}
00156 \textcolor{comment}{ *      if it is enabled.}
00157 \textcolor{comment}{ *}
00158 \textcolor{comment}{ *      At the end of an epoch, we discard prior hit rate data and start}
00159 \textcolor{comment}{ *      collecting afresh.  The epoch\_length must lie in the closed}
00160 \textcolor{comment}{ *      interval [H5C\_\_MIN\_AR\_EPOCH\_LENGTH, H5C\_\_MAX\_AR\_EPOCH\_LENGTH].}
00161 \textcolor{comment}{ *}
00162 \textcolor{comment}{ *}
00163 \textcolor{comment}{ * Cache size increase control fields:}
00164 \textcolor{comment}{ *}
00165 \textcolor{comment}{ * incr\_mode: Instance of the H5C\_cache\_incr\_mode enumerated type whose}
00166 \textcolor{comment}{ *      value indicates how we determine whether the cache size should be}
00167 \textcolor{comment}{ *      increased.  At present there are two possible values:}
00168 \textcolor{comment}{ *}
00169 \textcolor{comment}{ *      H5C\_incr\_\_off:  Don't attempt to increase the size of the cache}
00170 \textcolor{comment}{ *              automatically.}
00171 \textcolor{comment}{ *}
00172 \textcolor{comment}{ *              When this increment mode is selected, the remaining fields}
00173 \textcolor{comment}{ *              in the cache size increase section ar ignored.}
00174 \textcolor{comment}{ *}
00175 \textcolor{comment}{ *      H5C\_incr\_\_threshold: Attempt to increase the size of the cache}
00176 \textcolor{comment}{ *              whenever the average hit rate over the last epoch drops}
00177 \textcolor{comment}{ *              below the value supplied in the lower\_hr\_threshold}
00178 \textcolor{comment}{ *              field.}
00179 \textcolor{comment}{ *}
00180 \textcolor{comment}{ *              Note that this attempt will fail if the cache is already}
00181 \textcolor{comment}{ *              at its maximum size, or if the cache is not already using}
00182 \textcolor{comment}{ *              all available space.}
00183 \textcolor{comment}{ *}
00184 \textcolor{comment}{ *      Note that you must set decr\_mode to H5C\_incr\_\_off if you}
00185 \textcolor{comment}{ *      disable metadata cache entry evictions.}
00186 \textcolor{comment}{ *}
00187 \textcolor{comment}{ * lower\_hr\_threshold: Lower hit rate threshold.  If the increment mode}
00188 \textcolor{comment}{ *      (incr\_mode) is H5C\_incr\_\_threshold and the hit rate drops below the}
00189 \textcolor{comment}{ *      value supplied in this field in an epoch, increment the cache size by}
00190 \textcolor{comment}{ *      size\_increment.  Note that cache size may not be incremented above}
00191 \textcolor{comment}{ *      max\_size, and that the increment may be further restricted by the}
00192 \textcolor{comment}{ *      max\_increment field if it is enabled.}
00193 \textcolor{comment}{ *}
00194 \textcolor{comment}{ *      When enabled, this field must contain a value in the range [0.0, 1.0].}
00195 \textcolor{comment}{ *      Depending on the incr\_mode selected, it may also have to be less than}
00196 \textcolor{comment}{ *      upper\_hr\_threshold.}
00197 \textcolor{comment}{ *}
00198 \textcolor{comment}{ * increment:  Double containing the multiplier used to derive the new}
00199 \textcolor{comment}{ *      cache size from the old if a cache size increment is triggered.}
00200 \textcolor{comment}{ *      The increment must be greater than 1.0, and should not exceed 2.0.}
00201 \textcolor{comment}{ *}
00202 \textcolor{comment}{ *      The new cache size is obtained my multiplying the current max cache}
00203 \textcolor{comment}{ *      size by the increment, and then clamping to max\_size and to stay}
00204 \textcolor{comment}{ *      within the max\_increment as necessary.}
00205 \textcolor{comment}{ *}
00206 \textcolor{comment}{ * apply\_max\_increment:  Boolean flag indicating whether the max\_increment}
00207 \textcolor{comment}{ *      field should be used to limit the maximum cache size increment.}
00208 \textcolor{comment}{ *}
00209 \textcolor{comment}{ * max\_increment: If enabled by the apply\_max\_increment field described}
00210 \textcolor{comment}{ *      above, this field contains the maximum number of bytes by which the}
00211 \textcolor{comment}{ *      cache size can be increased in a single re-size.}
00212 \textcolor{comment}{ *}
00213 \textcolor{comment}{ * flash\_incr\_mode:  Instance of the H5C\_cache\_flash\_incr\_mode enumerated}
00214 \textcolor{comment}{ *      type whose value indicates whether and by which algorithm we should}
00215 \textcolor{comment}{ *      make flash increases in the size of the cache to accomodate insertion}
00216 \textcolor{comment}{ *      of large entries and large increases in the size of a single entry.}
00217 \textcolor{comment}{ *}
00218 \textcolor{comment}{ *      The addition of the flash increment mode was occasioned by performance}
00219 \textcolor{comment}{ *      problems that appear when a local heap is increased to a size in excess}
00220 \textcolor{comment}{ *      of the current cache size.  While the existing re-size code dealt with}
00221 \textcolor{comment}{ *      this eventually, performance was very bad for the remainder of the}
00222 \textcolor{comment}{ *      epoch.}
00223 \textcolor{comment}{ *}
00224 \textcolor{comment}{ *      At present, there are two possible values for the flash\_incr\_mode:}
00225 \textcolor{comment}{ *}
00226 \textcolor{comment}{ *      H5C\_flash\_incr\_\_off:  Don't perform flash increases in the size of}
00227 \textcolor{comment}{ *              the cache.}
00228 \textcolor{comment}{ *}
00229 \textcolor{comment}{ *      H5C\_flash\_incr\_\_add\_space:  Let x be either the size of a newly}
00230 \textcolor{comment}{ *              newly inserted entry, or the number of bytes by which the}
00231 \textcolor{comment}{ *              size of an existing entry has been increased.}
00232 \textcolor{comment}{ *}
00233 \textcolor{comment}{ *              If}
00234 \textcolor{comment}{ *                      x > flash\_threshold * current max cache size,}
00235 \textcolor{comment}{ *}
00236 \textcolor{comment}{ *              increase the current maximum cache size by x * flash\_multiple}
00237 \textcolor{comment}{ *              less any free space in the cache, and star a new epoch.  For}
00238 \textcolor{comment}{ *              now at least, pay no attention to the maximum increment.}
00239 \textcolor{comment}{ *}
00240 \textcolor{comment}{ *      In both of the above cases, the flash increment pays no attention to}
00241 \textcolor{comment}{ *      the maximum increment (at least in this first incarnation), but DOES}
00242 \textcolor{comment}{ *      stay within max\_size.}
00243 \textcolor{comment}{ *}
00244 \textcolor{comment}{ *      With a little thought, it should be obvious that the above flash}
00245 \textcolor{comment}{ *      cache size increase algorithm is not sufficient for all circumstances}
00246 \textcolor{comment}{ *      -- for example, suppose the user round robins through}
00247 \textcolor{comment}{ *      (1/flash\_threshold) +1 groups, adding one data set to each on each}
00248 \textcolor{comment}{ *      pass.  Then all will increase in size at about the same time, requiring}
00249 \textcolor{comment}{ *      the max cache size to at least double to maintain acceptable}
00250 \textcolor{comment}{ *      performance, however the above flash increment algorithm will not be}
00251 \textcolor{comment}{ *      triggered.}
00252 \textcolor{comment}{ *}
00253 \textcolor{comment}{ *      Hopefully, the add space algorithms detailed above will be sufficient}
00254 \textcolor{comment}{ *      for the performance problems encountered to date.  However, we should}
00255 \textcolor{comment}{ *      expect to revisit the issue.}
00256 \textcolor{comment}{ *}
00257 \textcolor{comment}{ * flash\_multiple: Double containing the multiple described above in the}
00258 \textcolor{comment}{ *      H5C\_flash\_incr\_\_add\_space section of the discussion of the}
00259 \textcolor{comment}{ *      flash\_incr\_mode section.  This field is ignored unless flash\_incr\_mode}
00260 \textcolor{comment}{ *      is H5C\_flash\_incr\_\_add\_space.}
00261 \textcolor{comment}{ *}
00262 \textcolor{comment}{ * flash\_threshold: Double containing the factor by which current max cache}
00263 \textcolor{comment}{ *      size is multiplied to obtain the size threshold for the add\_space flash}
00264 \textcolor{comment}{ *      increment algorithm.  The field is ignored unless flash\_incr\_mode is}
00265 \textcolor{comment}{ *      H5C\_flash\_incr\_\_add\_space.}
00266 \textcolor{comment}{ *}
00267 \textcolor{comment}{ *}
00268 \textcolor{comment}{ * Cache size decrease control fields:}
00269 \textcolor{comment}{ *}
00270 \textcolor{comment}{ * decr\_mode: Instance of the H5C\_cache\_decr\_mode enumerated type whose}
00271 \textcolor{comment}{ *      value indicates how we determine whether the cache size should be}
00272 \textcolor{comment}{ *      decreased.  At present there are four possibilities.}
00273 \textcolor{comment}{ *}
00274 \textcolor{comment}{ *      H5C\_decr\_\_off:  Don't attempt to decrease the size of the cache}
00275 \textcolor{comment}{ *              automatically.}
00276 \textcolor{comment}{ *}
00277 \textcolor{comment}{ *              When this increment mode is selected, the remaining fields}
00278 \textcolor{comment}{ *              in the cache size decrease section are ignored.}
00279 \textcolor{comment}{ *}
00280 \textcolor{comment}{ *      H5C\_decr\_\_threshold: Attempt to decrease the size of the cache}
00281 \textcolor{comment}{ *              whenever the average hit rate over the last epoch rises}
00282 \textcolor{comment}{ *              above the value supplied in the upper\_hr\_threshold}
00283 \textcolor{comment}{ *              field.}
00284 \textcolor{comment}{ *}
00285 \textcolor{comment}{ *      H5C\_decr\_\_age\_out:  At the end of each epoch, search the cache for}
00286 \textcolor{comment}{ *              entries that have not been accessed for at least the number}
00287 \textcolor{comment}{ *              of epochs specified in the epochs\_before\_eviction field, and}
00288 \textcolor{comment}{ *              evict these entries.  Conceptually, the maximum cache size}
00289 \textcolor{comment}{ *              is then decreased to match the new actual cache size.  However,}
00290 \textcolor{comment}{ *              this reduction may be modified by the min\_size, the}
00291 \textcolor{comment}{ *              max\_decrement, and/or the empty\_reserve.}
00292 \textcolor{comment}{ *}
00293 \textcolor{comment}{ *      H5C\_decr\_\_age\_out\_with\_threshold:  Same as age\_out, but we only}
00294 \textcolor{comment}{ *              attempt to reduce the cache size when the hit rate observed}
00295 \textcolor{comment}{ *              over the last epoch exceeds the value provided in the}
00296 \textcolor{comment}{ *              upper\_hr\_threshold field.}
00297 \textcolor{comment}{ *}
00298 \textcolor{comment}{ *      Note that you must set decr\_mode to H5C\_decr\_\_off if you}
00299 \textcolor{comment}{ *      disable metadata cache entry evictions.}
00300 \textcolor{comment}{ *}
00301 \textcolor{comment}{ * upper\_hr\_threshold: Upper hit rate threshold.  The use of this field}
00302 \textcolor{comment}{ *      varies according to the current decr\_mode:}
00303 \textcolor{comment}{ *}
00304 \textcolor{comment}{ *      H5C\_decr\_\_off or H5C\_decr\_\_age\_out:  The value of this field is}
00305 \textcolor{comment}{ *              ignored.}
00306 \textcolor{comment}{ *}
00307 \textcolor{comment}{ *      H5C\_decr\_\_threshold:  If the hit rate exceeds this threshold in any}
00308 \textcolor{comment}{ *              epoch, attempt to decrement the cache size by size\_decrement.}
00309 \textcolor{comment}{ *}
00310 \textcolor{comment}{ *              Note that cache size may not be decremented below min\_size.}
00311 \textcolor{comment}{ *}
00312 \textcolor{comment}{ *              Note also that if the upper\_threshold is 1.0, the cache size}
00313 \textcolor{comment}{ *              will never be reduced.}
00314 \textcolor{comment}{ *}
00315 \textcolor{comment}{ *      H5C\_decr\_\_age\_out\_with\_threshold:  If the hit rate exceeds this}
00316 \textcolor{comment}{ *              threshold in any epoch, attempt to reduce the cache size}
00317 \textcolor{comment}{ *              by evicting entries that have not been accessed for more}
00318 \textcolor{comment}{ *              than the specified number of epochs.}
00319 \textcolor{comment}{ *}
00320 \textcolor{comment}{ * decrement: This field is only used when the decr\_mode is}
00321 \textcolor{comment}{ *      H5C\_decr\_\_threshold.}
00322 \textcolor{comment}{ *}
00323 \textcolor{comment}{ *      The field is a double containing the multiplier used to derive the}
00324 \textcolor{comment}{ *      new cache size from the old if a cache size decrement is triggered.}
00325 \textcolor{comment}{ *      The decrement must be in the range 0.0 (in which case the cache will}
00326 \textcolor{comment}{ *      try to contract to its minimum size) to 1.0 (in which case the}
00327 \textcolor{comment}{ *      cache will never shrink).}
00328 \textcolor{comment}{ *}
00329 \textcolor{comment}{ * apply\_max\_decrement:  Boolean flag used to determine whether decrements}
00330 \textcolor{comment}{ *      in cache size are to be limited by the max\_decrement field.}
00331 \textcolor{comment}{ *}
00332 \textcolor{comment}{ * max\_decrement: Maximum number of bytes by which the cache size can be}
00333 \textcolor{comment}{ *      decreased in a single re-size.  Note that decrements may also be}
00334 \textcolor{comment}{ *      restricted by the min\_size of the cache, and (in age out modes) by}
00335 \textcolor{comment}{ *      the empty\_reserve field.}
00336 \textcolor{comment}{ *}
00337 \textcolor{comment}{ * epochs\_before\_eviction:  Integer field used in H5C\_decr\_\_age\_out and}
00338 \textcolor{comment}{ *      H5C\_decr\_\_age\_out\_with\_threshold decrement modes.}
00339 \textcolor{comment}{ *}
00340 \textcolor{comment}{ *      This field contains the number of epochs an entry must remain}
00341 \textcolor{comment}{ *      unaccessed before it is evicted in an attempt to reduce the}
00342 \textcolor{comment}{ *      cache size.  If applicable, this field must lie in the range}
00343 \textcolor{comment}{ *      [1, H5C\_\_MAX\_EPOCH\_MARKERS].}
00344 \textcolor{comment}{ *}
00345 \textcolor{comment}{ * apply\_empty\_reserve:  Boolean field controlling whether the empty\_reserve}
00346 \textcolor{comment}{ *      field is to be used in computing the new cache size when the}
00347 \textcolor{comment}{ *      decr\_mode is H5C\_decr\_\_age\_out or H5C\_decr\_\_age\_out\_with\_threshold.}
00348 \textcolor{comment}{ *}
00349 \textcolor{comment}{ * empty\_reserve:  To avoid a constant racheting down of cache size by small}
00350 \textcolor{comment}{ *      amounts in the H5C\_decr\_\_age\_out and H5C\_decr\_\_age\_out\_with\_threshold}
00351 \textcolor{comment}{ *      modes, this field allows one to require that any cache size}
00352 \textcolor{comment}{ *      reductions leave the specified fraction of unused space in the cache.}
00353 \textcolor{comment}{ *}
00354 \textcolor{comment}{ *      The value of this field must be in the range [0.0, 1.0].  I would}
00355 \textcolor{comment}{ *      expect typical values to be in the range of 0.01 to 0.1.}
00356 \textcolor{comment}{ *}
00357 \textcolor{comment}{ *}
00358 \textcolor{comment}{ * Parallel Configuration Fields:}
00359 \textcolor{comment}{ *}
00360 \textcolor{comment}{ * In PHDF5, all operations that modify metadata must be executed collectively.}
00361 \textcolor{comment}{ *}
00362 \textcolor{comment}{ * We used to think that this was enough to ensure consistency across the}
00363 \textcolor{comment}{ * metadata caches, but since we allow processes to read metadata individually,}
00364 \textcolor{comment}{ * the order of dirty entries in the LRU list can vary across processes,}
00365 \textcolor{comment}{ * which can result in inconsistencies between the caches.}
00366 \textcolor{comment}{ *}
00367 \textcolor{comment}{ * PHDF5 uses several strategies to prevent such inconsistencies in metadata,}
00368 \textcolor{comment}{ * all of which use the fact that the same stream of dirty metadata is seen}
00369 \textcolor{comment}{ * by all processes for purposes of synchronization.  This is done by }
00370 \textcolor{comment}{ * having each process count the number of bytes of dirty metadata generated,}
00371 \textcolor{comment}{ * and then running a "sync point" whenever this count exceeds a user }
00372 \textcolor{comment}{ * specified threshold (see dirty\_bytes\_threshold below).}
00373 \textcolor{comment}{ *}
00374 \textcolor{comment}{ * The current metadata write strategy is indicated by the }
00375 \textcolor{comment}{ * metadata\_write\_strategy field.  The possible values of this field, along}
00376 \textcolor{comment}{ * with the associated metadata write strategies are discussed below.}
00377 \textcolor{comment}{ *}
00378 \textcolor{comment}{ * dirty\_bytes\_threshold:  Threshold of dirty byte creation used to}
00379 \textcolor{comment}{ *  synchronize updates between caches. (See above for outline and}
00380 \textcolor{comment}{ *  motivation.)}
00381 \textcolor{comment}{ *}
00382 \textcolor{comment}{ *  This value MUST be consistant across all processes accessing the}
00383 \textcolor{comment}{ *  file.  This field is ignored unless HDF5 has been compiled for}
00384 \textcolor{comment}{ *  parallel.}
00385 \textcolor{comment}{ *}
00386 \textcolor{comment}{ * metadata\_write\_strategy: Integer field containing a code indicating the}
00387 \textcolor{comment}{ *  desired metadata write strategy.  The valid values of this field}
00388 \textcolor{comment}{ *  are enumerated and discussed below:}
00389 \textcolor{comment}{ *}
00390 \textcolor{comment}{ *}
00391 \textcolor{comment}{ *  H5AC\_METADATA\_WRITE\_STRATEGY\_\_PROCESS\_0\_ONLY:}
00392 \textcolor{comment}{ *}
00393 \textcolor{comment}{ *  When metadata\_write\_strategy is set to this value, only process }
00394 \textcolor{comment}{ *  zero is allowed to write dirty metadata to disk.  All other }
00395 \textcolor{comment}{ *  processes must retain dirty metadata until they are informed at}
00396 \textcolor{comment}{ *  a sync point that the dirty metadata in question has been written}
00397 \textcolor{comment}{ *  to disk.}
00398 \textcolor{comment}{ *}
00399 \textcolor{comment}{ *  When the sync point is reached (or when there is a user generated}
00400 \textcolor{comment}{ *  flush), process zero flushes sufficient entries to bring it into}
00401 \textcolor{comment}{ *  complience with its min clean size (or flushes all dirty entries in}
00402 \textcolor{comment}{ *  the case of a user generated flush), broad casts the list of }
00403 \textcolor{comment}{ *  entries just cleaned to all the other processes, and then exits}
00404 \textcolor{comment}{ *  the sync point.}
00405 \textcolor{comment}{ *}
00406 \textcolor{comment}{ *  Upon receipt of the broadcast, the other processes mark the indicated}
00407 \textcolor{comment}{ *  entries as clean, and leave the sync point as well.}
00408 \textcolor{comment}{ *}
00409 \textcolor{comment}{ *}
00410 \textcolor{comment}{ *  H5AC\_METADATA\_WRITE\_STRATEGY\_\_DISTRIBUTED:}
00411 \textcolor{comment}{ *}
00412 \textcolor{comment}{ *  In the distributed metadata write strategy, process zero still makes}
00413 \textcolor{comment}{ *  the decisions as to what entries should be flushed, but the actual }
00414 \textcolor{comment}{ *  flushes are distributed across the processes in the computation to }
00415 \textcolor{comment}{ *  the extent possible.}
00416 \textcolor{comment}{ *}
00417 \textcolor{comment}{ *  In this strategy, when a sync point is triggered (either by dirty}
00418 \textcolor{comment}{ *  metadata creation or manual flush), all processes enter a barrier.}
00419 \textcolor{comment}{ *}
00420 \textcolor{comment}{ *  On the other side of the barrier, process 0 constructs an ordered}
00421 \textcolor{comment}{ *  list of the entries to be flushed, and then broadcasts this list}
00422 \textcolor{comment}{ *  to the caches in all the processes.}
00423 \textcolor{comment}{ *}
00424 \textcolor{comment}{ *  All processes then scan the list of entries to be flushed, flushing}
00425 \textcolor{comment}{ *  some, and marking the rest as clean.  The algorithm for this purpose}
00426 \textcolor{comment}{ *  ensures that each entry in the list is flushed exactly once, and }
00427 \textcolor{comment}{ *  all are marked clean in each cache.}
00428 \textcolor{comment}{ *}
00429 \textcolor{comment}{ *  Note that in the case of a flush of the cache, no message passing}
00430 \textcolor{comment}{ *  is necessary, as all processes have the same list of dirty entries, }
00431 \textcolor{comment}{ *  and all of these entries must be flushed.  Thus in this case it is }
00432 \textcolor{comment}{ *  sufficient for each process to sort its list of dirty entries after }
00433 \textcolor{comment}{ *  leaving the initial barrier, and use this list as if it had been }
00434 \textcolor{comment}{ *  received from process zero.}
00435 \textcolor{comment}{ *}
00436 \textcolor{comment}{ *  To avoid possible messages from the past/future, all caches must}
00437 \textcolor{comment}{ *  wait until all caches are done before leaving the sync point.}
00438 \textcolor{comment}{ *      }
00439 \textcolor{comment}{ ****************************************************************************/}
00440 
00441 \textcolor{preprocessor}{#define H5AC\_\_CURR\_CACHE\_CONFIG\_VERSION     1}
00442 \textcolor{preprocessor}{#define H5AC\_\_MAX\_TRACE\_FILE\_NAME\_LEN       1024}
00443 
00444 \textcolor{preprocessor}{#define H5AC\_METADATA\_WRITE\_STRATEGY\_\_PROCESS\_0\_ONLY    0}
00445 \textcolor{preprocessor}{#define H5AC\_METADATA\_WRITE\_STRATEGY\_\_DISTRIBUTED       1}
00446 
00447 \textcolor{keyword}{typedef} \textcolor{keyword}{struct }\hyperlink{struct_h5_a_c__cache__config__t}{H5AC\_cache\_config\_t}
00448 \{
00449     \textcolor{comment}{/* general configuration fields: */}
00450     \textcolor{keywordtype}{int}                      version;
00451 
00452     hbool\_t          rpt\_fcn\_enabled;
00453 
00454     hbool\_t          open\_trace\_file;
00455     hbool\_t                  close\_trace\_file;
00456     \textcolor{keywordtype}{char}                     trace\_file\_name[H5AC\_\_MAX\_TRACE\_FILE\_NAME\_LEN + 1];
00457 
00458     hbool\_t                  evictions\_enabled;
00459 
00460     hbool\_t                  set\_initial\_size;
00461     \textcolor{keywordtype}{size\_t}                   initial\_size;
00462 
00463     \textcolor{keywordtype}{double}                   min\_clean\_fraction;
00464 
00465     \textcolor{keywordtype}{size\_t}                   max\_size;
00466     \textcolor{keywordtype}{size\_t}                   min\_size;
00467 
00468     \textcolor{keywordtype}{long} \textcolor{keywordtype}{int}                 epoch\_length;
00469 
00470 
00471     \textcolor{comment}{/* size increase control fields: */}
00472     \textcolor{keyword}{enum} H5C\_cache\_incr\_mode incr\_mode;
00473 
00474     \textcolor{keywordtype}{double}                   lower\_hr\_threshold;
00475 
00476     \textcolor{keywordtype}{double}                   increment;
00477 
00478     hbool\_t                  apply\_max\_increment;
00479     \textcolor{keywordtype}{size\_t}                   max\_increment;
00480 
00481     \textcolor{keyword}{enum} H5C\_cache\_flash\_incr\_mode      flash\_incr\_mode;
00482     \textcolor{keywordtype}{double}                              flash\_multiple;
00483     \textcolor{keywordtype}{double}                              flash\_threshold;
00484 
00485 
00486     \textcolor{comment}{/* size decrease control fields: */}
00487     \textcolor{keyword}{enum} H5C\_cache\_decr\_mode decr\_mode;
00488 
00489     \textcolor{keywordtype}{double}                   upper\_hr\_threshold;
00490 
00491     \textcolor{keywordtype}{double}                   decrement;
00492 
00493     hbool\_t                  apply\_max\_decrement;
00494     \textcolor{keywordtype}{size\_t}                   max\_decrement;
00495 
00496     \textcolor{keywordtype}{int}                      epochs\_before\_eviction;
00497 
00498     hbool\_t                  apply\_empty\_reserve;
00499     \textcolor{keywordtype}{double}                   empty\_reserve;
00500 
00501 
00502     \textcolor{comment}{/* parallel configuration fields: */}
00503     \textcolor{keywordtype}{size\_t}                   dirty\_bytes\_threshold;
00504     \textcolor{keywordtype}{int}                      metadata\_write\_strategy;
00505 
00506 \} \hyperlink{struct_h5_a_c__cache__config__t}{H5AC\_cache\_config\_t};
00507 
00508 
00509 \textcolor{comment}{/****************************************************************************}
00510 \textcolor{comment}{ *}
00511 \textcolor{comment}{ * structure H5AC\_cache\_image\_config\_t}
00512 \textcolor{comment}{ *}
00513 \textcolor{comment}{ * H5AC\_cache\_image\_ctl\_t is a public structure intended for use in public }
00514 \textcolor{comment}{ * APIs.  At least in its initial incarnation, it is a copy of struct}
00515 \textcolor{comment}{ * H5C\_cache\_image\_ctl\_t.}
00516 \textcolor{comment}{ *}
00517 \textcolor{comment}{ * The fields of the structure are discussed individually below:}
00518 \textcolor{comment}{ *}
00519 \textcolor{comment}{ * version: Integer field containing the version number of this version}
00520 \textcolor{comment}{ *      of the H5C\_image\_ctl\_t structure.  Any instance of}
00521 \textcolor{comment}{ *      H5C\_image\_ctl\_t passed to the cache must have a known}
00522 \textcolor{comment}{ *      version number, or an error will be flagged.}
00523 \textcolor{comment}{ *}
00524 \textcolor{comment}{ * generate\_image:  Boolean flag indicating whether a cache image should}
00525 \textcolor{comment}{ *      be created on file close.}
00526 \textcolor{comment}{ *}
00527 \textcolor{comment}{ * save\_resize\_status:  Boolean flag indicating whether the cache image }
00528 \textcolor{comment}{ *  should include the adaptive cache resize configuration and status.}
00529 \textcolor{comment}{ *  Note that this field is ignored at present.}
00530 \textcolor{comment}{ *}
00531 \textcolor{comment}{ * entry\_ageout:    Integer field indicating the maximum number of }
00532 \textcolor{comment}{ *  times a prefetched entry can appear in subsequent cache images.}
00533 \textcolor{comment}{ *  This field exists to allow the user to avoid the buildup of }
00534 \textcolor{comment}{ *  infrequently used entries in long sequences of cache images.}
00535 \textcolor{comment}{ *}
00536 \textcolor{comment}{ *  The value of this field must lie in the range}
00537 \textcolor{comment}{ *  H5AC\_\_CACHE\_IMAGE\_\_ENTRY\_AGEOUT\_\_NONE (-1) to }
00538 \textcolor{comment}{ *  H5AC\_\_CACHE\_IMAGE\_\_ENTRY\_AGEOUT\_\_MAX (100).}
00539 \textcolor{comment}{ *}
00540 \textcolor{comment}{ *  H5AC\_\_CACHE\_IMAGE\_\_ENTRY\_AGEOUT\_\_NONE means that no limit }
00541 \textcolor{comment}{ *  is imposed on number of times a prefeteched entry can appear}
00542 \textcolor{comment}{ *  in subsequent cache images.}
00543 \textcolor{comment}{ *}
00544 \textcolor{comment}{ *  A value of 0 prevents prefetched entries from being included }
00545 \textcolor{comment}{ *  in cache images.}
00546 \textcolor{comment}{ *}
00547 \textcolor{comment}{ *  Positive integers restrict prefetched entries to the specified}
00548 \textcolor{comment}{ *  number of appearances.}
00549 \textcolor{comment}{ *}
00550 \textcolor{comment}{ *  Note that the number of subsequent cache images that a prefetched}
00551 \textcolor{comment}{ *  entry has appeared in is tracked in an 8 bit field.  Thus, while}
00552 \textcolor{comment}{ *  H5AC\_\_CACHE\_IMAGE\_\_ENTRY\_AGEOUT\_\_MAX can be increased from its }
00553 \textcolor{comment}{ *  current value, any value in excess of 255 will be the functional }
00554 \textcolor{comment}{ *  equivalent of H5AC\_\_CACHE\_IMAGE\_\_ENTRY\_AGEOUT\_\_NONE.}
00555 \textcolor{comment}{ *}
00556 \textcolor{comment}{ ****************************************************************************/}
00557 
00558 \textcolor{preprocessor}{#define H5AC\_\_CURR\_CACHE\_IMAGE\_CONFIG\_VERSION   1}
00559 
00560 \textcolor{preprocessor}{#define H5AC\_\_CACHE\_IMAGE\_\_ENTRY\_AGEOUT\_\_NONE   -1}
00561 \textcolor{preprocessor}{#define H5AC\_\_CACHE\_IMAGE\_\_ENTRY\_AGEOUT\_\_MAX    100}
00562 
00563 \textcolor{keyword}{typedef} \textcolor{keyword}{struct }\hyperlink{struct_h5_a_c__cache__image__config__t}{H5AC\_cache\_image\_config\_t} \{
00564     \textcolor{keywordtype}{int}                                 version;
00565     hbool\_t                             generate\_image;
00566     hbool\_t                             save\_resize\_status;
00567     \textcolor{keywordtype}{int}                                 entry\_ageout;
00568 \} \hyperlink{struct_h5_a_c__cache__image__config__t}{H5AC\_cache\_image\_config\_t};
00569 
00570 \textcolor{preprocessor}{#ifdef \_\_cplusplus}
00571 \}
00572 \textcolor{preprocessor}{#endif}
00573 \textcolor{preprocessor}{#endif}
00574 
\end{DoxyCode}
