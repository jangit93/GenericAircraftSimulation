\hypertarget{visual__studio_2_h_d_f5_21_810_81_2_h_d_f5_examples_2_c_2_h5_t_2h5ex__t__cmpdatt_8c_source}{}\section{visual\+\_\+studio/\+H\+D\+F5/1.10.1/\+H\+D\+F5\+Examples/\+C/\+H5\+T/h5ex\+\_\+t\+\_\+cmpdatt.c}
\label{visual__studio_2_h_d_f5_21_810_81_2_h_d_f5_examples_2_c_2_h5_t_2h5ex__t__cmpdatt_8c_source}\index{h5ex\+\_\+t\+\_\+cmpdatt.\+c@{h5ex\+\_\+t\+\_\+cmpdatt.\+c}}

\begin{DoxyCode}
00001 \textcolor{comment}{/************************************************************}
00002 \textcolor{comment}{}
00003 \textcolor{comment}{  This example shows how to read and write compound}
00004 \textcolor{comment}{  datatypes to an attribute.  The program first writes}
00005 \textcolor{comment}{  compound structures to an attribute with a dataspace of}
00006 \textcolor{comment}{  DIM0, then closes the file.  Next, it reopens the file,}
00007 \textcolor{comment}{  reads back the data, and outputs it to the screen.}
00008 \textcolor{comment}{}
00009 \textcolor{comment}{  This file is intended for use with HDF5 Library version 1.8}
00010 \textcolor{comment}{}
00011 \textcolor{comment}{ ************************************************************/}
00012 
00013 \textcolor{preprocessor}{#include "hdf5.h"}
00014 \textcolor{preprocessor}{#include <stdio.h>}
00015 \textcolor{preprocessor}{#include <stdlib.h>}
00016 
00017 \textcolor{preprocessor}{#define FILE            "h5ex\_t\_cmpdatt.h5"}
00018 \textcolor{preprocessor}{#define DATASET         "DS1"}
00019 \textcolor{preprocessor}{#define ATTRIBUTE       "A1"}
00020 \textcolor{preprocessor}{#define DIM0            4}
00021 
00022 \textcolor{keyword}{typedef} \textcolor{keyword}{struct }\{
00023     \textcolor{keywordtype}{int}     serial\_no;
00024     \textcolor{keywordtype}{char}    *location;
00025     \textcolor{keywordtype}{double}  temperature;
00026     \textcolor{keywordtype}{double}  pressure;
00027 \} \hyperlink{structsensor__t}{sensor\_t};                                 \textcolor{comment}{/* Compound type */}
00028 
00029 \textcolor{keywordtype}{int}
00030 main (\textcolor{keywordtype}{void})
00031 \{
00032     hid\_t       \hyperlink{structfile}{file}, filetype, memtype, strtype, space, dset, attr;
00033                                             \textcolor{comment}{/* Handles */}
00034     herr\_t      status;
00035     hsize\_t     dims[1] = \{DIM0\};
00036     \hyperlink{structsensor__t}{sensor\_t}    wdata[DIM0],                \textcolor{comment}{/* Write buffer */}
00037                 *rdata;                     \textcolor{comment}{/* Read buffer */}
00038     \textcolor{keywordtype}{int}         ndims,
00039                 i;
00040 
00041     \textcolor{comment}{/*}
00042 \textcolor{comment}{     * Initialize data.}
00043 \textcolor{comment}{     */}
00044     wdata[0].serial\_no = 1153;
00045     wdata[0].location = \textcolor{stringliteral}{"Exterior (static)"};
00046     wdata[0].temperature = 53.23;
00047     wdata[0].pressure = 24.57;
00048     wdata[1].serial\_no = 1184;
00049     wdata[1].location = \textcolor{stringliteral}{"Intake"};
00050     wdata[1].temperature = 55.12;
00051     wdata[1].pressure = 22.95;
00052     wdata[2].serial\_no = 1027;
00053     wdata[2].location = \textcolor{stringliteral}{"Intake manifold"};
00054     wdata[2].temperature = 103.55;
00055     wdata[2].pressure = 31.23;
00056     wdata[3].serial\_no = 1313;
00057     wdata[3].location = \textcolor{stringliteral}{"Exhaust manifold"};
00058     wdata[3].temperature = 1252.89;
00059     wdata[3].pressure = 84.11;
00060 
00061     \textcolor{comment}{/*}
00062 \textcolor{comment}{     * Create a new file using the default properties.}
00063 \textcolor{comment}{     */}
00064     file = H5Fcreate (FILE, H5F\_ACC\_TRUNC, H5P\_DEFAULT, H5P\_DEFAULT);
00065 
00066     \textcolor{comment}{/*}
00067 \textcolor{comment}{     * Create variable-length string datatype.}
00068 \textcolor{comment}{     */}
00069     strtype = H5Tcopy (H5T\_C\_S1);
00070     status = H5Tset\_size (strtype, H5T\_VARIABLE);
00071 
00072     \textcolor{comment}{/*}
00073 \textcolor{comment}{     * Create the compound datatype for memory.}
00074 \textcolor{comment}{     */}
00075     memtype = H5Tcreate (H5T\_COMPOUND, \textcolor{keyword}{sizeof} (\hyperlink{structsensor__t}{sensor\_t}));
00076     status = H5Tinsert (memtype, \textcolor{stringliteral}{"Serial number"},
00077                 HOFFSET (\hyperlink{structsensor__t}{sensor\_t}, serial\_no), H5T\_NATIVE\_INT);
00078     status = H5Tinsert (memtype, \textcolor{stringliteral}{"Location"}, HOFFSET (\hyperlink{structsensor__t}{sensor\_t}, location),
00079                 strtype);
00080     status = H5Tinsert (memtype, \textcolor{stringliteral}{"Temperature (F)"},
00081                 HOFFSET (\hyperlink{structsensor__t}{sensor\_t}, temperature), H5T\_NATIVE\_DOUBLE);
00082     status = H5Tinsert (memtype, \textcolor{stringliteral}{"Pressure (inHg)"},
00083                 HOFFSET (\hyperlink{structsensor__t}{sensor\_t}, pressure), H5T\_NATIVE\_DOUBLE);
00084 
00085     \textcolor{comment}{/*}
00086 \textcolor{comment}{     * Create the compound datatype for the file.  Because the standard}
00087 \textcolor{comment}{     * types we are using for the file may have different sizes than}
00088 \textcolor{comment}{     * the corresponding native types, we must manually calculate the}
00089 \textcolor{comment}{     * offset of each member.}
00090 \textcolor{comment}{     */}
00091     filetype = H5Tcreate (H5T\_COMPOUND, 8 + \textcolor{keyword}{sizeof} (\hyperlink{structhvl__t}{hvl\_t}) + 8 + 8);
00092     status = H5Tinsert (filetype, \textcolor{stringliteral}{"Serial number"}, 0, H5T\_STD\_I64BE);
00093     status = H5Tinsert (filetype, \textcolor{stringliteral}{"Location"}, 8, strtype);
00094     status = H5Tinsert (filetype, \textcolor{stringliteral}{"Temperature (F)"}, 8 + \textcolor{keyword}{sizeof} (\hyperlink{structhvl__t}{hvl\_t}),
00095                 H5T\_IEEE\_F64BE);
00096     status = H5Tinsert (filetype, \textcolor{stringliteral}{"Pressure (inHg)"}, 8 + \textcolor{keyword}{sizeof} (\hyperlink{structhvl__t}{hvl\_t}) + 8,
00097                 H5T\_IEEE\_F64BE);
00098 
00099     \textcolor{comment}{/*}
00100 \textcolor{comment}{     * Create dataset with a null dataspace.}
00101 \textcolor{comment}{     */}
00102     space = H5Screate (H5S\_NULL);
00103     dset = H5Dcreate (file, DATASET, H5T\_STD\_I32LE, space, H5P\_DEFAULT,
00104                 H5P\_DEFAULT, H5P\_DEFAULT);
00105     status = H5Sclose (space);
00106 
00107     \textcolor{comment}{/*}
00108 \textcolor{comment}{     * Create dataspace.  Setting maximum size to NULL sets the maximum}
00109 \textcolor{comment}{     * size to be the current size.}
00110 \textcolor{comment}{     */}
00111     space = H5Screate\_simple (1, dims, NULL);
00112 
00113     \textcolor{comment}{/*}
00114 \textcolor{comment}{     * Create the attribute and write the compound data to it.}
00115 \textcolor{comment}{     */}
00116     attr = H5Acreate (dset, ATTRIBUTE, filetype, space, H5P\_DEFAULT,
00117                 H5P\_DEFAULT);
00118     status = H5Awrite (attr, memtype, wdata);
00119 
00120     \textcolor{comment}{/*}
00121 \textcolor{comment}{     * Close and release resources.}
00122 \textcolor{comment}{     */}
00123     status = H5Aclose (attr);
00124     status = H5Dclose (dset);
00125     status = H5Sclose (space);
00126     status = H5Tclose (filetype);
00127     status = H5Fclose (file);
00128 
00129 
00130     \textcolor{comment}{/*}
00131 \textcolor{comment}{     * Now we begin the read section of this example.  Here we assume}
00132 \textcolor{comment}{     * the attribute has the same name and rank, but can have any size.}
00133 \textcolor{comment}{     * Therefore we must allocate a new array to read in data using}
00134 \textcolor{comment}{     * malloc().  For simplicity, we do not rebuild memtype.}
00135 \textcolor{comment}{     */}
00136 
00137     \textcolor{comment}{/*}
00138 \textcolor{comment}{     * Open file, dataset, and attribute.}
00139 \textcolor{comment}{     */}
00140     file = H5Fopen (FILE, H5F\_ACC\_RDONLY, H5P\_DEFAULT);
00141     dset = H5Dopen (file, DATASET, H5P\_DEFAULT);
00142     attr = H5Aopen (dset, ATTRIBUTE, H5P\_DEFAULT);
00143 
00144     \textcolor{comment}{/*}
00145 \textcolor{comment}{     * Get dataspace and allocate memory for read buffer.}
00146 \textcolor{comment}{     */}
00147     space = H5Aget\_space (attr);
00148     ndims = H5Sget\_simple\_extent\_dims (space, dims, NULL);
00149     rdata = (\hyperlink{structsensor__t}{sensor\_t} *) malloc (dims[0] * \textcolor{keyword}{sizeof} (\hyperlink{structsensor__t}{sensor\_t}));
00150 
00151     \textcolor{comment}{/*}
00152 \textcolor{comment}{     * Read the data.}
00153 \textcolor{comment}{     */}
00154     status = H5Aread (attr, memtype, rdata);
00155 
00156     \textcolor{comment}{/*}
00157 \textcolor{comment}{     * Output the data to the screen.}
00158 \textcolor{comment}{     */}
00159     \textcolor{keywordflow}{for} (i=0; i<dims[0]; i++) \{
00160         printf (\textcolor{stringliteral}{"%s[%d]:\(\backslash\)n"}, ATTRIBUTE, i);
00161         printf (\textcolor{stringliteral}{"Serial number   : %d\(\backslash\)n"}, rdata[i].serial\_no);
00162         printf (\textcolor{stringliteral}{"Location        : %s\(\backslash\)n"}, rdata[i].location);
00163         printf (\textcolor{stringliteral}{"Temperature (F) : %f\(\backslash\)n"}, rdata[i].temperature);
00164         printf (\textcolor{stringliteral}{"Pressure (inHg) : %f\(\backslash\)n\(\backslash\)n"}, rdata[i].pressure);
00165     \}
00166 
00167     \textcolor{comment}{/*}
00168 \textcolor{comment}{     * Close and release resources.  H5Dvlen\_reclaim will automatically}
00169 \textcolor{comment}{     * traverse the structure and free any vlen data (strings in this}
00170 \textcolor{comment}{     * case).}
00171 \textcolor{comment}{     */}
00172     status = H5Dvlen\_reclaim (memtype, space, H5P\_DEFAULT, rdata);
00173     free (rdata);
00174     status = H5Aclose (attr);
00175     status = H5Dclose (dset);
00176     status = H5Sclose (space);
00177     status = H5Tclose (memtype);
00178     status = H5Tclose (strtype);
00179     status = H5Fclose (file);
00180 
00181     \textcolor{keywordflow}{return} 0;
00182 \}
\end{DoxyCode}
