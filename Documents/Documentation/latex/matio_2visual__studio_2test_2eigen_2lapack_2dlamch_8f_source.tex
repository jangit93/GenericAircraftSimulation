\hypertarget{matio_2visual__studio_2test_2eigen_2lapack_2dlamch_8f_source}{}\section{matio/visual\+\_\+studio/test/eigen/lapack/dlamch.f}
\label{matio_2visual__studio_2test_2eigen_2lapack_2dlamch_8f_source}\index{dlamch.\+f@{dlamch.\+f}}

\begin{DoxyCode}
00001 \textcolor{comment}{*> \(\backslash\)brief \(\backslash\)b DLAMCH}
00002 \textcolor{comment}{*}
00003 \textcolor{comment}{*  =========== DOCUMENTATION ===========}
00004 \textcolor{comment}{*}
00005 \textcolor{comment}{* Online html documentation available at }
00006 \textcolor{comment}{*            http://www.netlib.org/lapack/explore-html/ }
00007 \textcolor{comment}{*}
00008 \textcolor{comment}{*  Definition:}
00009 \textcolor{comment}{*  ===========}
00010 \textcolor{comment}{*}
00011 \textcolor{comment}{*      DOUBLE PRECISION FUNCTION DLAMCH( CMACH )}
00012 \textcolor{comment}{*  }
00013 \textcolor{comment}{*}
00014 \textcolor{comment}{*> \(\backslash\)par Purpose:}
00015 \textcolor{comment}{*  =============}
00016 \textcolor{comment}{*>}
00017 \textcolor{comment}{*> \(\backslash\)verbatim}
00018 \textcolor{comment}{*>}
00019 \textcolor{comment}{*> DLAMCH determines double precision machine parameters.}
00020 \textcolor{comment}{*> \(\backslash\)endverbatim}
00021 \textcolor{comment}{*}
00022 \textcolor{comment}{*  Arguments:}
00023 \textcolor{comment}{*  ==========}
00024 \textcolor{comment}{*}
00025 \textcolor{comment}{*> \(\backslash\)param[in] CMACH}
00026 \textcolor{comment}{*> \(\backslash\)verbatim}
00027 \textcolor{comment}{*>          Specifies the value to be returned by DLAMCH:}
00028 \textcolor{comment}{*>          = 'E' or 'e',   DLAMCH := eps}
00029 \textcolor{comment}{*>          = 'S' or 's ,   DLAMCH := sfmin}
00030 \textcolor{comment}{*>          = 'B' or 'b',   DLAMCH := base}
00031 \textcolor{comment}{*>          = 'P' or 'p',   DLAMCH := eps*base}
00032 \textcolor{comment}{*>          = 'N' or 'n',   DLAMCH := t}
00033 \textcolor{comment}{*>          = 'R' or 'r',   DLAMCH := rnd}
00034 \textcolor{comment}{*>          = 'M' or 'm',   DLAMCH := emin}
00035 \textcolor{comment}{*>          = 'U' or 'u',   DLAMCH := rmin}
00036 \textcolor{comment}{*>          = 'L' or 'l',   DLAMCH := emax}
00037 \textcolor{comment}{*>          = 'O' or 'o',   DLAMCH := rmax}
00038 \textcolor{comment}{*>          where}
00039 \textcolor{comment}{*>          eps   = relative machine precision}
00040 \textcolor{comment}{*>          sfmin = safe minimum, such that 1/sfmin does not overflow}
00041 \textcolor{comment}{*>          base  = base of the machine}
00042 \textcolor{comment}{*>          prec  = eps*base}
00043 \textcolor{comment}{*>          t     = number of (base) digits in the mantissa}
00044 \textcolor{comment}{*>          rnd   = 1.0 when rounding occurs in addition, 0.0 otherwise}
00045 \textcolor{comment}{*>          emin  = minimum exponent before (gradual) underflow}
00046 \textcolor{comment}{*>          rmin  = underflow threshold - base**(emin-1)}
00047 \textcolor{comment}{*>          emax  = largest exponent before overflow}
00048 \textcolor{comment}{*>          rmax  = overflow threshold  - (base**emax)*(1-eps)}
00049 \textcolor{comment}{*> \(\backslash\)endverbatim}
00050 \textcolor{comment}{*}
00051 \textcolor{comment}{*  Authors:}
00052 \textcolor{comment}{*  ========}
00053 \textcolor{comment}{*}
00054 \textcolor{comment}{*> \(\backslash\)author Univ. of Tennessee }
00055 \textcolor{comment}{*> \(\backslash\)author Univ. of California Berkeley }
00056 \textcolor{comment}{*> \(\backslash\)author Univ. of Colorado Denver }
00057 \textcolor{comment}{*> \(\backslash\)author NAG Ltd. }
00058 \textcolor{comment}{*}
00059 \textcolor{comment}{*> \(\backslash\)date November 2011}
00060 \textcolor{comment}{*}
00061 \textcolor{comment}{*> \(\backslash\)ingroup auxOTHERauxiliary}
00062 \textcolor{comment}{*}
00063 \textcolor{comment}{*  =====================================================================}
00064 \textcolor{keyword}{      DOUBLE PRECISION }\textcolor{keyword}{FUNCTION }dlamch( CMACH )
00065 \textcolor{comment}{*}
00066 \textcolor{comment}{*  -- LAPACK auxiliary routine (version 3.4.0) --}
00067 \textcolor{comment}{*  -- LAPACK is a software package provided by Univ. of Tennessee,    --}
00068 \textcolor{comment}{*  -- Univ. of California Berkeley, Univ. of Colorado Denver and NAG Ltd..--}
00069 \textcolor{comment}{*     November 2011}
00070 \textcolor{comment}{*}
00071 \textcolor{comment}{*     .. Scalar Arguments ..}
00072       \textcolor{keywordtype}{CHARACTER}          cmach
00073 \textcolor{comment}{*     ..}
00074 \textcolor{comment}{*}
00075 \textcolor{comment}{* =====================================================================}
00076 \textcolor{comment}{*}
00077 \textcolor{comment}{*     .. Parameters ..}
00078       \textcolor{keywordtype}{DOUBLE PRECISION}   one, zero
00079       parameter( one = 1.0d+0, zero = 0.0d+0 )
00080 \textcolor{comment}{*     ..}
00081 \textcolor{comment}{*     .. Local Scalars ..}
00082       \textcolor{keywordtype}{DOUBLE PRECISION}   rnd, eps, sfmin, small, rmach
00083 \textcolor{comment}{*     ..}
00084 \textcolor{comment}{*     .. External Functions ..}
00085       \textcolor{keywordtype}{LOGICAL}            lsame
00086       \textcolor{keywordtype}{EXTERNAL}           lsame
00087 \textcolor{comment}{*     ..}
00088 \textcolor{comment}{*     .. Intrinsic Functions ..}
00089       \textcolor{keywordtype}{INTRINSIC}          digits, epsilon, huge, maxexponent,
00090      $                   minexponent, radix, tiny
00091 \textcolor{comment}{*     ..}
00092 \textcolor{comment}{*     .. Executable Statements ..}
00093 \textcolor{comment}{*}
00094 \textcolor{comment}{*}
00095 \textcolor{comment}{*     Assume rounding, not chopping. Always.}
00096 \textcolor{comment}{*}
00097       rnd = one
00098 \textcolor{comment}{*}
00099       \textcolor{keywordflow}{IF}( one.EQ.rnd ) \textcolor{keywordflow}{THEN}
00100          eps = epsilon(zero) * 0.5
00101       \textcolor{keywordflow}{ELSE}
00102          eps = epsilon(zero)
00103 \textcolor{keywordflow}{      END IF}
00104 \textcolor{comment}{*}
00105       \textcolor{keywordflow}{IF}( lsame( cmach, \textcolor{stringliteral}{'E'} ) ) \textcolor{keywordflow}{THEN}
00106          rmach = eps
00107       \textcolor{keywordflow}{ELSE} \textcolor{keywordflow}{IF}( lsame( cmach, \textcolor{stringliteral}{'S'} ) ) \textcolor{keywordflow}{THEN}
00108          sfmin = tiny(zero)
00109          small = one / huge(zero)
00110          \textcolor{keywordflow}{IF}( small.GE.sfmin ) \textcolor{keywordflow}{THEN}
00111 \textcolor{comment}{*}
00112 \textcolor{comment}{*           Use SMALL plus a bit, to avoid the possibility of rounding}
00113 \textcolor{comment}{*           causing overflow when computing  1/sfmin.}
00114 \textcolor{comment}{*}
00115             sfmin = small*( one+eps )
00116 \textcolor{keywordflow}{         END IF}
00117          rmach = sfmin
00118       \textcolor{keywordflow}{ELSE} \textcolor{keywordflow}{IF}( lsame( cmach, \textcolor{stringliteral}{'B'} ) ) \textcolor{keywordflow}{THEN}
00119          rmach = radix(zero)
00120       \textcolor{keywordflow}{ELSE} \textcolor{keywordflow}{IF}( lsame( cmach, \textcolor{stringliteral}{'P'} ) ) \textcolor{keywordflow}{THEN}
00121          rmach = eps * radix(zero)
00122       \textcolor{keywordflow}{ELSE} \textcolor{keywordflow}{IF}( lsame( cmach, \textcolor{stringliteral}{'N'} ) ) \textcolor{keywordflow}{THEN}
00123          rmach = digits(zero)
00124       \textcolor{keywordflow}{ELSE} \textcolor{keywordflow}{IF}( lsame( cmach, \textcolor{stringliteral}{'R'} ) ) \textcolor{keywordflow}{THEN}
00125          rmach = rnd
00126       \textcolor{keywordflow}{ELSE} \textcolor{keywordflow}{IF}( lsame( cmach, \textcolor{stringliteral}{'M'} ) ) \textcolor{keywordflow}{THEN}
00127          rmach = minexponent(zero)
00128       \textcolor{keywordflow}{ELSE} \textcolor{keywordflow}{IF}( lsame( cmach, \textcolor{stringliteral}{'U'} ) ) \textcolor{keywordflow}{THEN}
00129          rmach = tiny(zero)
00130       \textcolor{keywordflow}{ELSE} \textcolor{keywordflow}{IF}( lsame( cmach, \textcolor{stringliteral}{'L'} ) ) \textcolor{keywordflow}{THEN}
00131          rmach = maxexponent(zero)
00132       \textcolor{keywordflow}{ELSE} \textcolor{keywordflow}{IF}( lsame( cmach, \textcolor{stringliteral}{'O'} ) ) \textcolor{keywordflow}{THEN}
00133          rmach = huge(zero)
00134       \textcolor{keywordflow}{ELSE}
00135          rmach = zero
00136 \textcolor{keywordflow}{      END IF}
00137 \textcolor{comment}{*}
00138       dlamch = rmach
00139       \textcolor{keywordflow}{RETURN}
00140 \textcolor{comment}{*}
00141 \textcolor{comment}{*     End of DLAMCH}
00142 \textcolor{comment}{*}
00143 \textcolor{keyword}{      END}
00144 \textcolor{comment}{************************************************************************}
00145 \textcolor{comment}{*> \(\backslash\)brief \(\backslash\)b DLAMC3}
00146 \textcolor{comment}{*> \(\backslash\)details}
00147 \textcolor{comment}{*> \(\backslash\)b Purpose:}
00148 \textcolor{comment}{*> \(\backslash\)verbatim}
00149 \textcolor{comment}{*> DLAMC3  is intended to force  A  and  B  to be stored prior to doing}
00150 \textcolor{comment}{*> the addition of  A  and  B ,  for use in situations where optimizers}
00151 \textcolor{comment}{*> might hold one of these in a register.}
00152 \textcolor{comment}{*> \(\backslash\)endverbatim}
00153 \textcolor{comment}{*> \(\backslash\)author LAPACK is a software package provided by Univ. of Tennessee, Univ. of California Berkeley, Univ.
       of Colorado Denver and NAG Ltd..}
00154 \textcolor{comment}{*> \(\backslash\)date November 2011}
00155 \textcolor{comment}{*> \(\backslash\)ingroup auxOTHERauxiliary}
00156 \textcolor{comment}{*>}
00157 \textcolor{comment}{*> \(\backslash\)param[in] A}
00158 \textcolor{comment}{*> \(\backslash\)verbatim}
00159 \textcolor{comment}{*>          A is a DOUBLE PRECISION}
00160 \textcolor{comment}{*> \(\backslash\)endverbatim}
00161 \textcolor{comment}{*>}
00162 \textcolor{comment}{*> \(\backslash\)param[in] B}
00163 \textcolor{comment}{*> \(\backslash\)verbatim}
00164 \textcolor{comment}{*>          B is a DOUBLE PRECISION}
00165 \textcolor{comment}{*>          The values A and B.}
00166 \textcolor{comment}{*> \(\backslash\)endverbatim}
00167 \textcolor{comment}{*>}
00168 \textcolor{keyword}{      DOUBLE PRECISION }\textcolor{keyword}{FUNCTION }dlamc3( A, B )
00169 \textcolor{comment}{*}
00170 \textcolor{comment}{*  -- LAPACK auxiliary routine (version 3.4.0) --}
00171 \textcolor{comment}{*     Univ. of Tennessee, Univ. of California Berkeley and NAG Ltd..}
00172 \textcolor{comment}{*     November 2010}
00173 \textcolor{comment}{*}
00174 \textcolor{comment}{*     .. Scalar Arguments ..}
00175       \textcolor{keywordtype}{DOUBLE PRECISION}   a, b
00176 \textcolor{comment}{*     ..}
00177 \textcolor{comment}{* =====================================================================}
00178 \textcolor{comment}{*}
00179 \textcolor{comment}{*     .. Executable Statements ..}
00180 \textcolor{comment}{*}
00181       dlamc3 = a + b
00182 \textcolor{comment}{*}
00183       \textcolor{keywordflow}{RETURN}
00184 \textcolor{comment}{*}
00185 \textcolor{comment}{*     End of DLAMC3}
00186 \textcolor{comment}{*}
00187 \textcolor{keyword}{      END}
00188 \textcolor{comment}{*}
00189 \textcolor{comment}{************************************************************************}
\end{DoxyCode}
