\hypertarget{_h_d_f5_21_810_81_2_h_d_f5_examples_2_f_o_r_t_r_a_n_2_h5_t_2h5ex__t__string___f03_8f90_source}{}\section{H\+D\+F5/1.10.1/\+H\+D\+F5\+Examples/\+F\+O\+R\+T\+R\+A\+N/\+H5\+T/h5ex\+\_\+t\+\_\+string\+\_\+\+F03.f90}
\label{_h_d_f5_21_810_81_2_h_d_f5_examples_2_f_o_r_t_r_a_n_2_h5_t_2h5ex__t__string___f03_8f90_source}\index{h5ex\+\_\+t\+\_\+string\+\_\+\+F03.\+f90@{h5ex\+\_\+t\+\_\+string\+\_\+\+F03.\+f90}}

\begin{DoxyCode}
00001 \textcolor{comment}{!************************************************************}
00002 \textcolor{comment}{!}
00003 \textcolor{comment}{!  This example shows how to read and write fixed string size}
00004 \textcolor{comment}{!  datatypes to a dataset.  The program first writes strings to a}
00005 \textcolor{comment}{!  dataset with a dataspace of DIM0, then closes the file.}
00006 \textcolor{comment}{!  Next, it reopens the file, reads back the data, and}
00007 \textcolor{comment}{!  outputs it to the screen.}
00008 \textcolor{comment}{!}
00009 \textcolor{comment}{!  This file is intended for use with HDF5 Library version 1.8}
00010 \textcolor{comment}{!  with --enable-fortran2003 }
00011 \textcolor{comment}{!}
00012 \textcolor{comment}{! ************************************************************/}
00013 \textcolor{keyword}{PROGRAM} main
00014 
00015   \textcolor{keywordtype}{USE }hdf5
00016   \textcolor{keywordtype}{USE }iso\_c\_binding
00017 
00018   \textcolor{keywordtype}{IMPLICIT NONE}
00019 
00020   \textcolor{keywordtype}{CHARACTER(LEN=20)}, \textcolor{keywordtype}{PARAMETER} :: filename  = \textcolor{stringliteral}{"h5ex\_t\_string\_F03.h5"}
00021   \textcolor{keywordtype}{CHARACTER(LEN=3)} , \textcolor{keywordtype}{PARAMETER} :: dataset   = \textcolor{stringliteral}{"DS1"}
00022   \textcolor{keywordtype}{INTEGER}          , \textcolor{keywordtype}{PARAMETER} :: dim0      = 4
00023   \textcolor{keywordtype}{INTEGER(SIZE\_T)}  , \textcolor{keywordtype}{PARAMETER} :: sdim      = 8 
00024 
00025   \textcolor{keywordtype}{INTEGER(HID\_T)}  :: \hyperlink{structfile}{file}, filetype, memtype, space, dset \textcolor{comment}{! Handles}
00026   \textcolor{keywordtype}{INTEGER} :: hdferr
00027 
00028   \textcolor{keywordtype}{INTEGER(HSIZE\_T)}, \textcolor{keywordtype}{DIMENSION(1:1)} :: dims = (/dim0/)
00029   \textcolor{keywordtype}{INTEGER(HSIZE\_T)}, \textcolor{keywordtype}{DIMENSION(1:1)} :: maxdims
00030 
00031   \textcolor{keywordtype}{CHARACTER(LEN=sdim)}, \textcolor{keywordtype}{DIMENSION(1:dim0)}, \textcolor{keywordtype}{TARGET} :: &
00032        wdata = (/\textcolor{stringliteral}{"Parting"}, \textcolor{stringliteral}{"is such"}, \textcolor{stringliteral}{"sweet  "}, \textcolor{stringliteral}{"sorrow."}/)
00033   \textcolor{keywordtype}{CHARACTER(LEN=sdim)}, \textcolor{keywordtype}{DIMENSION(:)}, \textcolor{keywordtype}{ALLOCATABLE}, \textcolor{keywordtype}{TARGET} :: rdata
00034   \textcolor{keywordtype}{INTEGER} :: i, len
00035   \textcolor{keywordtype}{INTEGER(SIZE\_T)} :: size
00036   \textcolor{keywordtype}{TYPE}(c\_ptr) :: f\_ptr
00037   \textcolor{comment}{!}
00038   \textcolor{comment}{! Initialize FORTRAN interface.}
00039   \textcolor{comment}{!}
00040   \textcolor{keyword}{CALL }h5open\_f(hdferr)
00041   \textcolor{comment}{!}
00042   \textcolor{comment}{! Create a new file using the default properties.}
00043   \textcolor{comment}{!}
00044   \textcolor{keyword}{CALL }h5fcreate\_f(filename, h5f\_acc\_trunc\_f, \hyperlink{structfile}{file}, hdferr)
00045   \textcolor{comment}{!}
00046   \textcolor{comment}{! Create file datatypes.  For this example we will save}
00047   \textcolor{comment}{! the strings as FORTRAN strings}
00048   \textcolor{comment}{!}
00049   \textcolor{keyword}{CALL }h5tcopy\_f(h5t\_fortran\_s1, filetype, hdferr)
00050   \textcolor{keyword}{CALL }h5tset\_size\_f(filetype, sdim, hdferr)
00051   \textcolor{comment}{!}
00052   \textcolor{comment}{! Create dataspace.}
00053   \textcolor{comment}{!}
00054   \textcolor{keyword}{CALL }h5screate\_simple\_f(1, dims, space, hdferr)
00055   \textcolor{comment}{!}
00056   \textcolor{comment}{! Create the dataset and write the string data to it.}
00057   \textcolor{comment}{!}
00058   \textcolor{keyword}{CALL }h5dcreate\_f(\hyperlink{structfile}{file}, dataset, filetype, space, dset, hdferr)
00059 
00060   f\_ptr = c\_loc(wdata(1)(1:1))
00061   \textcolor{keyword}{CALL }h5dwrite\_f(dset, filetype, f\_ptr, hdferr);
00062   \textcolor{comment}{!}
00063   \textcolor{comment}{! Close and release resources.}
00064   \textcolor{comment}{!}
00065   \textcolor{keyword}{CALL }h5dclose\_f(dset , hdferr)
00066   \textcolor{keyword}{CALL }h5sclose\_f(space, hdferr)
00067   \textcolor{keyword}{CALL }h5tclose\_f(filetype, hdferr)
00068   \textcolor{keyword}{CALL }h5fclose\_f(\hyperlink{structfile}{file} , hdferr)
00069   \textcolor{comment}{!}
00070   \textcolor{comment}{! Now we begin the read section of this example.}
00071   \textcolor{comment}{!}
00072   \textcolor{comment}{! Open file and dataset.}
00073   \textcolor{comment}{!}
00074   \textcolor{keyword}{CALL }h5fopen\_f(filename, h5f\_acc\_rdonly\_f, \hyperlink{structfile}{file}, hdferr)
00075   \textcolor{keyword}{CALL }h5dopen\_f(\hyperlink{structfile}{file}, dataset, dset, hdferr)
00076   \textcolor{comment}{!}
00077   \textcolor{comment}{! Get the datatype and its size.}
00078   \textcolor{comment}{!}
00079   \textcolor{keyword}{CALL }h5dget\_type\_f(dset, filetype, hdferr)
00080   \textcolor{keyword}{CALL }h5tget\_size\_f(filetype, \textcolor{keyword}{size}, hdferr)
00081 
00082   \textcolor{comment}{! Make sure the declared length is large enough}
00083   \textcolor{keywordflow}{IF}(size.GT.sdim)\textcolor{keywordflow}{THEN}
00084      print*,\textcolor{stringliteral}{'ERROR: Character LEN is to small'}
00085      stop
00086 \textcolor{keywordflow}{  ENDIF}
00087   \textcolor{comment}{!}
00088   \textcolor{comment}{! Get dataspace.}
00089   \textcolor{comment}{!}
00090   \textcolor{keyword}{CALL }h5dget\_space\_f(dset, space, hdferr)
00091   \textcolor{keyword}{CALL }h5sget\_simple\_extent\_dims\_f(space, dims, maxdims, hdferr)
00092 
00093   \textcolor{keyword}{ALLOCATE}(rdata(1:dims(1)))
00094   \textcolor{comment}{!}
00095   \textcolor{comment}{! Create the memory datatype.}
00096   \textcolor{comment}{!}
00097   \textcolor{keyword}{CALL }h5tcopy\_f(h5t\_fortran\_s1, memtype, hdferr)
00098   \textcolor{keyword}{CALL }h5tset\_size\_f(memtype, sdim, hdferr)
00099   \textcolor{comment}{!}
00100   \textcolor{comment}{! Read the data.}
00101   \textcolor{comment}{!}
00102   f\_ptr = c\_loc(rdata(1)(1:1))
00103   \textcolor{keyword}{CALL }h5dread\_f(dset, memtype, f\_ptr, hdferr, space)
00104   \textcolor{comment}{!}
00105   \textcolor{comment}{! Output the data to the screen.}
00106   \textcolor{comment}{!}
00107   \textcolor{keywordflow}{DO} i = 1, dims(1)
00108      \textcolor{keyword}{WRITE}(*,\textcolor{stringliteral}{'(A,"(",I0,"): ", A)'}) dataset, i, trim(rdata(i))
00109 \textcolor{keywordflow}{  END DO}
00110   \textcolor{comment}{!}
00111   \textcolor{comment}{! Close and release resources.}
00112   \textcolor{comment}{!}
00113   \textcolor{keyword}{CALL }h5dclose\_f(dset, hdferr)
00114   \textcolor{keyword}{CALL }h5sclose\_f(space, hdferr)
00115   \textcolor{keyword}{CALL }h5tclose\_f(memtype, hdferr)
00116   \textcolor{keyword}{CALL }h5fclose\_f(\hyperlink{structfile}{file}, hdferr)
00117 
00118   \textcolor{keyword}{DEALLOCATE}(rdata)
00119 
00120 \textcolor{keyword}{END PROGRAM }main
\end{DoxyCode}
