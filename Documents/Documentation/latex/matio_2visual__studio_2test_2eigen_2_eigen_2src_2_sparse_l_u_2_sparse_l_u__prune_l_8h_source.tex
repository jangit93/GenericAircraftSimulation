\hypertarget{matio_2visual__studio_2test_2eigen_2_eigen_2src_2_sparse_l_u_2_sparse_l_u__prune_l_8h_source}{}\section{matio/visual\+\_\+studio/test/eigen/\+Eigen/src/\+Sparse\+L\+U/\+Sparse\+L\+U\+\_\+pruneL.h}
\label{matio_2visual__studio_2test_2eigen_2_eigen_2src_2_sparse_l_u_2_sparse_l_u__prune_l_8h_source}\index{Sparse\+L\+U\+\_\+prune\+L.\+h@{Sparse\+L\+U\+\_\+prune\+L.\+h}}

\begin{DoxyCode}
00001 \textcolor{comment}{// This file is part of Eigen, a lightweight C++ template library}
00002 \textcolor{comment}{// for linear algebra.}
00003 \textcolor{comment}{//}
00004 \textcolor{comment}{// Copyright (C) 2012 Désiré Nuentsa-Wakam <desire.nuentsa\_wakam@inria.fr>}
00005 \textcolor{comment}{//}
00006 \textcolor{comment}{// This Source Code Form is subject to the terms of the Mozilla}
00007 \textcolor{comment}{// Public License v. 2.0. If a copy of the MPL was not distributed}
00008 \textcolor{comment}{// with this file, You can obtain one at http://mozilla.org/MPL/2.0/.}
00009 
00010 \textcolor{comment}{/* }
00011 \textcolor{comment}{ }
00012 \textcolor{comment}{ * NOTE: This file is the modified version of [s,d,c,z]pruneL.c file in SuperLU }
00013 \textcolor{comment}{ }
00014 \textcolor{comment}{ * -- SuperLU routine (version 2.0) --}
00015 \textcolor{comment}{ * Univ. of California Berkeley, Xerox Palo Alto Research Center,}
00016 \textcolor{comment}{ * and Lawrence Berkeley National Lab.}
00017 \textcolor{comment}{ * November 15, 1997}
00018 \textcolor{comment}{ *}
00019 \textcolor{comment}{ * Copyright (c) 1994 by Xerox Corporation.  All rights reserved.}
00020 \textcolor{comment}{ *}
00021 \textcolor{comment}{ * THIS MATERIAL IS PROVIDED AS IS, WITH ABSOLUTELY NO WARRANTY}
00022 \textcolor{comment}{ * EXPRESSED OR IMPLIED.  ANY USE IS AT YOUR OWN RISK.}
00023 \textcolor{comment}{ *}
00024 \textcolor{comment}{ * Permission is hereby granted to use or copy this program for any}
00025 \textcolor{comment}{ * purpose, provided the above notices are retained on all copies.}
00026 \textcolor{comment}{ * Permission to modify the code and to distribute modified code is}
00027 \textcolor{comment}{ * granted, provided the above notices are retained, and a notice that}
00028 \textcolor{comment}{ * the code was modified is included with the above copyright notice.}
00029 \textcolor{comment}{ */}
00030 \textcolor{preprocessor}{#ifndef SPARSELU\_PRUNEL\_H}
00031 \textcolor{preprocessor}{#define SPARSELU\_PRUNEL\_H}
00032 
00033 \textcolor{keyword}{namespace }\hyperlink{namespace_eigen}{Eigen} \{
00034 \textcolor{keyword}{namespace }\hyperlink{namespaceinternal}{internal} \{
00035 
00052 \textcolor{keyword}{template} <\textcolor{keyword}{typename} Scalar, \textcolor{keyword}{typename} StorageIndex>
00053 \textcolor{keywordtype}{void} \hyperlink{group___sparse_l_u___module_a350464d1c83182fbd7da8a5a74bdfde8}{SparseLUImpl<Scalar,StorageIndex>::pruneL}(\textcolor{keyword}{const} 
      \hyperlink{namespace_eigen_a62e77e0933482dafde8fe197d9a2cfde}{Index} jcol, \textcolor{keyword}{const} IndexVector& perm\_r, \textcolor{keyword}{const} \hyperlink{namespace_eigen_a62e77e0933482dafde8fe197d9a2cfde}{Index} pivrow, \textcolor{keyword}{const} \hyperlink{namespace_eigen_a62e77e0933482dafde8fe197d9a2cfde}{Index} nseg,
00054                                                \textcolor{keyword}{const} IndexVector& segrep, BlockIndexVector repfnz, 
      IndexVector& xprune, GlobalLU\_t& glu)
00055 \{
00056   \textcolor{comment}{// For each supernode-rep irep in U(*,j]}
00057   \hyperlink{namespace_eigen_a62e77e0933482dafde8fe197d9a2cfde}{Index} jsupno = glu.supno(jcol); 
00058   \hyperlink{namespace_eigen_a62e77e0933482dafde8fe197d9a2cfde}{Index} i,irep,irep1; 
00059   \textcolor{keywordtype}{bool} movnum, do\_prune = \textcolor{keyword}{false}; 
00060   \hyperlink{namespace_eigen_a62e77e0933482dafde8fe197d9a2cfde}{Index} kmin = 0, kmax = 0, minloc, maxloc,krow; 
00061   \textcolor{keywordflow}{for} (i = 0; i < nseg; i++)
00062   \{
00063     irep = segrep(i); 
00064     irep1 = irep + 1; 
00065     do\_prune = \textcolor{keyword}{false}; 
00066     
00067     \textcolor{comment}{// Don't prune with a zero U-segment }
00068     \textcolor{keywordflow}{if} (repfnz(irep) == emptyIdxLU) \textcolor{keywordflow}{continue}; 
00069     
00070     \textcolor{comment}{// If a snode overlaps with the next panel, then the U-segment}
00071     \textcolor{comment}{// is fragmented into two parts -- irep and irep1. We should let }
00072     \textcolor{comment}{// pruning occur at the rep-column in irep1s snode. }
00073     \textcolor{keywordflow}{if} (glu.supno(irep) == glu.supno(irep1) ) \textcolor{keywordflow}{continue}; \textcolor{comment}{// don't prune }
00074     
00075     \textcolor{comment}{// If it has not been pruned & it has a nonz in row L(pivrow,i)}
00076     \textcolor{keywordflow}{if} (glu.supno(irep) != jsupno )
00077     \{
00078       \textcolor{keywordflow}{if} ( xprune (irep) >= glu.xlsub(irep1) )
00079       \{
00080         kmin = glu.xlsub(irep);
00081         kmax = glu.xlsub(irep1) - 1; 
00082         \textcolor{keywordflow}{for} (krow = kmin; krow <= kmax; krow++)
00083         \{
00084           \textcolor{keywordflow}{if} (glu.lsub(krow) == pivrow) 
00085           \{
00086             do\_prune = \textcolor{keyword}{true}; 
00087             \textcolor{keywordflow}{break}; 
00088           \}
00089         \}
00090       \}
00091       
00092       \textcolor{keywordflow}{if} (do\_prune) 
00093       \{
00094         \textcolor{comment}{// do a quicksort-type partition}
00095         \textcolor{comment}{// movnum=true means that the num values have to be exchanged}
00096         movnum = \textcolor{keyword}{false}; 
00097         \textcolor{keywordflow}{if} (irep == glu.xsup(glu.supno(irep)) ) \textcolor{comment}{// Snode of size 1 }
00098           movnum = \textcolor{keyword}{true}; 
00099         
00100         \textcolor{keywordflow}{while} (kmin <= kmax)
00101         \{
00102           \textcolor{keywordflow}{if} (perm\_r(glu.lsub(kmax)) == emptyIdxLU)
00103             kmax--; 
00104           \textcolor{keywordflow}{else} \textcolor{keywordflow}{if} ( perm\_r(glu.lsub(kmin)) != emptyIdxLU)
00105             kmin++;
00106           \textcolor{keywordflow}{else} 
00107           \{
00108             \textcolor{comment}{// kmin below pivrow (not yet pivoted), and kmax}
00109             \textcolor{comment}{// above pivrow: interchange the two suscripts}
00110             \hyperlink{endian_8c_a3ca5ecd34b04d6a243c054ac3a57f68d}{std::swap}(glu.lsub(kmin), glu.lsub(kmax)); 
00111             
00112             \textcolor{comment}{// If the supernode has only one column, then we }
00113             \textcolor{comment}{// only keep one set of subscripts. For any subscript}
00114             \textcolor{comment}{// intercnahge performed, similar interchange must be }
00115             \textcolor{comment}{// done on the numerical values. }
00116             \textcolor{keywordflow}{if} (movnum) 
00117             \{
00118               minloc = glu.xlusup(irep) + ( kmin - glu.xlsub(irep) ); 
00119               maxloc = glu.xlusup(irep) + ( kmax - glu.xlsub(irep) ); 
00120               \hyperlink{endian_8c_a3ca5ecd34b04d6a243c054ac3a57f68d}{std::swap}(glu.lusup(minloc), glu.lusup(maxloc)); 
00121             \}
00122             kmin++;
00123             kmax--;
00124           \}
00125         \} \textcolor{comment}{// end while }
00126         
00127         xprune(irep) = StorageIndex(kmin);  \textcolor{comment}{//Pruning }
00128       \} \textcolor{comment}{// end if do\_prune }
00129     \} \textcolor{comment}{// end pruning }
00130   \} \textcolor{comment}{// End for each U-segment}
00131 \}
00132 
00133 \} \textcolor{comment}{// end namespace internal}
00134 \} \textcolor{comment}{// end namespace Eigen}
00135 
00136 \textcolor{preprocessor}{#endif // SPARSELU\_PRUNEL\_H}
\end{DoxyCode}
