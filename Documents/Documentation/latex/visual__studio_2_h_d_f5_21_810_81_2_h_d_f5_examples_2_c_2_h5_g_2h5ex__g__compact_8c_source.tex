\hypertarget{visual__studio_2_h_d_f5_21_810_81_2_h_d_f5_examples_2_c_2_h5_g_2h5ex__g__compact_8c_source}{}\section{visual\+\_\+studio/\+H\+D\+F5/1.10.1/\+H\+D\+F5\+Examples/\+C/\+H5\+G/h5ex\+\_\+g\+\_\+compact.c}
\label{visual__studio_2_h_d_f5_21_810_81_2_h_d_f5_examples_2_c_2_h5_g_2h5ex__g__compact_8c_source}\index{h5ex\+\_\+g\+\_\+compact.\+c@{h5ex\+\_\+g\+\_\+compact.\+c}}

\begin{DoxyCode}
00001 \textcolor{comment}{/************************************************************}
00002 \textcolor{comment}{}
00003 \textcolor{comment}{  This example shows how to create "compact-or-indexed"}
00004 \textcolor{comment}{  format groups, new to 1.8.  This example also illustrates}
00005 \textcolor{comment}{  the space savings of compact groups by creating 2 files}
00006 \textcolor{comment}{  which are identical except for the group format, and}
00007 \textcolor{comment}{  displaying the file size of each.  Both files have one}
00008 \textcolor{comment}{  empty group in the root group.}
00009 \textcolor{comment}{}
00010 \textcolor{comment}{  This file is intended for use with HDF5 Library version 1.8}
00011 \textcolor{comment}{}
00012 \textcolor{comment}{ ************************************************************/}
00013 
00014 \textcolor{preprocessor}{#include "hdf5.h"}
00015 \textcolor{preprocessor}{#include <stdio.h>}
00016 
00017 \textcolor{preprocessor}{#define FILE1       "h5ex\_g\_compact1.h5"}
00018 \textcolor{preprocessor}{#define FILE2       "h5ex\_g\_compact2.h5"}
00019 \textcolor{preprocessor}{#define GROUP       "G1"}
00020 
00021 \textcolor{keywordtype}{int}
00022 main (\textcolor{keywordtype}{void})
00023 \{
00024     hid\_t       \hyperlink{structfile}{file}, group, fapl;         \textcolor{comment}{/* Handles */}
00025     herr\_t      status;
00026     \hyperlink{struct_h5_g__info__t}{H5G\_info\_t}  ginfo;
00027     hsize\_t     size;
00028 
00029     \textcolor{comment}{/*}
00030 \textcolor{comment}{     * Create file 1.  This file will use original format groups.}
00031 \textcolor{comment}{     */}
00032     file = H5Fcreate (FILE1, H5F\_ACC\_TRUNC, H5P\_DEFAULT, H5P\_DEFAULT);
00033     group = H5Gcreate (file, GROUP, H5P\_DEFAULT, H5P\_DEFAULT, H5P\_DEFAULT);
00034 
00035     \textcolor{comment}{/*}
00036 \textcolor{comment}{     * Obtain the group info and print the group storage type.}
00037 \textcolor{comment}{     */}
00038     status = H5Gget\_info (group, &ginfo);
00039     printf (\textcolor{stringliteral}{"Group storage type for %s is: "}, FILE1);
00040     \textcolor{keywordflow}{switch} (ginfo.storage\_type) \{
00041         \textcolor{keywordflow}{case} H5G\_STORAGE\_TYPE\_COMPACT:
00042             printf (\textcolor{stringliteral}{"H5G\_STORAGE\_TYPE\_COMPACT\(\backslash\)n"}); \textcolor{comment}{/* New compact format */}
00043             \textcolor{keywordflow}{break};
00044         \textcolor{keywordflow}{case} H5G\_STORAGE\_TYPE\_DENSE:
00045             printf (\textcolor{stringliteral}{"H5G\_STORAGE\_TYPE\_DENSE\(\backslash\)n"}); \textcolor{comment}{/* New dense (indexed) format */}
00046             \textcolor{keywordflow}{break};
00047         \textcolor{keywordflow}{case} H5G\_STORAGE\_TYPE\_SYMBOL\_TABLE:
00048             printf (\textcolor{stringliteral}{"H5G\_STORAGE\_TYPE\_SYMBOL\_TABLE\(\backslash\)n"}); \textcolor{comment}{/* Original format */}
00049             \textcolor{keywordflow}{break};
00050         \textcolor{keywordflow}{case} H5G\_STORAGE\_TYPE\_UNKNOWN:
00051             printf (\textcolor{stringliteral}{"H5G\_STORAGE\_TYPE\_UNKNOWN\(\backslash\)n"}); \textcolor{comment}{/* Unknown format */}
00052     \}
00053 
00054     \textcolor{comment}{/*}
00055 \textcolor{comment}{     * Close and re-open file.  Needed to get the correct file size.}
00056 \textcolor{comment}{     */}
00057     status = H5Gclose (group);
00058     status = H5Fclose (file);
00059     file = H5Fopen (FILE1, H5F\_ACC\_RDONLY, H5P\_DEFAULT);
00060 
00061     \textcolor{comment}{/*}
00062 \textcolor{comment}{     * Obtain and print the file size.}
00063 \textcolor{comment}{     */}
00064     status = H5Fget\_filesize (file, &size);
00065     printf (\textcolor{stringliteral}{"File size for %s is: %d bytes\(\backslash\)n\(\backslash\)n"}, FILE1, (\textcolor{keywordtype}{int})size);
00066 
00067     \textcolor{comment}{/*}
00068 \textcolor{comment}{     * Close FILE1.}
00069 \textcolor{comment}{     */}
00070     status = H5Fclose (file);
00071 
00072     \textcolor{comment}{/*}
00073 \textcolor{comment}{     * Set file access property list to allow the latest file format.}
00074 \textcolor{comment}{     * This will allow the library to create new compact format groups.}
00075 \textcolor{comment}{     */}
00076     fapl = H5Pcreate (H5P\_FILE\_ACCESS);
00077     status = H5Pset\_libver\_bounds (fapl, H5F\_LIBVER\_LATEST, H5F\_LIBVER\_LATEST);
00078 
00079     \textcolor{comment}{/*}
00080 \textcolor{comment}{     * Create file 2 using the new file access property list.}
00081 \textcolor{comment}{     */}
00082     file = H5Fcreate (FILE2, H5F\_ACC\_TRUNC, H5P\_DEFAULT, fapl );
00083     group = H5Gcreate (file, GROUP, H5P\_DEFAULT, H5P\_DEFAULT, H5P\_DEFAULT);
00084 
00085     \textcolor{comment}{/*}
00086 \textcolor{comment}{     * Obtain the group info and print the group storage type.}
00087 \textcolor{comment}{     */}
00088     status = H5Gget\_info (group, &ginfo);
00089     printf (\textcolor{stringliteral}{"Group storage type for %s is: "}, FILE2);
00090     \textcolor{keywordflow}{switch} (ginfo.storage\_type) \{
00091         \textcolor{keywordflow}{case} H5G\_STORAGE\_TYPE\_COMPACT:
00092             printf (\textcolor{stringliteral}{"H5G\_STORAGE\_TYPE\_COMPACT\(\backslash\)n"}); \textcolor{comment}{/* New compact format */}
00093             \textcolor{keywordflow}{break};
00094         \textcolor{keywordflow}{case} H5G\_STORAGE\_TYPE\_DENSE:
00095             printf (\textcolor{stringliteral}{"H5G\_STORAGE\_TYPE\_DENSE\(\backslash\)n"}); \textcolor{comment}{/* New dense (indexed) format */}
00096             \textcolor{keywordflow}{break};
00097         \textcolor{keywordflow}{case} H5G\_STORAGE\_TYPE\_SYMBOL\_TABLE:
00098             printf (\textcolor{stringliteral}{"H5G\_STORAGE\_TYPE\_SYMBOL\_TABLE\(\backslash\)n"}); \textcolor{comment}{/* Original format */}
00099             \textcolor{keywordflow}{break};
00100         \textcolor{keywordflow}{case} H5G\_STORAGE\_TYPE\_UNKNOWN:
00101             printf (\textcolor{stringliteral}{"H5G\_STORAGE\_TYPE\_UNKNOWN\(\backslash\)n"}); \textcolor{comment}{/* Unknown format */}
00102     \}
00103 
00104     \textcolor{comment}{/*}
00105 \textcolor{comment}{     * Close and re-open file.  Needed to get the correct file size.}
00106 \textcolor{comment}{     */}
00107     status = H5Gclose (group);
00108     status = H5Fclose (file);
00109     file = H5Fopen (FILE2, H5F\_ACC\_RDONLY, fapl);
00110 
00111     \textcolor{comment}{/*}
00112 \textcolor{comment}{     * Obtain and print the file size.}
00113 \textcolor{comment}{     */}
00114     status = H5Fget\_filesize (file, &size);
00115     printf (\textcolor{stringliteral}{"File size for %s is: %d bytes\(\backslash\)n"}, FILE2, (\textcolor{keywordtype}{int})size);
00116     printf (\textcolor{stringliteral}{"\(\backslash\)n"});
00117 
00118     \textcolor{comment}{/*}
00119 \textcolor{comment}{     * Close and release resources.}
00120 \textcolor{comment}{     */}
00121     status = H5Pclose (fapl);
00122     status = H5Fclose (file);
00123 
00124     \textcolor{keywordflow}{return} 0;
00125 \}
\end{DoxyCode}
