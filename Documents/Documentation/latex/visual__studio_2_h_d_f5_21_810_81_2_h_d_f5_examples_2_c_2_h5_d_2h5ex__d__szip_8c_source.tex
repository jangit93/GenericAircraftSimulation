\hypertarget{visual__studio_2_h_d_f5_21_810_81_2_h_d_f5_examples_2_c_2_h5_d_2h5ex__d__szip_8c_source}{}\section{visual\+\_\+studio/\+H\+D\+F5/1.10.1/\+H\+D\+F5\+Examples/\+C/\+H5\+D/h5ex\+\_\+d\+\_\+szip.c}
\label{visual__studio_2_h_d_f5_21_810_81_2_h_d_f5_examples_2_c_2_h5_d_2h5ex__d__szip_8c_source}\index{h5ex\+\_\+d\+\_\+szip.\+c@{h5ex\+\_\+d\+\_\+szip.\+c}}

\begin{DoxyCode}
00001 \textcolor{comment}{/************************************************************}
00002 \textcolor{comment}{}
00003 \textcolor{comment}{  This example shows how to read and write data to a dataset}
00004 \textcolor{comment}{  using szip compression.    The program first checks if}
00005 \textcolor{comment}{  szip compression is available, then if it is it writes}
00006 \textcolor{comment}{  integers to a dataset using szip, then closes the file.}
00007 \textcolor{comment}{  Next, it reopens the file, reads back the data, and}
00008 \textcolor{comment}{  outputs the type of compression and the maximum value in}
00009 \textcolor{comment}{  the dataset to the screen.}
00010 \textcolor{comment}{}
00011 \textcolor{comment}{  This file is intended for use with HDF5 Library version 1.8}
00012 \textcolor{comment}{}
00013 \textcolor{comment}{ ************************************************************/}
00014 
00015 
00016 \textcolor{preprocessor}{#include "hdf5.h"}
00017 \textcolor{preprocessor}{#include <stdio.h>}
00018 \textcolor{preprocessor}{#include <stdlib.h>}
00019 
00020 \textcolor{preprocessor}{#define FILE            "h5ex\_d\_szip.h5"}
00021 \textcolor{preprocessor}{#define DATASET         "DS1"}
00022 \textcolor{preprocessor}{#define DIM0            32}
00023 \textcolor{preprocessor}{#define DIM1            64}
00024 \textcolor{preprocessor}{#define CHUNK0          4}
00025 \textcolor{preprocessor}{#define CHUNK1          8}
00026 
00027 \textcolor{keywordtype}{int}
00028 main (\textcolor{keywordtype}{void})
00029 \{
00030     hid\_t           \hyperlink{structfile}{file}, space, dset, dcpl;    \textcolor{comment}{/* Handles */}
00031     herr\_t          status;
00032     htri\_t          avail;
00033     H5Z\_filter\_t    filter\_type;
00034     hsize\_t         dims[2] = \{DIM0, DIM1\},
00035                     chunk[2] = \{CHUNK0, CHUNK1\};
00036     \textcolor{keywordtype}{size\_t}          nelmts;
00037     \textcolor{keywordtype}{unsigned} \textcolor{keywordtype}{int}    flags,
00038                     filter\_info;
00039     \textcolor{keywordtype}{int}             wdata[DIM0][DIM1],          \textcolor{comment}{/* Write buffer */}
00040                     rdata[DIM0][DIM1],          \textcolor{comment}{/* Read buffer */}
00041                     max,
00042                     i, j;
00043 
00044     \textcolor{comment}{/*}
00045 \textcolor{comment}{     * Check if szip compression is available and can be used for both}
00046 \textcolor{comment}{     * compression and decompression.  Normally we do not perform error}
00047 \textcolor{comment}{     * checking in these examples for the sake of clarity, but in this}
00048 \textcolor{comment}{     * case we will make an exception because this filter is an}
00049 \textcolor{comment}{     * optional part of the hdf5 library.}
00050 \textcolor{comment}{     */}
00051     avail = H5Zfilter\_avail(H5Z\_FILTER\_SZIP);
00052     \textcolor{keywordflow}{if} (!avail) \{
00053         printf (\textcolor{stringliteral}{"szip filter not available.\(\backslash\)n"});
00054         \textcolor{keywordflow}{return} 1;
00055     \}
00056     status = H5Zget\_filter\_info (H5Z\_FILTER\_SZIP, &filter\_info);
00057     \textcolor{keywordflow}{if} ( !(filter\_info & H5Z\_FILTER\_CONFIG\_ENCODE\_ENABLED) ||
00058                 !(filter\_info & H5Z\_FILTER\_CONFIG\_DECODE\_ENABLED) ) \{
00059         printf (\textcolor{stringliteral}{"szip filter not available for encoding and decoding.\(\backslash\)n"});
00060         \textcolor{keywordflow}{return} 1;
00061     \}
00062 
00063     \textcolor{comment}{/*}
00064 \textcolor{comment}{     * Initialize data.}
00065 \textcolor{comment}{     */}
00066     \textcolor{keywordflow}{for} (i=0; i<DIM0; i++)
00067         \textcolor{keywordflow}{for} (j=0; j<DIM1; j++)
00068             wdata[i][j] = i * j - j;
00069 
00070     \textcolor{comment}{/*}
00071 \textcolor{comment}{     * Create a new file using the default properties.}
00072 \textcolor{comment}{     */}
00073     file = H5Fcreate (FILE, H5F\_ACC\_TRUNC, H5P\_DEFAULT, H5P\_DEFAULT);
00074 
00075     \textcolor{comment}{/*}
00076 \textcolor{comment}{     * Create dataspace.  Setting maximum size to NULL sets the maximum}
00077 \textcolor{comment}{     * size to be the current size.}
00078 \textcolor{comment}{     */}
00079     space = H5Screate\_simple (2, dims, NULL);
00080 
00081     \textcolor{comment}{/*}
00082 \textcolor{comment}{     * Create the dataset creation property list, add the szip}
00083 \textcolor{comment}{     * compression filter and set the chunk size.}
00084 \textcolor{comment}{     */}
00085     dcpl = H5Pcreate (H5P\_DATASET\_CREATE);
00086     status = H5Pset\_szip (dcpl, H5\_SZIP\_NN\_OPTION\_MASK, 8);
00087     status = H5Pset\_chunk (dcpl, 2, chunk);
00088 
00089     \textcolor{comment}{/*}
00090 \textcolor{comment}{     * Create the dataset.}
00091 \textcolor{comment}{     */}
00092     dset = H5Dcreate (file, DATASET, H5T\_STD\_I32LE, space, H5P\_DEFAULT, dcpl,
00093                 H5P\_DEFAULT);
00094 
00095     \textcolor{comment}{/*}
00096 \textcolor{comment}{     * Write the data to the dataset.}
00097 \textcolor{comment}{     */}
00098     status = H5Dwrite (dset, H5T\_NATIVE\_INT, H5S\_ALL, H5S\_ALL, H5P\_DEFAULT,
00099                 wdata[0]);
00100 
00101     \textcolor{comment}{/*}
00102 \textcolor{comment}{     * Close and release resources.}
00103 \textcolor{comment}{     */}
00104     status = H5Pclose (dcpl);
00105     status = H5Dclose (dset);
00106     status = H5Sclose (space);
00107     status = H5Fclose (file);
00108 
00109 
00110     \textcolor{comment}{/*}
00111 \textcolor{comment}{     * Now we begin the read section of this example.}
00112 \textcolor{comment}{     */}
00113 
00114     \textcolor{comment}{/*}
00115 \textcolor{comment}{     * Open file and dataset using the default properties.}
00116 \textcolor{comment}{     */}
00117     file = H5Fopen (FILE, H5F\_ACC\_RDONLY, H5P\_DEFAULT);
00118     dset = H5Dopen (file, DATASET, H5P\_DEFAULT);
00119 
00120     \textcolor{comment}{/*}
00121 \textcolor{comment}{     * Retrieve dataset creation property list.}
00122 \textcolor{comment}{     */}
00123     dcpl = H5Dget\_create\_plist (dset);
00124 
00125     \textcolor{comment}{/*}
00126 \textcolor{comment}{     * Retrieve and print the filter type.  Here we only retrieve the}
00127 \textcolor{comment}{     * first filter because we know that we only added one filter.}
00128 \textcolor{comment}{     */}
00129     nelmts = 0;
00130     filter\_type = H5Pget\_filter (dcpl, 0, &flags, &nelmts, NULL, 0, NULL,
00131                 &filter\_info);
00132     printf (\textcolor{stringliteral}{"Filter type is: "});
00133     \textcolor{keywordflow}{switch} (filter\_type) \{
00134         \textcolor{keywordflow}{case} H5Z\_FILTER\_DEFLATE:
00135             printf (\textcolor{stringliteral}{"H5Z\_FILTER\_DEFLATE\(\backslash\)n"});
00136             \textcolor{keywordflow}{break};
00137         \textcolor{keywordflow}{case} H5Z\_FILTER\_SHUFFLE:
00138             printf (\textcolor{stringliteral}{"H5Z\_FILTER\_SHUFFLE\(\backslash\)n"});
00139             \textcolor{keywordflow}{break};
00140         \textcolor{keywordflow}{case} H5Z\_FILTER\_FLETCHER32:
00141             printf (\textcolor{stringliteral}{"H5Z\_FILTER\_FLETCHER32\(\backslash\)n"});
00142             \textcolor{keywordflow}{break};
00143         \textcolor{keywordflow}{case} H5Z\_FILTER\_SZIP:
00144             printf (\textcolor{stringliteral}{"H5Z\_FILTER\_SZIP\(\backslash\)n"});
00145             \textcolor{keywordflow}{break};
00146         \textcolor{keywordflow}{case} H5Z\_FILTER\_NBIT:
00147             printf (\textcolor{stringliteral}{"H5Z\_FILTER\_NBIT\(\backslash\)n"});
00148             \textcolor{keywordflow}{break};
00149         \textcolor{keywordflow}{case} H5Z\_FILTER\_SCALEOFFSET:
00150             printf (\textcolor{stringliteral}{"H5Z\_FILTER\_SCALEOFFSET\(\backslash\)n"});
00151     \}
00152 
00153     \textcolor{comment}{/*}
00154 \textcolor{comment}{     * Read the data using the default properties.}
00155 \textcolor{comment}{     */}
00156     status = H5Dread (dset, H5T\_NATIVE\_INT, H5S\_ALL, H5S\_ALL, H5P\_DEFAULT,
00157                 rdata[0]);
00158 
00159     \textcolor{comment}{/*}
00160 \textcolor{comment}{     * Find the maximum value in the dataset, to verify that it was}
00161 \textcolor{comment}{     * read correctly.}
00162 \textcolor{comment}{     */}
00163     max = rdata[0][0];
00164     \textcolor{keywordflow}{for} (i=0; i<DIM0; i++)
00165         \textcolor{keywordflow}{for} (j = 0; j<DIM1; j++)
00166             \textcolor{keywordflow}{if} (max < rdata[i][j])
00167                 max=rdata[i][j];
00168 
00169     \textcolor{comment}{/*}
00170 \textcolor{comment}{     * Print the maximum value.}
00171 \textcolor{comment}{     */}
00172     printf (\textcolor{stringliteral}{"Maximum value in %s is: %d\(\backslash\)n"}, DATASET, max);
00173 
00174     \textcolor{comment}{/*}
00175 \textcolor{comment}{     * Close and release resources.}
00176 \textcolor{comment}{     */}
00177     status = H5Pclose (dcpl);
00178     status = H5Dclose (dset);
00179     status = H5Fclose (file);
00180 
00181     \textcolor{keywordflow}{return} 0;
00182 \}
\end{DoxyCode}
