\hypertarget{eigen_2unsupported_2test_2_non_linear_optimization_8cpp_source}{}\section{eigen/unsupported/test/\+Non\+Linear\+Optimization.cpp}
\label{eigen_2unsupported_2test_2_non_linear_optimization_8cpp_source}\index{Non\+Linear\+Optimization.\+cpp@{Non\+Linear\+Optimization.\+cpp}}

\begin{DoxyCode}
00001 \textcolor{comment}{// This file is part of Eigen, a lightweight C++ template library}
00002 \textcolor{comment}{// for linear algebra.}
00003 \textcolor{comment}{//}
00004 \textcolor{comment}{// Copyright (C) 2009 Thomas Capricelli <orzel@freehackers.org>}
00005 
00006 \textcolor{preprocessor}{#include <stdio.h>}
00007 
00008 \textcolor{preprocessor}{#include "main.h"}
00009 \textcolor{preprocessor}{#include <unsupported/Eigen/NonLinearOptimization>}
00010 
00011 \textcolor{comment}{// This disables some useless Warnings on MSVC.}
00012 \textcolor{comment}{// It is intended to be done for this test only.}
00013 \textcolor{preprocessor}{#include <Eigen/src/Core/util/DisableStupidWarnings.h>}
00014 
00015 \textcolor{comment}{// tolerance for chekcing number of iterations}
00016 \textcolor{preprocessor}{#define LM\_EVAL\_COUNT\_TOL 4/3}
00017 
00018 \textcolor{keywordtype}{int} fcn\_chkder(\textcolor{keyword}{const} VectorXd &x, VectorXd &fvec, MatrixXd &fjac, \textcolor{keywordtype}{int} iflag)
00019 \{
00020     \textcolor{comment}{/*      subroutine fcn for chkder example. */}
00021 
00022     \textcolor{keywordtype}{int} i;
00023     assert(15 ==  fvec.size());
00024     assert(3 ==  x.size());
00025     \textcolor{keywordtype}{double} tmp1, tmp2, tmp3, tmp4;
00026     \textcolor{keyword}{static} \textcolor{keyword}{const} \textcolor{keywordtype}{double} y[15]=\{1.4e-1, 1.8e-1, 2.2e-1, 2.5e-1, 2.9e-1, 3.2e-1, 3.5e-1,
00027         3.9e-1, 3.7e-1, 5.8e-1, 7.3e-1, 9.6e-1, 1.34, 2.1, 4.39\};
00028 
00029 
00030     \textcolor{keywordflow}{if} (iflag == 0)
00031         \textcolor{keywordflow}{return} 0;
00032 
00033     \textcolor{keywordflow}{if} (iflag != 2)
00034         \textcolor{keywordflow}{for} (i=0; i<15; i++) \{
00035             tmp1 = i+1;
00036             tmp2 = 16-i-1;
00037             tmp3 = tmp1;
00038             \textcolor{keywordflow}{if} (i >= 8) tmp3 = tmp2;
00039             fvec[i] = y[i] - (x[0] + tmp1/(x[1]*tmp2 + x[2]*tmp3));
00040         \}
00041     \textcolor{keywordflow}{else} \{
00042         \textcolor{keywordflow}{for} (i = 0; i < 15; i++) \{
00043             tmp1 = i+1;
00044             tmp2 = 16-i-1;
00045 
00046             \textcolor{comment}{/* error introduced into next statement for illustration. */}
00047             \textcolor{comment}{/* corrected statement should read    tmp3 = tmp1 . */}
00048 
00049             tmp3 = tmp2;
00050             \textcolor{keywordflow}{if} (i >= 8) tmp3 = tmp2;
00051             tmp4 = (x[1]*tmp2 + x[2]*tmp3); tmp4=tmp4*tmp4;
00052             fjac(i,0) = -1.;
00053             fjac(i,1) = tmp1*tmp2/tmp4;
00054             fjac(i,2) = tmp1*tmp3/tmp4;
00055         \}
00056     \}
00057     \textcolor{keywordflow}{return} 0;
00058 \}
00059 
00060 
00061 \textcolor{keywordtype}{void} testChkder()
00062 \{
00063   \textcolor{keyword}{const} \textcolor{keywordtype}{int} m=15, n=3;
00064   VectorXd x(n), fvec(m), xp, fvecp(m), err;
00065   MatrixXd fjac(m,n);
00066   VectorXi ipvt;
00067 
00068   \textcolor{comment}{/*      the following values should be suitable for */}
00069   \textcolor{comment}{/*      checking the jacobian matrix. */}
00070   x << 9.2e-1, 1.3e-1, 5.4e-1;
00071 
00072   internal::chkder(x, fvec, fjac, xp, fvecp, 1, err);
00073   fcn\_chkder(x, fvec, fjac, 1);
00074   fcn\_chkder(x, fvec, fjac, 2);
00075   fcn\_chkder(xp, fvecp, fjac, 1);
00076   internal::chkder(x, fvec, fjac, xp, fvecp, 2, err);
00077 
00078   fvecp -= fvec;
00079 
00080   \textcolor{comment}{// check those}
00081   VectorXd fvec\_ref(m), fvecp\_ref(m), err\_ref(m);
00082   fvec\_ref <<
00083       -1.181606, -1.429655, -1.606344,
00084       -1.745269, -1.840654, -1.921586,
00085       -1.984141, -2.022537, -2.468977,
00086       -2.827562, -3.473582, -4.437612,
00087       -6.047662, -9.267761, -18.91806;
00088   fvecp\_ref <<
00089       -7.724666e-09, -3.432406e-09, -2.034843e-10,
00090       2.313685e-09,  4.331078e-09,  5.984096e-09,
00091       7.363281e-09,   8.53147e-09,  1.488591e-08,
00092       2.33585e-08,  3.522012e-08,  5.301255e-08,
00093       8.26666e-08,  1.419747e-07,   3.19899e-07;
00094   err\_ref <<
00095       0.1141397,  0.09943516,  0.09674474,
00096       0.09980447,  0.1073116, 0.1220445,
00097       0.1526814, 1, 1,
00098       1, 1, 1,
00099       1, 1, 1;
00100 
00101   VERIFY\_IS\_APPROX(fvec, fvec\_ref);
00102   VERIFY\_IS\_APPROX(fvecp, fvecp\_ref);
00103   VERIFY\_IS\_APPROX(err, err\_ref);
00104 \}
00105 
00106 \textcolor{comment}{// Generic functor}
00107 \textcolor{keyword}{template}<\textcolor{keyword}{typename} \_Scalar, \textcolor{keywordtype}{int} NX=Dynamic, \textcolor{keywordtype}{int} NY=Dynamic>
\Hypertarget{eigen_2unsupported_2test_2_non_linear_optimization_8cpp_source_l00108}\hyperlink{struct_functor}{00108} \textcolor{keyword}{struct }\hyperlink{struct_functor}{Functor}
00109 \{
00110   \textcolor{keyword}{typedef} \_Scalar Scalar;
00111   \textcolor{keyword}{enum} \{
00112     InputsAtCompileTime = NX,
00113     ValuesAtCompileTime = NY
00114   \};
00115   \textcolor{keyword}{typedef} Matrix<Scalar,InputsAtCompileTime,1> InputType;
00116   \textcolor{keyword}{typedef} Matrix<Scalar,ValuesAtCompileTime,1> ValueType;
00117   \textcolor{keyword}{typedef} Matrix<Scalar,ValuesAtCompileTime,InputsAtCompileTime> JacobianType;
00118 
00119   \textcolor{keyword}{const} \textcolor{keywordtype}{int} m\_inputs, m\_values;
00120 
00121   \hyperlink{struct_functor}{Functor}() : m\_inputs(InputsAtCompileTime), m\_values(ValuesAtCompileTime) \{\}
00122   \hyperlink{struct_functor}{Functor}(\textcolor{keywordtype}{int} inputs, \textcolor{keywordtype}{int} values) : m\_inputs(inputs), m\_values(values) \{\}
00123 
00124   \textcolor{keywordtype}{int} inputs()\textcolor{keyword}{ const }\{ \textcolor{keywordflow}{return} m\_inputs; \}
00125   \textcolor{keywordtype}{int} values()\textcolor{keyword}{ const }\{ \textcolor{keywordflow}{return} m\_values; \}
00126 
00127   \textcolor{comment}{// you should define that in the subclass :}
00128 \textcolor{comment}{//  void operator() (const InputType& x, ValueType* v, JacobianType* \_j=0) const;}
00129 \};
00130 
00131 \textcolor{keyword}{struct }\hyperlink{structlmder__functor}{lmder\_functor} : \hyperlink{struct_functor}{Functor}<double>
00132 \{
00133     \hyperlink{structlmder__functor}{lmder\_functor}(\textcolor{keywordtype}{void}): \hyperlink{struct_functor}{Functor<double>}(3,15) \{\}
00134     \textcolor{keywordtype}{int} operator()(\textcolor{keyword}{const} VectorXd &x, VectorXd &fvec)\textcolor{keyword}{ const}
00135 \textcolor{keyword}{    }\{
00136         \textcolor{keywordtype}{double} tmp1, tmp2, tmp3;
00137         \textcolor{keyword}{static} \textcolor{keyword}{const} \textcolor{keywordtype}{double} y[15] = \{1.4e-1, 1.8e-1, 2.2e-1, 2.5e-1, 2.9e-1, 3.2e-1, 3.5e-1,
00138             3.9e-1, 3.7e-1, 5.8e-1, 7.3e-1, 9.6e-1, 1.34, 2.1, 4.39\};
00139 
00140         \textcolor{keywordflow}{for} (\textcolor{keywordtype}{int} i = 0; i < values(); i++)
00141         \{
00142             tmp1 = i+1;
00143             tmp2 = 16 - i - 1;
00144             tmp3 = (i>=8)? tmp2 : tmp1;
00145             fvec[i] = y[i] - (x[0] + tmp1/(x[1]*tmp2 + x[2]*tmp3));
00146         \}
00147         \textcolor{keywordflow}{return} 0;
00148     \}
00149 
00150     \textcolor{keywordtype}{int} df(\textcolor{keyword}{const} VectorXd &x, MatrixXd &fjac)\textcolor{keyword}{ const}
00151 \textcolor{keyword}{    }\{
00152         \textcolor{keywordtype}{double} tmp1, tmp2, tmp3, tmp4;
00153         \textcolor{keywordflow}{for} (\textcolor{keywordtype}{int} i = 0; i < values(); i++)
00154         \{
00155             tmp1 = i+1;
00156             tmp2 = 16 - i - 1;
00157             tmp3 = (i>=8)? tmp2 : tmp1;
00158             tmp4 = (x[1]*tmp2 + x[2]*tmp3); tmp4 = tmp4*tmp4;
00159             fjac(i,0) = -1;
00160             fjac(i,1) = tmp1*tmp2/tmp4;
00161             fjac(i,2) = tmp1*tmp3/tmp4;
00162         \}
00163         \textcolor{keywordflow}{return} 0;
00164     \}
00165 \};
00166 
00167 \textcolor{keywordtype}{void} testLmder1()
00168 \{
00169   \textcolor{keywordtype}{int} n=3, info;
00170 
00171   VectorXd x;
00172 
00173   \textcolor{comment}{/* the following starting values provide a rough fit. */}
00174   x.setConstant(n, 1.);
00175 
00176   \textcolor{comment}{// do the computation}
00177   \hyperlink{structlmder__functor}{lmder\_functor} functor;
00178   LevenbergMarquardt<lmder\_functor> lm(functor);
00179   info = lm.lmder1(x);
00180 
00181   \textcolor{comment}{// check return value}
00182   VERIFY\_IS\_EQUAL(info, 1);
00183   VERIFY\_IS\_EQUAL(lm.nfev, 6);
00184   VERIFY\_IS\_EQUAL(lm.njev, 5);
00185 
00186   \textcolor{comment}{// check norm}
00187   VERIFY\_IS\_APPROX(lm.fvec.blueNorm(), 0.09063596);
00188 
00189   \textcolor{comment}{// check x}
00190   VectorXd x\_ref(n);
00191   x\_ref << 0.08241058, 1.133037, 2.343695;
00192   VERIFY\_IS\_APPROX(x, x\_ref);
00193 \}
00194 
00195 \textcolor{keywordtype}{void} testLmder()
00196 \{
00197   \textcolor{keyword}{const} \textcolor{keywordtype}{int} m=15, n=3;
00198   \textcolor{keywordtype}{int} info;
00199   \textcolor{keywordtype}{double} fnorm, covfac;
00200   VectorXd x;
00201 
00202   \textcolor{comment}{/* the following starting values provide a rough fit. */}
00203   x.setConstant(n, 1.);
00204 
00205   \textcolor{comment}{// do the computation}
00206   \hyperlink{structlmder__functor}{lmder\_functor} functor;
00207   LevenbergMarquardt<lmder\_functor> lm(functor);
00208   info = lm.minimize(x);
00209 
00210   \textcolor{comment}{// check return values}
00211   VERIFY\_IS\_EQUAL(info, 1);
00212   VERIFY\_IS\_EQUAL(lm.nfev, 6);
00213   VERIFY\_IS\_EQUAL(lm.njev, 5);
00214 
00215   \textcolor{comment}{// check norm}
00216   fnorm = lm.fvec.blueNorm();
00217   VERIFY\_IS\_APPROX(fnorm, 0.09063596);
00218 
00219   \textcolor{comment}{// check x}
00220   VectorXd x\_ref(n);
00221   x\_ref << 0.08241058, 1.133037, 2.343695;
00222   VERIFY\_IS\_APPROX(x, x\_ref);
00223 
00224   \textcolor{comment}{// check covariance}
00225   covfac = fnorm*fnorm/(m-n);
00226   internal::covar(lm.fjac, lm.permutation.indices()); \textcolor{comment}{// TODO : move this as a function of lm}
00227 
00228   MatrixXd cov\_ref(n,n);
00229   cov\_ref <<
00230       0.0001531202,   0.002869941,  -0.002656662,
00231       0.002869941,    0.09480935,   -0.09098995,
00232       -0.002656662,   -0.09098995,    0.08778727;
00233 
00234 \textcolor{comment}{//  std::cout << fjac*covfac << std::endl;}
00235 
00236   MatrixXd cov;
00237   cov =  covfac*lm.fjac.topLeftCorner<n,n>();
00238   VERIFY\_IS\_APPROX( cov, cov\_ref);
00239   \textcolor{comment}{// TODO: why isn't this allowed ? :}
00240   \textcolor{comment}{// VERIFY\_IS\_APPROX( covfac*fjac.topLeftCorner<n,n>() , cov\_ref);}
00241 \}
00242 
\Hypertarget{eigen_2unsupported_2test_2_non_linear_optimization_8cpp_source_l00243}\hyperlink{structhybrj__functor}{00243} \textcolor{keyword}{struct }\hyperlink{structhybrj__functor}{hybrj\_functor} : \hyperlink{struct_functor}{Functor}<double>
00244 \{
00245     \hyperlink{structhybrj__functor}{hybrj\_functor}(\textcolor{keywordtype}{void}) : \hyperlink{struct_functor}{Functor<double>}(9,9) \{\}
00246 
00247     \textcolor{keywordtype}{int} operator()(\textcolor{keyword}{const} VectorXd &x, VectorXd &fvec)
00248     \{
00249         \textcolor{keywordtype}{double} temp, temp1, temp2;
00250         \textcolor{keyword}{const} VectorXd::Index n = x.size();
00251         assert(fvec.size()==n);
00252         \textcolor{keywordflow}{for} (VectorXd::Index k = 0; k < n; k++)
00253         \{
00254             temp = (3. - 2.*x[k])*x[k];
00255             temp1 = 0.;
00256             \textcolor{keywordflow}{if} (k) temp1 = x[k-1];
00257             temp2 = 0.;
00258             \textcolor{keywordflow}{if} (k != n-1) temp2 = x[k+1];
00259             fvec[k] = temp - temp1 - 2.*temp2 + 1.;
00260         \}
00261         \textcolor{keywordflow}{return} 0;
00262     \}
00263     \textcolor{keywordtype}{int} df(\textcolor{keyword}{const} VectorXd &x, MatrixXd &fjac)
00264     \{
00265         \textcolor{keyword}{const} VectorXd::Index n = x.size();
00266         assert(fjac.rows()==n);
00267         assert(fjac.cols()==n);
00268         \textcolor{keywordflow}{for} (VectorXd::Index k = 0; k < n; k++)
00269         \{
00270             \textcolor{keywordflow}{for} (VectorXd::Index j = 0; j < n; j++)
00271                 fjac(k,j) = 0.;
00272             fjac(k,k) = 3.- 4.*x[k];
00273             \textcolor{keywordflow}{if} (k) fjac(k,k-1) = -1.;
00274             \textcolor{keywordflow}{if} (k != n-1) fjac(k,k+1) = -2.;
00275         \}
00276         \textcolor{keywordflow}{return} 0;
00277     \}
00278 \};
00279 
00280 
00281 \textcolor{keywordtype}{void} testHybrj1()
00282 \{
00283   \textcolor{keyword}{const} \textcolor{keywordtype}{int} n=9;
00284   \textcolor{keywordtype}{int} info;
00285   VectorXd x(n);
00286 
00287   \textcolor{comment}{/* the following starting values provide a rough fit. */}
00288   x.setConstant(n, -1.);
00289 
00290   \textcolor{comment}{// do the computation}
00291   \hyperlink{structhybrj__functor}{hybrj\_functor} functor;
00292   HybridNonLinearSolver<hybrj\_functor> solver(functor);
00293   info = solver.hybrj1(x);
00294 
00295   \textcolor{comment}{// check return value}
00296   VERIFY\_IS\_EQUAL(info, 1);
00297   VERIFY\_IS\_EQUAL(solver.nfev, 11);
00298   VERIFY\_IS\_EQUAL(solver.njev, 1);
00299 
00300   \textcolor{comment}{// check norm}
00301   VERIFY\_IS\_APPROX(solver.fvec.blueNorm(), 1.192636e-08);
00302 
00303 
00304 \textcolor{comment}{// check x}
00305   VectorXd x\_ref(n);
00306   x\_ref <<
00307      -0.5706545,    -0.6816283,    -0.7017325,
00308      -0.7042129,     -0.701369,    -0.6918656,
00309      -0.665792,    -0.5960342,    -0.4164121;
00310   VERIFY\_IS\_APPROX(x, x\_ref);
00311 \}
00312 
00313 \textcolor{keywordtype}{void} testHybrj()
00314 \{
00315   \textcolor{keyword}{const} \textcolor{keywordtype}{int} n=9;
00316   \textcolor{keywordtype}{int} info;
00317   VectorXd x(n);
00318 
00319   \textcolor{comment}{/* the following starting values provide a rough fit. */}
00320   x.setConstant(n, -1.);
00321 
00322 
00323   \textcolor{comment}{// do the computation}
00324   \hyperlink{structhybrj__functor}{hybrj\_functor} functor;
00325   HybridNonLinearSolver<hybrj\_functor> solver(functor);
00326   solver.diag.setConstant(n, 1.);
00327   solver.useExternalScaling = \textcolor{keyword}{true};
00328   info = solver.solve(x);
00329 
00330   \textcolor{comment}{// check return value}
00331   VERIFY\_IS\_EQUAL(info, 1);
00332   VERIFY\_IS\_EQUAL(solver.nfev, 11);
00333   VERIFY\_IS\_EQUAL(solver.njev, 1);
00334 
00335   \textcolor{comment}{// check norm}
00336   VERIFY\_IS\_APPROX(solver.fvec.blueNorm(), 1.192636e-08);
00337 
00338 
00339 \textcolor{comment}{// check x}
00340   VectorXd x\_ref(n);
00341   x\_ref <<
00342      -0.5706545,    -0.6816283,    -0.7017325,
00343      -0.7042129,     -0.701369,    -0.6918656,
00344      -0.665792,    -0.5960342,    -0.4164121;
00345   VERIFY\_IS\_APPROX(x, x\_ref);
00346 
00347 \}
00348 
\Hypertarget{eigen_2unsupported_2test_2_non_linear_optimization_8cpp_source_l00349}\hyperlink{structhybrd__functor}{00349} \textcolor{keyword}{struct }\hyperlink{structhybrd__functor}{hybrd\_functor} : \hyperlink{struct_functor}{Functor}<double>
00350 \{
00351     \hyperlink{structhybrd__functor}{hybrd\_functor}(\textcolor{keywordtype}{void}) : \hyperlink{struct_functor}{Functor<double>}(9,9) \{\}
00352     \textcolor{keywordtype}{int} operator()(\textcolor{keyword}{const} VectorXd &x, VectorXd &fvec)\textcolor{keyword}{ const}
00353 \textcolor{keyword}{    }\{
00354         \textcolor{keywordtype}{double} temp, temp1, temp2;
00355         \textcolor{keyword}{const} VectorXd::Index n = x.size();
00356 
00357         assert(fvec.size()==n);
00358         \textcolor{keywordflow}{for} (VectorXd::Index k=0; k < n; k++)
00359         \{
00360             temp = (3. - 2.*x[k])*x[k];
00361             temp1 = 0.;
00362             \textcolor{keywordflow}{if} (k) temp1 = x[k-1];
00363             temp2 = 0.;
00364             \textcolor{keywordflow}{if} (k != n-1) temp2 = x[k+1];
00365             fvec[k] = temp - temp1 - 2.*temp2 + 1.;
00366         \}
00367         \textcolor{keywordflow}{return} 0;
00368     \}
00369 \};
00370 
00371 \textcolor{keywordtype}{void} testHybrd1()
00372 \{
00373   \textcolor{keywordtype}{int} n=9, info;
00374   VectorXd x(n);
00375 
00376   \textcolor{comment}{/* the following starting values provide a rough solution. */}
00377   x.setConstant(n, -1.);
00378 
00379   \textcolor{comment}{// do the computation}
00380   \hyperlink{structhybrd__functor}{hybrd\_functor} functor;
00381   HybridNonLinearSolver<hybrd\_functor> solver(functor);
00382   info = solver.hybrd1(x);
00383 
00384   \textcolor{comment}{// check return value}
00385   VERIFY\_IS\_EQUAL(info, 1);
00386   VERIFY\_IS\_EQUAL(solver.nfev, 20);
00387 
00388   \textcolor{comment}{// check norm}
00389   VERIFY\_IS\_APPROX(solver.fvec.blueNorm(), 1.192636e-08);
00390 
00391   \textcolor{comment}{// check x}
00392   VectorXd x\_ref(n);
00393   x\_ref << -0.5706545, -0.6816283, -0.7017325, -0.7042129, -0.701369, -0.6918656, -0.665792, -0.5960342, -0
      .4164121;
00394   VERIFY\_IS\_APPROX(x, x\_ref);
00395 \}
00396 
00397 \textcolor{keywordtype}{void} testHybrd()
00398 \{
00399   \textcolor{keyword}{const} \textcolor{keywordtype}{int} n=9;
00400   \textcolor{keywordtype}{int} info;
00401   VectorXd x;
00402 
00403   \textcolor{comment}{/* the following starting values provide a rough fit. */}
00404   x.setConstant(n, -1.);
00405 
00406   \textcolor{comment}{// do the computation}
00407   \hyperlink{structhybrd__functor}{hybrd\_functor} functor;
00408   HybridNonLinearSolver<hybrd\_functor> solver(functor);
00409   solver.parameters.nb\_of\_subdiagonals = 1;
00410   solver.parameters.nb\_of\_superdiagonals = 1;
00411   solver.diag.setConstant(n, 1.);
00412   solver.useExternalScaling = \textcolor{keyword}{true};
00413   info = solver.solveNumericalDiff(x);
00414 
00415   \textcolor{comment}{// check return value}
00416   VERIFY\_IS\_EQUAL(info, 1);
00417   VERIFY\_IS\_EQUAL(solver.nfev, 14);
00418 
00419   \textcolor{comment}{// check norm}
00420   VERIFY\_IS\_APPROX(solver.fvec.blueNorm(), 1.192636e-08);
00421 
00422   \textcolor{comment}{// check x}
00423   VectorXd x\_ref(n);
00424   x\_ref <<
00425       -0.5706545,    -0.6816283,    -0.7017325,
00426       -0.7042129,     -0.701369,    -0.6918656,
00427       -0.665792,    -0.5960342,    -0.4164121;
00428   VERIFY\_IS\_APPROX(x, x\_ref);
00429 \}
00430 
\Hypertarget{eigen_2unsupported_2test_2_non_linear_optimization_8cpp_source_l00431}\hyperlink{structlmstr__functor}{00431} \textcolor{keyword}{struct }\hyperlink{structlmstr__functor}{lmstr\_functor} : \hyperlink{struct_functor}{Functor}<double>
00432 \{
00433     \hyperlink{structlmstr__functor}{lmstr\_functor}(\textcolor{keywordtype}{void}) : \hyperlink{struct_functor}{Functor<double>}(3,15) \{\}
00434     \textcolor{keywordtype}{int} operator()(\textcolor{keyword}{const} VectorXd &x, VectorXd &fvec)
00435     \{
00436         \textcolor{comment}{/*  subroutine fcn for lmstr1 example. */}
00437         \textcolor{keywordtype}{double} tmp1, tmp2, tmp3;
00438         \textcolor{keyword}{static} \textcolor{keyword}{const} \textcolor{keywordtype}{double} y[15]=\{1.4e-1, 1.8e-1, 2.2e-1, 2.5e-1, 2.9e-1, 3.2e-1, 3.5e-1,
00439             3.9e-1, 3.7e-1, 5.8e-1, 7.3e-1, 9.6e-1, 1.34, 2.1, 4.39\};
00440 
00441         assert(15==fvec.size());
00442         assert(3==x.size());
00443 
00444         \textcolor{keywordflow}{for} (\textcolor{keywordtype}{int} i=0; i<15; i++)
00445         \{
00446             tmp1 = i+1;
00447             tmp2 = 16 - i - 1;
00448             tmp3 = (i>=8)? tmp2 : tmp1;
00449             fvec[i] = y[i] - (x[0] + tmp1/(x[1]*tmp2 + x[2]*tmp3));
00450         \}
00451         \textcolor{keywordflow}{return} 0;
00452     \}
00453     \textcolor{keywordtype}{int} df(\textcolor{keyword}{const} VectorXd &x, VectorXd &jac\_row, VectorXd::Index rownb)
00454     \{
00455         assert(x.size()==3);
00456         assert(jac\_row.size()==x.size());
00457         \textcolor{keywordtype}{double} tmp1, tmp2, tmp3, tmp4;
00458 
00459         VectorXd::Index i = rownb-2;
00460         tmp1 = i+1;
00461         tmp2 = 16 - i - 1;
00462         tmp3 = (i>=8)? tmp2 : tmp1;
00463         tmp4 = (x[1]*tmp2 + x[2]*tmp3); tmp4 = tmp4*tmp4;
00464         jac\_row[0] = -1;
00465         jac\_row[1] = tmp1*tmp2/tmp4;
00466         jac\_row[2] = tmp1*tmp3/tmp4;
00467         \textcolor{keywordflow}{return} 0;
00468     \}
00469 \};
00470 
00471 \textcolor{keywordtype}{void} testLmstr1()
00472 \{
00473   \textcolor{keyword}{const} \textcolor{keywordtype}{int} n=3;
00474   \textcolor{keywordtype}{int} info;
00475 
00476   VectorXd x(n);
00477 
00478   \textcolor{comment}{/* the following starting values provide a rough fit. */}
00479   x.setConstant(n, 1.);
00480 
00481   \textcolor{comment}{// do the computation}
00482   \hyperlink{structlmstr__functor}{lmstr\_functor} functor;
00483   LevenbergMarquardt<lmstr\_functor> lm(functor);
00484   info = lm.lmstr1(x);
00485 
00486   \textcolor{comment}{// check return value}
00487   VERIFY\_IS\_EQUAL(info, 1);
00488   VERIFY\_IS\_EQUAL(lm.nfev, 6);
00489   VERIFY\_IS\_EQUAL(lm.njev, 5);
00490 
00491   \textcolor{comment}{// check norm}
00492   VERIFY\_IS\_APPROX(lm.fvec.blueNorm(), 0.09063596);
00493 
00494   \textcolor{comment}{// check x}
00495   VectorXd x\_ref(n);
00496   x\_ref << 0.08241058, 1.133037, 2.343695 ;
00497   VERIFY\_IS\_APPROX(x, x\_ref);
00498 \}
00499 
00500 \textcolor{keywordtype}{void} testLmstr()
00501 \{
00502   \textcolor{keyword}{const} \textcolor{keywordtype}{int} n=3;
00503   \textcolor{keywordtype}{int} info;
00504   \textcolor{keywordtype}{double} fnorm;
00505   VectorXd x(n);
00506 
00507   \textcolor{comment}{/* the following starting values provide a rough fit. */}
00508   x.setConstant(n, 1.);
00509 
00510   \textcolor{comment}{// do the computation}
00511   \hyperlink{structlmstr__functor}{lmstr\_functor} functor;
00512   LevenbergMarquardt<lmstr\_functor> lm(functor);
00513   info = lm.minimizeOptimumStorage(x);
00514 
00515   \textcolor{comment}{// check return values}
00516   VERIFY\_IS\_EQUAL(info, 1);
00517   VERIFY\_IS\_EQUAL(lm.nfev, 6);
00518   VERIFY\_IS\_EQUAL(lm.njev, 5);
00519 
00520   \textcolor{comment}{// check norm}
00521   fnorm = lm.fvec.blueNorm();
00522   VERIFY\_IS\_APPROX(fnorm, 0.09063596);
00523 
00524   \textcolor{comment}{// check x}
00525   VectorXd x\_ref(n);
00526   x\_ref << 0.08241058, 1.133037, 2.343695;
00527   VERIFY\_IS\_APPROX(x, x\_ref);
00528 
00529 \}
00530 
00531 \textcolor{keyword}{struct }\hyperlink{structlmdif__functor}{lmdif\_functor} : \hyperlink{struct_functor}{Functor}<double>
00532 \{
00533     \hyperlink{structlmdif__functor}{lmdif\_functor}(\textcolor{keywordtype}{void}) : \hyperlink{struct_functor}{Functor<double>}(3,15) \{\}
00534     \textcolor{keywordtype}{int} operator()(\textcolor{keyword}{const} VectorXd &x, VectorXd &fvec)\textcolor{keyword}{ const}
00535 \textcolor{keyword}{    }\{
00536         \textcolor{keywordtype}{int} i;
00537         \textcolor{keywordtype}{double} tmp1,tmp2,tmp3;
00538         \textcolor{keyword}{static} \textcolor{keyword}{const} \textcolor{keywordtype}{double} y[15]=\{1.4e-1,1.8e-1,2.2e-1,2.5e-1,2.9e-1,3.2e-1,3.5e-1,3.9e-1,
00539             3.7e-1,5.8e-1,7.3e-1,9.6e-1,1.34e0,2.1e0,4.39e0\};
00540 
00541         assert(x.size()==3);
00542         assert(fvec.size()==15);
00543         \textcolor{keywordflow}{for} (i=0; i<15; i++)
00544         \{
00545             tmp1 = i+1;
00546             tmp2 = 15 - i;
00547             tmp3 = tmp1;
00548 
00549             \textcolor{keywordflow}{if} (i >= 8) tmp3 = tmp2;
00550             fvec[i] = y[i] - (x[0] + tmp1/(x[1]*tmp2 + x[2]*tmp3));
00551         \}
00552         \textcolor{keywordflow}{return} 0;
00553     \}
00554 \};
00555 
00556 \textcolor{keywordtype}{void} testLmdif1()
00557 \{
00558   \textcolor{keyword}{const} \textcolor{keywordtype}{int} n=3;
00559   \textcolor{keywordtype}{int} info;
00560 
00561   VectorXd x(n), fvec(15);
00562 
00563   \textcolor{comment}{/* the following starting values provide a rough fit. */}
00564   x.setConstant(n, 1.);
00565 
00566   \textcolor{comment}{// do the computation}
00567   \hyperlink{structlmdif__functor}{lmdif\_functor} functor;
00568   DenseIndex nfev;
00569   info = LevenbergMarquardt<lmdif\_functor>::lmdif1(functor, x, &nfev);
00570 
00571   \textcolor{comment}{// check return value}
00572   VERIFY\_IS\_EQUAL(info, 1);
00573   VERIFY\_IS\_EQUAL(nfev, 26);
00574 
00575   \textcolor{comment}{// check norm}
00576   functor(x, fvec);
00577   VERIFY\_IS\_APPROX(fvec.blueNorm(), 0.09063596);
00578 
00579   \textcolor{comment}{// check x}
00580   VectorXd x\_ref(n);
00581   x\_ref << 0.0824106, 1.1330366, 2.3436947;
00582   VERIFY\_IS\_APPROX(x, x\_ref);
00583 
00584 \}
00585 
00586 \textcolor{keywordtype}{void} testLmdif()
00587 \{
00588   \textcolor{keyword}{const} \textcolor{keywordtype}{int} m=15, n=3;
00589   \textcolor{keywordtype}{int} info;
00590   \textcolor{keywordtype}{double} fnorm, covfac;
00591   VectorXd x(n);
00592 
00593   \textcolor{comment}{/* the following starting values provide a rough fit. */}
00594   x.setConstant(n, 1.);
00595 
00596   \textcolor{comment}{// do the computation}
00597   \hyperlink{structlmdif__functor}{lmdif\_functor} functor;
00598   NumericalDiff<lmdif\_functor> numDiff(functor);
00599   LevenbergMarquardt<NumericalDiff<lmdif\_functor> > lm(numDiff);
00600   info = lm.minimize(x);
00601 
00602   \textcolor{comment}{// check return values}
00603   VERIFY\_IS\_EQUAL(info, 1);
00604   VERIFY\_IS\_EQUAL(lm.nfev, 26);
00605 
00606   \textcolor{comment}{// check norm}
00607   fnorm = lm.fvec.blueNorm();
00608   VERIFY\_IS\_APPROX(fnorm, 0.09063596);
00609 
00610   \textcolor{comment}{// check x}
00611   VectorXd x\_ref(n);
00612   x\_ref << 0.08241058, 1.133037, 2.343695;
00613   VERIFY\_IS\_APPROX(x, x\_ref);
00614 
00615   \textcolor{comment}{// check covariance}
00616   covfac = fnorm*fnorm/(m-n);
00617   internal::covar(lm.fjac, lm.permutation.indices()); \textcolor{comment}{// TODO : move this as a function of lm}
00618 
00619   MatrixXd cov\_ref(n,n);
00620   cov\_ref <<
00621       0.0001531202,   0.002869942,  -0.002656662,
00622       0.002869942,    0.09480937,   -0.09098997,
00623       -0.002656662,   -0.09098997,    0.08778729;
00624 
00625 \textcolor{comment}{//  std::cout << fjac*covfac << std::endl;}
00626 
00627   MatrixXd cov;
00628   cov =  covfac*lm.fjac.topLeftCorner<n,n>();
00629   VERIFY\_IS\_APPROX( cov, cov\_ref);
00630   \textcolor{comment}{// TODO: why isn't this allowed ? :}
00631   \textcolor{comment}{// VERIFY\_IS\_APPROX( covfac*fjac.topLeftCorner<n,n>() , cov\_ref);}
00632 \}
00633 
00634 \textcolor{keyword}{struct }\hyperlink{structchwirut2__functor}{chwirut2\_functor} : \hyperlink{struct_functor}{Functor}<double>
00635 \{
00636     \hyperlink{structchwirut2__functor}{chwirut2\_functor}(\textcolor{keywordtype}{void}) : \hyperlink{struct_functor}{Functor<double>}(3,54) \{\}
00637     \textcolor{keyword}{static} \textcolor{keyword}{const} \textcolor{keywordtype}{double} m\_x[54];
00638     \textcolor{keyword}{static} \textcolor{keyword}{const} \textcolor{keywordtype}{double} m\_y[54];
00639     \textcolor{keywordtype}{int} operator()(\textcolor{keyword}{const} VectorXd &b, VectorXd &fvec)
00640     \{
00641         \textcolor{keywordtype}{int} i;
00642 
00643         assert(b.size()==3);
00644         assert(fvec.size()==54);
00645         \textcolor{keywordflow}{for}(i=0; i<54; i++) \{
00646             \textcolor{keywordtype}{double} x = m\_x[i];
00647             fvec[i] = exp(-b[0]*x)/(b[1]+b[2]*x) - m\_y[i];
00648         \}
00649         \textcolor{keywordflow}{return} 0;
00650     \}
00651     \textcolor{keywordtype}{int} df(\textcolor{keyword}{const} VectorXd &b, MatrixXd &fjac)
00652     \{
00653         assert(b.size()==3);
00654         assert(fjac.rows()==54);
00655         assert(fjac.cols()==3);
00656         \textcolor{keywordflow}{for}(\textcolor{keywordtype}{int} i=0; i<54; i++) \{
00657             \textcolor{keywordtype}{double} x = m\_x[i];
00658             \textcolor{keywordtype}{double} factor = 1./(b[1]+b[2]*x);
00659             \textcolor{keywordtype}{double} e = exp(-b[0]*x);
00660             fjac(i,0) = -x*e*factor;
00661             fjac(i,1) = -e*factor*factor;
00662             fjac(i,2) = -x*e*factor*factor;
00663         \}
00664         \textcolor{keywordflow}{return} 0;
00665     \}
00666 \};
00667 \textcolor{keyword}{const} \textcolor{keywordtype}{double} chwirut2\_functor::m\_x[54] = \{ 0.500E0, 1.000E0, 1.750E0, 3.750E0, 5.750E0, 0.875E0, 2.250E0, 3
      .250E0, 5.250E0, 0.750E0, 1.750E0, 2.750E0, 4.750E0, 0.625E0, 1.250E0, 2.250E0, 4.250E0, .500E0, 3.000E0, .7
      50E0, 3.000E0, 1.500E0, 6.000E0, 3.000E0, 6.000E0, 1.500E0, 3.000E0, .500E0, 2.000E0, 4.000E0, .750E0, 2.000
      E0, 5.000E0, .750E0, 2.250E0, 3.750E0, 5.750E0, 3.000E0, .750E0, 2.500E0, 4.000E0, .750E0, 2.500E0, 4.000E0,
       .750E0, 2.500E0, 4.000E0, .500E0, 6.000E0, 3.000E0, .500E0, 2.750E0, .500E0, 1.750E0\};
00668 \textcolor{keyword}{const} \textcolor{keywordtype}{double} chwirut2\_functor::m\_y[54] = \{ 92.9000E0 ,57.1000E0 ,31.0500E0 ,11.5875E0 ,8.0250E0 ,63.6000E0 
      ,21.4000E0 ,14.2500E0 ,8.4750E0 ,63.8000E0 ,26.8000E0 ,16.4625E0 ,7.1250E0 ,67.3000E0 ,41.0000E0 ,21.1500E0 
      ,8.1750E0 ,81.5000E0 ,13.1200E0 ,59.9000E0 ,14.6200E0 ,32.9000E0 ,5.4400E0 ,12.5600E0 ,5.4400E0 ,32.0000E0 ,
      13.9500E0 ,75.8000E0 ,20.0000E0 ,10.4200E0 ,59.5000E0 ,21.6700E0 ,8.5500E0 ,62.0000E0 ,20.2000E0 ,7.7600E0 ,
      3.7500E0 ,11.8100E0 ,54.7000E0 ,23.7000E0 ,11.5500E0 ,61.3000E0 ,17.7000E0 ,8.7400E0 ,59.2000E0 ,16.3000E0 ,
      8.6200E0 ,81.0000E0 ,4.8700E0 ,14.6200E0 ,81.7000E0 ,17.1700E0 ,81.3000E0 ,28.9000E0  \};
00669 
00670 \textcolor{comment}{// http://www.itl.nist.gov/div898/strd/nls/data/chwirut2.shtml}
00671 \textcolor{keywordtype}{void} testNistChwirut2(\textcolor{keywordtype}{void})
00672 \{
00673   \textcolor{keyword}{const} \textcolor{keywordtype}{int} n=3;
00674   \textcolor{keywordtype}{int} info;
00675 
00676   VectorXd x(n);
00677 
00678   \textcolor{comment}{/*}
00679 \textcolor{comment}{   * First try}
00680 \textcolor{comment}{   */}
00681   x<< 0.1, 0.01, 0.02;
00682   \textcolor{comment}{// do the computation}
00683   \hyperlink{structchwirut2__functor}{chwirut2\_functor} functor;
00684   LevenbergMarquardt<chwirut2\_functor> lm(functor);
00685   info = lm.minimize(x);
00686 
00687   \textcolor{comment}{// check return value}
00688   VERIFY\_IS\_EQUAL(info, 1);
00689   VERIFY\_IS\_EQUAL(lm.nfev, 10);
00690   VERIFY\_IS\_EQUAL(lm.njev, 8);
00691   \textcolor{comment}{// check norm^2}
00692   VERIFY\_IS\_APPROX(lm.fvec.squaredNorm(), 5.1304802941E+02);
00693   \textcolor{comment}{// check x}
00694   VERIFY\_IS\_APPROX(x[0], 1.6657666537E-01);
00695   VERIFY\_IS\_APPROX(x[1], 5.1653291286E-03);
00696   VERIFY\_IS\_APPROX(x[2], 1.2150007096E-02);
00697 
00698   \textcolor{comment}{/*}
00699 \textcolor{comment}{   * Second try}
00700 \textcolor{comment}{   */}
00701   x<< 0.15, 0.008, 0.010;
00702   \textcolor{comment}{// do the computation}
00703   lm.resetParameters();
00704   lm.parameters.ftol = 1.E6*NumTraits<double>::epsilon();
00705   lm.parameters.xtol = 1.E6*NumTraits<double>::epsilon();
00706   info = lm.minimize(x);
00707 
00708   \textcolor{comment}{// check return value}
00709   VERIFY\_IS\_EQUAL(info, 1);
00710   VERIFY\_IS\_EQUAL(lm.nfev, 7);
00711   VERIFY\_IS\_EQUAL(lm.njev, 6);
00712   \textcolor{comment}{// check norm^2}
00713   VERIFY\_IS\_APPROX(lm.fvec.squaredNorm(), 5.1304802941E+02);
00714   \textcolor{comment}{// check x}
00715   VERIFY\_IS\_APPROX(x[0], 1.6657666537E-01);
00716   VERIFY\_IS\_APPROX(x[1], 5.1653291286E-03);
00717   VERIFY\_IS\_APPROX(x[2], 1.2150007096E-02);
00718 \}
00719 
00720 
00721 \textcolor{keyword}{struct }\hyperlink{structmisra1a__functor}{misra1a\_functor} : \hyperlink{struct_functor}{Functor}<double>
00722 \{
00723     \hyperlink{structmisra1a__functor}{misra1a\_functor}(\textcolor{keywordtype}{void}) : \hyperlink{struct_functor}{Functor<double>}(2,14) \{\}
00724     \textcolor{keyword}{static} \textcolor{keyword}{const} \textcolor{keywordtype}{double} m\_x[14];
00725     \textcolor{keyword}{static} \textcolor{keyword}{const} \textcolor{keywordtype}{double} m\_y[14];
00726     \textcolor{keywordtype}{int} operator()(\textcolor{keyword}{const} VectorXd &b, VectorXd &fvec)
00727     \{
00728         assert(b.size()==2);
00729         assert(fvec.size()==14);
00730         \textcolor{keywordflow}{for}(\textcolor{keywordtype}{int} i=0; i<14; i++) \{
00731             fvec[i] = b[0]*(1.-exp(-b[1]*m\_x[i])) - m\_y[i] ;
00732         \}
00733         \textcolor{keywordflow}{return} 0;
00734     \}
00735     \textcolor{keywordtype}{int} df(\textcolor{keyword}{const} VectorXd &b, MatrixXd &fjac)
00736     \{
00737         assert(b.size()==2);
00738         assert(fjac.rows()==14);
00739         assert(fjac.cols()==2);
00740         \textcolor{keywordflow}{for}(\textcolor{keywordtype}{int} i=0; i<14; i++) \{
00741             fjac(i,0) = (1.-exp(-b[1]*m\_x[i]));
00742             fjac(i,1) = (b[0]*m\_x[i]*exp(-b[1]*m\_x[i]));
00743         \}
00744         \textcolor{keywordflow}{return} 0;
00745     \}
00746 \};
00747 \textcolor{keyword}{const} \textcolor{keywordtype}{double} misra1a\_functor::m\_x[14] = \{ 77.6E0, 114.9E0, 141.1E0, 190.8E0, 239.9E0, 289.0E0, 332.8E0, 378
      .4E0, 434.8E0, 477.3E0, 536.8E0, 593.1E0, 689.1E0, 760.0E0\};
00748 \textcolor{keyword}{const} \textcolor{keywordtype}{double} misra1a\_functor::m\_y[14] = \{ 10.07E0, 14.73E0, 17.94E0, 23.93E0, 29.61E0, 35.18E0, 40.02E0, 44
      .82E0, 50.76E0, 55.05E0, 61.01E0, 66.40E0, 75.47E0, 81.78E0\};
00749 
00750 \textcolor{comment}{// http://www.itl.nist.gov/div898/strd/nls/data/misra1a.shtml}
00751 \textcolor{keywordtype}{void} testNistMisra1a(\textcolor{keywordtype}{void})
00752 \{
00753   \textcolor{keyword}{const} \textcolor{keywordtype}{int} n=2;
00754   \textcolor{keywordtype}{int} info;
00755 
00756   VectorXd x(n);
00757 
00758   \textcolor{comment}{/*}
00759 \textcolor{comment}{   * First try}
00760 \textcolor{comment}{   */}
00761   x<< 500., 0.0001;
00762   \textcolor{comment}{// do the computation}
00763   \hyperlink{structmisra1a__functor}{misra1a\_functor} functor;
00764   LevenbergMarquardt<misra1a\_functor> lm(functor);
00765   info = lm.minimize(x);
00766 
00767   \textcolor{comment}{// check return value}
00768   VERIFY\_IS\_EQUAL(info, 1);
00769   VERIFY\_IS\_EQUAL(lm.nfev, 19);
00770   VERIFY\_IS\_EQUAL(lm.njev, 15);
00771   \textcolor{comment}{// check norm^2}
00772   VERIFY\_IS\_APPROX(lm.fvec.squaredNorm(), 1.2455138894E-01);
00773   \textcolor{comment}{// check x}
00774   VERIFY\_IS\_APPROX(x[0], 2.3894212918E+02);
00775   VERIFY\_IS\_APPROX(x[1], 5.5015643181E-04);
00776 
00777   \textcolor{comment}{/*}
00778 \textcolor{comment}{   * Second try}
00779 \textcolor{comment}{   */}
00780   x<< 250., 0.0005;
00781   \textcolor{comment}{// do the computation}
00782   info = lm.minimize(x);
00783 
00784   \textcolor{comment}{// check return value}
00785   VERIFY\_IS\_EQUAL(info, 1);
00786   VERIFY\_IS\_EQUAL(lm.nfev, 5);
00787   VERIFY\_IS\_EQUAL(lm.njev, 4);
00788   \textcolor{comment}{// check norm^2}
00789   VERIFY\_IS\_APPROX(lm.fvec.squaredNorm(), 1.2455138894E-01);
00790   \textcolor{comment}{// check x}
00791   VERIFY\_IS\_APPROX(x[0], 2.3894212918E+02);
00792   VERIFY\_IS\_APPROX(x[1], 5.5015643181E-04);
00793 \}
00794 
00795 \textcolor{keyword}{struct }\hyperlink{structhahn1__functor}{hahn1\_functor} : \hyperlink{struct_functor}{Functor}<double>
00796 \{
00797     \hyperlink{structhahn1__functor}{hahn1\_functor}(\textcolor{keywordtype}{void}) : \hyperlink{struct_functor}{Functor<double>}(7,236) \{\}
00798     \textcolor{keyword}{static} \textcolor{keyword}{const} \textcolor{keywordtype}{double} m\_x[236];
00799     \textcolor{keywordtype}{int} operator()(\textcolor{keyword}{const} VectorXd &b, VectorXd &fvec)
00800     \{
00801         \textcolor{keyword}{static} \textcolor{keyword}{const} \textcolor{keywordtype}{double} m\_y[236] = \{ .591E0 , 1.547E0 , 2.902E0 , 2.894E0 , 4.703E0 , 6.307E0 , 7.03E0 
       , 7.898E0 , 9.470E0 , 9.484E0 , 10.072E0 , 10.163E0 , 11.615E0 , 12.005E0 , 12.478E0 , 12.982E0 , 12.970E0 
      , 13.926E0 , 14.452E0 , 14.404E0 , 15.190E0 , 15.550E0 , 15.528E0 , 15.499E0 , 16.131E0 , 16.438E0 , 16.387
      E0 , 16.549E0 , 16.872E0 , 16.830E0 , 16.926E0 , 16.907E0 , 16.966E0 , 17.060E0 , 17.122E0 , 17.311E0 , 17.35
      5E0 , 17.668E0 , 17.767E0 , 17.803E0 , 17.765E0 , 17.768E0 , 17.736E0 , 17.858E0 , 17.877E0 , 17.912E0 , 18.
      046E0 , 18.085E0 , 18.291E0 , 18.357E0 , 18.426E0 , 18.584E0 , 18.610E0 , 18.870E0 , 18.795E0 , 19.111E0 , .
      367E0 , .796E0 , 0.892E0 , 1.903E0 , 2.150E0 , 3.697E0 , 5.870E0 , 6.421E0 , 7.422E0 , 9.944E0 , 11.023E0 , 
      11.87E0  , 12.786E0 , 14.067E0 , 13.974E0 , 14.462E0 , 14.464E0 , 15.381E0 , 15.483E0 , 15.59E0  , 16.075E0 
      , 16.347E0 , 16.181E0 , 16.915E0 , 17.003E0 , 16.978E0 , 17.756E0 , 17.808E0 , 17.868E0 , 18.481E0 , 18.486
      E0 , 19.090E0 , 16.062E0 , 16.337E0 , 16.345E0 ,
00802         16.388E0 , 17.159E0 , 17.116E0 , 17.164E0 , 17.123E0 , 17.979E0 , 17.974E0 , 18.007E0 , 17.993E0 , 
      18.523E0 , 18.669E0 , 18.617E0 , 19.371E0 , 19.330E0 , 0.080E0 , 0.248E0 , 1.089E0 , 1.418E0 , 2.278E0 , 3.6
      24E0 , 4.574E0 , 5.556E0 , 7.267E0 , 7.695E0 , 9.136E0 , 9.959E0 , 9.957E0 , 11.600E0 , 13.138E0 , 13.564E0 
      , 13.871E0 , 13.994E0 , 14.947E0 , 15.473E0 , 15.379E0 , 15.455E0 , 15.908E0 , 16.114E0 , 17.071E0 , 17.135
      E0 , 17.282E0 , 17.368E0 , 17.483E0 , 17.764E0 , 18.185E0 , 18.271E0 , 18.236E0 , 18.237E0 , 18.523E0 , 18.62
      7E0 , 18.665E0 , 19.086E0 , 0.214E0 , 0.943E0 , 1.429E0 , 2.241E0 , 2.951E0 , 3.782E0 , 4.757E0 , 5.602E0 , 
      7.169E0 , 8.920E0 , 10.055E0 , 12.035E0 , 12.861E0 , 13.436E0 , 14.167E0 , 14.755E0 , 15.168E0 , 15.651E0 , 
      15.746E0 , 16.216E0 , 16.445E0 , 16.965E0 , 17.121E0 , 17.206E0 , 17.250E0 , 17.339E0 , 17.793E0 , 18.123E0 
      , 18.49E0  , 18.566E0 , 18.645E0 , 18.706E0 , 18.924E0 , 19.1E0   , 0.375E0 , 0.471E0 , 1.504E0 , 2.204E0 , 
      2.813E0 , 4.765E0 , 9.835E0 , 10.040E0 , 11.946E0 , 12.596E0 , 
00803 13.303E0 , 13.922E0 , 14.440E0 , 14.951E0 , 15.627E0 , 15.639E0 , 15.814E0 , 16.315E0 , 16.334E0 , 16.430E0
       , 16.423E0 , 17.024E0 , 17.009E0 , 17.165E0 , 17.134E0 , 17.349E0 , 17.576E0 , 17.848E0 , 18.090E0 , 18.276
      E0 , 18.404E0 , 18.519E0 , 19.133E0 , 19.074E0 , 19.239E0 , 19.280E0 , 19.101E0 , 19.398E0 , 19.252E0 , 19.8
      9E0  , 20.007E0 , 19.929E0 , 19.268E0 , 19.324E0 , 20.049E0 , 20.107E0 , 20.062E0 , 20.065E0 , 19.286E0 , 19
      .972E0 , 20.088E0 , 20.743E0 , 20.83E0  , 20.935E0 , 21.035E0 , 20.93E0  , 21.074E0 , 21.085E0 , 20.935E0 \};
00804 
00805         \textcolor{comment}{//        int called=0; printf("call hahn1\_functor with  iflag=%d, called=%d\(\backslash\)n", iflag, called); if
       (iflag==1) called++;}
00806 
00807         assert(b.size()==7);
00808         assert(fvec.size()==236);
00809         \textcolor{keywordflow}{for}(\textcolor{keywordtype}{int} i=0; i<236; i++) \{
00810             \textcolor{keywordtype}{double} x=m\_x[i], xx=x*x, xxx=xx*x;
00811             fvec[i] = (b[0]+b[1]*x+b[2]*xx+b[3]*xxx) / (1.+b[4]*x+b[5]*xx+b[6]*xxx) - m\_y[i];
00812         \}
00813         \textcolor{keywordflow}{return} 0;
00814     \}
00815 
00816     \textcolor{keywordtype}{int} df(\textcolor{keyword}{const} VectorXd &b, MatrixXd &fjac)
00817     \{
00818         assert(b.size()==7);
00819         assert(fjac.rows()==236);
00820         assert(fjac.cols()==7);
00821         \textcolor{keywordflow}{for}(\textcolor{keywordtype}{int} i=0; i<236; i++) \{
00822             \textcolor{keywordtype}{double} x=m\_x[i], xx=x*x, xxx=xx*x;
00823             \textcolor{keywordtype}{double} fact = 1./(1.+b[4]*x+b[5]*xx+b[6]*xxx);
00824             fjac(i,0) = 1.*fact;
00825             fjac(i,1) = x*fact;
00826             fjac(i,2) = xx*fact;
00827             fjac(i,3) = xxx*fact;
00828             fact = - (b[0]+b[1]*x+b[2]*xx+b[3]*xxx) * fact * fact;
00829             fjac(i,4) = x*fact;
00830             fjac(i,5) = xx*fact;
00831             fjac(i,6) = xxx*fact;
00832         \}
00833         \textcolor{keywordflow}{return} 0;
00834     \}
00835 \};
00836 \textcolor{keyword}{const} \textcolor{keywordtype}{double} hahn1\_functor::m\_x[236] = \{ 24.41E0 , 34.82E0 , 44.09E0 , 45.07E0 , 54.98E0 , 65.51E0 , 70.53
      E0 , 75.70E0 , 89.57E0 , 91.14E0 , 96.40E0 , 97.19E0 , 114.26E0 , 120.25E0 , 127.08E0 , 133.55E0 , 133.61E0 ,
       158.67E0 , 172.74E0 , 171.31E0 , 202.14E0 , 220.55E0 , 221.05E0 , 221.39E0 , 250.99E0 , 268.99E0 , 271.80E0
       , 271.97E0 , 321.31E0 , 321.69E0 , 330.14E0 , 333.03E0 , 333.47E0 , 340.77E0 , 345.65E0 , 373.11E0 , 373.79
      E0 , 411.82E0 , 419.51E0 , 421.59E0 , 422.02E0 , 422.47E0 , 422.61E0 , 441.75E0 , 447.41E0 , 448.7E0  , 472.
      89E0 , 476.69E0 , 522.47E0 , 522.62E0 , 524.43E0 , 546.75E0 , 549.53E0 , 575.29E0 , 576.00E0 , 625.55E0 , 20
      .15E0 , 28.78E0 , 29.57E0 , 37.41E0 , 39.12E0 , 50.24E0 , 61.38E0 , 66.25E0 , 73.42E0 , 95.52E0 , 107.32E0 ,
       122.04E0 , 134.03E0 , 163.19E0 , 163.48E0 , 175.70E0 , 179.86E0 , 211.27E0 , 217.78E0 , 219.14E0 , 262.52E0
       , 268.01E0 , 268.62E0 , 336.25E0 , 337.23E0 , 339.33E0 , 427.38E0 , 428.58E0 , 432.68E0 , 528.99E0 , 531.08
      E0 , 628.34E0 , 253.24E0 , 273.13E0 , 273.66E0 ,
00837 282.10E0 , 346.62E0 , 347.19E0 , 348.78E0 , 351.18E0 , 450.10E0 , 450.35E0 , 451.92E0 , 455.56E0 , 552.22E0
       , 553.56E0 , 555.74E0 , 652.59E0 , 656.20E0 , 14.13E0 , 20.41E0 , 31.30E0 , 33.84E0 , 39.70E0 , 48.83E0 , 
      54.50E0 , 60.41E0 , 72.77E0 , 75.25E0 , 86.84E0 , 94.88E0 , 96.40E0 , 117.37E0 , 139.08E0 , 147.73E0 , 158.63
      E0 , 161.84E0 , 192.11E0 , 206.76E0 , 209.07E0 , 213.32E0 , 226.44E0 , 237.12E0 , 330.90E0 , 358.72E0 , 370.
      77E0 , 372.72E0 , 396.24E0 , 416.59E0 , 484.02E0 , 495.47E0 , 514.78E0 , 515.65E0 , 519.47E0 , 544.47E0 , 
      560.11E0 , 620.77E0 , 18.97E0 , 28.93E0 , 33.91E0 , 40.03E0 , 44.66E0 , 49.87E0 , 55.16E0 , 60.90E0 , 72.08E0 
      , 85.15E0 , 97.06E0 , 119.63E0 , 133.27E0 , 143.84E0 , 161.91E0 , 180.67E0 , 198.44E0 , 226.86E0 , 229.65E0 
      , 258.27E0 , 273.77E0 , 339.15E0 , 350.13E0 , 362.75E0 , 371.03E0 , 393.32E0 , 448.53E0 , 473.78E0 , 511.12
      E0 , 524.70E0 , 548.75E0 , 551.64E0 , 574.02E0 , 623.86E0 , 21.46E0 , 24.33E0 , 33.43E0 , 39.22E0 , 44.18E0 ,
       55.02E0 , 94.33E0 , 96.44E0 , 118.82E0 , 128.48E0 ,
00838 141.94E0 , 156.92E0 , 171.65E0 , 190.00E0 , 223.26E0 , 223.88E0 , 231.50E0 , 265.05E0 , 269.44E0 , 271.78E0
       , 273.46E0 , 334.61E0 , 339.79E0 , 349.52E0 , 358.18E0 , 377.98E0 , 394.77E0 , 429.66E0 , 468.22E0 , 487.27
      E0 , 519.54E0 , 523.03E0 , 612.99E0 , 638.59E0 , 641.36E0 , 622.05E0 , 631.50E0 , 663.97E0 , 646.9E0  , 748.
      29E0 , 749.21E0 , 750.14E0 , 647.04E0 , 646.89E0 , 746.9E0  , 748.43E0 , 747.35E0 , 749.27E0 , 647.61E0 , 
      747.78E0 , 750.51E0 , 851.37E0 , 845.97E0 , 847.54E0 , 849.93E0 , 851.61E0 , 849.75E0 , 850.98E0 , 848.23E0\};
00839 
00840 \textcolor{comment}{// http://www.itl.nist.gov/div898/strd/nls/data/hahn1.shtml}
00841 \textcolor{keywordtype}{void} testNistHahn1(\textcolor{keywordtype}{void})
00842 \{
00843   \textcolor{keyword}{const} \textcolor{keywordtype}{int}  n=7;
00844   \textcolor{keywordtype}{int} info;
00845 
00846   VectorXd x(n);
00847 
00848   \textcolor{comment}{/*}
00849 \textcolor{comment}{   * First try}
00850 \textcolor{comment}{   */}
00851   x<< 10., -1., .05, -.00001, -.05, .001, -.000001;
00852   \textcolor{comment}{// do the computation}
00853   \hyperlink{structhahn1__functor}{hahn1\_functor} functor;
00854   LevenbergMarquardt<hahn1\_functor> lm(functor);
00855   info = lm.minimize(x);
00856 
00857   \textcolor{comment}{// check return value}
00858   VERIFY\_IS\_EQUAL(info, 1);
00859   VERIFY\_IS\_EQUAL(lm.nfev, 11);
00860   VERIFY\_IS\_EQUAL(lm.njev, 10);
00861   \textcolor{comment}{// check norm^2}
00862   VERIFY\_IS\_APPROX(lm.fvec.squaredNorm(), 1.5324382854E+00);
00863   \textcolor{comment}{// check x}
00864   VERIFY\_IS\_APPROX(x[0], 1.0776351733E+00);
00865   VERIFY\_IS\_APPROX(x[1],-1.2269296921E-01);
00866   VERIFY\_IS\_APPROX(x[2], 4.0863750610E-03);
00867   VERIFY\_IS\_APPROX(x[3],-1.426264e-06); \textcolor{comment}{// shoulde be : -1.4262662514E-06}
00868   VERIFY\_IS\_APPROX(x[4],-5.7609940901E-03);
00869   VERIFY\_IS\_APPROX(x[5], 2.4053735503E-04);
00870   VERIFY\_IS\_APPROX(x[6],-1.2314450199E-07);
00871 
00872   \textcolor{comment}{/*}
00873 \textcolor{comment}{   * Second try}
00874 \textcolor{comment}{   */}
00875   x<< .1, -.1, .005, -.000001, -.005, .0001, -.0000001;
00876   \textcolor{comment}{// do the computation}
00877   info = lm.minimize(x);
00878 
00879   \textcolor{comment}{// check return value}
00880   VERIFY\_IS\_EQUAL(info, 1);
00881   VERIFY\_IS\_EQUAL(lm.nfev, 11);
00882   VERIFY\_IS\_EQUAL(lm.njev, 10);
00883   \textcolor{comment}{// check norm^2}
00884   VERIFY\_IS\_APPROX(lm.fvec.squaredNorm(), 1.5324382854E+00);
00885   \textcolor{comment}{// check x}
00886   VERIFY\_IS\_APPROX(x[0], 1.077640); \textcolor{comment}{// should be :  1.0776351733E+00}
00887   VERIFY\_IS\_APPROX(x[1], -0.1226933); \textcolor{comment}{// should be : -1.2269296921E-01}
00888   VERIFY\_IS\_APPROX(x[2], 0.004086383); \textcolor{comment}{// should be : 4.0863750610E-03}
00889   VERIFY\_IS\_APPROX(x[3], -1.426277e-06); \textcolor{comment}{// shoulde be : -1.4262662514E-06}
00890   VERIFY\_IS\_APPROX(x[4],-5.7609940901E-03);
00891   VERIFY\_IS\_APPROX(x[5], 0.00024053772); \textcolor{comment}{// should be : 2.4053735503E-04}
00892   VERIFY\_IS\_APPROX(x[6], -1.231450e-07); \textcolor{comment}{// should be : -1.2314450199E-07}
00893 
00894 \}
00895 
00896 \textcolor{keyword}{struct }\hyperlink{structmisra1d__functor}{misra1d\_functor} : \hyperlink{struct_functor}{Functor}<double>
00897 \{
00898     \hyperlink{structmisra1d__functor}{misra1d\_functor}(\textcolor{keywordtype}{void}) : \hyperlink{struct_functor}{Functor<double>}(2,14) \{\}
00899     \textcolor{keyword}{static} \textcolor{keyword}{const} \textcolor{keywordtype}{double} x[14];
00900     \textcolor{keyword}{static} \textcolor{keyword}{const} \textcolor{keywordtype}{double} y[14];
00901     \textcolor{keywordtype}{int} operator()(\textcolor{keyword}{const} VectorXd &b, VectorXd &fvec)
00902     \{
00903         assert(b.size()==2);
00904         assert(fvec.size()==14);
00905         \textcolor{keywordflow}{for}(\textcolor{keywordtype}{int} i=0; i<14; i++) \{
00906             fvec[i] = b[0]*b[1]*x[i]/(1.+b[1]*x[i]) - y[i];
00907         \}
00908         \textcolor{keywordflow}{return} 0;
00909     \}
00910     \textcolor{keywordtype}{int} df(\textcolor{keyword}{const} VectorXd &b, MatrixXd &fjac)
00911     \{
00912         assert(b.size()==2);
00913         assert(fjac.rows()==14);
00914         assert(fjac.cols()==2);
00915         \textcolor{keywordflow}{for}(\textcolor{keywordtype}{int} i=0; i<14; i++) \{
00916             \textcolor{keywordtype}{double} den = 1.+b[1]*x[i];
00917             fjac(i,0) = b[1]*x[i] / den;
00918             fjac(i,1) = b[0]*x[i]*(den-b[1]*x[i])/den/den;
00919         \}
00920         \textcolor{keywordflow}{return} 0;
00921     \}
00922 \};
00923 \textcolor{keyword}{const} \textcolor{keywordtype}{double} misra1d\_functor::x[14] = \{ 77.6E0, 114.9E0, 141.1E0, 190.8E0, 239.9E0, 289.0E0, 332.8E0, 378.4
      E0, 434.8E0, 477.3E0, 536.8E0, 593.1E0, 689.1E0, 760.0E0\};
00924 \textcolor{keyword}{const} \textcolor{keywordtype}{double} misra1d\_functor::y[14] = \{ 10.07E0, 14.73E0, 17.94E0, 23.93E0, 29.61E0, 35.18E0, 40.02E0, 44.8
      2E0, 50.76E0, 55.05E0, 61.01E0, 66.40E0, 75.47E0, 81.78E0\};
00925 
00926 \textcolor{comment}{// http://www.itl.nist.gov/div898/strd/nls/data/misra1d.shtml}
00927 \textcolor{keywordtype}{void} testNistMisra1d(\textcolor{keywordtype}{void})
00928 \{
00929   \textcolor{keyword}{const} \textcolor{keywordtype}{int} n=2;
00930   \textcolor{keywordtype}{int} info;
00931 
00932   VectorXd x(n);
00933 
00934   \textcolor{comment}{/*}
00935 \textcolor{comment}{   * First try}
00936 \textcolor{comment}{   */}
00937   x<< 500., 0.0001;
00938   \textcolor{comment}{// do the computation}
00939   \hyperlink{structmisra1d__functor}{misra1d\_functor} functor;
00940   LevenbergMarquardt<misra1d\_functor> lm(functor);
00941   info = lm.minimize(x);
00942 
00943   \textcolor{comment}{// check return value}
00944   VERIFY\_IS\_EQUAL(info, 3);
00945   VERIFY\_IS\_EQUAL(lm.nfev, 9);
00946   VERIFY\_IS\_EQUAL(lm.njev, 7);
00947   \textcolor{comment}{// check norm^2}
00948   VERIFY\_IS\_APPROX(lm.fvec.squaredNorm(), 5.6419295283E-02);
00949   \textcolor{comment}{// check x}
00950   VERIFY\_IS\_APPROX(x[0], 4.3736970754E+02);
00951   VERIFY\_IS\_APPROX(x[1], 3.0227324449E-04);
00952 
00953   \textcolor{comment}{/*}
00954 \textcolor{comment}{   * Second try}
00955 \textcolor{comment}{   */}
00956   x<< 450., 0.0003;
00957   \textcolor{comment}{// do the computation}
00958   info = lm.minimize(x);
00959 
00960   \textcolor{comment}{// check return value}
00961   VERIFY\_IS\_EQUAL(info, 1);
00962   VERIFY\_IS\_EQUAL(lm.nfev, 4);
00963   VERIFY\_IS\_EQUAL(lm.njev, 3);
00964   \textcolor{comment}{// check norm^2}
00965   VERIFY\_IS\_APPROX(lm.fvec.squaredNorm(), 5.6419295283E-02);
00966   \textcolor{comment}{// check x}
00967   VERIFY\_IS\_APPROX(x[0], 4.3736970754E+02);
00968   VERIFY\_IS\_APPROX(x[1], 3.0227324449E-04);
00969 \}
00970 
00971 
00972 \textcolor{keyword}{struct }\hyperlink{structlanczos1__functor}{lanczos1\_functor} : \hyperlink{struct_functor}{Functor}<double>
00973 \{
00974     \hyperlink{structlanczos1__functor}{lanczos1\_functor}(\textcolor{keywordtype}{void}) : \hyperlink{struct_functor}{Functor<double>}(6,24) \{\}
00975     \textcolor{keyword}{static} \textcolor{keyword}{const} \textcolor{keywordtype}{double} x[24];
00976     \textcolor{keyword}{static} \textcolor{keyword}{const} \textcolor{keywordtype}{double} y[24];
00977     \textcolor{keywordtype}{int} operator()(\textcolor{keyword}{const} VectorXd &b, VectorXd &fvec)
00978     \{
00979         assert(b.size()==6);
00980         assert(fvec.size()==24);
00981         \textcolor{keywordflow}{for}(\textcolor{keywordtype}{int} i=0; i<24; i++)
00982             fvec[i] = b[0]*exp(-b[1]*x[i]) + b[2]*exp(-b[3]*x[i]) + b[4]*exp(-b[5]*x[i])  - y[i];
00983         \textcolor{keywordflow}{return} 0;
00984     \}
00985     \textcolor{keywordtype}{int} df(\textcolor{keyword}{const} VectorXd &b, MatrixXd &fjac)
00986     \{
00987         assert(b.size()==6);
00988         assert(fjac.rows()==24);
00989         assert(fjac.cols()==6);
00990         \textcolor{keywordflow}{for}(\textcolor{keywordtype}{int} i=0; i<24; i++) \{
00991             fjac(i,0) = exp(-b[1]*x[i]);
00992             fjac(i,1) = -b[0]*x[i]*exp(-b[1]*x[i]);
00993             fjac(i,2) = exp(-b[3]*x[i]);
00994             fjac(i,3) = -b[2]*x[i]*exp(-b[3]*x[i]);
00995             fjac(i,4) = exp(-b[5]*x[i]);
00996             fjac(i,5) = -b[4]*x[i]*exp(-b[5]*x[i]);
00997         \}
00998         \textcolor{keywordflow}{return} 0;
00999     \}
01000 \};
01001 \textcolor{keyword}{const} \textcolor{keywordtype}{double} lanczos1\_functor::x[24] = \{ 0.000000000000E+00, 5.000000000000E-02, 1.000000000000E-01, 1.5000
      00000000E-01, 2.000000000000E-01, 2.500000000000E-01, 3.000000000000E-01, 3.500000000000E-01, 4.000000000000
      E-01, 4.500000000000E-01, 5.000000000000E-01, 5.500000000000E-01, 6.000000000000E-01, 6.500000000000E-01, 7.
      000000000000E-01, 7.500000000000E-01, 8.000000000000E-01, 8.500000000000E-01, 9.000000000000E-01, 9.50000000
      0000E-01, 1.000000000000E+00, 1.050000000000E+00, 1.100000000000E+00, 1.150000000000E+00 \};
01002 \textcolor{keyword}{const} \textcolor{keywordtype}{double} lanczos1\_functor::y[24] = \{ 2.513400000000E+00 ,2.044333373291E+00 ,1.668404436564E+00 ,1.3664
      18021208E+00 ,1.123232487372E+00 ,9.268897180037E-01 ,7.679338563728E-01 ,6.388775523106E-01 ,5.337835317402
      E-01 ,4.479363617347E-01 ,3.775847884350E-01 ,3.197393199326E-01 ,2.720130773746E-01 ,2.324965529032E-01 ,1.
      996589546065E-01 ,1.722704126914E-01 ,1.493405660168E-01 ,1.300700206922E-01 ,1.138119324644E-01 ,1.00041558
      7559E-01 ,8.833209084540E-02 ,7.833544019350E-02 ,6.976693743449E-02 ,6.239312536719E-02 \};
01003 
01004 \textcolor{comment}{// http://www.itl.nist.gov/div898/strd/nls/data/lanczos1.shtml}
01005 \textcolor{keywordtype}{void} testNistLanczos1(\textcolor{keywordtype}{void})
01006 \{
01007   \textcolor{keyword}{const} \textcolor{keywordtype}{int} n=6;
01008   \textcolor{keywordtype}{int} info;
01009 
01010   VectorXd x(n);
01011 
01012   \textcolor{comment}{/*}
01013 \textcolor{comment}{   * First try}
01014 \textcolor{comment}{   */}
01015   x<< 1.2, 0.3, 5.6, 5.5, 6.5, 7.6;
01016   \textcolor{comment}{// do the computation}
01017   \hyperlink{structlanczos1__functor}{lanczos1\_functor} functor;
01018   LevenbergMarquardt<lanczos1\_functor> lm(functor);
01019   info = lm.minimize(x);
01020 
01021   \textcolor{comment}{// check return value}
01022   VERIFY\_IS\_EQUAL(info, 2);
01023   VERIFY\_IS\_EQUAL(lm.nfev, 79);
01024   VERIFY\_IS\_EQUAL(lm.njev, 72);
01025   \textcolor{comment}{// check norm^2}
01026   std::cout.precision(30);
01027   std::cout << lm.fvec.squaredNorm() << \textcolor{stringliteral}{"\(\backslash\)n"};
01028   VERIFY(lm.fvec.squaredNorm() <= 1.4307867721E-25);
01029   \textcolor{comment}{// check x}
01030   VERIFY\_IS\_APPROX(x[0], 9.5100000027E-02);
01031   VERIFY\_IS\_APPROX(x[1], 1.0000000001E+00);
01032   VERIFY\_IS\_APPROX(x[2], 8.6070000013E-01);
01033   VERIFY\_IS\_APPROX(x[3], 3.0000000002E+00);
01034   VERIFY\_IS\_APPROX(x[4], 1.5575999998E+00);
01035   VERIFY\_IS\_APPROX(x[5], 5.0000000001E+00);
01036 
01037   \textcolor{comment}{/*}
01038 \textcolor{comment}{   * Second try}
01039 \textcolor{comment}{   */}
01040   x<< 0.5, 0.7, 3.6, 4.2, 4., 6.3;
01041   \textcolor{comment}{// do the computation}
01042   info = lm.minimize(x);
01043 
01044   \textcolor{comment}{// check return value}
01045   VERIFY\_IS\_EQUAL(info, 2);
01046   VERIFY\_IS\_EQUAL(lm.nfev, 9);
01047   VERIFY\_IS\_EQUAL(lm.njev, 8);
01048   \textcolor{comment}{// check norm^2}
01049   VERIFY(lm.fvec.squaredNorm() <= 1.4307867721E-25);
01050   \textcolor{comment}{// check x}
01051   VERIFY\_IS\_APPROX(x[0], 9.5100000027E-02);
01052   VERIFY\_IS\_APPROX(x[1], 1.0000000001E+00);
01053   VERIFY\_IS\_APPROX(x[2], 8.6070000013E-01);
01054   VERIFY\_IS\_APPROX(x[3], 3.0000000002E+00);
01055   VERIFY\_IS\_APPROX(x[4], 1.5575999998E+00);
01056   VERIFY\_IS\_APPROX(x[5], 5.0000000001E+00);
01057 
01058 \}
01059 
01060 \textcolor{keyword}{struct }\hyperlink{structrat42__functor}{rat42\_functor} : \hyperlink{struct_functor}{Functor}<double>
01061 \{
01062     \hyperlink{structrat42__functor}{rat42\_functor}(\textcolor{keywordtype}{void}) : \hyperlink{struct_functor}{Functor<double>}(3,9) \{\}
01063     \textcolor{keyword}{static} \textcolor{keyword}{const} \textcolor{keywordtype}{double} x[9];
01064     \textcolor{keyword}{static} \textcolor{keyword}{const} \textcolor{keywordtype}{double} y[9];
01065     \textcolor{keywordtype}{int} operator()(\textcolor{keyword}{const} VectorXd &b, VectorXd &fvec)
01066     \{
01067         assert(b.size()==3);
01068         assert(fvec.size()==9);
01069         \textcolor{keywordflow}{for}(\textcolor{keywordtype}{int} i=0; i<9; i++) \{
01070             fvec[i] = b[0] / (1.+exp(b[1]-b[2]*x[i])) - y[i];
01071         \}
01072         \textcolor{keywordflow}{return} 0;
01073     \}
01074 
01075     \textcolor{keywordtype}{int} df(\textcolor{keyword}{const} VectorXd &b, MatrixXd &fjac)
01076     \{
01077         assert(b.size()==3);
01078         assert(fjac.rows()==9);
01079         assert(fjac.cols()==3);
01080         \textcolor{keywordflow}{for}(\textcolor{keywordtype}{int} i=0; i<9; i++) \{
01081             \textcolor{keywordtype}{double} e = exp(b[1]-b[2]*x[i]);
01082             fjac(i,0) = 1./(1.+e);
01083             fjac(i,1) = -b[0]*e/(1.+e)/(1.+e);
01084             fjac(i,2) = +b[0]*e*x[i]/(1.+e)/(1.+e);
01085         \}
01086         \textcolor{keywordflow}{return} 0;
01087     \}
01088 \};
01089 \textcolor{keyword}{const} \textcolor{keywordtype}{double} rat42\_functor::x[9] = \{ 9.000E0, 14.000E0, 21.000E0, 28.000E0, 42.000E0, 57.000E0, 63.000E0, 
      70.000E0, 79.000E0 \};
01090 \textcolor{keyword}{const} \textcolor{keywordtype}{double} rat42\_functor::y[9] = \{ 8.930E0 ,10.800E0 ,18.590E0 ,22.330E0 ,39.350E0 ,56.110E0 ,61.730E0 ,
      64.620E0 ,67.080E0 \};
01091 
01092 \textcolor{comment}{// http://www.itl.nist.gov/div898/strd/nls/data/ratkowsky2.shtml}
01093 \textcolor{keywordtype}{void} testNistRat42(\textcolor{keywordtype}{void})
01094 \{
01095   \textcolor{keyword}{const} \textcolor{keywordtype}{int} n=3;
01096   \textcolor{keywordtype}{int} info;
01097 
01098   VectorXd x(n);
01099 
01100   \textcolor{comment}{/*}
01101 \textcolor{comment}{   * First try}
01102 \textcolor{comment}{   */}
01103   x<< 100., 1., 0.1;
01104   \textcolor{comment}{// do the computation}
01105   \hyperlink{structrat42__functor}{rat42\_functor} functor;
01106   LevenbergMarquardt<rat42\_functor> lm(functor);
01107   info = lm.minimize(x);
01108 
01109   \textcolor{comment}{// check return value}
01110   VERIFY\_IS\_EQUAL(info, 1);
01111   VERIFY\_IS\_EQUAL(lm.nfev, 10);
01112   VERIFY\_IS\_EQUAL(lm.njev, 8);
01113   \textcolor{comment}{// check norm^2}
01114   VERIFY\_IS\_APPROX(lm.fvec.squaredNorm(), 8.0565229338E+00);
01115   \textcolor{comment}{// check x}
01116   VERIFY\_IS\_APPROX(x[0], 7.2462237576E+01);
01117   VERIFY\_IS\_APPROX(x[1], 2.6180768402E+00);
01118   VERIFY\_IS\_APPROX(x[2], 6.7359200066E-02);
01119 
01120   \textcolor{comment}{/*}
01121 \textcolor{comment}{   * Second try}
01122 \textcolor{comment}{   */}
01123   x<< 75., 2.5, 0.07;
01124   \textcolor{comment}{// do the computation}
01125   info = lm.minimize(x);
01126 
01127   \textcolor{comment}{// check return value}
01128   VERIFY\_IS\_EQUAL(info, 1);
01129   VERIFY\_IS\_EQUAL(lm.nfev, 6);
01130   VERIFY\_IS\_EQUAL(lm.njev, 5);
01131   \textcolor{comment}{// check norm^2}
01132   VERIFY\_IS\_APPROX(lm.fvec.squaredNorm(), 8.0565229338E+00);
01133   \textcolor{comment}{// check x}
01134   VERIFY\_IS\_APPROX(x[0], 7.2462237576E+01);
01135   VERIFY\_IS\_APPROX(x[1], 2.6180768402E+00);
01136   VERIFY\_IS\_APPROX(x[2], 6.7359200066E-02);
01137 \}
01138 
01139 \textcolor{keyword}{struct }\hyperlink{struct_m_g_h10__functor}{MGH10\_functor} : \hyperlink{struct_functor}{Functor}<double>
01140 \{
01141     \hyperlink{struct_m_g_h10__functor}{MGH10\_functor}(\textcolor{keywordtype}{void}) : \hyperlink{struct_functor}{Functor<double>}(3,16) \{\}
01142     \textcolor{keyword}{static} \textcolor{keyword}{const} \textcolor{keywordtype}{double} x[16];
01143     \textcolor{keyword}{static} \textcolor{keyword}{const} \textcolor{keywordtype}{double} y[16];
01144     \textcolor{keywordtype}{int} operator()(\textcolor{keyword}{const} VectorXd &b, VectorXd &fvec)
01145     \{
01146         assert(b.size()==3);
01147         assert(fvec.size()==16);
01148         \textcolor{keywordflow}{for}(\textcolor{keywordtype}{int} i=0; i<16; i++)
01149             fvec[i] =  b[0] * exp(b[1]/(x[i]+b[2])) - y[i];
01150         \textcolor{keywordflow}{return} 0;
01151     \}
01152     \textcolor{keywordtype}{int} df(\textcolor{keyword}{const} VectorXd &b, MatrixXd &fjac)
01153     \{
01154         assert(b.size()==3);
01155         assert(fjac.rows()==16);
01156         assert(fjac.cols()==3);
01157         \textcolor{keywordflow}{for}(\textcolor{keywordtype}{int} i=0; i<16; i++) \{
01158             \textcolor{keywordtype}{double} factor = 1./(x[i]+b[2]);
01159             \textcolor{keywordtype}{double} e = exp(b[1]*factor);
01160             fjac(i,0) = e;
01161             fjac(i,1) = b[0]*factor*e;
01162             fjac(i,2) = -b[1]*b[0]*factor*factor*e;
01163         \}
01164         \textcolor{keywordflow}{return} 0;
01165     \}
01166 \};
01167 \textcolor{keyword}{const} \textcolor{keywordtype}{double} MGH10\_functor::x[16] = \{ 5.000000E+01, 5.500000E+01, 6.000000E+01, 6.500000E+01, 7.000000E+01,
       7.500000E+01, 8.000000E+01, 8.500000E+01, 9.000000E+01, 9.500000E+01, 1.000000E+02, 1.050000E+02, 1.100000E
      +02, 1.150000E+02, 1.200000E+02, 1.250000E+02 \};
01168 \textcolor{keyword}{const} \textcolor{keywordtype}{double} MGH10\_functor::y[16] = \{ 3.478000E+04, 2.861000E+04, 2.365000E+04, 1.963000E+04, 1.637000E+04,
       1.372000E+04, 1.154000E+04, 9.744000E+03, 8.261000E+03, 7.030000E+03, 6.005000E+03, 5.147000E+03, 4.427000E
      +03, 3.820000E+03, 3.307000E+03, 2.872000E+03 \};
01169 
01170 \textcolor{comment}{// http://www.itl.nist.gov/div898/strd/nls/data/mgh10.shtml}
01171 \textcolor{keywordtype}{void} testNistMGH10(\textcolor{keywordtype}{void})
01172 \{
01173   \textcolor{keyword}{const} \textcolor{keywordtype}{int} n=3;
01174   \textcolor{keywordtype}{int} info;
01175 
01176   VectorXd x(n);
01177 
01178   \textcolor{comment}{/*}
01179 \textcolor{comment}{   * First try}
01180 \textcolor{comment}{   */}
01181   x<< 2., 400000., 25000.;
01182   \textcolor{comment}{// do the computation}
01183   \hyperlink{struct_m_g_h10__functor}{MGH10\_functor} functor;
01184   LevenbergMarquardt<MGH10\_functor> lm(functor);
01185   info = lm.minimize(x);
01186 
01187   \textcolor{comment}{// check return value}
01188   VERIFY\_IS\_EQUAL(info, 2); 
01189   VERIFY\_IS\_EQUAL(lm.nfev, 284 ); 
01190   VERIFY\_IS\_EQUAL(lm.njev, 249 ); 
01191   \textcolor{comment}{// check norm^2}
01192   VERIFY\_IS\_APPROX(lm.fvec.squaredNorm(), 8.7945855171E+01);
01193   \textcolor{comment}{// check x}
01194   VERIFY\_IS\_APPROX(x[0], 5.6096364710E-03);
01195   VERIFY\_IS\_APPROX(x[1], 6.1813463463E+03);
01196   VERIFY\_IS\_APPROX(x[2], 3.4522363462E+02);
01197 
01198   \textcolor{comment}{/*}
01199 \textcolor{comment}{   * Second try}
01200 \textcolor{comment}{   */}
01201   x<< 0.02, 4000., 250.;
01202   \textcolor{comment}{// do the computation}
01203   info = lm.minimize(x);
01204 
01205   \textcolor{comment}{// check return value}
01206   VERIFY\_IS\_EQUAL(info, 3);
01207   VERIFY\_IS\_EQUAL(lm.nfev, 126);
01208   VERIFY\_IS\_EQUAL(lm.njev, 116);
01209   \textcolor{comment}{// check norm^2}
01210   VERIFY\_IS\_APPROX(lm.fvec.squaredNorm(), 8.7945855171E+01);
01211   \textcolor{comment}{// check x}
01212   VERIFY\_IS\_APPROX(x[0], 5.6096364710E-03);
01213   VERIFY\_IS\_APPROX(x[1], 6.1813463463E+03);
01214   VERIFY\_IS\_APPROX(x[2], 3.4522363462E+02);
01215 \}
01216 
01217 
01218 \textcolor{keyword}{struct }\hyperlink{struct_box_b_o_d__functor}{BoxBOD\_functor} : \hyperlink{struct_functor}{Functor}<double>
01219 \{
01220     \hyperlink{struct_box_b_o_d__functor}{BoxBOD\_functor}(\textcolor{keywordtype}{void}) : \hyperlink{struct_functor}{Functor<double>}(2,6) \{\}
01221     \textcolor{keyword}{static} \textcolor{keyword}{const} \textcolor{keywordtype}{double} x[6];
01222     \textcolor{keywordtype}{int} operator()(\textcolor{keyword}{const} VectorXd &b, VectorXd &fvec)
01223     \{
01224         \textcolor{keyword}{static} \textcolor{keyword}{const} \textcolor{keywordtype}{double} y[6] = \{ 109., 149., 149., 191., 213., 224. \};
01225         assert(b.size()==2);
01226         assert(fvec.size()==6);
01227         \textcolor{keywordflow}{for}(\textcolor{keywordtype}{int} i=0; i<6; i++)
01228             fvec[i] =  b[0]*(1.-exp(-b[1]*x[i])) - y[i];
01229         \textcolor{keywordflow}{return} 0;
01230     \}
01231     \textcolor{keywordtype}{int} df(\textcolor{keyword}{const} VectorXd &b, MatrixXd &fjac)
01232     \{
01233         assert(b.size()==2);
01234         assert(fjac.rows()==6);
01235         assert(fjac.cols()==2);
01236         \textcolor{keywordflow}{for}(\textcolor{keywordtype}{int} i=0; i<6; i++) \{
01237             \textcolor{keywordtype}{double} e = exp(-b[1]*x[i]);
01238             fjac(i,0) = 1.-e;
01239             fjac(i,1) = b[0]*x[i]*e;
01240         \}
01241         \textcolor{keywordflow}{return} 0;
01242     \}
01243 \};
01244 \textcolor{keyword}{const} \textcolor{keywordtype}{double} BoxBOD\_functor::x[6] = \{ 1., 2., 3., 5., 7., 10. \};
01245 
01246 \textcolor{comment}{// http://www.itl.nist.gov/div898/strd/nls/data/boxbod.shtml}
01247 \textcolor{keywordtype}{void} testNistBoxBOD(\textcolor{keywordtype}{void})
01248 \{
01249   \textcolor{keyword}{const} \textcolor{keywordtype}{int} n=2;
01250   \textcolor{keywordtype}{int} info;
01251 
01252   VectorXd x(n);
01253 
01254   \textcolor{comment}{/*}
01255 \textcolor{comment}{   * First try}
01256 \textcolor{comment}{   */}
01257   x<< 1., 1.;
01258   \textcolor{comment}{// do the computation}
01259   \hyperlink{struct_box_b_o_d__functor}{BoxBOD\_functor} functor;
01260   LevenbergMarquardt<BoxBOD\_functor> lm(functor);
01261   lm.parameters.ftol = 1.E6*NumTraits<double>::epsilon();
01262   lm.parameters.xtol = 1.E6*NumTraits<double>::epsilon();
01263   lm.parameters.factor = 10.;
01264   info = lm.minimize(x);
01265 
01266   \textcolor{comment}{// check return value}
01267   VERIFY\_IS\_EQUAL(info, 1);
01268   VERIFY(lm.nfev < 31); \textcolor{comment}{// 31}
01269   VERIFY(lm.njev < 25); \textcolor{comment}{// 25}
01270   \textcolor{comment}{// check norm^2}
01271   VERIFY\_IS\_APPROX(lm.fvec.squaredNorm(), 1.1680088766E+03);
01272   \textcolor{comment}{// check x}
01273   VERIFY\_IS\_APPROX(x[0], 2.1380940889E+02);
01274   VERIFY\_IS\_APPROX(x[1], 5.4723748542E-01);
01275 
01276   \textcolor{comment}{/*}
01277 \textcolor{comment}{   * Second try}
01278 \textcolor{comment}{   */}
01279   x<< 100., 0.75;
01280   \textcolor{comment}{// do the computation}
01281   lm.resetParameters();
01282   lm.parameters.ftol = NumTraits<double>::epsilon();
01283   lm.parameters.xtol = NumTraits<double>::epsilon();
01284   info = lm.minimize(x);
01285 
01286   \textcolor{comment}{// check return value}
01287   VERIFY\_IS\_EQUAL(info, 1); 
01288   VERIFY\_IS\_EQUAL(lm.nfev, 15 ); 
01289   VERIFY\_IS\_EQUAL(lm.njev, 14 ); 
01290   \textcolor{comment}{// check norm^2}
01291   VERIFY\_IS\_APPROX(lm.fvec.squaredNorm(), 1.1680088766E+03);
01292   \textcolor{comment}{// check x}
01293   VERIFY\_IS\_APPROX(x[0], 2.1380940889E+02);
01294   VERIFY\_IS\_APPROX(x[1], 5.4723748542E-01);
01295 \}
01296 
01297 \textcolor{keyword}{struct }\hyperlink{struct_m_g_h17__functor}{MGH17\_functor} : \hyperlink{struct_functor}{Functor}<double>
01298 \{
01299     \hyperlink{struct_m_g_h17__functor}{MGH17\_functor}(\textcolor{keywordtype}{void}) : \hyperlink{struct_functor}{Functor<double>}(5,33) \{\}
01300     \textcolor{keyword}{static} \textcolor{keyword}{const} \textcolor{keywordtype}{double} x[33];
01301     \textcolor{keyword}{static} \textcolor{keyword}{const} \textcolor{keywordtype}{double} y[33];
01302     \textcolor{keywordtype}{int} operator()(\textcolor{keyword}{const} VectorXd &b, VectorXd &fvec)
01303     \{
01304         assert(b.size()==5);
01305         assert(fvec.size()==33);
01306         \textcolor{keywordflow}{for}(\textcolor{keywordtype}{int} i=0; i<33; i++)
01307             fvec[i] =  b[0] + b[1]*exp(-b[3]*x[i]) +  b[2]*exp(-b[4]*x[i]) - y[i];
01308         \textcolor{keywordflow}{return} 0;
01309     \}
01310     \textcolor{keywordtype}{int} df(\textcolor{keyword}{const} VectorXd &b, MatrixXd &fjac)
01311     \{
01312         assert(b.size()==5);
01313         assert(fjac.rows()==33);
01314         assert(fjac.cols()==5);
01315         \textcolor{keywordflow}{for}(\textcolor{keywordtype}{int} i=0; i<33; i++) \{
01316             fjac(i,0) = 1.;
01317             fjac(i,1) = exp(-b[3]*x[i]);
01318             fjac(i,2) = exp(-b[4]*x[i]);
01319             fjac(i,3) = -x[i]*b[1]*exp(-b[3]*x[i]);
01320             fjac(i,4) = -x[i]*b[2]*exp(-b[4]*x[i]);
01321         \}
01322         \textcolor{keywordflow}{return} 0;
01323     \}
01324 \};
01325 \textcolor{keyword}{const} \textcolor{keywordtype}{double} MGH17\_functor::x[33] = \{ 0.000000E+00, 1.000000E+01, 2.000000E+01, 3.000000E+01, 4.000000E+01,
       5.000000E+01, 6.000000E+01, 7.000000E+01, 8.000000E+01, 9.000000E+01, 1.000000E+02, 1.100000E+02, 1.200000E
      +02, 1.300000E+02, 1.400000E+02, 1.500000E+02, 1.600000E+02, 1.700000E+02, 1.800000E+02, 1.900000E+02, 2.000
      000E+02, 2.100000E+02, 2.200000E+02, 2.300000E+02, 2.400000E+02, 2.500000E+02, 2.600000E+02, 2.700000E+02, 2
      .800000E+02, 2.900000E+02, 3.000000E+02, 3.100000E+02, 3.200000E+02 \};
01326 \textcolor{keyword}{const} \textcolor{keywordtype}{double} MGH17\_functor::y[33] = \{ 8.440000E-01, 9.080000E-01, 9.320000E-01, 9.360000E-01, 9.250000E-01,
       9.080000E-01, 8.810000E-01, 8.500000E-01, 8.180000E-01, 7.840000E-01, 7.510000E-01, 7.180000E-01, 6.850000E
      -01, 6.580000E-01, 6.280000E-01, 6.030000E-01, 5.800000E-01, 5.580000E-01, 5.380000E-01, 5.220000E-01, 5.060
      000E-01, 4.900000E-01, 4.780000E-01, 4.670000E-01, 4.570000E-01, 4.480000E-01, 4.380000E-01, 4.310000E-01, 4
      .240000E-01, 4.200000E-01, 4.140000E-01, 4.110000E-01, 4.060000E-01 \};
01327 
01328 \textcolor{comment}{// http://www.itl.nist.gov/div898/strd/nls/data/mgh17.shtml}
01329 \textcolor{keywordtype}{void} testNistMGH17(\textcolor{keywordtype}{void})
01330 \{
01331   \textcolor{keyword}{const} \textcolor{keywordtype}{int} n=5;
01332   \textcolor{keywordtype}{int} info;
01333 
01334   VectorXd x(n);
01335 
01336   \textcolor{comment}{/*}
01337 \textcolor{comment}{   * First try}
01338 \textcolor{comment}{   */}
01339   x<< 50., 150., -100., 1., 2.;
01340   \textcolor{comment}{// do the computation}
01341   \hyperlink{struct_m_g_h17__functor}{MGH17\_functor} functor;
01342   LevenbergMarquardt<MGH17\_functor> lm(functor);
01343   lm.parameters.ftol = NumTraits<double>::epsilon();
01344   lm.parameters.xtol = NumTraits<double>::epsilon();
01345   lm.parameters.maxfev = 1000;
01346   info = lm.minimize(x);
01347 
01348   \textcolor{comment}{// check norm^2}
01349   VERIFY\_IS\_APPROX(lm.fvec.squaredNorm(), 5.4648946975E-05);
01350   \textcolor{comment}{// check x}
01351   VERIFY\_IS\_APPROX(x[0], 3.7541005211E-01);
01352   VERIFY\_IS\_APPROX(x[1], 1.9358469127E+00);
01353   VERIFY\_IS\_APPROX(x[2], -1.4646871366E+00);
01354   VERIFY\_IS\_APPROX(x[3], 1.2867534640E-02);
01355   VERIFY\_IS\_APPROX(x[4], 2.2122699662E-02);
01356   
01357   \textcolor{comment}{// check return value}
01358   VERIFY\_IS\_EQUAL(info, 2); 
01359   ++g\_test\_level;
01360   VERIFY\_IS\_EQUAL(lm.nfev, 602);  \textcolor{comment}{// 602}
01361   VERIFY\_IS\_EQUAL(lm.njev, 545);  \textcolor{comment}{// 545}
01362   --g\_test\_level;
01363   VERIFY(lm.nfev < 602 * LM\_EVAL\_COUNT\_TOL);
01364   VERIFY(lm.njev < 545 * LM\_EVAL\_COUNT\_TOL);
01365 
01366   \textcolor{comment}{/*}
01367 \textcolor{comment}{   * Second try}
01368 \textcolor{comment}{   */}
01369   x<< 0.5  ,1.5  ,-1   ,0.01 ,0.02;
01370   \textcolor{comment}{// do the computation}
01371   lm.resetParameters();
01372   info = lm.minimize(x);
01373 
01374   \textcolor{comment}{// check return value}
01375   VERIFY\_IS\_EQUAL(info, 1);
01376   VERIFY\_IS\_EQUAL(lm.nfev, 18);
01377   VERIFY\_IS\_EQUAL(lm.njev, 15);
01378   \textcolor{comment}{// check norm^2}
01379   VERIFY\_IS\_APPROX(lm.fvec.squaredNorm(), 5.4648946975E-05);
01380   \textcolor{comment}{// check x}
01381   VERIFY\_IS\_APPROX(x[0], 3.7541005211E-01);
01382   VERIFY\_IS\_APPROX(x[1], 1.9358469127E+00);
01383   VERIFY\_IS\_APPROX(x[2], -1.4646871366E+00);
01384   VERIFY\_IS\_APPROX(x[3], 1.2867534640E-02);
01385   VERIFY\_IS\_APPROX(x[4], 2.2122699662E-02);
01386 \}
01387 
01388 \textcolor{keyword}{struct }\hyperlink{struct_m_g_h09__functor}{MGH09\_functor} : \hyperlink{struct_functor}{Functor}<double>
01389 \{
01390     \hyperlink{struct_m_g_h09__functor}{MGH09\_functor}(\textcolor{keywordtype}{void}) : \hyperlink{struct_functor}{Functor<double>}(4,11) \{\}
01391     \textcolor{keyword}{static} \textcolor{keyword}{const} \textcolor{keywordtype}{double} \_x[11];
01392     \textcolor{keyword}{static} \textcolor{keyword}{const} \textcolor{keywordtype}{double} y[11];
01393     \textcolor{keywordtype}{int} operator()(\textcolor{keyword}{const} VectorXd &b, VectorXd &fvec)
01394     \{
01395         assert(b.size()==4);
01396         assert(fvec.size()==11);
01397         \textcolor{keywordflow}{for}(\textcolor{keywordtype}{int} i=0; i<11; i++) \{
01398             \textcolor{keywordtype}{double} x = \_x[i], xx=x*x;
01399             fvec[i] = b[0]*(xx+x*b[1])/(xx+x*b[2]+b[3]) - y[i];
01400         \}
01401         \textcolor{keywordflow}{return} 0;
01402     \}
01403     \textcolor{keywordtype}{int} df(\textcolor{keyword}{const} VectorXd &b, MatrixXd &fjac)
01404     \{
01405         assert(b.size()==4);
01406         assert(fjac.rows()==11);
01407         assert(fjac.cols()==4);
01408         \textcolor{keywordflow}{for}(\textcolor{keywordtype}{int} i=0; i<11; i++) \{
01409             \textcolor{keywordtype}{double} x = \_x[i], xx=x*x;
01410             \textcolor{keywordtype}{double} factor = 1./(xx+x*b[2]+b[3]);
01411             fjac(i,0) = (xx+x*b[1]) * factor;
01412             fjac(i,1) = b[0]*x* factor;
01413             fjac(i,2) = - b[0]*(xx+x*b[1]) * x * factor * factor;
01414             fjac(i,3) = - b[0]*(xx+x*b[1]) * factor * factor;
01415         \}
01416         \textcolor{keywordflow}{return} 0;
01417     \}
01418 \};
01419 \textcolor{keyword}{const} \textcolor{keywordtype}{double} MGH09\_functor::\_x[11] = \{ 4., 2., 1., 5.E-1 , 2.5E-01, 1.670000E-01, 1.250000E-01,  1.E-01, 8.
      330000E-02, 7.140000E-02, 6.250000E-02 \};
01420 \textcolor{keyword}{const} \textcolor{keywordtype}{double} MGH09\_functor::y[11] = \{ 1.957000E-01, 1.947000E-01, 1.735000E-01, 1.600000E-01, 8.440000E-02,
       6.270000E-02, 4.560000E-02, 3.420000E-02, 3.230000E-02, 2.350000E-02, 2.460000E-02 \};
01421 
01422 \textcolor{comment}{// http://www.itl.nist.gov/div898/strd/nls/data/mgh09.shtml}
01423 \textcolor{keywordtype}{void} testNistMGH09(\textcolor{keywordtype}{void})
01424 \{
01425   \textcolor{keyword}{const} \textcolor{keywordtype}{int} n=4;
01426   \textcolor{keywordtype}{int} info;
01427 
01428   VectorXd x(n);
01429 
01430   \textcolor{comment}{/*}
01431 \textcolor{comment}{   * First try}
01432 \textcolor{comment}{   */}
01433   x<< 25., 39, 41.5, 39.;
01434   \textcolor{comment}{// do the computation}
01435   \hyperlink{struct_m_g_h09__functor}{MGH09\_functor} functor;
01436   LevenbergMarquardt<MGH09\_functor> lm(functor);
01437   lm.parameters.maxfev = 1000;
01438   info = lm.minimize(x);
01439 
01440   \textcolor{comment}{// check return value}
01441   VERIFY\_IS\_EQUAL(info, 1); 
01442   VERIFY\_IS\_EQUAL(lm.nfev, 490 ); 
01443   VERIFY\_IS\_EQUAL(lm.njev, 376 ); 
01444   \textcolor{comment}{// check norm^2}
01445   VERIFY\_IS\_APPROX(lm.fvec.squaredNorm(), 3.0750560385E-04);
01446   \textcolor{comment}{// check x}
01447   VERIFY\_IS\_APPROX(x[0], 0.1928077089); \textcolor{comment}{// should be 1.9280693458E-01}
01448   VERIFY\_IS\_APPROX(x[1], 0.19126423573); \textcolor{comment}{// should be 1.9128232873E-01}
01449   VERIFY\_IS\_APPROX(x[2], 0.12305309914); \textcolor{comment}{// should be 1.2305650693E-01}
01450   VERIFY\_IS\_APPROX(x[3], 0.13605395375); \textcolor{comment}{// should be 1.3606233068E-01}
01451 
01452   \textcolor{comment}{/*}
01453 \textcolor{comment}{   * Second try}
01454 \textcolor{comment}{   */}
01455   x<< 0.25, 0.39, 0.415, 0.39;
01456   \textcolor{comment}{// do the computation}
01457   lm.resetParameters();
01458   info = lm.minimize(x);
01459 
01460   \textcolor{comment}{// check return value}
01461   VERIFY\_IS\_EQUAL(info, 1);
01462   VERIFY\_IS\_EQUAL(lm.nfev, 18);
01463   VERIFY\_IS\_EQUAL(lm.njev, 16);
01464   \textcolor{comment}{// check norm^2}
01465   VERIFY\_IS\_APPROX(lm.fvec.squaredNorm(), 3.0750560385E-04);
01466   \textcolor{comment}{// check x}
01467   VERIFY\_IS\_APPROX(x[0], 0.19280781); \textcolor{comment}{// should be 1.9280693458E-01}
01468   VERIFY\_IS\_APPROX(x[1], 0.19126265); \textcolor{comment}{// should be 1.9128232873E-01}
01469   VERIFY\_IS\_APPROX(x[2], 0.12305280); \textcolor{comment}{// should be 1.2305650693E-01}
01470   VERIFY\_IS\_APPROX(x[3], 0.13605322); \textcolor{comment}{// should be 1.3606233068E-01}
01471 \}
01472 
01473 
01474 
01475 \textcolor{keyword}{struct }\hyperlink{struct_bennett5__functor}{Bennett5\_functor} : \hyperlink{struct_functor}{Functor}<double>
01476 \{
01477     \hyperlink{struct_bennett5__functor}{Bennett5\_functor}(\textcolor{keywordtype}{void}) : \hyperlink{struct_functor}{Functor<double>}(3,154) \{\}
01478     \textcolor{keyword}{static} \textcolor{keyword}{const} \textcolor{keywordtype}{double} x[154];
01479     \textcolor{keyword}{static} \textcolor{keyword}{const} \textcolor{keywordtype}{double} y[154];
01480     \textcolor{keywordtype}{int} operator()(\textcolor{keyword}{const} VectorXd &b, VectorXd &fvec)
01481     \{
01482         assert(b.size()==3);
01483         assert(fvec.size()==154);
01484         \textcolor{keywordflow}{for}(\textcolor{keywordtype}{int} i=0; i<154; i++)
01485             fvec[i] = b[0]* pow(b[1]+x[i],-1./b[2]) - y[i];
01486         \textcolor{keywordflow}{return} 0;
01487     \}
01488     \textcolor{keywordtype}{int} df(\textcolor{keyword}{const} VectorXd &b, MatrixXd &fjac)
01489     \{
01490         assert(b.size()==3);
01491         assert(fjac.rows()==154);
01492         assert(fjac.cols()==3);
01493         \textcolor{keywordflow}{for}(\textcolor{keywordtype}{int} i=0; i<154; i++) \{
01494             \textcolor{keywordtype}{double} e = pow(b[1]+x[i],-1./b[2]);
01495             fjac(i,0) = e;
01496             fjac(i,1) = - b[0]*e/b[2]/(b[1]+x[i]);
01497             fjac(i,2) = b[0]*e*\hyperlink{structlog}{log}(b[1]+x[i])/b[2]/b[2];
01498         \}
01499         \textcolor{keywordflow}{return} 0;
01500     \}
01501 \};
01502 \textcolor{keyword}{const} \textcolor{keywordtype}{double} Bennett5\_functor::x[154] = \{ 7.447168E0, 8.102586E0, 8.452547E0, 8.711278E0, 8.916774E0, 9.087
      155E0, 9.232590E0, 9.359535E0, 9.472166E0, 9.573384E0, 9.665293E0, 9.749461E0, 9.827092E0, 9.899128E0, 9.966
      321E0, 10.029280E0, 10.088510E0, 10.144430E0, 10.197380E0, 10.247670E0, 10.295560E0, 10.341250E0, 10.384950
      E0, 10.426820E0, 10.467000E0, 10.505640E0, 10.542830E0, 10.578690E0, 10.613310E0, 10.646780E0, 10.679150E0, 
      10.710520E0, 10.740920E0, 10.770440E0, 10.799100E0, 10.826970E0, 10.854080E0, 10.880470E0, 10.906190E0, 10.93
      1260E0, 10.955720E0, 10.979590E0, 11.002910E0, 11.025700E0, 11.047980E0, 11.069770E0, 11.091100E0, 11.111980
      E0, 11.132440E0, 11.152480E0, 11.172130E0, 11.191410E0, 11.210310E0, 11.228870E0, 11.247090E0, 11.264980E0, 
      11.282560E0, 11.299840E0, 11.316820E0, 11.333520E0, 11.349940E0, 11.366100E0, 11.382000E0, 11.397660E0, 11.4
      13070E0, 11.428240E0, 11.443200E0, 11.457930E0, 11.472440E0, 11.486750E0, 11.500860E0, 11.514770E0, 11.52849
      0E0, 11.542020E0, 11.555380E0, 11.568550E0,
01503 11.581560E0, 11.594420E0, 11.607121E0, 11.619640E0, 11.632000E0, 11.644210E0, 11.656280E0, 11.668200E0, 11.
      679980E0, 11.691620E0, 11.703130E0, 11.714510E0, 11.725760E0, 11.736880E0, 11.747890E0, 11.758780E0, 11.7695
      50E0, 11.780200E0, 11.790730E0, 11.801160E0, 11.811480E0, 11.821700E0, 11.831810E0, 11.841820E0, 11.851730E0
      , 11.861550E0, 11.871270E0, 11.880890E0, 11.890420E0, 11.899870E0, 11.909220E0, 11.918490E0, 11.927680E0, 11
      .936780E0, 11.945790E0, 11.954730E0, 11.963590E0, 11.972370E0, 11.981070E0, 11.989700E0, 11.998260E0, 12.006
      740E0, 12.015150E0, 12.023490E0, 12.031760E0, 12.039970E0, 12.048100E0, 12.056170E0, 12.064180E0, 12.072120
      E0, 12.080010E0, 12.087820E0, 12.095580E0, 12.103280E0, 12.110920E0, 12.118500E0, 12.126030E0, 12.133500E0, 
      12.140910E0, 12.148270E0, 12.155570E0, 12.162830E0, 12.170030E0, 12.177170E0, 12.184270E0, 12.191320E0, 12.19
      8320E0, 12.205270E0, 12.212170E0, 12.219030E0, 12.225840E0, 12.232600E0, 12.239320E0, 12.245990E0, 12.252620
      E0, 12.259200E0, 12.265750E0, 12.272240E0 \};
01504 \textcolor{keyword}{const} \textcolor{keywordtype}{double} Bennett5\_functor::y[154] = \{ -34.834702E0 ,-34.393200E0 ,-34.152901E0 ,-33.979099E0 ,-33.84590
      1E0 ,-33.732899E0 ,-33.640301E0 ,-33.559200E0 ,-33.486801E0 ,-33.423100E0 ,-33.365101E0 ,-33.313000E0 ,-33.2
      60899E0 ,-33.217400E0 ,-33.176899E0 ,-33.139198E0 ,-33.101601E0 ,-33.066799E0 ,-33.035000E0 ,-33.003101E0 ,-
      32.971298E0 ,-32.942299E0 ,-32.916302E0 ,-32.890202E0 ,-32.864101E0 ,-32.841000E0 ,-32.817799E0 ,-32.797501
      E0 ,-32.774300E0 ,-32.757000E0 ,-32.733799E0 ,-32.716400E0 ,-32.699100E0 ,-32.678799E0 ,-32.661400E0 ,-32.644
      001E0 ,-32.626701E0 ,-32.612202E0 ,-32.597698E0 ,-32.583199E0 ,-32.568699E0 ,-32.554298E0 ,-32.539799E0 ,-32
      .525299E0 ,-32.510799E0 ,-32.499199E0 ,-32.487598E0 ,-32.473202E0 ,-32.461601E0 ,-32.435501E0 ,-32.435501E0 
      ,-32.426800E0 ,-32.412300E0 ,-32.400799E0 ,-32.392101E0 ,-32.380501E0 ,-32.366001E0 ,-32.357300E0 ,-32.34859
      8E0 ,-32.339901E0 ,-32.328400E0 ,-32.319698E0 ,-32.311001E0 ,-32.299400E0 ,-32.290699E0 ,-32.282001E0 ,-32.2
      73300E0 ,-32.264599E0 ,-32.256001E0 ,-32.247299E0
01505 ,-32.238602E0 ,-32.229900E0 ,-32.224098E0 ,-32.215401E0 ,-32.203800E0 ,-32.198002E0 ,-32.189400E0 ,-32.1836
      01E0 ,-32.174900E0 ,-32.169102E0 ,-32.163300E0 ,-32.154598E0 ,-32.145901E0 ,-32.140099E0 ,-32.131401E0 ,-32.
      125599E0 ,-32.119801E0 ,-32.111198E0 ,-32.105400E0 ,-32.096699E0 ,-32.090900E0 ,-32.088001E0 ,-32.079300E0 ,
      -32.073502E0 ,-32.067699E0 ,-32.061901E0 ,-32.056099E0 ,-32.050301E0 ,-32.044498E0 ,-32.038799E0 ,-32.033001
      E0 ,-32.027199E0 ,-32.024300E0 ,-32.018501E0 ,-32.012699E0 ,-32.004002E0 ,-32.001099E0 ,-31.995300E0 ,-31.98
      9500E0 ,-31.983700E0 ,-31.977900E0 ,-31.972099E0 ,-31.969299E0 ,-31.963501E0 ,-31.957701E0 ,-31.951900E0 ,-
      31.946100E0 ,-31.940300E0 ,-31.937401E0 ,-31.931601E0 ,-31.925800E0 ,-31.922899E0 ,-31.917101E0 ,-31.911301E0
       ,-31.908400E0 ,-31.902599E0 ,-31.896900E0 ,-31.893999E0 ,-31.888201E0 ,-31.885300E0 ,-31.882401E0 ,-31.8766
      00E0 ,-31.873699E0 ,-31.867901E0 ,-31.862101E0 ,-31.859200E0 ,-31.856300E0 ,-31.850500E0 ,-31.844700E0 ,-31.
      841801E0 ,-31.838900E0 ,-31.833099E0 ,-31.830200E0 ,
01506 -31.827299E0 ,-31.821600E0 ,-31.818701E0 ,-31.812901E0 ,-31.809999E0 ,-31.807100E0 ,-31.801300E0 ,-31.79840
      1E0 ,-31.795500E0 ,-31.789700E0 ,-31.786800E0 \};
01507 
01508 \textcolor{comment}{// http://www.itl.nist.gov/div898/strd/nls/data/bennett5.shtml}
01509 \textcolor{keywordtype}{void} testNistBennett5(\textcolor{keywordtype}{void})
01510 \{
01511   \textcolor{keyword}{const} \textcolor{keywordtype}{int}  n=3;
01512   \textcolor{keywordtype}{int} info;
01513 
01514   VectorXd x(n);
01515 
01516   \textcolor{comment}{/*}
01517 \textcolor{comment}{   * First try}
01518 \textcolor{comment}{   */}
01519   x<< -2000., 50., 0.8;
01520   \textcolor{comment}{// do the computation}
01521   \hyperlink{struct_bennett5__functor}{Bennett5\_functor} functor;
01522   LevenbergMarquardt<Bennett5\_functor> lm(functor);
01523   lm.parameters.maxfev = 1000;
01524   info = lm.minimize(x);
01525 
01526   \textcolor{comment}{// check return value}
01527   VERIFY\_IS\_EQUAL(info, 1);
01528   VERIFY\_IS\_EQUAL(lm.nfev, 758);
01529   VERIFY\_IS\_EQUAL(lm.njev, 744);
01530   \textcolor{comment}{// check norm^2}
01531   VERIFY\_IS\_APPROX(lm.fvec.squaredNorm(), 5.2404744073E-04);
01532   \textcolor{comment}{// check x}
01533   VERIFY\_IS\_APPROX(x[0], -2.5235058043E+03);
01534   VERIFY\_IS\_APPROX(x[1], 4.6736564644E+01);
01535   VERIFY\_IS\_APPROX(x[2], 9.3218483193E-01);
01536   \textcolor{comment}{/*}
01537 \textcolor{comment}{   * Second try}
01538 \textcolor{comment}{   */}
01539   x<< -1500., 45., 0.85;
01540   \textcolor{comment}{// do the computation}
01541   lm.resetParameters();
01542   info = lm.minimize(x);
01543 
01544   \textcolor{comment}{// check return value}
01545   VERIFY\_IS\_EQUAL(info, 1);
01546   VERIFY\_IS\_EQUAL(lm.nfev, 203);
01547   VERIFY\_IS\_EQUAL(lm.njev, 192);
01548   \textcolor{comment}{// check norm^2}
01549   VERIFY\_IS\_APPROX(lm.fvec.squaredNorm(), 5.2404744073E-04);
01550   \textcolor{comment}{// check x}
01551   VERIFY\_IS\_APPROX(x[0], -2523.3007865); \textcolor{comment}{// should be -2.5235058043E+03}
01552   VERIFY\_IS\_APPROX(x[1], 46.735705771); \textcolor{comment}{// should be 4.6736564644E+01);}
01553   VERIFY\_IS\_APPROX(x[2], 0.93219881891); \textcolor{comment}{// should be 9.3218483193E-01);}
01554 \}
01555 
01556 \textcolor{keyword}{struct }\hyperlink{structthurber__functor}{thurber\_functor} : \hyperlink{struct_functor}{Functor}<double>
01557 \{
01558     \hyperlink{structthurber__functor}{thurber\_functor}(\textcolor{keywordtype}{void}) : \hyperlink{struct_functor}{Functor<double>}(7,37) \{\}
01559     \textcolor{keyword}{static} \textcolor{keyword}{const} \textcolor{keywordtype}{double} \_x[37];
01560     \textcolor{keyword}{static} \textcolor{keyword}{const} \textcolor{keywordtype}{double} \_y[37];
01561     \textcolor{keywordtype}{int} operator()(\textcolor{keyword}{const} VectorXd &b, VectorXd &fvec)
01562     \{
01563         \textcolor{comment}{//        int called=0; printf("call hahn1\_functor with  iflag=%d, called=%d\(\backslash\)n", iflag, called); if
       (iflag==1) called++;}
01564         assert(b.size()==7);
01565         assert(fvec.size()==37);
01566         \textcolor{keywordflow}{for}(\textcolor{keywordtype}{int} i=0; i<37; i++) \{
01567             \textcolor{keywordtype}{double} x=\_x[i], xx=x*x, xxx=xx*x;
01568             fvec[i] = (b[0]+b[1]*x+b[2]*xx+b[3]*xxx) / (1.+b[4]*x+b[5]*xx+b[6]*xxx) - \_y[i];
01569         \}
01570         \textcolor{keywordflow}{return} 0;
01571     \}
01572     \textcolor{keywordtype}{int} df(\textcolor{keyword}{const} VectorXd &b, MatrixXd &fjac)
01573     \{
01574         assert(b.size()==7);
01575         assert(fjac.rows()==37);
01576         assert(fjac.cols()==7);
01577         \textcolor{keywordflow}{for}(\textcolor{keywordtype}{int} i=0; i<37; i++) \{
01578             \textcolor{keywordtype}{double} x=\_x[i], xx=x*x, xxx=xx*x;
01579             \textcolor{keywordtype}{double} fact = 1./(1.+b[4]*x+b[5]*xx+b[6]*xxx);
01580             fjac(i,0) = 1.*fact;
01581             fjac(i,1) = x*fact;
01582             fjac(i,2) = xx*fact;
01583             fjac(i,3) = xxx*fact;
01584             fact = - (b[0]+b[1]*x+b[2]*xx+b[3]*xxx) * fact * fact;
01585             fjac(i,4) = x*fact;
01586             fjac(i,5) = xx*fact;
01587             fjac(i,6) = xxx*fact;
01588         \}
01589         \textcolor{keywordflow}{return} 0;
01590     \}
01591 \};
01592 \textcolor{keyword}{const} \textcolor{keywordtype}{double} thurber\_functor::\_x[37] = \{ -3.067E0, -2.981E0, -2.921E0, -2.912E0, -2.840E0, -2.797E0, -2.702
      E0, -2.699E0, -2.633E0, -2.481E0, -2.363E0, -2.322E0, -1.501E0, -1.460E0, -1.274E0, -1.212E0, -1.100E0, -1.0
      46E0, -0.915E0, -0.714E0, -0.566E0, -0.545E0, -0.400E0, -0.309E0, -0.109E0, -0.103E0, 0.010E0, 0.119E0, 0.37
      7E0, 0.790E0, 0.963E0, 1.006E0, 1.115E0, 1.572E0, 1.841E0, 2.047E0, 2.200E0 \};
01593 \textcolor{keyword}{const} \textcolor{keywordtype}{double} thurber\_functor::\_y[37] = \{ 80.574E0, 84.248E0, 87.264E0, 87.195E0, 89.076E0, 89.608E0, 89.868
      E0, 90.101E0, 92.405E0, 95.854E0, 100.696E0, 101.060E0, 401.672E0, 390.724E0, 567.534E0, 635.316E0, 733.054
      E0, 759.087E0, 894.206E0, 990.785E0, 1090.109E0, 1080.914E0, 1122.643E0, 1178.351E0, 1260.531E0, 1273.514E0, 
      1288.339E0, 1327.543E0, 1353.863E0, 1414.509E0, 1425.208E0, 1421.384E0, 1442.962E0, 1464.350E0, 1468.705E0, 
      1447.894E0, 1457.628E0\};
01594 
01595 \textcolor{comment}{// http://www.itl.nist.gov/div898/strd/nls/data/thurber.shtml}
01596 \textcolor{keywordtype}{void} testNistThurber(\textcolor{keywordtype}{void})
01597 \{
01598   \textcolor{keyword}{const} \textcolor{keywordtype}{int} n=7;
01599   \textcolor{keywordtype}{int} info;
01600 
01601   VectorXd x(n);
01602 
01603   \textcolor{comment}{/*}
01604 \textcolor{comment}{   * First try}
01605 \textcolor{comment}{   */}
01606   x<< 1000 ,1000 ,400 ,40 ,0.7,0.3,0.0 ;
01607   \textcolor{comment}{// do the computation}
01608   \hyperlink{structthurber__functor}{thurber\_functor} functor;
01609   LevenbergMarquardt<thurber\_functor> lm(functor);
01610   lm.parameters.ftol = 1.E4*NumTraits<double>::epsilon();
01611   lm.parameters.xtol = 1.E4*NumTraits<double>::epsilon();
01612   info = lm.minimize(x);
01613 
01614   \textcolor{comment}{// check return value}
01615   VERIFY\_IS\_EQUAL(info, 1);
01616   VERIFY\_IS\_EQUAL(lm.nfev, 39);
01617   VERIFY\_IS\_EQUAL(lm.njev, 36);
01618   \textcolor{comment}{// check norm^2}
01619   VERIFY\_IS\_APPROX(lm.fvec.squaredNorm(), 5.6427082397E+03);
01620   \textcolor{comment}{// check x}
01621   VERIFY\_IS\_APPROX(x[0], 1.2881396800E+03);
01622   VERIFY\_IS\_APPROX(x[1], 1.4910792535E+03);
01623   VERIFY\_IS\_APPROX(x[2], 5.8323836877E+02);
01624   VERIFY\_IS\_APPROX(x[3], 7.5416644291E+01);
01625   VERIFY\_IS\_APPROX(x[4], 9.6629502864E-01);
01626   VERIFY\_IS\_APPROX(x[5], 3.9797285797E-01);
01627   VERIFY\_IS\_APPROX(x[6], 4.9727297349E-02);
01628 
01629   \textcolor{comment}{/*}
01630 \textcolor{comment}{   * Second try}
01631 \textcolor{comment}{   */}
01632   x<< 1300 ,1500 ,500  ,75   ,1    ,0.4  ,0.05  ;
01633   \textcolor{comment}{// do the computation}
01634   lm.resetParameters();
01635   lm.parameters.ftol = 1.E4*NumTraits<double>::epsilon();
01636   lm.parameters.xtol = 1.E4*NumTraits<double>::epsilon();
01637   info = lm.minimize(x);
01638 
01639   \textcolor{comment}{// check return value}
01640   VERIFY\_IS\_EQUAL(info, 1);
01641   VERIFY\_IS\_EQUAL(lm.nfev, 29);
01642   VERIFY\_IS\_EQUAL(lm.njev, 28);
01643   \textcolor{comment}{// check norm^2}
01644   VERIFY\_IS\_APPROX(lm.fvec.squaredNorm(), 5.6427082397E+03);
01645   \textcolor{comment}{// check x}
01646   VERIFY\_IS\_APPROX(x[0], 1.2881396800E+03);
01647   VERIFY\_IS\_APPROX(x[1], 1.4910792535E+03);
01648   VERIFY\_IS\_APPROX(x[2], 5.8323836877E+02);
01649   VERIFY\_IS\_APPROX(x[3], 7.5416644291E+01);
01650   VERIFY\_IS\_APPROX(x[4], 9.6629502864E-01);
01651   VERIFY\_IS\_APPROX(x[5], 3.9797285797E-01);
01652   VERIFY\_IS\_APPROX(x[6], 4.9727297349E-02);
01653 \}
01654 
01655 \textcolor{keyword}{struct }\hyperlink{structrat43__functor}{rat43\_functor} : \hyperlink{struct_functor}{Functor}<double>
01656 \{
01657     \hyperlink{structrat43__functor}{rat43\_functor}(\textcolor{keywordtype}{void}) : \hyperlink{struct_functor}{Functor<double>}(4,15) \{\}
01658     \textcolor{keyword}{static} \textcolor{keyword}{const} \textcolor{keywordtype}{double} x[15];
01659     \textcolor{keyword}{static} \textcolor{keyword}{const} \textcolor{keywordtype}{double} y[15];
01660     \textcolor{keywordtype}{int} operator()(\textcolor{keyword}{const} VectorXd &b, VectorXd &fvec)
01661     \{
01662         assert(b.size()==4);
01663         assert(fvec.size()==15);
01664         \textcolor{keywordflow}{for}(\textcolor{keywordtype}{int} i=0; i<15; i++)
01665             fvec[i] = b[0] * pow(1.+exp(b[1]-b[2]*x[i]),-1./b[3]) - y[i];
01666         \textcolor{keywordflow}{return} 0;
01667     \}
01668     \textcolor{keywordtype}{int} df(\textcolor{keyword}{const} VectorXd &b, MatrixXd &fjac)
01669     \{
01670         assert(b.size()==4);
01671         assert(fjac.rows()==15);
01672         assert(fjac.cols()==4);
01673         \textcolor{keywordflow}{for}(\textcolor{keywordtype}{int} i=0; i<15; i++) \{
01674             \textcolor{keywordtype}{double} e = exp(b[1]-b[2]*x[i]);
01675             \textcolor{keywordtype}{double} power = -1./b[3];
01676             fjac(i,0) = pow(1.+e, power);
01677             fjac(i,1) = power*b[0]*e*pow(1.+e, power-1.);
01678             fjac(i,2) = -power*b[0]*e*x[i]*pow(1.+e, power-1.);
01679             fjac(i,3) = b[0]*power*power*\hyperlink{structlog}{log}(1.+e)*pow(1.+e, power);
01680         \}
01681         \textcolor{keywordflow}{return} 0;
01682     \}
01683 \};
01684 \textcolor{keyword}{const} \textcolor{keywordtype}{double} rat43\_functor::x[15] = \{ 1., 2., 3., 4., 5., 6., 7., 8., 9., 10., 11., 12., 13., 14., 15. \};
01685 \textcolor{keyword}{const} \textcolor{keywordtype}{double} rat43\_functor::y[15] = \{ 16.08, 33.83, 65.80, 97.20, 191.55, 326.20, 386.87, 520.53, 590.03, 
      651.92, 724.93, 699.56, 689.96, 637.56, 717.41 \};
01686 
01687 \textcolor{comment}{// http://www.itl.nist.gov/div898/strd/nls/data/ratkowsky3.shtml}
01688 \textcolor{keywordtype}{void} testNistRat43(\textcolor{keywordtype}{void})
01689 \{
01690   \textcolor{keyword}{const} \textcolor{keywordtype}{int} n=4;
01691   \textcolor{keywordtype}{int} info;
01692 
01693   VectorXd x(n);
01694 
01695   \textcolor{comment}{/*}
01696 \textcolor{comment}{   * First try}
01697 \textcolor{comment}{   */}
01698   x<< 100., 10., 1., 1.;
01699   \textcolor{comment}{// do the computation}
01700   \hyperlink{structrat43__functor}{rat43\_functor} functor;
01701   LevenbergMarquardt<rat43\_functor> lm(functor);
01702   lm.parameters.ftol = 1.E6*NumTraits<double>::epsilon();
01703   lm.parameters.xtol = 1.E6*NumTraits<double>::epsilon();
01704   info = lm.minimize(x);
01705 
01706   \textcolor{comment}{// check return value}
01707   VERIFY\_IS\_EQUAL(info, 1);
01708   VERIFY\_IS\_EQUAL(lm.nfev, 27);
01709   VERIFY\_IS\_EQUAL(lm.njev, 20);
01710   \textcolor{comment}{// check norm^2}
01711   VERIFY\_IS\_APPROX(lm.fvec.squaredNorm(), 8.7864049080E+03);
01712   \textcolor{comment}{// check x}
01713   VERIFY\_IS\_APPROX(x[0], 6.9964151270E+02);
01714   VERIFY\_IS\_APPROX(x[1], 5.2771253025E+00);
01715   VERIFY\_IS\_APPROX(x[2], 7.5962938329E-01);
01716   VERIFY\_IS\_APPROX(x[3], 1.2792483859E+00);
01717 
01718   \textcolor{comment}{/*}
01719 \textcolor{comment}{   * Second try}
01720 \textcolor{comment}{   */}
01721   x<< 700., 5., 0.75, 1.3;
01722   \textcolor{comment}{// do the computation}
01723   lm.resetParameters();
01724   lm.parameters.ftol = 1.E5*NumTraits<double>::epsilon();
01725   lm.parameters.xtol = 1.E5*NumTraits<double>::epsilon();
01726   info = lm.minimize(x);
01727 
01728   \textcolor{comment}{// check return value}
01729   VERIFY\_IS\_EQUAL(info, 1);
01730   VERIFY\_IS\_EQUAL(lm.nfev, 9);
01731   VERIFY\_IS\_EQUAL(lm.njev, 8);
01732   \textcolor{comment}{// check norm^2}
01733   VERIFY\_IS\_APPROX(lm.fvec.squaredNorm(), 8.7864049080E+03);
01734   \textcolor{comment}{// check x}
01735   VERIFY\_IS\_APPROX(x[0], 6.9964151270E+02);
01736   VERIFY\_IS\_APPROX(x[1], 5.2771253025E+00);
01737   VERIFY\_IS\_APPROX(x[2], 7.5962938329E-01);
01738   VERIFY\_IS\_APPROX(x[3], 1.2792483859E+00);
01739 \}
01740 
01741 
01742 
01743 \textcolor{keyword}{struct }\hyperlink{structeckerle4__functor}{eckerle4\_functor} : \hyperlink{struct_functor}{Functor}<double>
01744 \{
01745     \hyperlink{structeckerle4__functor}{eckerle4\_functor}(\textcolor{keywordtype}{void}) : \hyperlink{struct_functor}{Functor<double>}(3,35) \{\}
01746     \textcolor{keyword}{static} \textcolor{keyword}{const} \textcolor{keywordtype}{double} x[35];
01747     \textcolor{keyword}{static} \textcolor{keyword}{const} \textcolor{keywordtype}{double} y[35];
01748     \textcolor{keywordtype}{int} operator()(\textcolor{keyword}{const} VectorXd &b, VectorXd &fvec)
01749     \{
01750         assert(b.size()==3);
01751         assert(fvec.size()==35);
01752         \textcolor{keywordflow}{for}(\textcolor{keywordtype}{int} i=0; i<35; i++)
01753             fvec[i] = b[0]/b[1] * exp(-0.5*(x[i]-b[2])*(x[i]-b[2])/(b[1]*b[1])) - y[i];
01754         \textcolor{keywordflow}{return} 0;
01755     \}
01756     \textcolor{keywordtype}{int} df(\textcolor{keyword}{const} VectorXd &b, MatrixXd &fjac)
01757     \{
01758         assert(b.size()==3);
01759         assert(fjac.rows()==35);
01760         assert(fjac.cols()==3);
01761         \textcolor{keywordflow}{for}(\textcolor{keywordtype}{int} i=0; i<35; i++) \{
01762             \textcolor{keywordtype}{double} b12 = b[1]*b[1];
01763             \textcolor{keywordtype}{double} e = exp(-0.5*(x[i]-b[2])*(x[i]-b[2])/b12);
01764             fjac(i,0) = e / b[1];
01765             fjac(i,1) = ((x[i]-b[2])*(x[i]-b[2])/b12-1.) * b[0]*e/b12;
01766             fjac(i,2) = (x[i]-b[2])*e*b[0]/b[1]/b12;
01767         \}
01768         \textcolor{keywordflow}{return} 0;
01769     \}
01770 \};
01771 \textcolor{keyword}{const} \textcolor{keywordtype}{double} eckerle4\_functor::x[35] = \{ 400.0, 405.0, 410.0, 415.0, 420.0, 425.0, 430.0, 435.0, 436.5, 438
      .0, 439.5, 441.0, 442.5, 444.0, 445.5, 447.0, 448.5, 450.0, 451.5, 453.0, 454.5, 456.0, 457.5, 459.0, 460.5,
       462.0, 463.5, 465.0, 470.0, 475.0, 480.0, 485.0, 490.0, 495.0, 500.0\};
01772 \textcolor{keyword}{const} \textcolor{keywordtype}{double} eckerle4\_functor::y[35] = \{ 0.0001575, 0.0001699, 0.0002350, 0.0003102, 0.0004917, 0.0008710, 
      0.0017418, 0.0046400, 0.0065895, 0.0097302, 0.0149002, 0.0237310, 0.0401683, 0.0712559, 0.1264458, 0.2073413
      , 0.2902366, 0.3445623, 0.3698049, 0.3668534, 0.3106727, 0.2078154, 0.1164354, 0.0616764, 0.0337200, 0.01940
      23, 0.0117831, 0.0074357, 0.0022732, 0.0008800, 0.0004579, 0.0002345, 0.0001586, 0.0001143, 0.0000710 \};
01773 
01774 \textcolor{comment}{// http://www.itl.nist.gov/div898/strd/nls/data/eckerle4.shtml}
01775 \textcolor{keywordtype}{void} testNistEckerle4(\textcolor{keywordtype}{void})
01776 \{
01777   \textcolor{keyword}{const} \textcolor{keywordtype}{int} n=3;
01778   \textcolor{keywordtype}{int} info;
01779 
01780   VectorXd x(n);
01781 
01782   \textcolor{comment}{/*}
01783 \textcolor{comment}{   * First try}
01784 \textcolor{comment}{   */}
01785   x<< 1., 10., 500.;
01786   \textcolor{comment}{// do the computation}
01787   \hyperlink{structeckerle4__functor}{eckerle4\_functor} functor;
01788   LevenbergMarquardt<eckerle4\_functor> lm(functor);
01789   info = lm.minimize(x);
01790 
01791   \textcolor{comment}{// check return value}
01792   VERIFY\_IS\_EQUAL(info, 1);
01793   VERIFY\_IS\_EQUAL(lm.nfev, 18);
01794   VERIFY\_IS\_EQUAL(lm.njev, 15);
01795   \textcolor{comment}{// check norm^2}
01796   VERIFY\_IS\_APPROX(lm.fvec.squaredNorm(), 1.4635887487E-03);
01797   \textcolor{comment}{// check x}
01798   VERIFY\_IS\_APPROX(x[0], 1.5543827178);
01799   VERIFY\_IS\_APPROX(x[1], 4.0888321754);
01800   VERIFY\_IS\_APPROX(x[2], 4.5154121844E+02);
01801 
01802   \textcolor{comment}{/*}
01803 \textcolor{comment}{   * Second try}
01804 \textcolor{comment}{   */}
01805   x<< 1.5, 5., 450.;
01806   \textcolor{comment}{// do the computation}
01807   info = lm.minimize(x);
01808 
01809   \textcolor{comment}{// check return value}
01810   VERIFY\_IS\_EQUAL(info, 1);
01811   VERIFY\_IS\_EQUAL(lm.nfev, 7);
01812   VERIFY\_IS\_EQUAL(lm.njev, 6);
01813   \textcolor{comment}{// check norm^2}
01814   VERIFY\_IS\_APPROX(lm.fvec.squaredNorm(), 1.4635887487E-03);
01815   \textcolor{comment}{// check x}
01816   VERIFY\_IS\_APPROX(x[0], 1.5543827178);
01817   VERIFY\_IS\_APPROX(x[1], 4.0888321754);
01818   VERIFY\_IS\_APPROX(x[2], 4.5154121844E+02);
01819 \}
01820 
01821 \textcolor{keywordtype}{void} test\_NonLinearOptimization()
01822 \{
01823     \textcolor{comment}{// Tests using the examples provided by (c)minpack}
01824     CALL\_SUBTEST\textcolor{comment}{/*\_1*/}(testChkder());
01825     CALL\_SUBTEST\textcolor{comment}{/*\_1*/}(testLmder1());
01826     CALL\_SUBTEST\textcolor{comment}{/*\_1*/}(testLmder());
01827     CALL\_SUBTEST\textcolor{comment}{/*\_2*/}(testHybrj1());
01828     CALL\_SUBTEST\textcolor{comment}{/*\_2*/}(testHybrj());
01829     CALL\_SUBTEST\textcolor{comment}{/*\_2*/}(testHybrd1());
01830     CALL\_SUBTEST\textcolor{comment}{/*\_2*/}(testHybrd());
01831     CALL\_SUBTEST\textcolor{comment}{/*\_3*/}(testLmstr1());
01832     CALL\_SUBTEST\textcolor{comment}{/*\_3*/}(testLmstr());
01833     CALL\_SUBTEST\textcolor{comment}{/*\_3*/}(testLmdif1());
01834     CALL\_SUBTEST\textcolor{comment}{/*\_3*/}(testLmdif());
01835 
01836     \textcolor{comment}{// NIST tests, level of difficulty = "Lower"}
01837     CALL\_SUBTEST\textcolor{comment}{/*\_4*/}(testNistMisra1a());
01838     CALL\_SUBTEST\textcolor{comment}{/*\_4*/}(testNistChwirut2());
01839 
01840     \textcolor{comment}{// NIST tests, level of difficulty = "Average"}
01841     CALL\_SUBTEST\textcolor{comment}{/*\_5*/}(testNistHahn1());
01842     CALL\_SUBTEST\textcolor{comment}{/*\_6*/}(testNistMisra1d());
01843     CALL\_SUBTEST\textcolor{comment}{/*\_7*/}(testNistMGH17());
01844     CALL\_SUBTEST\textcolor{comment}{/*\_8*/}(testNistLanczos1());
01845 
01846 \textcolor{comment}{//     // NIST tests, level of difficulty = "Higher"}
01847     CALL\_SUBTEST\textcolor{comment}{/*\_9*/}(testNistRat42());
01848 \textcolor{comment}{//     CALL\_SUBTEST/*\_10*/(testNistMGH10());}
01849     CALL\_SUBTEST\textcolor{comment}{/*\_11*/}(testNistBoxBOD());
01850 \textcolor{comment}{//     CALL\_SUBTEST/*\_12*/(testNistMGH09());}
01851     CALL\_SUBTEST\textcolor{comment}{/*\_13*/}(testNistBennett5());
01852     CALL\_SUBTEST\textcolor{comment}{/*\_14*/}(testNistThurber());
01853     CALL\_SUBTEST\textcolor{comment}{/*\_15*/}(testNistRat43());
01854     CALL\_SUBTEST\textcolor{comment}{/*\_16*/}(testNistEckerle4());
01855 \}
01856 
01857 \textcolor{comment}{/*}
01858 \textcolor{comment}{ * Can be useful for debugging...}
01859 \textcolor{comment}{  printf("info, nfev : %d, %d\(\backslash\)n", info, lm.nfev);}
01860 \textcolor{comment}{  printf("info, nfev, njev : %d, %d, %d\(\backslash\)n", info, solver.nfev, solver.njev);}
01861 \textcolor{comment}{  printf("info, nfev : %d, %d\(\backslash\)n", info, solver.nfev);}
01862 \textcolor{comment}{  printf("x[0] : %.32g\(\backslash\)n", x[0]);}
01863 \textcolor{comment}{  printf("x[1] : %.32g\(\backslash\)n", x[1]);}
01864 \textcolor{comment}{  printf("x[2] : %.32g\(\backslash\)n", x[2]);}
01865 \textcolor{comment}{  printf("x[3] : %.32g\(\backslash\)n", x[3]);}
01866 \textcolor{comment}{  printf("fvec.blueNorm() : %.32g\(\backslash\)n", solver.fvec.blueNorm());}
01867 \textcolor{comment}{  printf("fvec.blueNorm() : %.32g\(\backslash\)n", lm.fvec.blueNorm());}
01868 \textcolor{comment}{}
01869 \textcolor{comment}{  printf("info, nfev, njev : %d, %d, %d\(\backslash\)n", info, lm.nfev, lm.njev);}
01870 \textcolor{comment}{  printf("fvec.squaredNorm() : %.13g\(\backslash\)n", lm.fvec.squaredNorm());}
01871 \textcolor{comment}{  std::cout << x << std::endl;}
01872 \textcolor{comment}{  std::cout.precision(9);}
01873 \textcolor{comment}{  std::cout << x[0] << std::endl;}
01874 \textcolor{comment}{  std::cout << x[1] << std::endl;}
01875 \textcolor{comment}{  std::cout << x[2] << std::endl;}
01876 \textcolor{comment}{  std::cout << x[3] << std::endl;}
01877 \textcolor{comment}{*/}
01878 
\end{DoxyCode}
