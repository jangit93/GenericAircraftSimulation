\hypertarget{_h_d_f5_21_810_81_2_h_d_f5_examples_2_c_2_h5_t_2h5ex__t__vlstring_8c_source}{}\section{H\+D\+F5/1.10.1/\+H\+D\+F5\+Examples/\+C/\+H5\+T/h5ex\+\_\+t\+\_\+vlstring.c}
\label{_h_d_f5_21_810_81_2_h_d_f5_examples_2_c_2_h5_t_2h5ex__t__vlstring_8c_source}\index{h5ex\+\_\+t\+\_\+vlstring.\+c@{h5ex\+\_\+t\+\_\+vlstring.\+c}}

\begin{DoxyCode}
00001 \textcolor{comment}{/************************************************************}
00002 \textcolor{comment}{}
00003 \textcolor{comment}{  This example shows how to read and write variable-length}
00004 \textcolor{comment}{  string datatypes to a dataset.  The program first writes}
00005 \textcolor{comment}{  variable-length strings to a dataset with a dataspace of}
00006 \textcolor{comment}{  DIM0, then closes the file.  Next, it reopens the file,}
00007 \textcolor{comment}{  reads back the data, and outputs it to the screen.}
00008 \textcolor{comment}{}
00009 \textcolor{comment}{  This file is intended for use with HDF5 Library version 1.8}
00010 \textcolor{comment}{}
00011 \textcolor{comment}{ ************************************************************/}
00012 
00013 \textcolor{preprocessor}{#include "hdf5.h"}
00014 \textcolor{preprocessor}{#include <stdio.h>}
00015 \textcolor{preprocessor}{#include <stdlib.h>}
00016 
00017 \textcolor{preprocessor}{#define FILE            "h5ex\_t\_vlstring.h5"}
00018 \textcolor{preprocessor}{#define DATASET         "DS1"}
00019 \textcolor{preprocessor}{#define DIM0            4}
00020 
00021 \textcolor{keywordtype}{int}
00022 main (\textcolor{keywordtype}{void})
00023 \{
00024     hid\_t       \hyperlink{structfile}{file}, filetype, memtype, space, dset;
00025                                             \textcolor{comment}{/* Handles */}
00026     herr\_t      status;
00027     hsize\_t     dims[1] = \{DIM0\};
00028     \textcolor{keywordtype}{char}        *wdata[DIM0] = \{\textcolor{stringliteral}{"Parting"}, \textcolor{stringliteral}{"is such"}, \textcolor{stringliteral}{"sweet"}, \textcolor{stringliteral}{"sorrow."}\},
00029                                             \textcolor{comment}{/* Write buffer */}
00030                 **rdata;                    \textcolor{comment}{/* Read buffer */}
00031     \textcolor{keywordtype}{int}         ndims,
00032                 i;
00033 
00034     \textcolor{comment}{/*}
00035 \textcolor{comment}{     * Create a new file using the default properties.}
00036 \textcolor{comment}{     */}
00037     file = H5Fcreate (FILE, H5F\_ACC\_TRUNC, H5P\_DEFAULT, H5P\_DEFAULT);
00038 
00039     \textcolor{comment}{/*}
00040 \textcolor{comment}{     * Create file and memory datatypes.  For this example we will save}
00041 \textcolor{comment}{     * the strings as FORTRAN strings.}
00042 \textcolor{comment}{     */}
00043     filetype = H5Tcopy (H5T\_FORTRAN\_S1);
00044     status = H5Tset\_size (filetype, H5T\_VARIABLE);
00045     memtype = H5Tcopy (H5T\_C\_S1);
00046     status = H5Tset\_size (memtype, H5T\_VARIABLE);
00047 
00048     \textcolor{comment}{/*}
00049 \textcolor{comment}{     * Create dataspace.  Setting maximum size to NULL sets the maximum}
00050 \textcolor{comment}{     * size to be the current size.}
00051 \textcolor{comment}{     */}
00052     space = H5Screate\_simple (1, dims, NULL);
00053 
00054     \textcolor{comment}{/*}
00055 \textcolor{comment}{     * Create the dataset and write the variable-length string data to}
00056 \textcolor{comment}{     * it.}
00057 \textcolor{comment}{     */}
00058     dset = H5Dcreate (file, DATASET, filetype, space, H5P\_DEFAULT, H5P\_DEFAULT,
00059                 H5P\_DEFAULT);
00060     status = H5Dwrite (dset, memtype, H5S\_ALL, H5S\_ALL, H5P\_DEFAULT, wdata);
00061 
00062     \textcolor{comment}{/*}
00063 \textcolor{comment}{     * Close and release resources.}
00064 \textcolor{comment}{     */}
00065     status = H5Dclose (dset);
00066     status = H5Sclose (space);
00067     status = H5Tclose (filetype);
00068     status = H5Tclose (memtype);
00069     status = H5Fclose (file);
00070 
00071 
00072     \textcolor{comment}{/*}
00073 \textcolor{comment}{     * Now we begin the read section of this example.  Here we assume}
00074 \textcolor{comment}{     * the dataset has the same name and rank, but can have any size.}
00075 \textcolor{comment}{     * Therefore we must allocate a new array to read in data using}
00076 \textcolor{comment}{     * malloc().}
00077 \textcolor{comment}{     */}
00078 
00079     \textcolor{comment}{/*}
00080 \textcolor{comment}{     * Open file and dataset.}
00081 \textcolor{comment}{     */}
00082     file = H5Fopen (FILE, H5F\_ACC\_RDONLY, H5P\_DEFAULT);
00083     dset = H5Dopen (file, DATASET, H5P\_DEFAULT);
00084 
00085     \textcolor{comment}{/*}
00086 \textcolor{comment}{     * Get the datatype.}
00087 \textcolor{comment}{     */}
00088     filetype = H5Dget\_type (dset);
00089 
00090     \textcolor{comment}{/*}
00091 \textcolor{comment}{     * Get dataspace and allocate memory for read buffer.}
00092 \textcolor{comment}{     */}
00093     space = H5Dget\_space (dset);
00094     ndims = H5Sget\_simple\_extent\_dims (space, dims, NULL);
00095     rdata = (\textcolor{keywordtype}{char} **) malloc (dims[0] * \textcolor{keyword}{sizeof} (\textcolor{keywordtype}{char} *));
00096 
00097     \textcolor{comment}{/*}
00098 \textcolor{comment}{     * Create the memory datatype.}
00099 \textcolor{comment}{     */}
00100     memtype = H5Tcopy (H5T\_C\_S1);
00101     status = H5Tset\_size (memtype, H5T\_VARIABLE);
00102 
00103     \textcolor{comment}{/*}
00104 \textcolor{comment}{     * Read the data.}
00105 \textcolor{comment}{     */}
00106     status = H5Dread (dset, memtype, H5S\_ALL, H5S\_ALL, H5P\_DEFAULT, rdata);
00107 
00108     \textcolor{comment}{/*}
00109 \textcolor{comment}{     * Output the data to the screen.}
00110 \textcolor{comment}{     */}
00111     \textcolor{keywordflow}{for} (i=0; i<dims[0]; i++)
00112         printf (\textcolor{stringliteral}{"%s[%d]: %s\(\backslash\)n"}, DATASET, i, rdata[i]);
00113 
00114     \textcolor{comment}{/*}
00115 \textcolor{comment}{     * Close and release resources.  Note that H5Dvlen\_reclaim works}
00116 \textcolor{comment}{     * for variable-length strings as well as variable-length arrays.}
00117 \textcolor{comment}{     * Also note that we must still free the array of pointers stored}
00118 \textcolor{comment}{     * in rdata, as H5Tvlen\_reclaim only frees the data these point to.}
00119 \textcolor{comment}{     */}
00120     status = H5Dvlen\_reclaim (memtype, space, H5P\_DEFAULT, rdata);
00121     free (rdata);
00122     status = H5Dclose (dset);
00123     status = H5Sclose (space);
00124     status = H5Tclose (filetype);
00125     status = H5Tclose (memtype);
00126     status = H5Fclose (file);
00127 
00128     \textcolor{keywordflow}{return} 0;
00129 \}
\end{DoxyCode}
