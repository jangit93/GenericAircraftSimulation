\hypertarget{_h_d_f5_21_810_81_2_h_d_f5_examples_2_c_2_h5_d_2h5ex__d__chunk_8c_source}{}\section{H\+D\+F5/1.10.1/\+H\+D\+F5\+Examples/\+C/\+H5\+D/h5ex\+\_\+d\+\_\+chunk.c}
\label{_h_d_f5_21_810_81_2_h_d_f5_examples_2_c_2_h5_d_2h5ex__d__chunk_8c_source}\index{h5ex\+\_\+d\+\_\+chunk.\+c@{h5ex\+\_\+d\+\_\+chunk.\+c}}

\begin{DoxyCode}
00001 \textcolor{comment}{/************************************************************}
00002 \textcolor{comment}{}
00003 \textcolor{comment}{  This example shows how to create a chunked dataset.  The}
00004 \textcolor{comment}{  program first writes integers in a hyperslab selection to}
00005 \textcolor{comment}{  a chunked dataset with dataspace dimensions of DIM0xDIM1}
00006 \textcolor{comment}{  and chunk size of CHUNK0xCHUNK1, then closes the file.}
00007 \textcolor{comment}{  Next, it reopens the file, reads back the data, and}
00008 \textcolor{comment}{  outputs it to the screen.  Finally it reads the data again}
00009 \textcolor{comment}{  using a different hyperslab selection, and outputs}
00010 \textcolor{comment}{  the result to the screen.}
00011 \textcolor{comment}{}
00012 \textcolor{comment}{  This file is intended for use with HDF5 Library version 1.8}
00013 \textcolor{comment}{}
00014 \textcolor{comment}{ ************************************************************/}
00015 
00016 \textcolor{preprocessor}{#include "hdf5.h"}
00017 \textcolor{preprocessor}{#include <stdio.h>}
00018 \textcolor{preprocessor}{#include <stdlib.h>}
00019 
00020 \textcolor{preprocessor}{#define FILE            "h5ex\_d\_chunk.h5"}
00021 \textcolor{preprocessor}{#define DATASET         "DS1"}
00022 \textcolor{preprocessor}{#define DIM0            6}
00023 \textcolor{preprocessor}{#define DIM1            8}
00024 \textcolor{preprocessor}{#define CHUNK0          4}
00025 \textcolor{preprocessor}{#define CHUNK1          4}
00026 
00027 \textcolor{keywordtype}{int}
00028 main (\textcolor{keywordtype}{void})
00029 \{
00030     hid\_t           \hyperlink{structfile}{file}, space, dset, dcpl;    \textcolor{comment}{/* Handles */}
00031     herr\_t          status;
00032     H5D\_layout\_t    layout;
00033     hsize\_t         dims[2] = \{DIM0, DIM1\},
00034                     chunk[2] = \{CHUNK0, CHUNK1\},
00035                     start[2],
00036                     stride[2],
00037                     count[2],
00038                     block[2];
00039     \textcolor{keywordtype}{int}             wdata[DIM0][DIM1],          \textcolor{comment}{/* Write buffer */}
00040                     rdata[DIM0][DIM1],          \textcolor{comment}{/* Read buffer */}
00041                     i, j;
00042 
00043     \textcolor{comment}{/*}
00044 \textcolor{comment}{     * Initialize data to "1", to make it easier to see the selections.}
00045 \textcolor{comment}{     */}
00046     \textcolor{keywordflow}{for} (i=0; i<DIM0; i++)
00047         \textcolor{keywordflow}{for} (j=0; j<DIM1; j++)
00048             wdata[i][j] = 1;
00049 
00050     \textcolor{comment}{/*}
00051 \textcolor{comment}{     * Print the data to the screen.}
00052 \textcolor{comment}{     */}
00053     printf (\textcolor{stringliteral}{"Original Data:\(\backslash\)n"});
00054     \textcolor{keywordflow}{for} (i=0; i<DIM0; i++) \{
00055         printf (\textcolor{stringliteral}{" ["});
00056         \textcolor{keywordflow}{for} (j=0; j<DIM1; j++)
00057             printf (\textcolor{stringliteral}{" %3d"}, wdata[i][j]);
00058         printf (\textcolor{stringliteral}{"]\(\backslash\)n"});
00059     \}
00060 
00061     \textcolor{comment}{/*}
00062 \textcolor{comment}{     * Create a new file using the default properties.}
00063 \textcolor{comment}{     */}
00064     file = H5Fcreate (FILE, H5F\_ACC\_TRUNC, H5P\_DEFAULT, H5P\_DEFAULT);
00065 
00066     \textcolor{comment}{/*}
00067 \textcolor{comment}{     * Create dataspace.  Setting maximum size to NULL sets the maximum}
00068 \textcolor{comment}{     * size to be the current size.}
00069 \textcolor{comment}{     */}
00070     space = H5Screate\_simple (2, dims, NULL);
00071 
00072     \textcolor{comment}{/*}
00073 \textcolor{comment}{     * Create the dataset creation property list, and set the chunk}
00074 \textcolor{comment}{     * size.}
00075 \textcolor{comment}{     */}
00076     dcpl = H5Pcreate (H5P\_DATASET\_CREATE);
00077     status = H5Pset\_chunk (dcpl, 2, chunk);
00078 
00079     \textcolor{comment}{/*}
00080 \textcolor{comment}{     * Create the chunked dataset.}
00081 \textcolor{comment}{     */}
00082     dset = H5Dcreate (file, DATASET, H5T\_STD\_I32LE, space, H5P\_DEFAULT, dcpl,
00083                 H5P\_DEFAULT);
00084 
00085     \textcolor{comment}{/*}
00086 \textcolor{comment}{     * Define and select the first part of the hyperslab selection.}
00087 \textcolor{comment}{     */}
00088     start[0] = 0;
00089     start[1] = 0;
00090     stride[0] = 3;
00091     stride[1] = 3;
00092     count[0] = 2;
00093     count[1] = 3;
00094     block[0] = 2;
00095     block[1] = 2;
00096     status = H5Sselect\_hyperslab (space, H5S\_SELECT\_SET, start, stride, count,
00097                 block);
00098 
00099     \textcolor{comment}{/*}
00100 \textcolor{comment}{     * Define and select the second part of the hyperslab selection,}
00101 \textcolor{comment}{     * which is subtracted from the first selection by the use of}
00102 \textcolor{comment}{     * H5S\_SELECT\_NOTB}
00103 \textcolor{comment}{     */}
00104     block[0] = 1;
00105     block[1] = 1;
00106     status = H5Sselect\_hyperslab (space, H5S\_SELECT\_NOTB, start, stride, count,
00107                 block);
00108 
00109     \textcolor{comment}{/*}
00110 \textcolor{comment}{     * Write the data to the dataset.}
00111 \textcolor{comment}{     */}
00112     status = H5Dwrite (dset, H5T\_NATIVE\_INT, H5S\_ALL, space, H5P\_DEFAULT,
00113                 wdata[0]);
00114 
00115     \textcolor{comment}{/*}
00116 \textcolor{comment}{     * Close and release resources.}
00117 \textcolor{comment}{     */}
00118     status = H5Pclose (dcpl);
00119     status = H5Dclose (dset);
00120     status = H5Sclose (space);
00121     status = H5Fclose (file);
00122 
00123 
00124     \textcolor{comment}{/*}
00125 \textcolor{comment}{     * Now we begin the read section of this example.}
00126 \textcolor{comment}{     */}
00127 
00128     \textcolor{comment}{/*}
00129 \textcolor{comment}{     * Open file and dataset using the default properties.}
00130 \textcolor{comment}{     */}
00131     file = H5Fopen (FILE, H5F\_ACC\_RDONLY, H5P\_DEFAULT);
00132     dset = H5Dopen (file, DATASET, H5P\_DEFAULT);
00133 
00134     \textcolor{comment}{/*}
00135 \textcolor{comment}{     * Retrieve the dataset creation property list, and print the}
00136 \textcolor{comment}{     * storage layout.}
00137 \textcolor{comment}{     */}
00138     dcpl = H5Dget\_create\_plist (dset);
00139     layout = H5Pget\_layout (dcpl);
00140     printf (\textcolor{stringliteral}{"\(\backslash\)nStorage layout for %s is: "}, DATASET);
00141     \textcolor{keywordflow}{switch} (layout) \{
00142         \textcolor{keywordflow}{case} H5D\_COMPACT:
00143             printf (\textcolor{stringliteral}{"H5D\_COMPACT\(\backslash\)n"});
00144             \textcolor{keywordflow}{break};
00145         \textcolor{keywordflow}{case} H5D\_CONTIGUOUS:
00146             printf (\textcolor{stringliteral}{"H5D\_CONTIGUOUS\(\backslash\)n"});
00147             \textcolor{keywordflow}{break};
00148         \textcolor{keywordflow}{case} H5D\_CHUNKED:
00149             printf (\textcolor{stringliteral}{"H5D\_CHUNKED\(\backslash\)n"});
00150             \textcolor{keywordflow}{break};
00151         \textcolor{keywordflow}{case} H5D\_VIRTUAL:
00152             printf (\textcolor{stringliteral}{"H5D\_VIRTUAL\(\backslash\)n"});
00153             \textcolor{keywordflow}{break};
00154         \textcolor{keywordflow}{case} H5D\_LAYOUT\_ERROR:
00155         \textcolor{keywordflow}{case} H5D\_NLAYOUTS:
00156             printf (\textcolor{stringliteral}{"H5D\_LAYOUT\_ERROR\(\backslash\)n"});
00157     \}
00158 
00159     \textcolor{comment}{/*}
00160 \textcolor{comment}{     * Read the data using the default properties.}
00161 \textcolor{comment}{     */}
00162     status = H5Dread (dset, H5T\_NATIVE\_INT, H5S\_ALL, H5S\_ALL, H5P\_DEFAULT,
00163                 rdata[0]);
00164 
00165     \textcolor{comment}{/*}
00166 \textcolor{comment}{     * Output the data to the screen.}
00167 \textcolor{comment}{     */}
00168     printf (\textcolor{stringliteral}{"\(\backslash\)nData as written to disk by hyberslabs:\(\backslash\)n"});
00169     \textcolor{keywordflow}{for} (i=0; i<DIM0; i++) \{
00170         printf (\textcolor{stringliteral}{" ["});
00171         \textcolor{keywordflow}{for} (j=0; j<DIM1; j++)
00172             printf (\textcolor{stringliteral}{" %3d"}, rdata[i][j]);
00173         printf (\textcolor{stringliteral}{"]\(\backslash\)n"});
00174     \}
00175 
00176     \textcolor{comment}{/*}
00177 \textcolor{comment}{     * Initialize the read array.}
00178 \textcolor{comment}{     */}
00179     \textcolor{keywordflow}{for} (i=0; i<DIM0; i++)
00180         \textcolor{keywordflow}{for} (j=0; j<DIM1; j++)
00181             rdata[i][j] = 0;
00182 
00183     \textcolor{comment}{/*}
00184 \textcolor{comment}{     * Define and select the hyperslab to use for reading.}
00185 \textcolor{comment}{     */}
00186     space = H5Dget\_space (dset);
00187     start[0] = 0;
00188     start[1] = 1;
00189     stride[0] = 4;
00190     stride[1] = 4;
00191     count[0] = 2;
00192     count[1] = 2;
00193     block[0] = 2;
00194     block[1] = 3;
00195     status = H5Sselect\_hyperslab (space, H5S\_SELECT\_SET, start, stride, count, block);
00196 
00197     \textcolor{comment}{/*}
00198 \textcolor{comment}{     * Read the data using the previously defined hyperslab.}
00199 \textcolor{comment}{     */}
00200     status = H5Dread (dset, H5T\_NATIVE\_INT, H5S\_ALL, space, H5P\_DEFAULT,
00201                 rdata[0]);
00202 
00203     \textcolor{comment}{/*}
00204 \textcolor{comment}{     * Output the data to the screen.}
00205 \textcolor{comment}{     */}
00206     printf (\textcolor{stringliteral}{"\(\backslash\)nData as read from disk by hyperslab:\(\backslash\)n"});
00207     \textcolor{keywordflow}{for} (i=0; i<DIM0; i++) \{
00208         printf (\textcolor{stringliteral}{" ["});
00209         \textcolor{keywordflow}{for} (j=0; j<DIM1; j++)
00210             printf (\textcolor{stringliteral}{" %3d"}, rdata[i][j]);
00211         printf (\textcolor{stringliteral}{"]\(\backslash\)n"});
00212     \}
00213 
00214     \textcolor{comment}{/*}
00215 \textcolor{comment}{     * Close and release resources.}
00216 \textcolor{comment}{     */}
00217     status = H5Pclose (dcpl);
00218     status = H5Dclose (dset);
00219     status = H5Sclose (space);
00220     status = H5Fclose (file);
00221 
00222     \textcolor{keywordflow}{return} 0;
00223 \}
\end{DoxyCode}
