\hypertarget{_h_d_f5_21_810_81_2_h_d_f5_examples_2_f_o_r_t_r_a_n_2_h5_t_2h5ex__t__cmpdatt___f03_8f90_source}{}\section{H\+D\+F5/1.10.1/\+H\+D\+F5\+Examples/\+F\+O\+R\+T\+R\+A\+N/\+H5\+T/h5ex\+\_\+t\+\_\+cmpdatt\+\_\+\+F03.f90}
\label{_h_d_f5_21_810_81_2_h_d_f5_examples_2_f_o_r_t_r_a_n_2_h5_t_2h5ex__t__cmpdatt___f03_8f90_source}\index{h5ex\+\_\+t\+\_\+cmpdatt\+\_\+\+F03.\+f90@{h5ex\+\_\+t\+\_\+cmpdatt\+\_\+\+F03.\+f90}}

\begin{DoxyCode}
00001 \textcolor{comment}{!************************************************************}
00002 \textcolor{comment}{!}
00003 \textcolor{comment}{!  This example shows how to read and write compound}
00004 \textcolor{comment}{!  datatypes to an attribute.  The program first writes}
00005 \textcolor{comment}{!  compound structures to an attribute with a dataspace of}
00006 \textcolor{comment}{!  DIM0, then closes the file.  Next, it reopens the file,}
00007 \textcolor{comment}{!  reads back the data, and outputs it to the screen.}
00008 \textcolor{comment}{!}
00009 \textcolor{comment}{!  This file is intended for use with HDF5 Library version 1.8}
00010 \textcolor{comment}{!  with --enable-fortran2003}
00011 \textcolor{comment}{!}
00012 \textcolor{comment}{!************************************************************}
00013 \textcolor{keyword}{PROGRAM} main
00014 
00015   \textcolor{keywordtype}{USE }iso\_c\_binding
00016   \textcolor{keywordtype}{USE }hdf5
00017 
00018   \textcolor{keywordtype}{IMPLICIT NONE}
00019 
00020   \textcolor{comment}{! This should map to REAL*8 on most modern processors}
00021   \textcolor{keywordtype}{INTEGER}, \textcolor{keywordtype}{PARAMETER} :: real\_kind\_15 = selected\_real\_kind(15,307)
00022 
00023   \textcolor{keywordtype}{CHARACTER(LEN=21)}, \textcolor{keywordtype}{PARAMETER} :: filename     = \textcolor{stringliteral}{"h5ex\_t\_cmpdatt\_F03.h5"}
00024   \textcolor{keywordtype}{CHARACTER(LEN=3)} , \textcolor{keywordtype}{PARAMETER} :: dataset      = \textcolor{stringliteral}{"DS1"}
00025   \textcolor{keywordtype}{CHARACTER(LEN=2)} , \textcolor{keywordtype}{PARAMETER} :: attribute    = \textcolor{stringliteral}{"A1"}
00026   \textcolor{keywordtype}{INTEGER}          , \textcolor{keywordtype}{PARAMETER} :: dim0         = 4
00027   \textcolor{keywordtype}{INTEGER}          , \textcolor{keywordtype}{PARAMETER} :: maxstringlen = 80
00028 
00029   \textcolor{keyword}{TYPE} \hyperlink{structsensor__t}{sensor\_t} \textcolor{comment}{! Compound data type}
00030      \textcolor{keywordtype}{INTEGER} :: serial\_no
00031      \textcolor{keywordtype}{CHARACTER(LEN=maxstringlen)} :: location
00032      \textcolor{keywordtype}{REAL(real\_kind\_15)} :: temperature
00033      \textcolor{keywordtype}{REAL(real\_kind\_15)} :: pressure
00034 \textcolor{keyword}{  END TYPE }\hyperlink{structsensor__t}{sensor\_t}
00035 
00036   \textcolor{keywordtype}{TYPE}(\hyperlink{structsensor__t}{sensor\_t}), \textcolor{keywordtype}{DIMENSION(1:dim0)}, \textcolor{keywordtype}{TARGET} ::  wdata \textcolor{comment}{! Write buffer}
00037   \textcolor{keywordtype}{TYPE}(\hyperlink{structsensor__t}{sensor\_t}), \textcolor{keywordtype}{DIMENSION(1:dim0)}, \textcolor{keywordtype}{TARGET} ::  rdata \textcolor{comment}{! Read buffer}
00038   \textcolor{keywordtype}{INTEGER(HID\_T)}  :: \hyperlink{structfile}{file}, filetype, memtype, space, dset, attr, strtype \textcolor{comment}{! Handles}
00039   \textcolor{keywordtype}{INTEGER} :: hdferr
00040   \textcolor{keywordtype}{INTEGER(HSIZE\_T)}, \textcolor{keywordtype}{DIMENSION(1:1)}   :: dims = (/dim0/), ndims
00041   \textcolor{keywordtype}{TYPE}(c\_ptr) :: f\_ptr
00042   \textcolor{keywordtype}{INTEGER} :: i
00043   \textcolor{comment}{!}
00044   \textcolor{comment}{! Initialize FORTRAN interface.}
00045   \textcolor{comment}{!}
00046   \textcolor{keyword}{CALL }h5open\_f(hdferr)
00047   \textcolor{comment}{!}
00048   \textcolor{comment}{! Initialize data.}
00049   \textcolor{comment}{!}
00050   wdata(1)%serial\_no = 1153
00051   wdata(1)%location = \textcolor{stringliteral}{"Exterior (static)"}
00052   wdata(1)%temperature = 53.23\_real\_kind\_15
00053   wdata(1)%pressure = 24.57\_real\_kind\_15
00054   wdata(2)%serial\_no = 1184
00055   wdata(2)%location = \textcolor{stringliteral}{"Intake"}
00056   wdata(2)%temperature = 55.12\_real\_kind\_15
00057   wdata(2)%pressure = 22.95\_real\_kind\_15
00058   wdata(3)%serial\_no = 1027
00059   wdata(3)%location = \textcolor{stringliteral}{"Intake manifold"}
00060   wdata(3)%temperature = 103.55\_real\_kind\_15
00061   wdata(3)%pressure = 31.23\_real\_kind\_15
00062   wdata(4)%serial\_no = 1313
00063   wdata(4)%location = \textcolor{stringliteral}{"Exhaust manifold"}
00064   wdata(4)%temperature = 1252.89\_real\_kind\_15
00065   wdata(4)%pressure = 84.11\_real\_kind\_15
00066   \textcolor{comment}{!}
00067   \textcolor{comment}{! Create a new file using the default properties.}
00068   \textcolor{comment}{!}
00069   \textcolor{keyword}{CALL }h5fcreate\_f(filename, h5f\_acc\_trunc\_f, \hyperlink{structfile}{file}, hdferr)
00070   \textcolor{comment}{!}
00071   \textcolor{comment}{! Create the compound datatype for memory.}
00072   \textcolor{comment}{!}
00073   \textcolor{keyword}{CALL }h5tcreate\_f(h5t\_compound\_f, h5offsetof(c\_loc(wdata(1)), c\_loc(wdata(2))), memtype, hdferr)
00074   
00075   \textcolor{keyword}{CALL }h5tinsert\_f(memtype, \textcolor{stringliteral}{"Serial number"}, &
00076        h5offsetof(c\_loc(wdata(1)),c\_loc(wdata(1)%serial\_no)), h5t\_native\_integer, hdferr)
00077   \textcolor{comment}{!}
00078   \textcolor{comment}{! Create datatype for the String attribute.}
00079   \textcolor{comment}{!}
00080   \textcolor{keyword}{CALL }h5tcopy\_f(h5t\_native\_character, strtype, hdferr)
00081   \textcolor{keyword}{CALL }h5tset\_size\_f(strtype, int(maxstringlen,size\_t), hdferr)  
00082 
00083   \textcolor{keyword}{CALL }h5tinsert\_f(memtype, \textcolor{stringliteral}{"Location"}, &
00084        h5offsetof(c\_loc(wdata(1)),c\_loc(wdata(1)%location)), strtype, hdferr)
00085 
00086   \textcolor{keyword}{CALL }h5tinsert\_f(memtype, \textcolor{stringliteral}{"Temperature (F)"}, &
00087        h5offsetof(c\_loc(wdata(1)),c\_loc(wdata(1)%temperature)), &
00088        h5kind\_to\_type(real\_kind\_15,h5\_real\_kind), hdferr)
00089 
00090   \textcolor{keyword}{CALL }h5tinsert\_f(memtype, \textcolor{stringliteral}{"Pressure (inHg)"}, &
00091        h5offsetof(c\_loc(wdata(1)),c\_loc(wdata(1)%pressure)), &
00092        h5kind\_to\_type(real\_kind\_15,h5\_real\_kind), hdferr)
00093   \textcolor{comment}{!}
00094   \textcolor{comment}{! Create the compound datatype for the file.  Because the standard}
00095   \textcolor{comment}{! types we are using for the file may have different sizes than}
00096   \textcolor{comment}{! the corresponding native types, we must manually calculate the}
00097   \textcolor{comment}{! offset of each member.}
00098   \textcolor{comment}{!}
00099   \textcolor{keyword}{CALL }h5tcreate\_f(h5t\_compound\_f, int(8 + maxstringlen + 8 + 8 , size\_t), filetype, hdferr)
00100   
00101   \textcolor{keyword}{CALL }h5tinsert\_f(filetype, \textcolor{stringliteral}{"Serial number"}, 0\_size\_t, h5t\_std\_i64be, hdferr)
00102 
00103   \textcolor{keyword}{CALL }h5tinsert\_f(filetype, \textcolor{stringliteral}{"Location"}, 8\_size\_t, strtype, hdferr)
00104 
00105   \textcolor{keyword}{CALL }h5tinsert\_f(filetype, \textcolor{stringliteral}{"Temperature (F)"}, int(8 + maxstringlen,size\_t), &
00106        h5t\_ieee\_f64be, hdferr)
00107 
00108   \textcolor{keyword}{CALL }h5tinsert\_f(filetype, \textcolor{stringliteral}{"Pressure (inHg)"}, int(8 + maxstringlen + 8, size\_t), &
00109        h5t\_ieee\_f64be, hdferr)
00110   \textcolor{comment}{!}
00111   \textcolor{comment}{! Create dataset with a null dataspace.}
00112   \textcolor{comment}{!}
00113   \textcolor{keyword}{CALL }h5screate\_f(h5s\_null\_f, space, hdferr)
00114 
00115   \textcolor{keyword}{CALL }h5dcreate\_f(\hyperlink{structfile}{file},dataset, h5t\_std\_i32le, space, dset, hdferr)
00116 
00117   \textcolor{keyword}{CALL }h5sclose\_f(space, hdferr)
00118   \textcolor{comment}{!}
00119   \textcolor{comment}{! Create dataspace.  Set the size to be the current size.}
00120   \textcolor{comment}{!}
00121   \textcolor{keyword}{CALL }h5screate\_simple\_f(1, dims, space, hdferr)
00122   \textcolor{comment}{!}
00123   \textcolor{comment}{! Create the attribute and write the compound data to it.}
00124   \textcolor{comment}{!}
00125   \textcolor{keyword}{CALL }h5acreate\_f(dset, attribute, filetype, space, attr, hdferr)
00126 
00127   f\_ptr = c\_loc(wdata(1))
00128   \textcolor{keyword}{CALL }h5awrite\_f(attr, memtype, f\_ptr, hdferr)
00129   \textcolor{comment}{!}
00130   \textcolor{comment}{! Close and release resources.}
00131   \textcolor{comment}{!}
00132   \textcolor{keyword}{CALL }h5aclose\_f(attr, hdferr)
00133   \textcolor{keyword}{CALL }h5dclose\_f(dset, hdferr)
00134   \textcolor{keyword}{CALL }h5sclose\_f(space, hdferr)
00135   \textcolor{keyword}{CALL }h5tclose\_f(filetype, hdferr)
00136   \textcolor{keyword}{CALL }h5fclose\_f(\hyperlink{structfile}{file}, hdferr)
00137   \textcolor{comment}{!}
00138   \textcolor{comment}{! Now we begin the read section of this example.}
00139   \textcolor{comment}{!}
00140   \textcolor{comment}{!}
00141   \textcolor{comment}{! Open file, dataset, and attribute.}
00142   \textcolor{comment}{!}
00143   \textcolor{keyword}{CALL }h5fopen\_f(filename, h5f\_acc\_rdonly\_f, \hyperlink{structfile}{file}, hdferr)
00144   \textcolor{keyword}{CALL }h5dopen\_f(\hyperlink{structfile}{file}, dataset, dset, hdferr)
00145   \textcolor{keyword}{CALL }h5aopen\_f(dset, attribute, attr, hdferr)
00146   \textcolor{comment}{!}
00147   \textcolor{comment}{! Get dataspace and allocate memory for read buffer.}
00148   \textcolor{comment}{!}
00149   \textcolor{keyword}{CALL }h5aget\_space\_f(attr,space, hdferr)
00150   \textcolor{keyword}{CALL }h5sget\_simple\_extent\_dims\_f(space, dims, ndims, hdferr)
00151   \textcolor{comment}{!}
00152   \textcolor{comment}{! Read the data.}
00153   \textcolor{comment}{!}
00154   f\_ptr = c\_loc(rdata(1))
00155   \textcolor{keyword}{CALL }h5aread\_f( attr, memtype, f\_ptr, hdferr)
00156   \textcolor{comment}{!}
00157   \textcolor{comment}{! Output the data to the screen.}
00158   \textcolor{comment}{!}
00159   \textcolor{keywordflow}{DO} i = 1, ndims(1)
00160      \textcolor{keyword}{WRITE}(*,\textcolor{stringliteral}{'(A,I1,":")'}) attribute, i
00161      \textcolor{keyword}{WRITE}(*,\textcolor{stringliteral}{'("Serial number   : ", I6)'}) rdata(i)%serial\_no
00162      \textcolor{keyword}{WRITE}(*,\textcolor{stringliteral}{'("Location        : ", A)'} ) trim(rdata(i)%location)
00163      \textcolor{keyword}{WRITE}(*,\textcolor{stringliteral}{'("Temperature (F) : ", f8.2)'}) rdata(i)%temperature
00164      \textcolor{keyword}{WRITE}(*,\textcolor{stringliteral}{'("Pressure (inHg) : ", f8.2)'}) rdata(i)%pressure
00165 \textcolor{keywordflow}{  END DO}
00166   \textcolor{comment}{!}
00167   \textcolor{comment}{! Close and release resources}
00168   \textcolor{comment}{!}
00169   \textcolor{keyword}{CALL }h5aclose\_f(attr, hdferr)
00170   \textcolor{keyword}{CALL }h5dclose\_f(dset, hdferr)
00171   \textcolor{keyword}{CALL }h5sclose\_f(space, hdferr)
00172   \textcolor{keyword}{CALL }h5tclose\_f(strtype, hdferr)
00173   \textcolor{keyword}{CALL }h5fclose\_f(\hyperlink{structfile}{file}, hdferr)
00174 
00175 \textcolor{keyword}{END PROGRAM }main
\end{DoxyCode}
