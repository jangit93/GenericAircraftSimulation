\hypertarget{visual__studio_2_h_d_f5_21_810_81_2_h_d_f5_examples_2_c_2_h5_t_2h5ex__t__string_8c_source}{}\section{visual\+\_\+studio/\+H\+D\+F5/1.10.1/\+H\+D\+F5\+Examples/\+C/\+H5\+T/h5ex\+\_\+t\+\_\+string.c}
\label{visual__studio_2_h_d_f5_21_810_81_2_h_d_f5_examples_2_c_2_h5_t_2h5ex__t__string_8c_source}\index{h5ex\+\_\+t\+\_\+string.\+c@{h5ex\+\_\+t\+\_\+string.\+c}}

\begin{DoxyCode}
00001 \textcolor{comment}{/************************************************************}
00002 \textcolor{comment}{}
00003 \textcolor{comment}{  This example shows how to read and write string datatypes}
00004 \textcolor{comment}{  to a dataset.  The program first writes strings to a}
00005 \textcolor{comment}{  dataset with a dataspace of DIM0, then closes the file.}
00006 \textcolor{comment}{  Next, it reopens the file, reads back the data, and}
00007 \textcolor{comment}{  outputs it to the screen.}
00008 \textcolor{comment}{}
00009 \textcolor{comment}{  This file is intended for use with HDF5 Library version 1.8}
00010 \textcolor{comment}{}
00011 \textcolor{comment}{ ************************************************************/}
00012 
00013 \textcolor{preprocessor}{#include "hdf5.h"}
00014 \textcolor{preprocessor}{#include <stdio.h>}
00015 \textcolor{preprocessor}{#include <stdlib.h>}
00016 
00017 \textcolor{preprocessor}{#define FILE            "h5ex\_t\_string.h5"}
00018 \textcolor{preprocessor}{#define DATASET         "DS1"}
00019 \textcolor{preprocessor}{#define DIM0            4}
00020 \textcolor{preprocessor}{#define SDIM            8}
00021 
00022 \textcolor{keywordtype}{int}
00023 main (\textcolor{keywordtype}{void})
00024 \{
00025     hid\_t       \hyperlink{structfile}{file}, filetype, memtype, space, dset;
00026                                             \textcolor{comment}{/* Handles */}
00027     herr\_t      status;
00028     hsize\_t     dims[1] = \{DIM0\};
00029     \textcolor{keywordtype}{size\_t}      sdim;
00030     \textcolor{keywordtype}{char}        wdata[DIM0][SDIM] = \{\textcolor{stringliteral}{"Parting"}, \textcolor{stringliteral}{"is such"}, \textcolor{stringliteral}{"sweet"}, \textcolor{stringliteral}{"sorrow."}\},
00031                                             \textcolor{comment}{/* Write buffer */}
00032                 **rdata;                    \textcolor{comment}{/* Read buffer */}
00033     \textcolor{keywordtype}{int}         ndims,
00034                 i;
00035 
00036     \textcolor{comment}{/*}
00037 \textcolor{comment}{     * Create a new file using the default properties.}
00038 \textcolor{comment}{     */}
00039     file = H5Fcreate (FILE, H5F\_ACC\_TRUNC, H5P\_DEFAULT, H5P\_DEFAULT);
00040 
00041     \textcolor{comment}{/*}
00042 \textcolor{comment}{     * Create file and memory datatypes.  For this example we will save}
00043 \textcolor{comment}{     * the strings as FORTRAN strings, therefore they do not need space}
00044 \textcolor{comment}{     * for the null terminator in the file.}
00045 \textcolor{comment}{     */}
00046     filetype = H5Tcopy (H5T\_FORTRAN\_S1);
00047     status = H5Tset\_size (filetype, SDIM - 1);
00048     memtype = H5Tcopy (H5T\_C\_S1);
00049     status = H5Tset\_size (memtype, SDIM);
00050 
00051     \textcolor{comment}{/*}
00052 \textcolor{comment}{     * Create dataspace.  Setting maximum size to NULL sets the maximum}
00053 \textcolor{comment}{     * size to be the current size.}
00054 \textcolor{comment}{     */}
00055     space = H5Screate\_simple (1, dims, NULL);
00056 
00057     \textcolor{comment}{/*}
00058 \textcolor{comment}{     * Create the dataset and write the string data to it.}
00059 \textcolor{comment}{     */}
00060     dset = H5Dcreate (file, DATASET, filetype, space, H5P\_DEFAULT, H5P\_DEFAULT,
00061                 H5P\_DEFAULT);
00062     status = H5Dwrite (dset, memtype, H5S\_ALL, H5S\_ALL, H5P\_DEFAULT, wdata[0]);
00063 
00064     \textcolor{comment}{/*}
00065 \textcolor{comment}{     * Close and release resources.}
00066 \textcolor{comment}{     */}
00067     status = H5Dclose (dset);
00068     status = H5Sclose (space);
00069     status = H5Tclose (filetype);
00070     status = H5Tclose (memtype);
00071     status = H5Fclose (file);
00072 
00073 
00074     \textcolor{comment}{/*}
00075 \textcolor{comment}{     * Now we begin the read section of this example.  Here we assume}
00076 \textcolor{comment}{     * the dataset and string have the same name and rank, but can have}
00077 \textcolor{comment}{     * any size.  Therefore we must allocate a new array to read in}
00078 \textcolor{comment}{     * data using malloc().}
00079 \textcolor{comment}{     */}
00080 
00081     \textcolor{comment}{/*}
00082 \textcolor{comment}{     * Open file and dataset.}
00083 \textcolor{comment}{     */}
00084     file = H5Fopen (FILE, H5F\_ACC\_RDONLY, H5P\_DEFAULT);
00085     dset = H5Dopen (file, DATASET, H5P\_DEFAULT);
00086 
00087     \textcolor{comment}{/*}
00088 \textcolor{comment}{     * Get the datatype and its size.}
00089 \textcolor{comment}{     */}
00090     filetype = H5Dget\_type (dset);
00091     sdim = H5Tget\_size (filetype);
00092     sdim++;                         \textcolor{comment}{/* Make room for null terminator */}
00093 
00094     \textcolor{comment}{/*}
00095 \textcolor{comment}{     * Get dataspace and allocate memory for read buffer.  This is a}
00096 \textcolor{comment}{     * two dimensional dataset so the dynamic allocation must be done}
00097 \textcolor{comment}{     * in steps.}
00098 \textcolor{comment}{     */}
00099     space = H5Dget\_space (dset);
00100     ndims = H5Sget\_simple\_extent\_dims (space, dims, NULL);
00101 
00102     \textcolor{comment}{/*}
00103 \textcolor{comment}{     * Allocate array of pointers to rows.}
00104 \textcolor{comment}{     */}
00105     rdata = (\textcolor{keywordtype}{char} **) malloc (dims[0] * \textcolor{keyword}{sizeof} (\textcolor{keywordtype}{char} *));
00106 
00107     \textcolor{comment}{/*}
00108 \textcolor{comment}{     * Allocate space for integer data.}
00109 \textcolor{comment}{     */}
00110     rdata[0] = (\textcolor{keywordtype}{char} *) malloc (dims[0] * sdim * \textcolor{keyword}{sizeof} (\textcolor{keywordtype}{char}));
00111 
00112     \textcolor{comment}{/*}
00113 \textcolor{comment}{     * Set the rest of the pointers to rows to the correct addresses.}
00114 \textcolor{comment}{     */}
00115     \textcolor{keywordflow}{for} (i=1; i<dims[0]; i++)
00116         rdata[i] = rdata[0] + i * sdim;
00117 
00118     \textcolor{comment}{/*}
00119 \textcolor{comment}{     * Create the memory datatype.}
00120 \textcolor{comment}{     */}
00121     memtype = H5Tcopy (H5T\_C\_S1);
00122     status = H5Tset\_size (memtype, sdim);
00123 
00124     \textcolor{comment}{/*}
00125 \textcolor{comment}{     * Read the data.}
00126 \textcolor{comment}{     */}
00127     status = H5Dread (dset, memtype, H5S\_ALL, H5S\_ALL, H5P\_DEFAULT, rdata[0]);
00128 
00129     \textcolor{comment}{/*}
00130 \textcolor{comment}{     * Output the data to the screen.}
00131 \textcolor{comment}{     */}
00132     \textcolor{keywordflow}{for} (i=0; i<dims[0]; i++)
00133         printf (\textcolor{stringliteral}{"%s[%d]: %s\(\backslash\)n"}, DATASET, i, rdata[i]);
00134 
00135     \textcolor{comment}{/*}
00136 \textcolor{comment}{     * Close and release resources.}
00137 \textcolor{comment}{     */}
00138     free (rdata[0]);
00139     free (rdata);
00140     status = H5Dclose (dset);
00141     status = H5Sclose (space);
00142     status = H5Tclose (filetype);
00143     status = H5Tclose (memtype);
00144     status = H5Fclose (file);
00145 
00146     \textcolor{keywordflow}{return} 0;
00147 \}
\end{DoxyCode}
