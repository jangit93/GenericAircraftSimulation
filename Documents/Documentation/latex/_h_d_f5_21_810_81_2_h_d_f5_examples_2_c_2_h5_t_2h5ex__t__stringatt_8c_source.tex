\hypertarget{_h_d_f5_21_810_81_2_h_d_f5_examples_2_c_2_h5_t_2h5ex__t__stringatt_8c_source}{}\section{H\+D\+F5/1.10.1/\+H\+D\+F5\+Examples/\+C/\+H5\+T/h5ex\+\_\+t\+\_\+stringatt.c}
\label{_h_d_f5_21_810_81_2_h_d_f5_examples_2_c_2_h5_t_2h5ex__t__stringatt_8c_source}\index{h5ex\+\_\+t\+\_\+stringatt.\+c@{h5ex\+\_\+t\+\_\+stringatt.\+c}}

\begin{DoxyCode}
00001 \textcolor{comment}{/************************************************************}
00002 \textcolor{comment}{}
00003 \textcolor{comment}{  This example shows how to read and write string datatypes}
00004 \textcolor{comment}{  to an attribute.  The program first writes strings to an}
00005 \textcolor{comment}{  attribute with a dataspace of DIM0, then closes the file.}
00006 \textcolor{comment}{  Next, it reopens the file, reads back the data, and}
00007 \textcolor{comment}{  outputs it to the screen.}
00008 \textcolor{comment}{}
00009 \textcolor{comment}{  This file is intended for use with HDF5 Library version 1.8}
00010 \textcolor{comment}{}
00011 \textcolor{comment}{ ************************************************************/}
00012 
00013 \textcolor{preprocessor}{#include "hdf5.h"}
00014 \textcolor{preprocessor}{#include <stdio.h>}
00015 \textcolor{preprocessor}{#include <stdlib.h>}
00016 
00017 \textcolor{preprocessor}{#define FILE            "h5ex\_t\_stringatt.h5"}
00018 \textcolor{preprocessor}{#define DATASET         "DS1"}
00019 \textcolor{preprocessor}{#define ATTRIBUTE       "A1"}
00020 \textcolor{preprocessor}{#define DIM0            4}
00021 \textcolor{preprocessor}{#define SDIM            8}
00022 
00023 \textcolor{keywordtype}{int}
00024 main (\textcolor{keywordtype}{void})
00025 \{
00026     hid\_t       \hyperlink{structfile}{file}, filetype, memtype, space, dset, attr;
00027                                             \textcolor{comment}{/* Handles */}
00028     herr\_t      status;
00029     hsize\_t     dims[1] = \{DIM0\};
00030     \textcolor{keywordtype}{size\_t}      sdim;
00031     \textcolor{keywordtype}{char}        wdata[DIM0][SDIM] = \{\textcolor{stringliteral}{"Parting"}, \textcolor{stringliteral}{"is such"}, \textcolor{stringliteral}{"sweet"}, \textcolor{stringliteral}{"sorrow."}\},
00032                                             \textcolor{comment}{/* Write buffer */}
00033                 **rdata;                    \textcolor{comment}{/* Read buffer */}
00034     \textcolor{keywordtype}{int}         ndims,
00035                 i;
00036 
00037     \textcolor{comment}{/*}
00038 \textcolor{comment}{     * Create a new file using the default properties.}
00039 \textcolor{comment}{     */}
00040     file = H5Fcreate (FILE, H5F\_ACC\_TRUNC, H5P\_DEFAULT, H5P\_DEFAULT);
00041 
00042     \textcolor{comment}{/*}
00043 \textcolor{comment}{     * Create file and memory datatypes.  For this example we will save}
00044 \textcolor{comment}{     * the strings as FORTRAN strings, therefore they do not need space}
00045 \textcolor{comment}{     * for the null terminator in the file.}
00046 \textcolor{comment}{     */}
00047     filetype = H5Tcopy (H5T\_FORTRAN\_S1);
00048     status = H5Tset\_size (filetype, SDIM - 1);
00049     memtype = H5Tcopy (H5T\_C\_S1);
00050     status = H5Tset\_size (memtype, SDIM);
00051 
00052     \textcolor{comment}{/*}
00053 \textcolor{comment}{     * Create dataset with a null dataspace.}
00054 \textcolor{comment}{     */}
00055     space = H5Screate (H5S\_NULL);
00056     dset = H5Dcreate (file, DATASET, H5T\_STD\_I32LE, space, H5P\_DEFAULT,
00057                 H5P\_DEFAULT, H5P\_DEFAULT);
00058     status = H5Sclose (space);
00059 
00060     \textcolor{comment}{/*}
00061 \textcolor{comment}{     * Create dataspace.  Setting maximum size to NULL sets the maximum}
00062 \textcolor{comment}{     * size to be the current size.}
00063 \textcolor{comment}{     */}
00064     space = H5Screate\_simple (1, dims, NULL);
00065 
00066     \textcolor{comment}{/*}
00067 \textcolor{comment}{     * Create the attribute and write the string data to it.}
00068 \textcolor{comment}{     */}
00069     attr = H5Acreate (dset, ATTRIBUTE, filetype, space, H5P\_DEFAULT,
00070                 H5P\_DEFAULT);
00071     status = H5Awrite (attr, memtype, wdata[0]);
00072 
00073     \textcolor{comment}{/*}
00074 \textcolor{comment}{     * Close and release resources.}
00075 \textcolor{comment}{     */}
00076     status = H5Aclose (attr);
00077     status = H5Dclose (dset);
00078     status = H5Sclose (space);
00079     status = H5Tclose (filetype);
00080     status = H5Tclose (memtype);
00081     status = H5Fclose (file);
00082 
00083 
00084     \textcolor{comment}{/*}
00085 \textcolor{comment}{     * Now we begin the read section of this example.  Here we assume}
00086 \textcolor{comment}{     * the attribute and string have the same name and rank, but can}
00087 \textcolor{comment}{     * have any size.  Therefore we must allocate a new array to read}
00088 \textcolor{comment}{     * in data using malloc().}
00089 \textcolor{comment}{     */}
00090 
00091     \textcolor{comment}{/*}
00092 \textcolor{comment}{     * Open file, dataset, and attribute.}
00093 \textcolor{comment}{     */}
00094     file = H5Fopen (FILE, H5F\_ACC\_RDONLY, H5P\_DEFAULT);
00095     dset = H5Dopen (file, DATASET, H5P\_DEFAULT);
00096     attr = H5Aopen (dset, ATTRIBUTE, H5P\_DEFAULT);
00097 
00098     \textcolor{comment}{/*}
00099 \textcolor{comment}{     * Get the datatype and its size.}
00100 \textcolor{comment}{     */}
00101     filetype = H5Aget\_type (attr);
00102     sdim = H5Tget\_size (filetype);
00103     sdim++;                         \textcolor{comment}{/* Make room for null terminator */}
00104 
00105     \textcolor{comment}{/*}
00106 \textcolor{comment}{     * Get dataspace and allocate memory for read buffer.  This is a}
00107 \textcolor{comment}{     * two dimensional attribute so the dynamic allocation must be done}
00108 \textcolor{comment}{     * in steps.}
00109 \textcolor{comment}{     */}
00110     space = H5Aget\_space (attr);
00111     ndims = H5Sget\_simple\_extent\_dims (space, dims, NULL);
00112 
00113     \textcolor{comment}{/*}
00114 \textcolor{comment}{     * Allocate array of pointers to rows.}
00115 \textcolor{comment}{     */}
00116     rdata = (\textcolor{keywordtype}{char} **) malloc (dims[0] * \textcolor{keyword}{sizeof} (\textcolor{keywordtype}{char} *));
00117 
00118     \textcolor{comment}{/*}
00119 \textcolor{comment}{     * Allocate space for integer data.}
00120 \textcolor{comment}{     */}
00121     rdata[0] = (\textcolor{keywordtype}{char} *) malloc (dims[0] * sdim * \textcolor{keyword}{sizeof} (\textcolor{keywordtype}{char}));
00122 
00123     \textcolor{comment}{/*}
00124 \textcolor{comment}{     * Set the rest of the pointers to rows to the correct addresses.}
00125 \textcolor{comment}{     */}
00126     \textcolor{keywordflow}{for} (i=1; i<dims[0]; i++)
00127         rdata[i] = rdata[0] + i * sdim;
00128 
00129     \textcolor{comment}{/*}
00130 \textcolor{comment}{     * Create the memory datatype.}
00131 \textcolor{comment}{     */}
00132     memtype = H5Tcopy (H5T\_C\_S1);
00133     status = H5Tset\_size (memtype, sdim);
00134 
00135     \textcolor{comment}{/*}
00136 \textcolor{comment}{     * Read the data.}
00137 \textcolor{comment}{     */}
00138     status = H5Aread (attr, memtype, rdata[0]);
00139 
00140     \textcolor{comment}{/*}
00141 \textcolor{comment}{     * Output the data to the screen.}
00142 \textcolor{comment}{     */}
00143     \textcolor{keywordflow}{for} (i=0; i<dims[0]; i++)
00144         printf (\textcolor{stringliteral}{"%s[%d]: %s\(\backslash\)n"}, ATTRIBUTE, i, rdata[i]);
00145 
00146     \textcolor{comment}{/*}
00147 \textcolor{comment}{     * Close and release resources.}
00148 \textcolor{comment}{     */}
00149     free (rdata[0]);
00150     free (rdata);
00151     status = H5Aclose (attr);
00152     status = H5Dclose (dset);
00153     status = H5Sclose (space);
00154     status = H5Tclose (filetype);
00155     status = H5Tclose (memtype);
00156     status = H5Fclose (file);
00157 
00158     \textcolor{keywordflow}{return} 0;
00159 \}
\end{DoxyCode}
