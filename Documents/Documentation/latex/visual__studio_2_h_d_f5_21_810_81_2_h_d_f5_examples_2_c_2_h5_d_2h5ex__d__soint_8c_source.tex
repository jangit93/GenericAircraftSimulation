\hypertarget{visual__studio_2_h_d_f5_21_810_81_2_h_d_f5_examples_2_c_2_h5_d_2h5ex__d__soint_8c_source}{}\section{visual\+\_\+studio/\+H\+D\+F5/1.10.1/\+H\+D\+F5\+Examples/\+C/\+H5\+D/h5ex\+\_\+d\+\_\+soint.c}
\label{visual__studio_2_h_d_f5_21_810_81_2_h_d_f5_examples_2_c_2_h5_d_2h5ex__d__soint_8c_source}\index{h5ex\+\_\+d\+\_\+soint.\+c@{h5ex\+\_\+d\+\_\+soint.\+c}}

\begin{DoxyCode}
00001 \textcolor{comment}{/************************************************************}
00002 \textcolor{comment}{}
00003 \textcolor{comment}{  This example shows how to read and write data to a dataset}
00004 \textcolor{comment}{  using the Scale-Offset filter.  The program first checks}
00005 \textcolor{comment}{  if the Scale-Offset filter is available, then if it is it}
00006 \textcolor{comment}{  writes integers to a dataset using Scale-Offset, then}
00007 \textcolor{comment}{  closes the file Next, it reopens the file, reads back the}
00008 \textcolor{comment}{  data, and outputs the type of filter and the maximum value}
00009 \textcolor{comment}{  in the dataset to the screen.}
00010 \textcolor{comment}{}
00011 \textcolor{comment}{  This file is intended for use with HDF5 Library version 1.8}
00012 \textcolor{comment}{}
00013 \textcolor{comment}{ ************************************************************/}
00014 
00015 \textcolor{preprocessor}{#include "hdf5.h"}
00016 \textcolor{preprocessor}{#include <stdio.h>}
00017 \textcolor{preprocessor}{#include <stdlib.h>}
00018 
00019 \textcolor{preprocessor}{#define FILE            "h5ex\_d\_soint.h5"}
00020 \textcolor{preprocessor}{#define DATASET         "DS1"}
00021 \textcolor{preprocessor}{#define DIM0            32}
00022 \textcolor{preprocessor}{#define DIM1            64}
00023 \textcolor{preprocessor}{#define CHUNK0          4}
00024 \textcolor{preprocessor}{#define CHUNK1          8}
00025 
00026 \textcolor{keywordtype}{int}
00027 main (\textcolor{keywordtype}{void})
00028 \{
00029     hid\_t           \hyperlink{structfile}{file}, space, dset, dcpl;
00030                                                 \textcolor{comment}{/* Handles */}
00031     herr\_t          status;
00032     htri\_t          avail;
00033     H5Z\_filter\_t    filter\_type;
00034     hsize\_t         dims[2] = \{DIM0, DIM1\},
00035                     chunk[2] = \{CHUNK0, CHUNK1\};
00036     \textcolor{keywordtype}{size\_t}          nelmts;
00037     \textcolor{keywordtype}{unsigned} \textcolor{keywordtype}{int}    flags,
00038                     filter\_info;
00039     \textcolor{keywordtype}{int}             wdata[DIM0][DIM1],          \textcolor{comment}{/* Write buffer */}
00040                     rdata[DIM0][DIM1],          \textcolor{comment}{/* Read buffer */}
00041                     max,
00042                     i, j;
00043 
00044     \textcolor{comment}{/*}
00045 \textcolor{comment}{     * Check if Scale-Offset compression is available and can be used}
00046 \textcolor{comment}{     * for both compression and decompression.  Normally we do not}
00047 \textcolor{comment}{     * perform error checking in these examples for the sake of}
00048 \textcolor{comment}{     * clarity, but in this case we will make an exception because this}
00049 \textcolor{comment}{     * filter is an optional part of the hdf5 library.}
00050 \textcolor{comment}{     */}
00051     avail = H5Zfilter\_avail(H5Z\_FILTER\_SCALEOFFSET);
00052     \textcolor{keywordflow}{if} (!avail) \{
00053         printf (\textcolor{stringliteral}{"Scale-Offset filter not available.\(\backslash\)n"});
00054         \textcolor{keywordflow}{return} 1;
00055     \}
00056     status = H5Zget\_filter\_info (H5Z\_FILTER\_SCALEOFFSET, &filter\_info);
00057     \textcolor{keywordflow}{if} ( !(filter\_info & H5Z\_FILTER\_CONFIG\_ENCODE\_ENABLED) ||
00058                 !(filter\_info & H5Z\_FILTER\_CONFIG\_DECODE\_ENABLED) ) \{
00059         printf (\textcolor{stringliteral}{"Scale-Offset filter not available for encoding and decoding.\(\backslash\)n"});
00060         \textcolor{keywordflow}{return} 1;
00061     \}
00062 
00063     \textcolor{comment}{/*}
00064 \textcolor{comment}{     * Initialize data.}
00065 \textcolor{comment}{     */}
00066     \textcolor{keywordflow}{for} (i=0; i<DIM0; i++)
00067         \textcolor{keywordflow}{for} (j=0; j<DIM1; j++)
00068             wdata[i][j] = i * j - j;
00069 
00070     \textcolor{comment}{/*}
00071 \textcolor{comment}{     * Create a new file using the default properties.}
00072 \textcolor{comment}{     */}
00073     file = H5Fcreate (FILE, H5F\_ACC\_TRUNC, H5P\_DEFAULT, H5P\_DEFAULT);
00074 
00075     \textcolor{comment}{/*}
00076 \textcolor{comment}{     * Create dataspace.  Setting maximum size to NULL sets the maximum}
00077 \textcolor{comment}{     * size to be the current size.}
00078 \textcolor{comment}{     */}
00079     space = H5Screate\_simple (2, dims, NULL);
00080 
00081     \textcolor{comment}{/*}
00082 \textcolor{comment}{     * Create the dataset creation property list, add the Scale-Offset}
00083 \textcolor{comment}{     * filter and set the chunk size.}
00084 \textcolor{comment}{     */}
00085     dcpl = H5Pcreate (H5P\_DATASET\_CREATE);
00086     status = H5Pset\_scaleoffset (dcpl, H5Z\_SO\_INT, H5Z\_SO\_INT\_MINBITS\_DEFAULT);
00087     status = H5Pset\_chunk (dcpl, 2, chunk);
00088 
00089     \textcolor{comment}{/*}
00090 \textcolor{comment}{     * Create the dataset.}
00091 \textcolor{comment}{     */}
00092     dset = H5Dcreate (file, DATASET, H5T\_STD\_I32LE, space, H5P\_DEFAULT, dcpl,
00093                 H5P\_DEFAULT);
00094 
00095     \textcolor{comment}{/*}
00096 \textcolor{comment}{     * Write the data to the dataset.}
00097 \textcolor{comment}{     */}
00098     status = H5Dwrite (dset, H5T\_NATIVE\_INT, H5S\_ALL, H5S\_ALL, H5P\_DEFAULT,
00099                 wdata[0]);
00100 
00101     \textcolor{comment}{/*}
00102 \textcolor{comment}{     * Close and release resources.}
00103 \textcolor{comment}{     */}
00104     status = H5Pclose (dcpl);
00105     status = H5Dclose (dset);
00106     status = H5Sclose (space);
00107     status = H5Fclose (file);
00108 
00109 
00110     \textcolor{comment}{/*}
00111 \textcolor{comment}{     * Now we begin the read section of this example.}
00112 \textcolor{comment}{     */}
00113 
00114     \textcolor{comment}{/*}
00115 \textcolor{comment}{     * Open file and dataset using the default properties.}
00116 \textcolor{comment}{     */}
00117     file = H5Fopen (FILE, H5F\_ACC\_RDONLY, H5P\_DEFAULT);
00118     dset = H5Dopen (file, DATASET, H5P\_DEFAULT);
00119 
00120     \textcolor{comment}{/*}
00121 \textcolor{comment}{     * Retrieve dataset creation property list.}
00122 \textcolor{comment}{     */}
00123     dcpl = H5Dget\_create\_plist (dset);
00124 
00125     \textcolor{comment}{/*}
00126 \textcolor{comment}{     * Retrieve and print the filter type.  Here we only retrieve the}
00127 \textcolor{comment}{     * first filter because we know that we only added one filter.}
00128 \textcolor{comment}{     */}
00129     nelmts = 0;
00130     filter\_type = H5Pget\_filter (dcpl, 0, &flags, &nelmts, NULL, 0, NULL, &filter\_info);
00131     printf (\textcolor{stringliteral}{"Filter type is: "});
00132     \textcolor{keywordflow}{switch} (filter\_type) \{
00133         \textcolor{keywordflow}{case} H5Z\_FILTER\_DEFLATE:
00134             printf (\textcolor{stringliteral}{"H5Z\_FILTER\_DEFLATE\(\backslash\)n"});
00135             \textcolor{keywordflow}{break};
00136         \textcolor{keywordflow}{case} H5Z\_FILTER\_SHUFFLE:
00137             printf (\textcolor{stringliteral}{"H5Z\_FILTER\_SHUFFLE\(\backslash\)n"});
00138             \textcolor{keywordflow}{break};
00139         \textcolor{keywordflow}{case} H5Z\_FILTER\_FLETCHER32:
00140             printf (\textcolor{stringliteral}{"H5Z\_FILTER\_FLETCHER32\(\backslash\)n"});
00141             \textcolor{keywordflow}{break};
00142         \textcolor{keywordflow}{case} H5Z\_FILTER\_SZIP:
00143             printf (\textcolor{stringliteral}{"H5Z\_FILTER\_SZIP\(\backslash\)n"});
00144             \textcolor{keywordflow}{break};
00145         \textcolor{keywordflow}{case} H5Z\_FILTER\_NBIT:
00146             printf (\textcolor{stringliteral}{"H5Z\_FILTER\_NBIT\(\backslash\)n"});
00147             \textcolor{keywordflow}{break};
00148         \textcolor{keywordflow}{case} H5Z\_FILTER\_SCALEOFFSET:
00149             printf (\textcolor{stringliteral}{"H5Z\_FILTER\_SCALEOFFSET\(\backslash\)n"});
00150     \}
00151 
00152     \textcolor{comment}{/*}
00153 \textcolor{comment}{     * Read the data using the default properties.}
00154 \textcolor{comment}{     */}
00155     status = H5Dread (dset, H5T\_NATIVE\_INT, H5S\_ALL, H5S\_ALL, H5P\_DEFAULT,
00156                 rdata[0]);
00157 
00158     \textcolor{comment}{/*}
00159 \textcolor{comment}{     * Find the maximum value in the dataset, to verify that it was}
00160 \textcolor{comment}{     * read correctly.}
00161 \textcolor{comment}{     */}
00162     max = rdata[0][0];
00163     \textcolor{keywordflow}{for} (i=0; i<DIM0; i++)
00164         \textcolor{keywordflow}{for} (j=0; j<DIM1; j++)
00165             \textcolor{keywordflow}{if} (max < rdata[i][j])
00166                 max = rdata[i][j];
00167 
00168     \textcolor{comment}{/*}
00169 \textcolor{comment}{     * Print the maximum value.}
00170 \textcolor{comment}{     */}
00171     printf (\textcolor{stringliteral}{"Maximum value in %s is: %d\(\backslash\)n"}, DATASET, max);
00172 
00173     \textcolor{comment}{/*}
00174 \textcolor{comment}{     * Close and release resources.}
00175 \textcolor{comment}{     */}
00176     status = H5Pclose (dcpl);
00177     status = H5Dclose (dset);
00178     status = H5Fclose (file);
00179 
00180     \textcolor{keywordflow}{return} 0;
00181 \}
\end{DoxyCode}
