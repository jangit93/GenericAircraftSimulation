\hypertarget{visual__studio_2_h_d_f5_21_810_81_2_h_d_f5_examples_2_f_o_r_t_r_a_n_2_h5_t_2h5ex__t__float___f03_8f90_source}{}\section{visual\+\_\+studio/\+H\+D\+F5/1.10.1/\+H\+D\+F5\+Examples/\+F\+O\+R\+T\+R\+A\+N/\+H5\+T/h5ex\+\_\+t\+\_\+float\+\_\+\+F03.f90}
\label{visual__studio_2_h_d_f5_21_810_81_2_h_d_f5_examples_2_f_o_r_t_r_a_n_2_h5_t_2h5ex__t__float___f03_8f90_source}\index{h5ex\+\_\+t\+\_\+float\+\_\+\+F03.\+f90@{h5ex\+\_\+t\+\_\+float\+\_\+\+F03.\+f90}}

\begin{DoxyCode}
00001 \textcolor{comment}{!************************************************************}
00002 \textcolor{comment}{!}
00003 \textcolor{comment}{!  This example shows how to read and write REAL datatypes}
00004 \textcolor{comment}{!  (using SELECTED\_REAL\_KIND) to a dataset.  The program first }
00005 \textcolor{comment}{!  writes REAL datatypes to a dataset with a dataspace of }
00006 \textcolor{comment}{!  DIM0xDIM1, then closes the file.  Next, it reopens the file, }
00007 \textcolor{comment}{!  reads back the REAL data, and outputs it to the screen.}
00008 \textcolor{comment}{!}
00009 \textcolor{comment}{!  This file is intended for use with HDF5 Library version 1.8}
00010 \textcolor{comment}{!  with --enable-fortran2003}
00011 \textcolor{comment}{!}
00012 \textcolor{comment}{!************************************************************}
00013 \textcolor{keyword}{PROGRAM} main
00014 
00015   \textcolor{keywordtype}{USE }hdf5
00016   \textcolor{keywordtype}{USE }iso\_c\_binding
00017 
00018   \textcolor{keywordtype}{IMPLICIT NONE}
00019   \textcolor{comment}{! This should map to REAL*8 on most modern processors}
00020   \textcolor{keywordtype}{INTEGER}, \textcolor{keywordtype}{PARAMETER} :: real\_kind\_15 = selected\_real\_kind(15,307)
00021 
00022   \textcolor{keywordtype}{CHARACTER(LEN=19)}, \textcolor{keywordtype}{PARAMETER} :: filename  = \textcolor{stringliteral}{"h5ex\_t\_float\_F03.h5"}
00023   \textcolor{keywordtype}{CHARACTER(LEN=3)} , \textcolor{keywordtype}{PARAMETER} :: dataset   = \textcolor{stringliteral}{"DS1"}
00024   \textcolor{keywordtype}{INTEGER}          , \textcolor{keywordtype}{PARAMETER} :: dim0      = 4
00025   \textcolor{keywordtype}{INTEGER}          , \textcolor{keywordtype}{PARAMETER} :: dim1      = 7
00026 
00027   \textcolor{keywordtype}{INTEGER(HID\_T)}  :: \hyperlink{structfile}{file}, space, dset \textcolor{comment}{! Handles}
00028   \textcolor{keywordtype}{INTEGER} :: hdferr
00029 
00030   \textcolor{keywordtype}{INTEGER(hsize\_t)},   \textcolor{keywordtype}{DIMENSION(1:2)} :: dims = (/dim0, dim1/)
00031   \textcolor{keywordtype}{REAL(KIND=real\_kind\_15)}, \textcolor{keywordtype}{DIMENSION(1:dim0, 1:dim1)}, \textcolor{keywordtype}{TARGET} :: wdata \textcolor{comment}{! Write buffer}
00032   \textcolor{keywordtype}{REAL(KIND=real\_kind\_15)}, \textcolor{keywordtype}{DIMENSION(:,:)}, \textcolor{keywordtype}{ALLOCATABLE}, \textcolor{keywordtype}{TARGET} :: rdata \textcolor{comment}{! Read buffer}
00033   \textcolor{keywordtype}{INTEGER(HSIZE\_T)}, \textcolor{keywordtype}{DIMENSION(1:2)} :: maxdims
00034   \textcolor{keywordtype}{INTEGER} :: i, j
00035   \textcolor{keywordtype}{TYPE}(c\_ptr) :: f\_ptr
00036   \textcolor{comment}{!}
00037   \textcolor{comment}{! Initialize FORTRAN interface.}
00038   \textcolor{comment}{!}
00039   \textcolor{keyword}{CALL }h5open\_f(hdferr)
00040   \textcolor{comment}{!}
00041   \textcolor{comment}{! Initialize DATA.}
00042   \textcolor{comment}{!}
00043   \textcolor{keywordflow}{DO} i = 1, dim0
00044      \textcolor{keywordflow}{DO} j = 1, dim1
00045         wdata(i,j) = \textcolor{keywordtype}{REAL(i-1,real\_kind\_15)} / ( \textcolor{keywordtype}{REAL}(j-1,real\_kind\_15)+0.5\_real\_kind\_15) + j-1
00046 \textcolor{keywordflow}{     ENDDO}
00047 \textcolor{keywordflow}{  ENDDO}
00048   \textcolor{comment}{!}
00049   \textcolor{comment}{! Create a new file using the default properties.}
00050   \textcolor{comment}{!}
00051   \textcolor{keyword}{CALL }h5fcreate\_f(filename, h5f\_acc\_trunc\_f, \hyperlink{structfile}{file}, hdferr)
00052   \textcolor{comment}{!}
00053   \textcolor{comment}{! Create dataspace.  Setting maximum size to be the current size.}
00054   \textcolor{comment}{!}
00055   \textcolor{keyword}{CALL }h5screate\_simple\_f(2, dims, space, hdferr)
00056   \textcolor{comment}{!}
00057   \textcolor{comment}{! Create the dataset and write the floating point data to it.  In}
00058   \textcolor{comment}{! this example we will save the data as 64 bit little endian IEEE}
00059   \textcolor{comment}{! floating point numbers, regardless of the native type.  The HDF5}
00060   \textcolor{comment}{! library automatically converts between different floating point}
00061   \textcolor{comment}{! types.}
00062   \textcolor{comment}{!}
00063   \textcolor{keyword}{CALL }h5dcreate\_f(\hyperlink{structfile}{file}, dataset, h5t\_ieee\_f64le, space, dset, hdferr)
00064   f\_ptr = c\_loc(wdata(1,1))
00065   \textcolor{keyword}{CALL }h5dwrite\_f(dset, h5t\_native\_double, f\_ptr, hdferr)
00066   \textcolor{comment}{!}
00067   \textcolor{comment}{! Close and release resources.}
00068   \textcolor{comment}{!}
00069   \textcolor{keyword}{CALL }h5dclose\_f(dset , hdferr)
00070   \textcolor{keyword}{CALL }h5sclose\_f(space, hdferr)
00071   \textcolor{keyword}{CALL }h5fclose\_f(\hyperlink{structfile}{file} , hdferr)
00072   \textcolor{comment}{!}
00073   \textcolor{comment}{! Open file and dataset.}
00074   \textcolor{comment}{!}
00075   \textcolor{keyword}{CALL }h5fopen\_f(filename, h5f\_acc\_rdonly\_f, \hyperlink{structfile}{file}, hdferr)
00076   \textcolor{keyword}{CALL }h5dopen\_f (\hyperlink{structfile}{file}, dataset, dset, hdferr)
00077   \textcolor{comment}{!}
00078   \textcolor{comment}{! Get dataspace and allocate memory for read buffer.}
00079   \textcolor{comment}{!}
00080   \textcolor{keyword}{CALL }h5dget\_space\_f(dset,space, hdferr)
00081   \textcolor{keyword}{CALL }h5sget\_simple\_extent\_dims\_f (space, dims, maxdims, hdferr)
00082 
00083   \textcolor{keyword}{ALLOCATE}(rdata(1:dims(1),1:dims(2)))
00084   \textcolor{comment}{!}
00085   \textcolor{comment}{! Read the data.}
00086   \textcolor{comment}{!}
00087   f\_ptr = c\_loc(rdata(1,1))
00088   \textcolor{keyword}{CALL }h5dread\_f( dset, h5t\_native\_double, f\_ptr, hdferr)
00089   \textcolor{comment}{!}
00090   \textcolor{comment}{! Output the data to the screen.}
00091   \textcolor{comment}{!}
00092   \textcolor{keyword}{WRITE}(*, \textcolor{stringliteral}{'(A,":")'}) dataset
00093   \textcolor{keywordflow}{DO} i=1, dims(1)
00094      \textcolor{keyword}{WRITE}(*,\textcolor{stringliteral}{'(" [")'}, advance=\textcolor{stringliteral}{'NO'})
00095      \textcolor{keyword}{WRITE}(*,\textcolor{stringliteral}{'(80(" ",f9.4))'}, advance=\textcolor{stringliteral}{'NO'}) rdata(i,1:dims(2))
00096      \textcolor{keyword}{WRITE}(*,\textcolor{stringliteral}{'(" ]")'})
00097 \textcolor{keywordflow}{  ENDDO}
00098   \textcolor{comment}{!}
00099   \textcolor{comment}{! Close and release resources.}
00100   \textcolor{comment}{!}
00101   \textcolor{keyword}{DEALLOCATE}(rdata)
00102   \textcolor{keyword}{CALL }h5dclose\_f(dset , hdferr)
00103   \textcolor{keyword}{CALL }h5sclose\_f(space, hdferr)
00104   \textcolor{keyword}{CALL }h5fclose\_f(\hyperlink{structfile}{file} , hdferr)
00105 \textcolor{keyword}{END PROGRAM }main
\end{DoxyCode}
