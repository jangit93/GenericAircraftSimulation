\hypertarget{_h_d_f5_21_810_81_2_h_d_f5_examples_2_c_2_h5_t_2h5ex__t__vlen_8c_source}{}\section{H\+D\+F5/1.10.1/\+H\+D\+F5\+Examples/\+C/\+H5\+T/h5ex\+\_\+t\+\_\+vlen.c}
\label{_h_d_f5_21_810_81_2_h_d_f5_examples_2_c_2_h5_t_2h5ex__t__vlen_8c_source}\index{h5ex\+\_\+t\+\_\+vlen.\+c@{h5ex\+\_\+t\+\_\+vlen.\+c}}

\begin{DoxyCode}
00001 \textcolor{comment}{/************************************************************}
00002 \textcolor{comment}{}
00003 \textcolor{comment}{  This example shows how to read and write variable-length}
00004 \textcolor{comment}{  datatypes to a dataset.  The program first writes two}
00005 \textcolor{comment}{  variable-length integer arrays to a dataset then closes}
00006 \textcolor{comment}{  the file.  Next, it reopens the file, reads back the data,}
00007 \textcolor{comment}{  and outputs it to the screen.}
00008 \textcolor{comment}{}
00009 \textcolor{comment}{  This file is intended for use with HDF5 Library version 1.8}
00010 \textcolor{comment}{}
00011 \textcolor{comment}{ ************************************************************/}
00012 
00013 \textcolor{preprocessor}{#include "hdf5.h"}
00014 \textcolor{preprocessor}{#include <stdio.h>}
00015 \textcolor{preprocessor}{#include <stdlib.h>}
00016 
00017 \textcolor{preprocessor}{#define FILE            "h5ex\_t\_vlen.h5"}
00018 \textcolor{preprocessor}{#define DATASET         "DS1"}
00019 \textcolor{preprocessor}{#define LEN0            3}
00020 \textcolor{preprocessor}{#define LEN1            12}
00021 
00022 \textcolor{keywordtype}{int}
00023 main (\textcolor{keywordtype}{void})
00024 \{
00025     hid\_t       \hyperlink{structfile}{file}, filetype, memtype, space, dset;
00026                                     \textcolor{comment}{/* Handles */}
00027     herr\_t      status;
00028     \hyperlink{structhvl__t}{hvl\_t}       wdata[2],           \textcolor{comment}{/* Array of vlen structures */}
00029                 *rdata;             \textcolor{comment}{/* Pointer to vlen structures */}
00030     hsize\_t     dims[1] = \{2\};
00031     \textcolor{keywordtype}{int}         *ptr,
00032                 ndims,
00033                 i, j;
00034 
00035     \textcolor{comment}{/*}
00036 \textcolor{comment}{     * Initialize variable-length data.  wdata[0] is a countdown of}
00037 \textcolor{comment}{     * length LEN0, wdata[1] is a Fibonacci sequence of length LEN1.}
00038 \textcolor{comment}{     */}
00039     wdata[0].len = LEN0;
00040     ptr = (\textcolor{keywordtype}{int} *) malloc (wdata[0].len * \textcolor{keyword}{sizeof} (\textcolor{keywordtype}{int}));
00041     \textcolor{keywordflow}{for} (i=0; i<wdata[0].len; i++)
00042         ptr[i] = wdata[0].len - (\textcolor{keywordtype}{size\_t})i;       \textcolor{comment}{/* 3 2 1 */}
00043     wdata[0].p = (\textcolor{keywordtype}{void} *) ptr;
00044 
00045     wdata[1].len = LEN1;
00046     ptr = (\textcolor{keywordtype}{int} *) malloc (wdata[1].len * \textcolor{keyword}{sizeof} (\textcolor{keywordtype}{int}));
00047     ptr[0] = 1;
00048     ptr[1] = 1;
00049     \textcolor{keywordflow}{for} (i=2; i<wdata[1].len; i++)
00050         ptr[i] = ptr[i-1] + ptr[i-2];   \textcolor{comment}{/* 1 1 2 3 5 8 etc. */}
00051     wdata[1].p = (\textcolor{keywordtype}{void} *) ptr;
00052 
00053     \textcolor{comment}{/*}
00054 \textcolor{comment}{     * Create a new file using the default properties.}
00055 \textcolor{comment}{     */}
00056     file = H5Fcreate (FILE, H5F\_ACC\_TRUNC, H5P\_DEFAULT, H5P\_DEFAULT);
00057 
00058     \textcolor{comment}{/*}
00059 \textcolor{comment}{     * Create variable-length datatype for file and memory.}
00060 \textcolor{comment}{     */}
00061     filetype = H5Tvlen\_create (H5T\_STD\_I32LE);
00062     memtype = H5Tvlen\_create (H5T\_NATIVE\_INT);
00063 
00064     \textcolor{comment}{/*}
00065 \textcolor{comment}{     * Create dataspace.  Setting maximum size to NULL sets the maximum}
00066 \textcolor{comment}{     * size to be the current size.}
00067 \textcolor{comment}{     */}
00068     space = H5Screate\_simple (1, dims, NULL);
00069 
00070     \textcolor{comment}{/*}
00071 \textcolor{comment}{     * Create the dataset and write the variable-length data to it.}
00072 \textcolor{comment}{     */}
00073     dset = H5Dcreate (file, DATASET, filetype, space, H5P\_DEFAULT, H5P\_DEFAULT,
00074                 H5P\_DEFAULT);
00075     status = H5Dwrite (dset, memtype, H5S\_ALL, H5S\_ALL, H5P\_DEFAULT, wdata);
00076 
00077     \textcolor{comment}{/*}
00078 \textcolor{comment}{     * Close and release resources.  Note the use of H5Dvlen\_reclaim}
00079 \textcolor{comment}{     * removes the need to manually free() the previously malloc'ed}
00080 \textcolor{comment}{     * data.}
00081 \textcolor{comment}{     */}
00082     status = H5Dvlen\_reclaim (memtype, space, H5P\_DEFAULT, wdata);
00083     status = H5Dclose (dset);
00084     status = H5Sclose (space);
00085     status = H5Tclose (filetype);
00086     status = H5Tclose (memtype);
00087     status = H5Fclose (file);
00088 
00089 
00090     \textcolor{comment}{/*}
00091 \textcolor{comment}{     * Now we begin the read section of this example.  Here we assume}
00092 \textcolor{comment}{     * the dataset has the same name and rank, but can have any size.}
00093 \textcolor{comment}{     * Therefore we must allocate a new array to read in data using}
00094 \textcolor{comment}{     * malloc().}
00095 \textcolor{comment}{     */}
00096 
00097     \textcolor{comment}{/*}
00098 \textcolor{comment}{     * Open file and dataset.}
00099 \textcolor{comment}{     */}
00100     file = H5Fopen (FILE, H5F\_ACC\_RDONLY, H5P\_DEFAULT);
00101     dset = H5Dopen (file, DATASET, H5P\_DEFAULT);
00102 
00103     \textcolor{comment}{/*}
00104 \textcolor{comment}{     * Get dataspace and allocate memory for array of vlen structures.}
00105 \textcolor{comment}{     * This does not actually allocate memory for the vlen data, that}
00106 \textcolor{comment}{     * will be done by the library.}
00107 \textcolor{comment}{     */}
00108     space = H5Dget\_space (dset);
00109     ndims = H5Sget\_simple\_extent\_dims (space, dims, NULL);
00110     rdata = (\hyperlink{structhvl__t}{hvl\_t} *) malloc (dims[0] * \textcolor{keyword}{sizeof} (\hyperlink{structhvl__t}{hvl\_t}));
00111 
00112     \textcolor{comment}{/*}
00113 \textcolor{comment}{     * Create the memory datatype.}
00114 \textcolor{comment}{     */}
00115     memtype = H5Tvlen\_create (H5T\_NATIVE\_INT);
00116 
00117     \textcolor{comment}{/*}
00118 \textcolor{comment}{     * Read the data.}
00119 \textcolor{comment}{     */}
00120     status = H5Dread (dset, memtype, H5S\_ALL, H5S\_ALL, H5P\_DEFAULT, rdata);
00121 
00122     \textcolor{comment}{/*}
00123 \textcolor{comment}{     * Output the variable-length data to the screen.}
00124 \textcolor{comment}{     */}
00125     \textcolor{keywordflow}{for} (i=0; i<dims[0]; i++) \{
00126         printf (\textcolor{stringliteral}{"%s[%u]:\(\backslash\)n  \{"},DATASET,i);
00127         ptr = rdata[i].p;
00128         \textcolor{keywordflow}{for} (j=0; j<rdata[i].len; j++) \{
00129             printf (\textcolor{stringliteral}{" %d"}, ptr[j]);
00130             \textcolor{keywordflow}{if} ( (j+1) < rdata[i].len )
00131                 printf (\textcolor{stringliteral}{","});
00132         \}
00133         printf (\textcolor{stringliteral}{" \}\(\backslash\)n"});
00134     \}
00135 
00136     \textcolor{comment}{/*}
00137 \textcolor{comment}{     * Close and release resources.  Note we must still free the}
00138 \textcolor{comment}{     * top-level pointer "rdata", as H5Dvlen\_reclaim only frees the}
00139 \textcolor{comment}{     * actual variable-length data, and not the structures themselves.}
00140 \textcolor{comment}{     */}
00141     status = H5Dvlen\_reclaim (memtype, space, H5P\_DEFAULT, rdata);
00142     free (rdata);
00143     status = H5Dclose (dset);
00144     status = H5Sclose (space);
00145     status = H5Tclose (memtype);
00146     status = H5Fclose (file);
00147 
00148     \textcolor{keywordflow}{return} 0;
00149 \}
\end{DoxyCode}
