\hypertarget{visual__studio_2_h_d_f5_21_810_81_2include_2_h5_abstract_ds_8h_source}{}\section{visual\+\_\+studio/\+H\+D\+F5/1.10.1/include/\+H5\+Abstract\+Ds.h}
\label{visual__studio_2_h_d_f5_21_810_81_2include_2_h5_abstract_ds_8h_source}\index{H5\+Abstract\+Ds.\+h@{H5\+Abstract\+Ds.\+h}}

\begin{DoxyCode}
00001 \textcolor{comment}{// C++ informative line for the emacs editor: -*- C++ -*-}
00002 \textcolor{comment}{/* * * * * * * * * * * * * * * * * * * * * * * * * * * * * * * * * * * * * * *}
00003 \textcolor{comment}{ * Copyright by The HDF Group.                                               *}
00004 \textcolor{comment}{ * Copyright by the Board of Trustees of the University of Illinois.         *}
00005 \textcolor{comment}{ * All rights reserved.                                                      *}
00006 \textcolor{comment}{ *                                                                           *}
00007 \textcolor{comment}{ * This file is part of HDF5.  The full HDF5 copyright notice, including     *}
00008 \textcolor{comment}{ * terms governing use, modification, and redistribution, is contained in    *}
00009 \textcolor{comment}{ * the COPYING file, which can be found at the root of the source code       *}
00010 \textcolor{comment}{ * distribution tree, or in https://support.hdfgroup.org/ftp/HDF5/releases.  *}
00011 \textcolor{comment}{ * If you do not have access to either file, you may request a copy from     *}
00012 \textcolor{comment}{ * help@hdfgroup.org.                                                        *}
00013 \textcolor{comment}{ * * * * * * * * * * * * * * * * * * * * * * * * * * * * * * * * * * * * * * */}
00014 
00015 \textcolor{preprocessor}{#ifndef \_\_AbstractDs\_H}
00016 \textcolor{preprocessor}{#define \_\_AbstractDs\_H}
00017 
00018 \textcolor{keyword}{namespace }\hyperlink{namespace_h5}{H5} \{
00019 
00020 \textcolor{keyword}{class }ArrayType;
00021 \textcolor{keyword}{class }CompType;
00022 \textcolor{keyword}{class }EnumType;
00023 \textcolor{keyword}{class }FloatType;
00024 \textcolor{keyword}{class }IntType;
00025 \textcolor{keyword}{class }StrType;
00026 \textcolor{keyword}{class }VarLenType;
00027 \textcolor{keyword}{class }DataSpace;
00028 
00036 \textcolor{keyword}{class }H5\_DLLCPP AbstractDs \{
00037    \textcolor{keyword}{public}:
00038         \textcolor{comment}{// Gets a copy the datatype of that this abstract dataset uses.}
00039         \textcolor{comment}{// Note that this datatype is a generic one and can only be accessed}
00040         \textcolor{comment}{// via generic member functions, i.e., member functions belong}
00041         \textcolor{comment}{// to DataType.  To get specific datatype, i.e. EnumType, FloatType,}
00042         \textcolor{comment}{// etc..., use the specific functions, that follow, instead.}
00043         DataType getDataType() \textcolor{keyword}{const};
00044 
00045         \textcolor{comment}{// Gets a copy of the specific datatype of this abstract dataset.}
00046         ArrayType getArrayType() \textcolor{keyword}{const};
00047         CompType getCompType() \textcolor{keyword}{const};
00048         EnumType getEnumType() \textcolor{keyword}{const};
00049         IntType getIntType() \textcolor{keyword}{const};
00050         FloatType getFloatType() \textcolor{keyword}{const};
00051         StrType getStrType() \textcolor{keyword}{const};
00052         VarLenType getVarLenType() \textcolor{keyword}{const};
00053 
00055         \textcolor{keyword}{virtual} \textcolor{keywordtype}{size\_t} getInMemDataSize() \textcolor{keyword}{const} = 0;
00056 
00058         \textcolor{keyword}{virtual} DataSpace getSpace() \textcolor{keyword}{const} = 0;
00059 
00060         \textcolor{comment}{// Gets the class of the datatype that is used by this abstract}
00061         \textcolor{comment}{// dataset.}
00062         H5T\_class\_t getTypeClass() \textcolor{keyword}{const};
00063 
00065         \textcolor{keyword}{virtual} hsize\_t getStorageSize() \textcolor{keyword}{const} = 0;
00066 
00067         \textcolor{comment}{// Returns this class name - pure virtual.}
00068         \textcolor{keyword}{virtual} H5std\_string fromClass() \textcolor{keyword}{const} = 0;
00069 
00070         \textcolor{comment}{// Destructor}
00071         \textcolor{keyword}{virtual} ~AbstractDs();
00072 
00073    \textcolor{keyword}{protected}:
00074         \textcolor{comment}{// Default constructor}
00075         AbstractDs();
00076 
00077         \textcolor{comment}{// *** Deprecation warning ***}
00078         \textcolor{comment}{// The following two constructors are no longer appropriate after the}
00079         \textcolor{comment}{// data member "id" had been moved to the sub-classes.}
00080         \textcolor{comment}{// The copy constructor is a noop and is removed in 1.8.15 and the}
00081         \textcolor{comment}{// other will be removed from 1.10 release, and then from 1.8 if its}
00082         \textcolor{comment}{// removal does not raise any problems in two 1.10 releases.}
00083 
00084         \textcolor{comment}{// Mar 2016 -BMR, AbstractDs(const hid\_t h5\_id);}
00085 
00086         \textcolor{comment}{// Copy constructor}
00087         \textcolor{comment}{// AbstractDs( const AbstractDs& original );}
00088 
00089    \textcolor{keyword}{private}:
00090         \textcolor{comment}{// This member function is implemented by DataSet and Attribute - pure virtual.}
00091         \textcolor{keyword}{virtual} hid\_t p\_get\_type() \textcolor{keyword}{const} = 0;
00092 
00093 \}; \textcolor{comment}{// end of AbstractDs}
00094 \} \textcolor{comment}{// namespace H5}
00095 
00096 \textcolor{preprocessor}{#endif // \_\_AbstractDs\_H}
\end{DoxyCode}
