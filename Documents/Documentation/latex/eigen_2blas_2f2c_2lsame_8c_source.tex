\hypertarget{eigen_2blas_2f2c_2lsame_8c_source}{}\section{eigen/blas/f2c/lsame.c}
\label{eigen_2blas_2f2c_2lsame_8c_source}\index{lsame.\+c@{lsame.\+c}}

\begin{DoxyCode}
00001 \textcolor{comment}{/* lsame.f -- translated by f2c (version 20100827).}
00002 \textcolor{comment}{   You must link the resulting object file with libf2c:}
00003 \textcolor{comment}{    on Microsoft Windows system, link with libf2c.lib;}
00004 \textcolor{comment}{    on Linux or Unix systems, link with .../path/to/libf2c.a -lm}
00005 \textcolor{comment}{    or, if you install libf2c.a in a standard place, with -lf2c -lm}
00006 \textcolor{comment}{    -- in that order, at the end of the command line, as in}
00007 \textcolor{comment}{        cc *.o -lf2c -lm}
00008 \textcolor{comment}{    Source for libf2c is in /netlib/f2c/libf2c.zip, e.g.,}
00009 \textcolor{comment}{}
00010 \textcolor{comment}{        http://www.netlib.org/f2c/libf2c.zip}
00011 \textcolor{comment}{*/}
00012 
00013 \textcolor{preprocessor}{#include "datatypes.h"}
00014 
00015 logical lsame\_(\textcolor{keywordtype}{char} *ca, \textcolor{keywordtype}{char} *cb, ftnlen ca\_len, ftnlen cb\_len)
00016 \{
00017     \textcolor{comment}{/* System generated locals */}
00018     logical ret\_val;
00019 
00020     \textcolor{comment}{/* Local variables */}
00021     integer inta, intb, zcode;
00022 
00023 
00024 \textcolor{comment}{/*  -- LAPACK auxiliary routine (version 3.1) -- */}
00025 \textcolor{comment}{/*     Univ. of Tennessee, Univ. of California Berkeley and NAG Ltd.. */}
00026 \textcolor{comment}{/*     November 2006 */}
00027 
00028 \textcolor{comment}{/*     .. Scalar Arguments .. */}
00029 \textcolor{comment}{/*     .. */}
00030 
00031 \textcolor{comment}{/*  Purpose */}
00032 \textcolor{comment}{/*  ======= */}
00033 
00034 \textcolor{comment}{/*  LSAME returns .TRUE. if CA is the same letter as CB regardless of */}
00035 \textcolor{comment}{/*  case. */}
00036 
00037 \textcolor{comment}{/*  Arguments */}
00038 \textcolor{comment}{/*  ========= */}
00039 
00040 \textcolor{comment}{/*  CA      (input) CHARACTER*1 */}
00041 
00042 \textcolor{comment}{/*  CB      (input) CHARACTER*1 */}
00043 \textcolor{comment}{/*          CA and CB specify the single characters to be compared. */}
00044 
00045 \textcolor{comment}{/* ===================================================================== */}
00046 
00047 \textcolor{comment}{/*     .. Intrinsic Functions .. */}
00048 \textcolor{comment}{/*     .. */}
00049 \textcolor{comment}{/*     .. Local Scalars .. */}
00050 \textcolor{comment}{/*     .. */}
00051 
00052 \textcolor{comment}{/*     Test if the characters are equal */}
00053 
00054     ret\_val = *(\textcolor{keywordtype}{unsigned} \textcolor{keywordtype}{char} *)ca == *(\textcolor{keywordtype}{unsigned} \textcolor{keywordtype}{char} *)cb;
00055     \textcolor{keywordflow}{if} (ret\_val) \{
00056     \textcolor{keywordflow}{return} ret\_val;
00057     \}
00058 
00059 \textcolor{comment}{/*     Now test for equivalence if both characters are alphabetic. */}
00060 
00061     zcode = \textcolor{charliteral}{'Z'};
00062 
00063 \textcolor{comment}{/*     Use 'Z' rather than 'A' so that ASCII can be detected on Prime */}
00064 \textcolor{comment}{/*     machines, on which ICHAR returns a value with bit 8 set. */}
00065 \textcolor{comment}{/*     ICHAR('A') on Prime machines returns 193 which is the same as */}
00066 \textcolor{comment}{/*     ICHAR('A') on an EBCDIC machine. */}
00067 
00068     inta = *(\textcolor{keywordtype}{unsigned} \textcolor{keywordtype}{char} *)ca;
00069     intb = *(\textcolor{keywordtype}{unsigned} \textcolor{keywordtype}{char} *)cb;
00070 
00071     \textcolor{keywordflow}{if} (zcode == 90 || zcode == 122) \{
00072 
00073 \textcolor{comment}{/*        ASCII is assumed - ZCODE is the ASCII code of either lower or */}
00074 \textcolor{comment}{/*        upper case 'Z'. */}
00075 
00076     \textcolor{keywordflow}{if} (inta >= 97 && inta <= 122) \{
00077         inta += -32;
00078     \}
00079     \textcolor{keywordflow}{if} (intb >= 97 && intb <= 122) \{
00080         intb += -32;
00081     \}
00082 
00083     \} \textcolor{keywordflow}{else} \textcolor{keywordflow}{if} (zcode == 233 || zcode == 169) \{
00084 
00085 \textcolor{comment}{/*        EBCDIC is assumed - ZCODE is the EBCDIC code of either lower or */}
00086 \textcolor{comment}{/*        upper case 'Z'. */}
00087 
00088     \textcolor{keywordflow}{if} ((inta >= 129 && inta <= 137) || (inta >= 145 && inta <= 153) || 
00089             (inta >= 162 && inta <= 169)) \{
00090         inta += 64;
00091     \}
00092     \textcolor{keywordflow}{if} ((intb >= 129 && intb <= 137) || (intb >= 145 && intb <= 153) || 
00093             (intb >= 162 && intb <= 169)) \{
00094         intb += 64;
00095     \}
00096 
00097     \} \textcolor{keywordflow}{else} \textcolor{keywordflow}{if} (zcode == 218 || zcode == 250) \{
00098 
00099 \textcolor{comment}{/*        ASCII is assumed, on Prime machines - ZCODE is the ASCII code */}
00100 \textcolor{comment}{/*        plus 128 of either lower or upper case 'Z'. */}
00101 
00102     \textcolor{keywordflow}{if} (inta >= 225 && inta <= 250) \{
00103         inta += -32;
00104     \}
00105     \textcolor{keywordflow}{if} (intb >= 225 && intb <= 250) \{
00106         intb += -32;
00107     \}
00108     \}
00109     ret\_val = inta == intb;
00110 
00111 \textcolor{comment}{/*     RETURN */}
00112 
00113 \textcolor{comment}{/*     End of LSAME */}
00114 
00115     \textcolor{keywordflow}{return} ret\_val;
00116 \} \textcolor{comment}{/* lsame\_ */}
00117 
\end{DoxyCode}
