\hypertarget{_h_d_f5_21_810_81_2_h_d_f5_examples_2_f_o_r_t_r_a_n_2_h5_t_2h5ex__t__string_c___f03_8f90_source}{}\section{H\+D\+F5/1.10.1/\+H\+D\+F5\+Examples/\+F\+O\+R\+T\+R\+A\+N/\+H5\+T/h5ex\+\_\+t\+\_\+string\+C\+\_\+\+F03.f90}
\label{_h_d_f5_21_810_81_2_h_d_f5_examples_2_f_o_r_t_r_a_n_2_h5_t_2h5ex__t__string_c___f03_8f90_source}\index{h5ex\+\_\+t\+\_\+string\+C\+\_\+\+F03.\+f90@{h5ex\+\_\+t\+\_\+string\+C\+\_\+\+F03.\+f90}}

\begin{DoxyCode}
00001 \textcolor{comment}{!************************************************************}
00002 \textcolor{comment}{!}
00003 \textcolor{comment}{!  This example shows how to write a C string to a dataset }
00004 \textcolor{comment}{!  and read it back as a Fortran string.}
00005 \textcolor{comment}{!  The program first writes the C }
00006 \textcolor{comment}{!  strings to a dataset with a dataspace of DIM0, then closes the file.}
00007 \textcolor{comment}{!  Next, it reopens the file, reads back the data into a Fortran}
00008 \textcolor{comment}{!  fixed character string, and outputs it to the screen.}
00009 \textcolor{comment}{!}
00010 \textcolor{comment}{!  This file is intended for use with HDF5 Library version 1.8}
00011 \textcolor{comment}{!  with --enable-fortran2003 }
00012 \textcolor{comment}{!}
00013 \textcolor{comment}{! ************************************************************/}
00014 \textcolor{keyword}{PROGRAM} main
00015 
00016   \textcolor{keywordtype}{USE }hdf5
00017   \textcolor{keywordtype}{USE }iso\_c\_binding
00018 
00019   \textcolor{keywordtype}{IMPLICIT NONE}
00020 
00021   \textcolor{keywordtype}{CHARACTER(LEN=21)}, \textcolor{keywordtype}{PARAMETER} :: filename  = \textcolor{stringliteral}{"h5ex\_t\_stringC\_F03.h5"}
00022   \textcolor{keywordtype}{CHARACTER(LEN=3)} , \textcolor{keywordtype}{PARAMETER} :: dataset   = \textcolor{stringliteral}{"DS1"}
00023   \textcolor{keywordtype}{INTEGER}          , \textcolor{keywordtype}{PARAMETER} :: dim0      = 4
00024   \textcolor{keywordtype}{INTEGER(SIZE\_T)}  , \textcolor{keywordtype}{PARAMETER} :: sdim      = 7
00025 
00026   \textcolor{keywordtype}{INTEGER(HID\_T)}  :: \hyperlink{structfile}{file}, filetype, memtype, space, dset \textcolor{comment}{! Handles}
00027   \textcolor{keywordtype}{INTEGER} :: hdferr
00028 
00029   \textcolor{keywordtype}{INTEGER(HSIZE\_T)}, \textcolor{keywordtype}{DIMENSION(1:1)} :: dims = (/dim0/)
00030   \textcolor{keywordtype}{INTEGER(HSIZE\_T)}, \textcolor{keywordtype}{DIMENSION(1:1)} :: maxdims
00031 
00032   \textcolor{keywordtype}{CHARACTER(LEN=sdim)}, \textcolor{keywordtype}{DIMENSION(1:dim0)}, \textcolor{keywordtype}{TARGET} :: &
00033        wdata = (/\textcolor{stringliteral}{"Parting"}, \textcolor{stringliteral}{"is such"}, \textcolor{stringliteral}{"sweet  "}, \textcolor{stringliteral}{"sorrow."}/)
00034   \textcolor{keywordtype}{CHARACTER(LEN=sdim)}, \textcolor{keywordtype}{DIMENSION(:)}, \textcolor{keywordtype}{ALLOCATABLE}, \textcolor{keywordtype}{TARGET} :: rdata
00035   \textcolor{keywordtype}{INTEGER} :: i
00036   \textcolor{keywordtype}{INTEGER(SIZE\_T)} :: size
00037   \textcolor{keywordtype}{TYPE}(c\_ptr) :: f\_ptr
00038   \textcolor{comment}{!}
00039   \textcolor{comment}{! Initialize FORTRAN interface.}
00040   \textcolor{comment}{!}
00041   \textcolor{keyword}{CALL }h5open\_f(hdferr)
00042   \textcolor{comment}{!}
00043   \textcolor{comment}{! Create a new file using the default properties.}
00044   \textcolor{comment}{!}
00045   \textcolor{keyword}{CALL }h5fcreate\_f(filename, h5f\_acc\_trunc\_f, \hyperlink{structfile}{file}, hdferr)
00046 
00047   \textcolor{comment}{!}
00048   \textcolor{comment}{! Create file and memory datatypes.  For this example we will save}
00049   \textcolor{comment}{! the strings as C strings}
00050   \textcolor{comment}{!}
00051   \textcolor{comment}{! Include the NULL TERMINATION of string in C (i.e. add +1 to the length)}
00052 
00053   \textcolor{keyword}{CALL }h5tcopy\_f(h5t\_c\_s1, filetype, hdferr)
00054   \textcolor{keyword}{CALL }h5tset\_size\_f(filetype, sdim+1, hdferr)
00055 
00056   \textcolor{keyword}{CALL }h5tcopy\_f( h5t\_fortran\_s1, memtype, hdferr)
00057   \textcolor{keyword}{CALL }h5tset\_size\_f(memtype, sdim, hdferr)
00058   \textcolor{comment}{!}
00059   \textcolor{comment}{! Create dataspace.}
00060   \textcolor{comment}{!}
00061   \textcolor{keyword}{CALL }h5screate\_simple\_f(1, dims, space, hdferr)
00062   \textcolor{comment}{!}
00063   \textcolor{comment}{! Create the dataset and write the string data to it.}
00064   \textcolor{comment}{!}
00065   \textcolor{keyword}{CALL }h5dcreate\_f(\hyperlink{structfile}{file}, dataset, filetype, space, dset, hdferr)
00066 
00067   f\_ptr = c\_loc(wdata(1)(1:1))
00068   \textcolor{keyword}{CALL }h5dwrite\_f(dset, memtype, f\_ptr, hdferr)
00069 
00070   \textcolor{comment}{!}
00071   \textcolor{comment}{! Close and release resources.}
00072   \textcolor{comment}{!}
00073   \textcolor{keyword}{CALL }h5dclose\_f(dset , hdferr)
00074   \textcolor{keyword}{CALL }h5sclose\_f(space, hdferr)
00075   \textcolor{keyword}{CALL }h5tclose\_f(filetype, hdferr)
00076   \textcolor{keyword}{CALL }h5tclose\_f(memtype, hdferr)
00077   \textcolor{keyword}{CALL }h5fclose\_f(\hyperlink{structfile}{file} , hdferr)
00078   \textcolor{comment}{!}
00079   \textcolor{comment}{! Now we begin the read section of this example.}
00080   \textcolor{comment}{!}
00081   \textcolor{comment}{! Open file and dataset.}
00082   \textcolor{comment}{!}
00083   \textcolor{keyword}{CALL }h5fopen\_f(filename, h5f\_acc\_rdonly\_f, \hyperlink{structfile}{file}, hdferr)
00084   \textcolor{keyword}{CALL }h5dopen\_f(\hyperlink{structfile}{file}, dataset, dset, hdferr)
00085   \textcolor{comment}{!}
00086   \textcolor{comment}{! Get the datatype and its size.}
00087   \textcolor{comment}{!}
00088   \textcolor{keyword}{CALL }h5dget\_type\_f(dset, filetype, hdferr)
00089   \textcolor{keyword}{CALL }h5tget\_size\_f(filetype, \textcolor{keyword}{size}, hdferr)
00090 
00091   \textcolor{comment}{! Make sure the declared length is large enough,}
00092   \textcolor{comment}{! the C string contains the null character.}
00093   \textcolor{keywordflow}{IF}(size.GT.sdim+1)\textcolor{keywordflow}{THEN}
00094      print*,\textcolor{stringliteral}{'ERROR: Character LEN is to small'}
00095      stop
00096 \textcolor{keywordflow}{  ENDIF}
00097 
00098   \textcolor{comment}{! Get dataspace.}
00099   \textcolor{comment}{!}
00100   \textcolor{keyword}{CALL }h5dget\_space\_f(dset, space, hdferr)
00101   \textcolor{keyword}{CALL }h5sget\_simple\_extent\_dims\_f(space, dims, maxdims, hdferr)
00102 
00103   \textcolor{keyword}{ALLOCATE}(rdata(1:dims(1)))
00104   \textcolor{comment}{!}
00105   \textcolor{comment}{! Create the memory datatype.}
00106   \textcolor{comment}{!}
00107   \textcolor{keyword}{CALL }h5tcopy\_f(h5t\_fortran\_s1, memtype, hdferr)
00108   \textcolor{keyword}{CALL }h5tset\_size\_f(memtype, sdim, hdferr)
00109   \textcolor{comment}{!}
00110   \textcolor{comment}{! Read the data.}
00111   \textcolor{comment}{!}
00112   f\_ptr = c\_loc(rdata(1)(1:1))
00113   \textcolor{keyword}{CALL }h5dread\_f(dset, memtype, f\_ptr, hdferr, space)
00114   \textcolor{comment}{!}
00115   \textcolor{comment}{! Output the data to the screen.}
00116   \textcolor{comment}{!}
00117   \textcolor{keywordflow}{DO} i = 1, dims(1)
00118      \textcolor{keyword}{WRITE}(*,\textcolor{stringliteral}{'(A,"(",I0,"): ", A)'}) dataset, i, rdata(i)
00119 \textcolor{keywordflow}{  END DO}
00120   \textcolor{comment}{!}
00121   \textcolor{comment}{! Close and release resources.}
00122   \textcolor{comment}{!}
00123   \textcolor{keyword}{CALL }h5dclose\_f(dset, hdferr)
00124   \textcolor{keyword}{CALL }h5sclose\_f(space, hdferr)
00125   \textcolor{keyword}{CALL }h5tclose\_f(filetype, hdferr)
00126   \textcolor{keyword}{CALL }h5tclose\_f(memtype, hdferr)
00127   \textcolor{keyword}{CALL }h5fclose\_f(\hyperlink{structfile}{file}, hdferr)
00128 
00129   \textcolor{keyword}{DEALLOCATE}(rdata)
00130 
00131 
00132 \textcolor{keyword}{END PROGRAM }main
\end{DoxyCode}
