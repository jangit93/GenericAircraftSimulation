\hypertarget{_h_d_f5_21_810_81_2_h_d_f5_examples_2_c_2_h5_t_2h5ex__t__vlstringatt_8c_source}{}\section{H\+D\+F5/1.10.1/\+H\+D\+F5\+Examples/\+C/\+H5\+T/h5ex\+\_\+t\+\_\+vlstringatt.c}
\label{_h_d_f5_21_810_81_2_h_d_f5_examples_2_c_2_h5_t_2h5ex__t__vlstringatt_8c_source}\index{h5ex\+\_\+t\+\_\+vlstringatt.\+c@{h5ex\+\_\+t\+\_\+vlstringatt.\+c}}

\begin{DoxyCode}
00001 \textcolor{comment}{/************************************************************}
00002 \textcolor{comment}{}
00003 \textcolor{comment}{  This example shows how to read and write variable-length}
00004 \textcolor{comment}{  string datatypes to an attribute.  The program first}
00005 \textcolor{comment}{  writes variable-length strings to an attribute with a}
00006 \textcolor{comment}{  dataspace of DIM0, then closes the file.  Next, it reopens}
00007 \textcolor{comment}{  the file, reads back the data, and outputs it to the}
00008 \textcolor{comment}{  screen.}
00009 \textcolor{comment}{}
00010 \textcolor{comment}{  This file is intended for use with HDF5 Library version 1.8}
00011 \textcolor{comment}{}
00012 \textcolor{comment}{ ************************************************************/}
00013 
00014 \textcolor{preprocessor}{#include "hdf5.h"}
00015 \textcolor{preprocessor}{#include <stdio.h>}
00016 \textcolor{preprocessor}{#include <stdlib.h>}
00017 
00018 \textcolor{preprocessor}{#define FILE            "h5ex\_t\_vlstringatt.h5"}
00019 \textcolor{preprocessor}{#define DATASET         "DS1"}
00020 \textcolor{preprocessor}{#define ATTRIBUTE       "A1"}
00021 \textcolor{preprocessor}{#define DIM0            4}
00022 
00023 \textcolor{keywordtype}{int}
00024 main (\textcolor{keywordtype}{void})
00025 \{
00026     hid\_t       \hyperlink{structfile}{file}, filetype, memtype, space, dset, attr;
00027                                             \textcolor{comment}{/* Handles */}
00028     herr\_t      status;
00029     hsize\_t     dims[1] = \{DIM0\};
00030     \textcolor{keywordtype}{char}        *wdata[DIM0] = \{\textcolor{stringliteral}{"Parting"}, \textcolor{stringliteral}{"is such"}, \textcolor{stringliteral}{"sweet"}, \textcolor{stringliteral}{"sorrow."}\},
00031                                             \textcolor{comment}{/* Write buffer */}
00032                 **rdata;                    \textcolor{comment}{/* Read buffer */}
00033     \textcolor{keywordtype}{int}         ndims,
00034                 i;
00035 
00036     \textcolor{comment}{/*}
00037 \textcolor{comment}{     * Create a new file using the default properties.}
00038 \textcolor{comment}{     */}
00039     file = H5Fcreate (FILE, H5F\_ACC\_TRUNC, H5P\_DEFAULT, H5P\_DEFAULT);
00040 
00041     \textcolor{comment}{/*}
00042 \textcolor{comment}{     * Create file and memory datatypes.  For this example we will save}
00043 \textcolor{comment}{     * the strings as FORTRAN strings.}
00044 \textcolor{comment}{     */}
00045     filetype = H5Tcopy (H5T\_FORTRAN\_S1);
00046     status = H5Tset\_size (filetype, H5T\_VARIABLE);
00047     memtype = H5Tcopy (H5T\_C\_S1);
00048     status = H5Tset\_size (memtype, H5T\_VARIABLE);
00049 
00050     \textcolor{comment}{/*}
00051 \textcolor{comment}{     * Create dataset with a null dataspace.}
00052 \textcolor{comment}{     */}
00053     space = H5Screate (H5S\_NULL);
00054     dset = H5Dcreate (file, DATASET, H5T\_STD\_I32LE, space, H5P\_DEFAULT,
00055                 H5P\_DEFAULT, H5P\_DEFAULT);
00056     status = H5Sclose (space);
00057 
00058     \textcolor{comment}{/*}
00059 \textcolor{comment}{     * Create dataspace.  Setting maximum size to NULL sets the maximum}
00060 \textcolor{comment}{     * size to be the current size.}
00061 \textcolor{comment}{     */}
00062     space = H5Screate\_simple (1, dims, NULL);
00063 
00064     \textcolor{comment}{/*}
00065 \textcolor{comment}{     * Create the attribute and write the variable-length string data}
00066 \textcolor{comment}{     * to it.}
00067 \textcolor{comment}{     */}
00068     attr = H5Acreate (dset, ATTRIBUTE, filetype, space, H5P\_DEFAULT,
00069                 H5P\_DEFAULT);
00070     status = H5Awrite (attr, memtype, wdata);
00071 
00072     \textcolor{comment}{/*}
00073 \textcolor{comment}{     * Close and release resources.}
00074 \textcolor{comment}{     */}
00075     status = H5Aclose (attr);
00076     status = H5Dclose (dset);
00077     status = H5Sclose (space);
00078     status = H5Tclose (filetype);
00079     status = H5Tclose (memtype);
00080     status = H5Fclose (file);
00081 
00082 
00083     \textcolor{comment}{/*}
00084 \textcolor{comment}{     * Now we begin the read section of this example.  Here we assume}
00085 \textcolor{comment}{     * the attribute has the same name and rank, but can have any size.}
00086 \textcolor{comment}{     * Therefore we must allocate a new array to read in data using}
00087 \textcolor{comment}{     * malloc().}
00088 \textcolor{comment}{     */}
00089 
00090     \textcolor{comment}{/*}
00091 \textcolor{comment}{     * Open file, dataset, and attribute.}
00092 \textcolor{comment}{     */}
00093     file = H5Fopen (FILE, H5F\_ACC\_RDONLY, H5P\_DEFAULT);
00094     dset = H5Dopen (file, DATASET, H5P\_DEFAULT);
00095     attr = H5Aopen (dset, ATTRIBUTE, H5P\_DEFAULT);
00096 
00097     \textcolor{comment}{/*}
00098 \textcolor{comment}{     * Get the datatype.}
00099 \textcolor{comment}{     */}
00100     filetype = H5Aget\_type (attr);
00101 
00102     \textcolor{comment}{/*}
00103 \textcolor{comment}{     * Get dataspace and allocate memory for read buffer.}
00104 \textcolor{comment}{     */}
00105     space = H5Aget\_space (attr);
00106     ndims = H5Sget\_simple\_extent\_dims (space, dims, NULL);
00107     rdata = (\textcolor{keywordtype}{char} **) malloc (dims[0] * \textcolor{keyword}{sizeof} (\textcolor{keywordtype}{char} *));
00108 
00109     \textcolor{comment}{/*}
00110 \textcolor{comment}{     * Create the memory datatype.}
00111 \textcolor{comment}{     */}
00112     memtype = H5Tcopy (H5T\_C\_S1);
00113     status = H5Tset\_size (memtype, H5T\_VARIABLE);
00114 
00115     \textcolor{comment}{/*}
00116 \textcolor{comment}{     * Read the data.}
00117 \textcolor{comment}{     */}
00118     status = H5Aread (attr, memtype, rdata);
00119 
00120     \textcolor{comment}{/*}
00121 \textcolor{comment}{     * Output the data to the screen.}
00122 \textcolor{comment}{     */}
00123     \textcolor{keywordflow}{for} (i=0; i<dims[0]; i++)
00124         printf (\textcolor{stringliteral}{"%s[%d]: %s\(\backslash\)n"}, ATTRIBUTE, i, rdata[i]);
00125 
00126     \textcolor{comment}{/*}
00127 \textcolor{comment}{     * Close and release resources.  Note that H5Dvlen\_reclaim works}
00128 \textcolor{comment}{     * for variable-length strings as well as variable-length arrays.}
00129 \textcolor{comment}{     * Also note that we must still free the array of pointers stored}
00130 \textcolor{comment}{     * in rdata, as H5Tvlen\_reclaim only frees the data these point to.}
00131 \textcolor{comment}{     */}
00132     status = H5Dvlen\_reclaim (memtype, space, H5P\_DEFAULT, rdata);
00133     free (rdata);
00134     status = H5Aclose (attr);
00135     status = H5Dclose (dset);
00136     status = H5Sclose (space);
00137     status = H5Tclose (filetype);
00138     status = H5Tclose (memtype);
00139     status = H5Fclose (file);
00140 
00141     \textcolor{keywordflow}{return} 0;
00142 \}
\end{DoxyCode}
