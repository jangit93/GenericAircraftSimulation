\hypertarget{eigen_2blas_2f2c_2ssbmv_8c_source}{}\section{eigen/blas/f2c/ssbmv.c}
\label{eigen_2blas_2f2c_2ssbmv_8c_source}\index{ssbmv.\+c@{ssbmv.\+c}}

\begin{DoxyCode}
00001 \textcolor{comment}{/* ssbmv.f -- translated by f2c (version 20100827).}
00002 \textcolor{comment}{   You must link the resulting object file with libf2c:}
00003 \textcolor{comment}{    on Microsoft Windows system, link with libf2c.lib;}
00004 \textcolor{comment}{    on Linux or Unix systems, link with .../path/to/libf2c.a -lm}
00005 \textcolor{comment}{    or, if you install libf2c.a in a standard place, with -lf2c -lm}
00006 \textcolor{comment}{    -- in that order, at the end of the command line, as in}
00007 \textcolor{comment}{        cc *.o -lf2c -lm}
00008 \textcolor{comment}{    Source for libf2c is in /netlib/f2c/libf2c.zip, e.g.,}
00009 \textcolor{comment}{}
00010 \textcolor{comment}{        http://www.netlib.org/f2c/libf2c.zip}
00011 \textcolor{comment}{*/}
00012 
00013 \textcolor{preprocessor}{#include "datatypes.h"}
00014 
00015 \textcolor{comment}{/* Subroutine */} \textcolor{keywordtype}{int} ssbmv\_(\textcolor{keywordtype}{char} *uplo, integer *n, integer *k, real *alpha, 
00016     real *a, integer *lda, real *x, integer *incx, real *beta, real *y, 
00017     integer *incy, ftnlen uplo\_len)
00018 \{
00019     \textcolor{comment}{/* System generated locals */}
00020     integer a\_dim1, a\_offset, i\_\_1, i\_\_2, i\_\_3, i\_\_4;
00021 
00022     \textcolor{comment}{/* Local variables */}
00023     integer i\_\_, j, l, ix, iy, jx, jy, kx, ky, info;
00024     real temp1, temp2;
00025     \textcolor{keyword}{extern} logical lsame\_(\textcolor{keywordtype}{char} *, \textcolor{keywordtype}{char} *, ftnlen, ftnlen);
00026     integer kplus1;
00027     \textcolor{keyword}{extern} \textcolor{comment}{/* Subroutine */} \textcolor{keywordtype}{int} xerbla\_(\textcolor{keywordtype}{char} *, integer *, ftnlen);
00028 
00029 \textcolor{comment}{/*     .. Scalar Arguments .. */}
00030 \textcolor{comment}{/*     .. */}
00031 \textcolor{comment}{/*     .. Array Arguments .. */}
00032 \textcolor{comment}{/*     .. */}
00033 
00034 \textcolor{comment}{/*  Purpose */}
00035 \textcolor{comment}{/*  ======= */}
00036 
00037 \textcolor{comment}{/*  SSBMV  performs the matrix-vector  operation */}
00038 
00039 \textcolor{comment}{/*     y := alpha*A*x + beta*y, */}
00040 
00041 \textcolor{comment}{/*  where alpha and beta are scalars, x and y are n element vectors and */}
00042 \textcolor{comment}{/*  A is an n by n symmetric band matrix, with k super-diagonals. */}
00043 
00044 \textcolor{comment}{/*  Arguments */}
00045 \textcolor{comment}{/*  ========== */}
00046 
00047 \textcolor{comment}{/*  UPLO   - CHARACTER*1. */}
00048 \textcolor{comment}{/*           On entry, UPLO specifies whether the upper or lower */}
00049 \textcolor{comment}{/*           triangular part of the band matrix A is being supplied as */}
00050 \textcolor{comment}{/*           follows: */}
00051 
00052 \textcolor{comment}{/*              UPLO = 'U' or 'u'   The upper triangular part of A is */}
00053 \textcolor{comment}{/*                                  being supplied. */}
00054 
00055 \textcolor{comment}{/*              UPLO = 'L' or 'l'   The lower triangular part of A is */}
00056 \textcolor{comment}{/*                                  being supplied. */}
00057 
00058 \textcolor{comment}{/*           Unchanged on exit. */}
00059 
00060 \textcolor{comment}{/*  N      - INTEGER. */}
00061 \textcolor{comment}{/*           On entry, N specifies the order of the matrix A. */}
00062 \textcolor{comment}{/*           N must be at least zero. */}
00063 \textcolor{comment}{/*           Unchanged on exit. */}
00064 
00065 \textcolor{comment}{/*  K      - INTEGER. */}
00066 \textcolor{comment}{/*           On entry, K specifies the number of super-diagonals of the */}
00067 \textcolor{comment}{/*           matrix A. K must satisfy  0 .le. K. */}
00068 \textcolor{comment}{/*           Unchanged on exit. */}
00069 
00070 \textcolor{comment}{/*  ALPHA  - REAL            . */}
00071 \textcolor{comment}{/*           On entry, ALPHA specifies the scalar alpha. */}
00072 \textcolor{comment}{/*           Unchanged on exit. */}
00073 
00074 \textcolor{comment}{/*  A      - REAL             array of DIMENSION ( LDA, n ). */}
00075 \textcolor{comment}{/*           Before entry with UPLO = 'U' or 'u', the leading ( k + 1 ) */}
00076 \textcolor{comment}{/*           by n part of the array A must contain the upper triangular */}
00077 \textcolor{comment}{/*           band part of the symmetric matrix, supplied column by */}
00078 \textcolor{comment}{/*           column, with the leading diagonal of the matrix in row */}
00079 \textcolor{comment}{/*           ( k + 1 ) of the array, the first super-diagonal starting at */}
00080 \textcolor{comment}{/*           position 2 in row k, and so on. The top left k by k triangle */}
00081 \textcolor{comment}{/*           of the array A is not referenced. */}
00082 \textcolor{comment}{/*           The following program segment will transfer the upper */}
00083 \textcolor{comment}{/*           triangular part of a symmetric band matrix from conventional */}
00084 \textcolor{comment}{/*           full matrix storage to band storage: */}
00085 
00086 \textcolor{comment}{/*                 DO 20, J = 1, N */}
00087 \textcolor{comment}{/*                    M = K + 1 - J */}
00088 \textcolor{comment}{/*                    DO 10, I = MAX( 1, J - K ), J */}
00089 \textcolor{comment}{/*                       A( M + I, J ) = matrix( I, J ) */}
00090 \textcolor{comment}{/*              10    CONTINUE */}
00091 \textcolor{comment}{/*              20 CONTINUE */}
00092 
00093 \textcolor{comment}{/*           Before entry with UPLO = 'L' or 'l', the leading ( k + 1 ) */}
00094 \textcolor{comment}{/*           by n part of the array A must contain the lower triangular */}
00095 \textcolor{comment}{/*           band part of the symmetric matrix, supplied column by */}
00096 \textcolor{comment}{/*           column, with the leading diagonal of the matrix in row 1 of */}
00097 \textcolor{comment}{/*           the array, the first sub-diagonal starting at position 1 in */}
00098 \textcolor{comment}{/*           row 2, and so on. The bottom right k by k triangle of the */}
00099 \textcolor{comment}{/*           array A is not referenced. */}
00100 \textcolor{comment}{/*           The following program segment will transfer the lower */}
00101 \textcolor{comment}{/*           triangular part of a symmetric band matrix from conventional */}
00102 \textcolor{comment}{/*           full matrix storage to band storage: */}
00103 
00104 \textcolor{comment}{/*                 DO 20, J = 1, N */}
00105 \textcolor{comment}{/*                    M = 1 - J */}
00106 \textcolor{comment}{/*                    DO 10, I = J, MIN( N, J + K ) */}
00107 \textcolor{comment}{/*                       A( M + I, J ) = matrix( I, J ) */}
00108 \textcolor{comment}{/*              10    CONTINUE */}
00109 \textcolor{comment}{/*              20 CONTINUE */}
00110 
00111 \textcolor{comment}{/*           Unchanged on exit. */}
00112 
00113 \textcolor{comment}{/*  LDA    - INTEGER. */}
00114 \textcolor{comment}{/*           On entry, LDA specifies the first dimension of A as declared */}
00115 \textcolor{comment}{/*           in the calling (sub) program. LDA must be at least */}
00116 \textcolor{comment}{/*           ( k + 1 ). */}
00117 \textcolor{comment}{/*           Unchanged on exit. */}
00118 
00119 \textcolor{comment}{/*  X      - REAL             array of DIMENSION at least */}
00120 \textcolor{comment}{/*           ( 1 + ( n - 1 )*abs( INCX ) ). */}
00121 \textcolor{comment}{/*           Before entry, the incremented array X must contain the */}
00122 \textcolor{comment}{/*           vector x. */}
00123 \textcolor{comment}{/*           Unchanged on exit. */}
00124 
00125 \textcolor{comment}{/*  INCX   - INTEGER. */}
00126 \textcolor{comment}{/*           On entry, INCX specifies the increment for the elements of */}
00127 \textcolor{comment}{/*           X. INCX must not be zero. */}
00128 \textcolor{comment}{/*           Unchanged on exit. */}
00129 
00130 \textcolor{comment}{/*  BETA   - REAL            . */}
00131 \textcolor{comment}{/*           On entry, BETA specifies the scalar beta. */}
00132 \textcolor{comment}{/*           Unchanged on exit. */}
00133 
00134 \textcolor{comment}{/*  Y      - REAL             array of DIMENSION at least */}
00135 \textcolor{comment}{/*           ( 1 + ( n - 1 )*abs( INCY ) ). */}
00136 \textcolor{comment}{/*           Before entry, the incremented array Y must contain the */}
00137 \textcolor{comment}{/*           vector y. On exit, Y is overwritten by the updated vector y. */}
00138 
00139 \textcolor{comment}{/*  INCY   - INTEGER. */}
00140 \textcolor{comment}{/*           On entry, INCY specifies the increment for the elements of */}
00141 \textcolor{comment}{/*           Y. INCY must not be zero. */}
00142 \textcolor{comment}{/*           Unchanged on exit. */}
00143 
00144 \textcolor{comment}{/*  Further Details */}
00145 \textcolor{comment}{/*  =============== */}
00146 
00147 \textcolor{comment}{/*  Level 2 Blas routine. */}
00148 
00149 \textcolor{comment}{/*  -- Written on 22-October-1986. */}
00150 \textcolor{comment}{/*     Jack Dongarra, Argonne National Lab. */}
00151 \textcolor{comment}{/*     Jeremy Du Croz, Nag Central Office. */}
00152 \textcolor{comment}{/*     Sven Hammarling, Nag Central Office. */}
00153 \textcolor{comment}{/*     Richard Hanson, Sandia National Labs. */}
00154 
00155 \textcolor{comment}{/*  ===================================================================== */}
00156 
00157 \textcolor{comment}{/*     .. Parameters .. */}
00158 \textcolor{comment}{/*     .. */}
00159 \textcolor{comment}{/*     .. Local Scalars .. */}
00160 \textcolor{comment}{/*     .. */}
00161 \textcolor{comment}{/*     .. External Functions .. */}
00162 \textcolor{comment}{/*     .. */}
00163 \textcolor{comment}{/*     .. External Subroutines .. */}
00164 \textcolor{comment}{/*     .. */}
00165 \textcolor{comment}{/*     .. Intrinsic Functions .. */}
00166 \textcolor{comment}{/*     .. */}
00167 
00168 \textcolor{comment}{/*     Test the input parameters. */}
00169 
00170     \textcolor{comment}{/* Parameter adjustments */}
00171     a\_dim1 = *lda;
00172     a\_offset = 1 + a\_dim1;
00173     a -= a\_offset;
00174     --x;
00175     --y;
00176 
00177     \textcolor{comment}{/* Function Body */}
00178     info = 0;
00179     \textcolor{keywordflow}{if} (! lsame\_(uplo, \textcolor{stringliteral}{"U"}, (ftnlen)1, (ftnlen)1) && ! lsame\_(uplo, \textcolor{stringliteral}{"L"}, (
00180         ftnlen)1, (ftnlen)1)) \{
00181     info = 1;
00182     \} \textcolor{keywordflow}{else} \textcolor{keywordflow}{if} (*n < 0) \{
00183     info = 2;
00184     \} \textcolor{keywordflow}{else} \textcolor{keywordflow}{if} (*k < 0) \{
00185     info = 3;
00186     \} \textcolor{keywordflow}{else} \textcolor{keywordflow}{if} (*lda < *k + 1) \{
00187     info = 6;
00188     \} \textcolor{keywordflow}{else} \textcolor{keywordflow}{if} (*incx == 0) \{
00189     info = 8;
00190     \} \textcolor{keywordflow}{else} \textcolor{keywordflow}{if} (*incy == 0) \{
00191     info = 11;
00192     \}
00193     \textcolor{keywordflow}{if} (info != 0) \{
00194     xerbla\_(\textcolor{stringliteral}{"SSBMV "}, &info, (ftnlen)6);
00195     \textcolor{keywordflow}{return} 0;
00196     \}
00197 
00198 \textcolor{comment}{/*     Quick return if possible. */}
00199 
00200     \textcolor{keywordflow}{if} (*n == 0 || (*alpha == 0.f && *beta == 1.f)) \{
00201     \textcolor{keywordflow}{return} 0;
00202     \}
00203 
00204 \textcolor{comment}{/*     Set up the start points in  X  and  Y. */}
00205 
00206     \textcolor{keywordflow}{if} (*incx > 0) \{
00207     kx = 1;
00208     \} \textcolor{keywordflow}{else} \{
00209     kx = 1 - (*n - 1) * *incx;
00210     \}
00211     \textcolor{keywordflow}{if} (*incy > 0) \{
00212     ky = 1;
00213     \} \textcolor{keywordflow}{else} \{
00214     ky = 1 - (*n - 1) * *incy;
00215     \}
00216 
00217 \textcolor{comment}{/*     Start the operations. In this version the elements of the array A */}
00218 \textcolor{comment}{/*     are accessed sequentially with one pass through A. */}
00219 
00220 \textcolor{comment}{/*     First form  y := beta*y. */}
00221 
00222     \textcolor{keywordflow}{if} (*beta != 1.f) \{
00223     \textcolor{keywordflow}{if} (*incy == 1) \{
00224         \textcolor{keywordflow}{if} (*beta == 0.f) \{
00225         i\_\_1 = *n;
00226         \textcolor{keywordflow}{for} (i\_\_ = 1; i\_\_ <= i\_\_1; ++i\_\_) \{
00227             y[i\_\_] = 0.f;
00228 \textcolor{comment}{/* L10: */}
00229         \}
00230         \} \textcolor{keywordflow}{else} \{
00231         i\_\_1 = *n;
00232         \textcolor{keywordflow}{for} (i\_\_ = 1; i\_\_ <= i\_\_1; ++i\_\_) \{
00233             y[i\_\_] = *beta * y[i\_\_];
00234 \textcolor{comment}{/* L20: */}
00235         \}
00236         \}
00237     \} \textcolor{keywordflow}{else} \{
00238         iy = ky;
00239         \textcolor{keywordflow}{if} (*beta == 0.f) \{
00240         i\_\_1 = *n;
00241         \textcolor{keywordflow}{for} (i\_\_ = 1; i\_\_ <= i\_\_1; ++i\_\_) \{
00242             y[iy] = 0.f;
00243             iy += *incy;
00244 \textcolor{comment}{/* L30: */}
00245         \}
00246         \} \textcolor{keywordflow}{else} \{
00247         i\_\_1 = *n;
00248         \textcolor{keywordflow}{for} (i\_\_ = 1; i\_\_ <= i\_\_1; ++i\_\_) \{
00249             y[iy] = *beta * y[iy];
00250             iy += *incy;
00251 \textcolor{comment}{/* L40: */}
00252         \}
00253         \}
00254     \}
00255     \}
00256     \textcolor{keywordflow}{if} (*alpha == 0.f) \{
00257     \textcolor{keywordflow}{return} 0;
00258     \}
00259     \textcolor{keywordflow}{if} (lsame\_(uplo, \textcolor{stringliteral}{"U"}, (ftnlen)1, (ftnlen)1)) \{
00260 
00261 \textcolor{comment}{/*        Form  y  when upper triangle of A is stored. */}
00262 
00263     kplus1 = *k + 1;
00264     \textcolor{keywordflow}{if} (*incx == 1 && *incy == 1) \{
00265         i\_\_1 = *n;
00266         \textcolor{keywordflow}{for} (j = 1; j <= i\_\_1; ++j) \{
00267         temp1 = *alpha * x[j];
00268         temp2 = 0.f;
00269         l = kplus1 - j;
00270 \textcolor{comment}{/* Computing MAX */}
00271         i\_\_2 = 1, i\_\_3 = j - *k;
00272         i\_\_4 = j - 1;
00273         \textcolor{keywordflow}{for} (i\_\_ = max(i\_\_2,i\_\_3); i\_\_ <= i\_\_4; ++i\_\_) \{
00274             y[i\_\_] += temp1 * a[l + i\_\_ + j * a\_dim1];
00275             temp2 += a[l + i\_\_ + j * a\_dim1] * x[i\_\_];
00276 \textcolor{comment}{/* L50: */}
00277         \}
00278         y[j] = y[j] + temp1 * a[kplus1 + j * a\_dim1] + *alpha * temp2;
00279 \textcolor{comment}{/* L60: */}
00280         \}
00281     \} \textcolor{keywordflow}{else} \{
00282         jx = kx;
00283         jy = ky;
00284         i\_\_1 = *n;
00285         \textcolor{keywordflow}{for} (j = 1; j <= i\_\_1; ++j) \{
00286         temp1 = *alpha * x[jx];
00287         temp2 = 0.f;
00288         ix = kx;
00289         iy = ky;
00290         l = kplus1 - j;
00291 \textcolor{comment}{/* Computing MAX */}
00292         i\_\_4 = 1, i\_\_2 = j - *k;
00293         i\_\_3 = j - 1;
00294         \textcolor{keywordflow}{for} (i\_\_ = max(i\_\_4,i\_\_2); i\_\_ <= i\_\_3; ++i\_\_) \{
00295             y[iy] += temp1 * a[l + i\_\_ + j * a\_dim1];
00296             temp2 += a[l + i\_\_ + j * a\_dim1] * x[ix];
00297             ix += *incx;
00298             iy += *incy;
00299 \textcolor{comment}{/* L70: */}
00300         \}
00301         y[jy] = y[jy] + temp1 * a[kplus1 + j * a\_dim1] + *alpha * 
00302             temp2;
00303         jx += *incx;
00304         jy += *incy;
00305         \textcolor{keywordflow}{if} (j > *k) \{
00306             kx += *incx;
00307             ky += *incy;
00308         \}
00309 \textcolor{comment}{/* L80: */}
00310         \}
00311     \}
00312     \} \textcolor{keywordflow}{else} \{
00313 
00314 \textcolor{comment}{/*        Form  y  when lower triangle of A is stored. */}
00315 
00316     \textcolor{keywordflow}{if} (*incx == 1 && *incy == 1) \{
00317         i\_\_1 = *n;
00318         \textcolor{keywordflow}{for} (j = 1; j <= i\_\_1; ++j) \{
00319         temp1 = *alpha * x[j];
00320         temp2 = 0.f;
00321         y[j] += temp1 * a[j * a\_dim1 + 1];
00322         l = 1 - j;
00323 \textcolor{comment}{/* Computing MIN */}
00324         i\_\_4 = *n, i\_\_2 = j + *k;
00325         i\_\_3 = min(i\_\_4,i\_\_2);
00326         \textcolor{keywordflow}{for} (i\_\_ = j + 1; i\_\_ <= i\_\_3; ++i\_\_) \{
00327             y[i\_\_] += temp1 * a[l + i\_\_ + j * a\_dim1];
00328             temp2 += a[l + i\_\_ + j * a\_dim1] * x[i\_\_];
00329 \textcolor{comment}{/* L90: */}
00330         \}
00331         y[j] += *alpha * temp2;
00332 \textcolor{comment}{/* L100: */}
00333         \}
00334     \} \textcolor{keywordflow}{else} \{
00335         jx = kx;
00336         jy = ky;
00337         i\_\_1 = *n;
00338         \textcolor{keywordflow}{for} (j = 1; j <= i\_\_1; ++j) \{
00339         temp1 = *alpha * x[jx];
00340         temp2 = 0.f;
00341         y[jy] += temp1 * a[j * a\_dim1 + 1];
00342         l = 1 - j;
00343         ix = jx;
00344         iy = jy;
00345 \textcolor{comment}{/* Computing MIN */}
00346         i\_\_4 = *n, i\_\_2 = j + *k;
00347         i\_\_3 = min(i\_\_4,i\_\_2);
00348         \textcolor{keywordflow}{for} (i\_\_ = j + 1; i\_\_ <= i\_\_3; ++i\_\_) \{
00349             ix += *incx;
00350             iy += *incy;
00351             y[iy] += temp1 * a[l + i\_\_ + j * a\_dim1];
00352             temp2 += a[l + i\_\_ + j * a\_dim1] * x[ix];
00353 \textcolor{comment}{/* L110: */}
00354         \}
00355         y[jy] += *alpha * temp2;
00356         jx += *incx;
00357         jy += *incy;
00358 \textcolor{comment}{/* L120: */}
00359         \}
00360     \}
00361     \}
00362 
00363     \textcolor{keywordflow}{return} 0;
00364 
00365 \textcolor{comment}{/*     End of SSBMV . */}
00366 
00367 \} \textcolor{comment}{/* ssbmv\_ */}
00368 
\end{DoxyCode}
