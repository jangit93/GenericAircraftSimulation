\hypertarget{eigen_2blas_2f2c_2sspmv_8c_source}{}\section{eigen/blas/f2c/sspmv.c}
\label{eigen_2blas_2f2c_2sspmv_8c_source}\index{sspmv.\+c@{sspmv.\+c}}

\begin{DoxyCode}
00001 \textcolor{comment}{/* sspmv.f -- translated by f2c (version 20100827).}
00002 \textcolor{comment}{   You must link the resulting object file with libf2c:}
00003 \textcolor{comment}{    on Microsoft Windows system, link with libf2c.lib;}
00004 \textcolor{comment}{    on Linux or Unix systems, link with .../path/to/libf2c.a -lm}
00005 \textcolor{comment}{    or, if you install libf2c.a in a standard place, with -lf2c -lm}
00006 \textcolor{comment}{    -- in that order, at the end of the command line, as in}
00007 \textcolor{comment}{        cc *.o -lf2c -lm}
00008 \textcolor{comment}{    Source for libf2c is in /netlib/f2c/libf2c.zip, e.g.,}
00009 \textcolor{comment}{}
00010 \textcolor{comment}{        http://www.netlib.org/f2c/libf2c.zip}
00011 \textcolor{comment}{*/}
00012 
00013 \textcolor{preprocessor}{#include "datatypes.h"}
00014 
00015 \textcolor{comment}{/* Subroutine */} \textcolor{keywordtype}{int} sspmv\_(\textcolor{keywordtype}{char} *uplo, integer *n, real *alpha, real *ap, 
00016     real *x, integer *incx, real *beta, real *y, integer *incy, ftnlen 
00017     uplo\_len)
00018 \{
00019     \textcolor{comment}{/* System generated locals */}
00020     integer i\_\_1, i\_\_2;
00021 
00022     \textcolor{comment}{/* Local variables */}
00023     integer i\_\_, j, k, kk, ix, iy, jx, jy, kx, ky, info;
00024     real temp1, temp2;
00025     \textcolor{keyword}{extern} logical lsame\_(\textcolor{keywordtype}{char} *, \textcolor{keywordtype}{char} *, ftnlen, ftnlen);
00026     \textcolor{keyword}{extern} \textcolor{comment}{/* Subroutine */} \textcolor{keywordtype}{int} xerbla\_(\textcolor{keywordtype}{char} *, integer *, ftnlen);
00027 
00028 \textcolor{comment}{/*     .. Scalar Arguments .. */}
00029 \textcolor{comment}{/*     .. */}
00030 \textcolor{comment}{/*     .. Array Arguments .. */}
00031 \textcolor{comment}{/*     .. */}
00032 
00033 \textcolor{comment}{/*  Purpose */}
00034 \textcolor{comment}{/*  ======= */}
00035 
00036 \textcolor{comment}{/*  SSPMV  performs the matrix-vector operation */}
00037 
00038 \textcolor{comment}{/*     y := alpha*A*x + beta*y, */}
00039 
00040 \textcolor{comment}{/*  where alpha and beta are scalars, x and y are n element vectors and */}
00041 \textcolor{comment}{/*  A is an n by n symmetric matrix, supplied in packed form. */}
00042 
00043 \textcolor{comment}{/*  Arguments */}
00044 \textcolor{comment}{/*  ========== */}
00045 
00046 \textcolor{comment}{/*  UPLO   - CHARACTER*1. */}
00047 \textcolor{comment}{/*           On entry, UPLO specifies whether the upper or lower */}
00048 \textcolor{comment}{/*           triangular part of the matrix A is supplied in the packed */}
00049 \textcolor{comment}{/*           array AP as follows: */}
00050 
00051 \textcolor{comment}{/*              UPLO = 'U' or 'u'   The upper triangular part of A is */}
00052 \textcolor{comment}{/*                                  supplied in AP. */}
00053 
00054 \textcolor{comment}{/*              UPLO = 'L' or 'l'   The lower triangular part of A is */}
00055 \textcolor{comment}{/*                                  supplied in AP. */}
00056 
00057 \textcolor{comment}{/*           Unchanged on exit. */}
00058 
00059 \textcolor{comment}{/*  N      - INTEGER. */}
00060 \textcolor{comment}{/*           On entry, N specifies the order of the matrix A. */}
00061 \textcolor{comment}{/*           N must be at least zero. */}
00062 \textcolor{comment}{/*           Unchanged on exit. */}
00063 
00064 \textcolor{comment}{/*  ALPHA  - REAL            . */}
00065 \textcolor{comment}{/*           On entry, ALPHA specifies the scalar alpha. */}
00066 \textcolor{comment}{/*           Unchanged on exit. */}
00067 
00068 \textcolor{comment}{/*  AP     - REAL             array of DIMENSION at least */}
00069 \textcolor{comment}{/*           ( ( n*( n + 1 ) )/2 ). */}
00070 \textcolor{comment}{/*           Before entry with UPLO = 'U' or 'u', the array AP must */}
00071 \textcolor{comment}{/*           contain the upper triangular part of the symmetric matrix */}
00072 \textcolor{comment}{/*           packed sequentially, column by column, so that AP( 1 ) */}
00073 \textcolor{comment}{/*           contains a( 1, 1 ), AP( 2 ) and AP( 3 ) contain a( 1, 2 ) */}
00074 \textcolor{comment}{/*           and a( 2, 2 ) respectively, and so on. */}
00075 \textcolor{comment}{/*           Before entry with UPLO = 'L' or 'l', the array AP must */}
00076 \textcolor{comment}{/*           contain the lower triangular part of the symmetric matrix */}
00077 \textcolor{comment}{/*           packed sequentially, column by column, so that AP( 1 ) */}
00078 \textcolor{comment}{/*           contains a( 1, 1 ), AP( 2 ) and AP( 3 ) contain a( 2, 1 ) */}
00079 \textcolor{comment}{/*           and a( 3, 1 ) respectively, and so on. */}
00080 \textcolor{comment}{/*           Unchanged on exit. */}
00081 
00082 \textcolor{comment}{/*  X      - REAL             array of dimension at least */}
00083 \textcolor{comment}{/*           ( 1 + ( n - 1 )*abs( INCX ) ). */}
00084 \textcolor{comment}{/*           Before entry, the incremented array X must contain the n */}
00085 \textcolor{comment}{/*           element vector x. */}
00086 \textcolor{comment}{/*           Unchanged on exit. */}
00087 
00088 \textcolor{comment}{/*  INCX   - INTEGER. */}
00089 \textcolor{comment}{/*           On entry, INCX specifies the increment for the elements of */}
00090 \textcolor{comment}{/*           X. INCX must not be zero. */}
00091 \textcolor{comment}{/*           Unchanged on exit. */}
00092 
00093 \textcolor{comment}{/*  BETA   - REAL            . */}
00094 \textcolor{comment}{/*           On entry, BETA specifies the scalar beta. When BETA is */}
00095 \textcolor{comment}{/*           supplied as zero then Y need not be set on input. */}
00096 \textcolor{comment}{/*           Unchanged on exit. */}
00097 
00098 \textcolor{comment}{/*  Y      - REAL             array of dimension at least */}
00099 \textcolor{comment}{/*           ( 1 + ( n - 1 )*abs( INCY ) ). */}
00100 \textcolor{comment}{/*           Before entry, the incremented array Y must contain the n */}
00101 \textcolor{comment}{/*           element vector y. On exit, Y is overwritten by the updated */}
00102 \textcolor{comment}{/*           vector y. */}
00103 
00104 \textcolor{comment}{/*  INCY   - INTEGER. */}
00105 \textcolor{comment}{/*           On entry, INCY specifies the increment for the elements of */}
00106 \textcolor{comment}{/*           Y. INCY must not be zero. */}
00107 \textcolor{comment}{/*           Unchanged on exit. */}
00108 
00109 \textcolor{comment}{/*  Further Details */}
00110 \textcolor{comment}{/*  =============== */}
00111 
00112 \textcolor{comment}{/*  Level 2 Blas routine. */}
00113 
00114 \textcolor{comment}{/*  -- Written on 22-October-1986. */}
00115 \textcolor{comment}{/*     Jack Dongarra, Argonne National Lab. */}
00116 \textcolor{comment}{/*     Jeremy Du Croz, Nag Central Office. */}
00117 \textcolor{comment}{/*     Sven Hammarling, Nag Central Office. */}
00118 \textcolor{comment}{/*     Richard Hanson, Sandia National Labs. */}
00119 
00120 \textcolor{comment}{/*  ===================================================================== */}
00121 
00122 \textcolor{comment}{/*     .. Parameters .. */}
00123 \textcolor{comment}{/*     .. */}
00124 \textcolor{comment}{/*     .. Local Scalars .. */}
00125 \textcolor{comment}{/*     .. */}
00126 \textcolor{comment}{/*     .. External Functions .. */}
00127 \textcolor{comment}{/*     .. */}
00128 \textcolor{comment}{/*     .. External Subroutines .. */}
00129 \textcolor{comment}{/*     .. */}
00130 
00131 \textcolor{comment}{/*     Test the input parameters. */}
00132 
00133     \textcolor{comment}{/* Parameter adjustments */}
00134     --y;
00135     --x;
00136     --ap;
00137 
00138     \textcolor{comment}{/* Function Body */}
00139     info = 0;
00140     \textcolor{keywordflow}{if} (! lsame\_(uplo, \textcolor{stringliteral}{"U"}, (ftnlen)1, (ftnlen)1) && ! lsame\_(uplo, \textcolor{stringliteral}{"L"}, (
00141         ftnlen)1, (ftnlen)1)) \{
00142     info = 1;
00143     \} \textcolor{keywordflow}{else} \textcolor{keywordflow}{if} (*n < 0) \{
00144     info = 2;
00145     \} \textcolor{keywordflow}{else} \textcolor{keywordflow}{if} (*incx == 0) \{
00146     info = 6;
00147     \} \textcolor{keywordflow}{else} \textcolor{keywordflow}{if} (*incy == 0) \{
00148     info = 9;
00149     \}
00150     \textcolor{keywordflow}{if} (info != 0) \{
00151     xerbla\_(\textcolor{stringliteral}{"SSPMV "}, &info, (ftnlen)6);
00152     \textcolor{keywordflow}{return} 0;
00153     \}
00154 
00155 \textcolor{comment}{/*     Quick return if possible. */}
00156 
00157     \textcolor{keywordflow}{if} (*n == 0 || (*alpha == 0.f && *beta == 1.f)) \{
00158     \textcolor{keywordflow}{return} 0;
00159     \}
00160 
00161 \textcolor{comment}{/*     Set up the start points in  X  and  Y. */}
00162 
00163     \textcolor{keywordflow}{if} (*incx > 0) \{
00164     kx = 1;
00165     \} \textcolor{keywordflow}{else} \{
00166     kx = 1 - (*n - 1) * *incx;
00167     \}
00168     \textcolor{keywordflow}{if} (*incy > 0) \{
00169     ky = 1;
00170     \} \textcolor{keywordflow}{else} \{
00171     ky = 1 - (*n - 1) * *incy;
00172     \}
00173 
00174 \textcolor{comment}{/*     Start the operations. In this version the elements of the array AP */}
00175 \textcolor{comment}{/*     are accessed sequentially with one pass through AP. */}
00176 
00177 \textcolor{comment}{/*     First form  y := beta*y. */}
00178 
00179     \textcolor{keywordflow}{if} (*beta != 1.f) \{
00180     \textcolor{keywordflow}{if} (*incy == 1) \{
00181         \textcolor{keywordflow}{if} (*beta == 0.f) \{
00182         i\_\_1 = *n;
00183         \textcolor{keywordflow}{for} (i\_\_ = 1; i\_\_ <= i\_\_1; ++i\_\_) \{
00184             y[i\_\_] = 0.f;
00185 \textcolor{comment}{/* L10: */}
00186         \}
00187         \} \textcolor{keywordflow}{else} \{
00188         i\_\_1 = *n;
00189         \textcolor{keywordflow}{for} (i\_\_ = 1; i\_\_ <= i\_\_1; ++i\_\_) \{
00190             y[i\_\_] = *beta * y[i\_\_];
00191 \textcolor{comment}{/* L20: */}
00192         \}
00193         \}
00194     \} \textcolor{keywordflow}{else} \{
00195         iy = ky;
00196         \textcolor{keywordflow}{if} (*beta == 0.f) \{
00197         i\_\_1 = *n;
00198         \textcolor{keywordflow}{for} (i\_\_ = 1; i\_\_ <= i\_\_1; ++i\_\_) \{
00199             y[iy] = 0.f;
00200             iy += *incy;
00201 \textcolor{comment}{/* L30: */}
00202         \}
00203         \} \textcolor{keywordflow}{else} \{
00204         i\_\_1 = *n;
00205         \textcolor{keywordflow}{for} (i\_\_ = 1; i\_\_ <= i\_\_1; ++i\_\_) \{
00206             y[iy] = *beta * y[iy];
00207             iy += *incy;
00208 \textcolor{comment}{/* L40: */}
00209         \}
00210         \}
00211     \}
00212     \}
00213     \textcolor{keywordflow}{if} (*alpha == 0.f) \{
00214     \textcolor{keywordflow}{return} 0;
00215     \}
00216     kk = 1;
00217     \textcolor{keywordflow}{if} (lsame\_(uplo, \textcolor{stringliteral}{"U"}, (ftnlen)1, (ftnlen)1)) \{
00218 
00219 \textcolor{comment}{/*        Form  y  when AP contains the upper triangle. */}
00220 
00221     \textcolor{keywordflow}{if} (*incx == 1 && *incy == 1) \{
00222         i\_\_1 = *n;
00223         \textcolor{keywordflow}{for} (j = 1; j <= i\_\_1; ++j) \{
00224         temp1 = *alpha * x[j];
00225         temp2 = 0.f;
00226         k = kk;
00227         i\_\_2 = j - 1;
00228         \textcolor{keywordflow}{for} (i\_\_ = 1; i\_\_ <= i\_\_2; ++i\_\_) \{
00229             y[i\_\_] += temp1 * ap[k];
00230             temp2 += ap[k] * x[i\_\_];
00231             ++k;
00232 \textcolor{comment}{/* L50: */}
00233         \}
00234         y[j] = y[j] + temp1 * ap[kk + j - 1] + *alpha * temp2;
00235         kk += j;
00236 \textcolor{comment}{/* L60: */}
00237         \}
00238     \} \textcolor{keywordflow}{else} \{
00239         jx = kx;
00240         jy = ky;
00241         i\_\_1 = *n;
00242         \textcolor{keywordflow}{for} (j = 1; j <= i\_\_1; ++j) \{
00243         temp1 = *alpha * x[jx];
00244         temp2 = 0.f;
00245         ix = kx;
00246         iy = ky;
00247         i\_\_2 = kk + j - 2;
00248         \textcolor{keywordflow}{for} (k = kk; k <= i\_\_2; ++k) \{
00249             y[iy] += temp1 * ap[k];
00250             temp2 += ap[k] * x[ix];
00251             ix += *incx;
00252             iy += *incy;
00253 \textcolor{comment}{/* L70: */}
00254         \}
00255         y[jy] = y[jy] + temp1 * ap[kk + j - 1] + *alpha * temp2;
00256         jx += *incx;
00257         jy += *incy;
00258         kk += j;
00259 \textcolor{comment}{/* L80: */}
00260         \}
00261     \}
00262     \} \textcolor{keywordflow}{else} \{
00263 
00264 \textcolor{comment}{/*        Form  y  when AP contains the lower triangle. */}
00265 
00266     \textcolor{keywordflow}{if} (*incx == 1 && *incy == 1) \{
00267         i\_\_1 = *n;
00268         \textcolor{keywordflow}{for} (j = 1; j <= i\_\_1; ++j) \{
00269         temp1 = *alpha * x[j];
00270         temp2 = 0.f;
00271         y[j] += temp1 * ap[kk];
00272         k = kk + 1;
00273         i\_\_2 = *n;
00274         \textcolor{keywordflow}{for} (i\_\_ = j + 1; i\_\_ <= i\_\_2; ++i\_\_) \{
00275             y[i\_\_] += temp1 * ap[k];
00276             temp2 += ap[k] * x[i\_\_];
00277             ++k;
00278 \textcolor{comment}{/* L90: */}
00279         \}
00280         y[j] += *alpha * temp2;
00281         kk += *n - j + 1;
00282 \textcolor{comment}{/* L100: */}
00283         \}
00284     \} \textcolor{keywordflow}{else} \{
00285         jx = kx;
00286         jy = ky;
00287         i\_\_1 = *n;
00288         \textcolor{keywordflow}{for} (j = 1; j <= i\_\_1; ++j) \{
00289         temp1 = *alpha * x[jx];
00290         temp2 = 0.f;
00291         y[jy] += temp1 * ap[kk];
00292         ix = jx;
00293         iy = jy;
00294         i\_\_2 = kk + *n - j;
00295         \textcolor{keywordflow}{for} (k = kk + 1; k <= i\_\_2; ++k) \{
00296             ix += *incx;
00297             iy += *incy;
00298             y[iy] += temp1 * ap[k];
00299             temp2 += ap[k] * x[ix];
00300 \textcolor{comment}{/* L110: */}
00301         \}
00302         y[jy] += *alpha * temp2;
00303         jx += *incx;
00304         jy += *incy;
00305         kk += *n - j + 1;
00306 \textcolor{comment}{/* L120: */}
00307         \}
00308     \}
00309     \}
00310 
00311     \textcolor{keywordflow}{return} 0;
00312 
00313 \textcolor{comment}{/*     End of SSPMV . */}
00314 
00315 \} \textcolor{comment}{/* sspmv\_ */}
00316 
\end{DoxyCode}
