\hypertarget{matio_2visual__studio_2test_2eigen_2blas_2f2c_2dtbmv_8c_source}{}\section{matio/visual\+\_\+studio/test/eigen/blas/f2c/dtbmv.c}
\label{matio_2visual__studio_2test_2eigen_2blas_2f2c_2dtbmv_8c_source}\index{dtbmv.\+c@{dtbmv.\+c}}

\begin{DoxyCode}
00001 \textcolor{comment}{/* dtbmv.f -- translated by f2c (version 20100827).}
00002 \textcolor{comment}{   You must link the resulting object file with libf2c:}
00003 \textcolor{comment}{    on Microsoft Windows system, link with libf2c.lib;}
00004 \textcolor{comment}{    on Linux or Unix systems, link with .../path/to/libf2c.a -lm}
00005 \textcolor{comment}{    or, if you install libf2c.a in a standard place, with -lf2c -lm}
00006 \textcolor{comment}{    -- in that order, at the end of the command line, as in}
00007 \textcolor{comment}{        cc *.o -lf2c -lm}
00008 \textcolor{comment}{    Source for libf2c is in /netlib/f2c/libf2c.zip, e.g.,}
00009 \textcolor{comment}{}
00010 \textcolor{comment}{        http://www.netlib.org/f2c/libf2c.zip}
00011 \textcolor{comment}{*/}
00012 
00013 \textcolor{preprocessor}{#include "datatypes.h"}
00014 
00015 \textcolor{comment}{/* Subroutine */} \textcolor{keywordtype}{int} dtbmv\_(\textcolor{keywordtype}{char} *uplo, \textcolor{keywordtype}{char} *trans, \textcolor{keywordtype}{char} *diag, integer *n, 
00016     integer *k, doublereal *a, integer *lda, doublereal *x, integer *incx,
00017      ftnlen uplo\_len, ftnlen trans\_len, ftnlen diag\_len)
00018 \{
00019     \textcolor{comment}{/* System generated locals */}
00020     integer a\_dim1, a\_offset, i\_\_1, i\_\_2, i\_\_3, i\_\_4;
00021 
00022     \textcolor{comment}{/* Local variables */}
00023     integer i\_\_, j, l, ix, jx, kx, info;
00024     doublereal temp;
00025     \textcolor{keyword}{extern} logical lsame\_(\textcolor{keywordtype}{char} *, \textcolor{keywordtype}{char} *, ftnlen, ftnlen);
00026     integer kplus1;
00027     \textcolor{keyword}{extern} \textcolor{comment}{/* Subroutine */} \textcolor{keywordtype}{int} xerbla\_(\textcolor{keywordtype}{char} *, integer *, ftnlen);
00028     logical nounit;
00029 
00030 \textcolor{comment}{/*     .. Scalar Arguments .. */}
00031 \textcolor{comment}{/*     .. */}
00032 \textcolor{comment}{/*     .. Array Arguments .. */}
00033 \textcolor{comment}{/*     .. */}
00034 
00035 \textcolor{comment}{/*  Purpose */}
00036 \textcolor{comment}{/*  ======= */}
00037 
00038 \textcolor{comment}{/*  DTBMV  performs one of the matrix-vector operations */}
00039 
00040 \textcolor{comment}{/*     x := A*x,   or   x := A'*x, */}
00041 
00042 \textcolor{comment}{/*  where x is an n element vector and  A is an n by n unit, or non-unit, */}
00043 \textcolor{comment}{/*  upper or lower triangular band matrix, with ( k + 1 ) diagonals. */}
00044 
00045 \textcolor{comment}{/*  Arguments */}
00046 \textcolor{comment}{/*  ========== */}
00047 
00048 \textcolor{comment}{/*  UPLO   - CHARACTER*1. */}
00049 \textcolor{comment}{/*           On entry, UPLO specifies whether the matrix is an upper or */}
00050 \textcolor{comment}{/*           lower triangular matrix as follows: */}
00051 
00052 \textcolor{comment}{/*              UPLO = 'U' or 'u'   A is an upper triangular matrix. */}
00053 
00054 \textcolor{comment}{/*              UPLO = 'L' or 'l'   A is a lower triangular matrix. */}
00055 
00056 \textcolor{comment}{/*           Unchanged on exit. */}
00057 
00058 \textcolor{comment}{/*  TRANS  - CHARACTER*1. */}
00059 \textcolor{comment}{/*           On entry, TRANS specifies the operation to be performed as */}
00060 \textcolor{comment}{/*           follows: */}
00061 
00062 \textcolor{comment}{/*              TRANS = 'N' or 'n'   x := A*x. */}
00063 
00064 \textcolor{comment}{/*              TRANS = 'T' or 't'   x := A'*x. */}
00065 
00066 \textcolor{comment}{/*              TRANS = 'C' or 'c'   x := A'*x. */}
00067 
00068 \textcolor{comment}{/*           Unchanged on exit. */}
00069 
00070 \textcolor{comment}{/*  DIAG   - CHARACTER*1. */}
00071 \textcolor{comment}{/*           On entry, DIAG specifies whether or not A is unit */}
00072 \textcolor{comment}{/*           triangular as follows: */}
00073 
00074 \textcolor{comment}{/*              DIAG = 'U' or 'u'   A is assumed to be unit triangular. */}
00075 
00076 \textcolor{comment}{/*              DIAG = 'N' or 'n'   A is not assumed to be unit */}
00077 \textcolor{comment}{/*                                  triangular. */}
00078 
00079 \textcolor{comment}{/*           Unchanged on exit. */}
00080 
00081 \textcolor{comment}{/*  N      - INTEGER. */}
00082 \textcolor{comment}{/*           On entry, N specifies the order of the matrix A. */}
00083 \textcolor{comment}{/*           N must be at least zero. */}
00084 \textcolor{comment}{/*           Unchanged on exit. */}
00085 
00086 \textcolor{comment}{/*  K      - INTEGER. */}
00087 \textcolor{comment}{/*           On entry with UPLO = 'U' or 'u', K specifies the number of */}
00088 \textcolor{comment}{/*           super-diagonals of the matrix A. */}
00089 \textcolor{comment}{/*           On entry with UPLO = 'L' or 'l', K specifies the number of */}
00090 \textcolor{comment}{/*           sub-diagonals of the matrix A. */}
00091 \textcolor{comment}{/*           K must satisfy  0 .le. K. */}
00092 \textcolor{comment}{/*           Unchanged on exit. */}
00093 
00094 \textcolor{comment}{/*  A      - DOUBLE PRECISION array of DIMENSION ( LDA, n ). */}
00095 \textcolor{comment}{/*           Before entry with UPLO = 'U' or 'u', the leading ( k + 1 ) */}
00096 \textcolor{comment}{/*           by n part of the array A must contain the upper triangular */}
00097 \textcolor{comment}{/*           band part of the matrix of coefficients, supplied column by */}
00098 \textcolor{comment}{/*           column, with the leading diagonal of the matrix in row */}
00099 \textcolor{comment}{/*           ( k + 1 ) of the array, the first super-diagonal starting at */}
00100 \textcolor{comment}{/*           position 2 in row k, and so on. The top left k by k triangle */}
00101 \textcolor{comment}{/*           of the array A is not referenced. */}
00102 \textcolor{comment}{/*           The following program segment will transfer an upper */}
00103 \textcolor{comment}{/*           triangular band matrix from conventional full matrix storage */}
00104 \textcolor{comment}{/*           to band storage: */}
00105 
00106 \textcolor{comment}{/*                 DO 20, J = 1, N */}
00107 \textcolor{comment}{/*                    M = K + 1 - J */}
00108 \textcolor{comment}{/*                    DO 10, I = MAX( 1, J - K ), J */}
00109 \textcolor{comment}{/*                       A( M + I, J ) = matrix( I, J ) */}
00110 \textcolor{comment}{/*              10    CONTINUE */}
00111 \textcolor{comment}{/*              20 CONTINUE */}
00112 
00113 \textcolor{comment}{/*           Before entry with UPLO = 'L' or 'l', the leading ( k + 1 ) */}
00114 \textcolor{comment}{/*           by n part of the array A must contain the lower triangular */}
00115 \textcolor{comment}{/*           band part of the matrix of coefficients, supplied column by */}
00116 \textcolor{comment}{/*           column, with the leading diagonal of the matrix in row 1 of */}
00117 \textcolor{comment}{/*           the array, the first sub-diagonal starting at position 1 in */}
00118 \textcolor{comment}{/*           row 2, and so on. The bottom right k by k triangle of the */}
00119 \textcolor{comment}{/*           array A is not referenced. */}
00120 \textcolor{comment}{/*           The following program segment will transfer a lower */}
00121 \textcolor{comment}{/*           triangular band matrix from conventional full matrix storage */}
00122 \textcolor{comment}{/*           to band storage: */}
00123 
00124 \textcolor{comment}{/*                 DO 20, J = 1, N */}
00125 \textcolor{comment}{/*                    M = 1 - J */}
00126 \textcolor{comment}{/*                    DO 10, I = J, MIN( N, J + K ) */}
00127 \textcolor{comment}{/*                       A( M + I, J ) = matrix( I, J ) */}
00128 \textcolor{comment}{/*              10    CONTINUE */}
00129 \textcolor{comment}{/*              20 CONTINUE */}
00130 
00131 \textcolor{comment}{/*           Note that when DIAG = 'U' or 'u' the elements of the array A */}
00132 \textcolor{comment}{/*           corresponding to the diagonal elements of the matrix are not */}
00133 \textcolor{comment}{/*           referenced, but are assumed to be unity. */}
00134 \textcolor{comment}{/*           Unchanged on exit. */}
00135 
00136 \textcolor{comment}{/*  LDA    - INTEGER. */}
00137 \textcolor{comment}{/*           On entry, LDA specifies the first dimension of A as declared */}
00138 \textcolor{comment}{/*           in the calling (sub) program. LDA must be at least */}
00139 \textcolor{comment}{/*           ( k + 1 ). */}
00140 \textcolor{comment}{/*           Unchanged on exit. */}
00141 
00142 \textcolor{comment}{/*  X      - DOUBLE PRECISION array of dimension at least */}
00143 \textcolor{comment}{/*           ( 1 + ( n - 1 )*abs( INCX ) ). */}
00144 \textcolor{comment}{/*           Before entry, the incremented array X must contain the n */}
00145 \textcolor{comment}{/*           element vector x. On exit, X is overwritten with the */}
00146 \textcolor{comment}{/*           tranformed vector x. */}
00147 
00148 \textcolor{comment}{/*  INCX   - INTEGER. */}
00149 \textcolor{comment}{/*           On entry, INCX specifies the increment for the elements of */}
00150 \textcolor{comment}{/*           X. INCX must not be zero. */}
00151 \textcolor{comment}{/*           Unchanged on exit. */}
00152 
00153 \textcolor{comment}{/*  Further Details */}
00154 \textcolor{comment}{/*  =============== */}
00155 
00156 \textcolor{comment}{/*  Level 2 Blas routine. */}
00157 
00158 \textcolor{comment}{/*  -- Written on 22-October-1986. */}
00159 \textcolor{comment}{/*     Jack Dongarra, Argonne National Lab. */}
00160 \textcolor{comment}{/*     Jeremy Du Croz, Nag Central Office. */}
00161 \textcolor{comment}{/*     Sven Hammarling, Nag Central Office. */}
00162 \textcolor{comment}{/*     Richard Hanson, Sandia National Labs. */}
00163 
00164 \textcolor{comment}{/*  ===================================================================== */}
00165 
00166 \textcolor{comment}{/*     .. Parameters .. */}
00167 \textcolor{comment}{/*     .. */}
00168 \textcolor{comment}{/*     .. Local Scalars .. */}
00169 \textcolor{comment}{/*     .. */}
00170 \textcolor{comment}{/*     .. External Functions .. */}
00171 \textcolor{comment}{/*     .. */}
00172 \textcolor{comment}{/*     .. External Subroutines .. */}
00173 \textcolor{comment}{/*     .. */}
00174 \textcolor{comment}{/*     .. Intrinsic Functions .. */}
00175 \textcolor{comment}{/*     .. */}
00176 
00177 \textcolor{comment}{/*     Test the input parameters. */}
00178 
00179     \textcolor{comment}{/* Parameter adjustments */}
00180     a\_dim1 = *lda;
00181     a\_offset = 1 + a\_dim1;
00182     a -= a\_offset;
00183     --x;
00184 
00185     \textcolor{comment}{/* Function Body */}
00186     info = 0;
00187     \textcolor{keywordflow}{if} (! lsame\_(uplo, \textcolor{stringliteral}{"U"}, (ftnlen)1, (ftnlen)1) && ! lsame\_(uplo, \textcolor{stringliteral}{"L"}, (
00188         ftnlen)1, (ftnlen)1)) \{
00189     info = 1;
00190     \} \textcolor{keywordflow}{else} \textcolor{keywordflow}{if} (! lsame\_(trans, \textcolor{stringliteral}{"N"}, (ftnlen)1, (ftnlen)1) && ! lsame\_(trans, 
00191         \textcolor{stringliteral}{"T"}, (ftnlen)1, (ftnlen)1) && ! lsame\_(trans, \textcolor{stringliteral}{"C"}, (ftnlen)1, (
00192         ftnlen)1)) \{
00193     info = 2;
00194     \} \textcolor{keywordflow}{else} \textcolor{keywordflow}{if} (! lsame\_(diag, \textcolor{stringliteral}{"U"}, (ftnlen)1, (ftnlen)1) && ! lsame\_(diag, 
00195         \textcolor{stringliteral}{"N"}, (ftnlen)1, (ftnlen)1)) \{
00196     info = 3;
00197     \} \textcolor{keywordflow}{else} \textcolor{keywordflow}{if} (*n < 0) \{
00198     info = 4;
00199     \} \textcolor{keywordflow}{else} \textcolor{keywordflow}{if} (*k < 0) \{
00200     info = 5;
00201     \} \textcolor{keywordflow}{else} \textcolor{keywordflow}{if} (*lda < *k + 1) \{
00202     info = 7;
00203     \} \textcolor{keywordflow}{else} \textcolor{keywordflow}{if} (*incx == 0) \{
00204     info = 9;
00205     \}
00206     \textcolor{keywordflow}{if} (info != 0) \{
00207     xerbla\_(\textcolor{stringliteral}{"DTBMV "}, &info, (ftnlen)6);
00208     \textcolor{keywordflow}{return} 0;
00209     \}
00210 
00211 \textcolor{comment}{/*     Quick return if possible. */}
00212 
00213     \textcolor{keywordflow}{if} (*n == 0) \{
00214     \textcolor{keywordflow}{return} 0;
00215     \}
00216 
00217     nounit = lsame\_(diag, \textcolor{stringliteral}{"N"}, (ftnlen)1, (ftnlen)1);
00218 
00219 \textcolor{comment}{/*     Set up the start point in X if the increment is not unity. This */}
00220 \textcolor{comment}{/*     will be  ( N - 1 )*INCX   too small for descending loops. */}
00221 
00222     \textcolor{keywordflow}{if} (*incx <= 0) \{
00223     kx = 1 - (*n - 1) * *incx;
00224     \} \textcolor{keywordflow}{else} \textcolor{keywordflow}{if} (*incx != 1) \{
00225     kx = 1;
00226     \}
00227 
00228 \textcolor{comment}{/*     Start the operations. In this version the elements of A are */}
00229 \textcolor{comment}{/*     accessed sequentially with one pass through A. */}
00230 
00231     \textcolor{keywordflow}{if} (lsame\_(trans, \textcolor{stringliteral}{"N"}, (ftnlen)1, (ftnlen)1)) \{
00232 
00233 \textcolor{comment}{/*         Form  x := A*x. */}
00234 
00235     \textcolor{keywordflow}{if} (lsame\_(uplo, \textcolor{stringliteral}{"U"}, (ftnlen)1, (ftnlen)1)) \{
00236         kplus1 = *k + 1;
00237         \textcolor{keywordflow}{if} (*incx == 1) \{
00238         i\_\_1 = *n;
00239         \textcolor{keywordflow}{for} (j = 1; j <= i\_\_1; ++j) \{
00240             \textcolor{keywordflow}{if} (x[j] != 0.) \{
00241             temp = x[j];
00242             l = kplus1 - j;
00243 \textcolor{comment}{/* Computing MAX */}
00244             i\_\_2 = 1, i\_\_3 = j - *k;
00245             i\_\_4 = j - 1;
00246             \textcolor{keywordflow}{for} (i\_\_ = max(i\_\_2,i\_\_3); i\_\_ <= i\_\_4; ++i\_\_) \{
00247                 x[i\_\_] += temp * a[l + i\_\_ + j * a\_dim1];
00248 \textcolor{comment}{/* L10: */}
00249             \}
00250             \textcolor{keywordflow}{if} (nounit) \{
00251                 x[j] *= a[kplus1 + j * a\_dim1];
00252             \}
00253             \}
00254 \textcolor{comment}{/* L20: */}
00255         \}
00256         \} \textcolor{keywordflow}{else} \{
00257         jx = kx;
00258         i\_\_1 = *n;
00259         \textcolor{keywordflow}{for} (j = 1; j <= i\_\_1; ++j) \{
00260             \textcolor{keywordflow}{if} (x[jx] != 0.) \{
00261             temp = x[jx];
00262             ix = kx;
00263             l = kplus1 - j;
00264 \textcolor{comment}{/* Computing MAX */}
00265             i\_\_4 = 1, i\_\_2 = j - *k;
00266             i\_\_3 = j - 1;
00267             \textcolor{keywordflow}{for} (i\_\_ = max(i\_\_4,i\_\_2); i\_\_ <= i\_\_3; ++i\_\_) \{
00268                 x[ix] += temp * a[l + i\_\_ + j * a\_dim1];
00269                 ix += *incx;
00270 \textcolor{comment}{/* L30: */}
00271             \}
00272             \textcolor{keywordflow}{if} (nounit) \{
00273                 x[jx] *= a[kplus1 + j * a\_dim1];
00274             \}
00275             \}
00276             jx += *incx;
00277             \textcolor{keywordflow}{if} (j > *k) \{
00278             kx += *incx;
00279             \}
00280 \textcolor{comment}{/* L40: */}
00281         \}
00282         \}
00283     \} \textcolor{keywordflow}{else} \{
00284         \textcolor{keywordflow}{if} (*incx == 1) \{
00285         \textcolor{keywordflow}{for} (j = *n; j >= 1; --j) \{
00286             \textcolor{keywordflow}{if} (x[j] != 0.) \{
00287             temp = x[j];
00288             l = 1 - j;
00289 \textcolor{comment}{/* Computing MIN */}
00290             i\_\_1 = *n, i\_\_3 = j + *k;
00291             i\_\_4 = j + 1;
00292             \textcolor{keywordflow}{for} (i\_\_ = min(i\_\_1,i\_\_3); i\_\_ >= i\_\_4; --i\_\_) \{
00293                 x[i\_\_] += temp * a[l + i\_\_ + j * a\_dim1];
00294 \textcolor{comment}{/* L50: */}
00295             \}
00296             \textcolor{keywordflow}{if} (nounit) \{
00297                 x[j] *= a[j * a\_dim1 + 1];
00298             \}
00299             \}
00300 \textcolor{comment}{/* L60: */}
00301         \}
00302         \} \textcolor{keywordflow}{else} \{
00303         kx += (*n - 1) * *incx;
00304         jx = kx;
00305         \textcolor{keywordflow}{for} (j = *n; j >= 1; --j) \{
00306             \textcolor{keywordflow}{if} (x[jx] != 0.) \{
00307             temp = x[jx];
00308             ix = kx;
00309             l = 1 - j;
00310 \textcolor{comment}{/* Computing MIN */}
00311             i\_\_4 = *n, i\_\_1 = j + *k;
00312             i\_\_3 = j + 1;
00313             \textcolor{keywordflow}{for} (i\_\_ = min(i\_\_4,i\_\_1); i\_\_ >= i\_\_3; --i\_\_) \{
00314                 x[ix] += temp * a[l + i\_\_ + j * a\_dim1];
00315                 ix -= *incx;
00316 \textcolor{comment}{/* L70: */}
00317             \}
00318             \textcolor{keywordflow}{if} (nounit) \{
00319                 x[jx] *= a[j * a\_dim1 + 1];
00320             \}
00321             \}
00322             jx -= *incx;
00323             \textcolor{keywordflow}{if} (*n - j >= *k) \{
00324             kx -= *incx;
00325             \}
00326 \textcolor{comment}{/* L80: */}
00327         \}
00328         \}
00329     \}
00330     \} \textcolor{keywordflow}{else} \{
00331 
00332 \textcolor{comment}{/*        Form  x := A'*x. */}
00333 
00334     \textcolor{keywordflow}{if} (lsame\_(uplo, \textcolor{stringliteral}{"U"}, (ftnlen)1, (ftnlen)1)) \{
00335         kplus1 = *k + 1;
00336         \textcolor{keywordflow}{if} (*incx == 1) \{
00337         \textcolor{keywordflow}{for} (j = *n; j >= 1; --j) \{
00338             temp = x[j];
00339             l = kplus1 - j;
00340             \textcolor{keywordflow}{if} (nounit) \{
00341             temp *= a[kplus1 + j * a\_dim1];
00342             \}
00343 \textcolor{comment}{/* Computing MAX */}
00344             i\_\_4 = 1, i\_\_1 = j - *k;
00345             i\_\_3 = max(i\_\_4,i\_\_1);
00346             \textcolor{keywordflow}{for} (i\_\_ = j - 1; i\_\_ >= i\_\_3; --i\_\_) \{
00347             temp += a[l + i\_\_ + j * a\_dim1] * x[i\_\_];
00348 \textcolor{comment}{/* L90: */}
00349             \}
00350             x[j] = temp;
00351 \textcolor{comment}{/* L100: */}
00352         \}
00353         \} \textcolor{keywordflow}{else} \{
00354         kx += (*n - 1) * *incx;
00355         jx = kx;
00356         \textcolor{keywordflow}{for} (j = *n; j >= 1; --j) \{
00357             temp = x[jx];
00358             kx -= *incx;
00359             ix = kx;
00360             l = kplus1 - j;
00361             \textcolor{keywordflow}{if} (nounit) \{
00362             temp *= a[kplus1 + j * a\_dim1];
00363             \}
00364 \textcolor{comment}{/* Computing MAX */}
00365             i\_\_4 = 1, i\_\_1 = j - *k;
00366             i\_\_3 = max(i\_\_4,i\_\_1);
00367             \textcolor{keywordflow}{for} (i\_\_ = j - 1; i\_\_ >= i\_\_3; --i\_\_) \{
00368             temp += a[l + i\_\_ + j * a\_dim1] * x[ix];
00369             ix -= *incx;
00370 \textcolor{comment}{/* L110: */}
00371             \}
00372             x[jx] = temp;
00373             jx -= *incx;
00374 \textcolor{comment}{/* L120: */}
00375         \}
00376         \}
00377     \} \textcolor{keywordflow}{else} \{
00378         \textcolor{keywordflow}{if} (*incx == 1) \{
00379         i\_\_3 = *n;
00380         \textcolor{keywordflow}{for} (j = 1; j <= i\_\_3; ++j) \{
00381             temp = x[j];
00382             l = 1 - j;
00383             \textcolor{keywordflow}{if} (nounit) \{
00384             temp *= a[j * a\_dim1 + 1];
00385             \}
00386 \textcolor{comment}{/* Computing MIN */}
00387             i\_\_1 = *n, i\_\_2 = j + *k;
00388             i\_\_4 = min(i\_\_1,i\_\_2);
00389             \textcolor{keywordflow}{for} (i\_\_ = j + 1; i\_\_ <= i\_\_4; ++i\_\_) \{
00390             temp += a[l + i\_\_ + j * a\_dim1] * x[i\_\_];
00391 \textcolor{comment}{/* L130: */}
00392             \}
00393             x[j] = temp;
00394 \textcolor{comment}{/* L140: */}
00395         \}
00396         \} \textcolor{keywordflow}{else} \{
00397         jx = kx;
00398         i\_\_3 = *n;
00399         \textcolor{keywordflow}{for} (j = 1; j <= i\_\_3; ++j) \{
00400             temp = x[jx];
00401             kx += *incx;
00402             ix = kx;
00403             l = 1 - j;
00404             \textcolor{keywordflow}{if} (nounit) \{
00405             temp *= a[j * a\_dim1 + 1];
00406             \}
00407 \textcolor{comment}{/* Computing MIN */}
00408             i\_\_1 = *n, i\_\_2 = j + *k;
00409             i\_\_4 = min(i\_\_1,i\_\_2);
00410             \textcolor{keywordflow}{for} (i\_\_ = j + 1; i\_\_ <= i\_\_4; ++i\_\_) \{
00411             temp += a[l + i\_\_ + j * a\_dim1] * x[ix];
00412             ix += *incx;
00413 \textcolor{comment}{/* L150: */}
00414             \}
00415             x[jx] = temp;
00416             jx += *incx;
00417 \textcolor{comment}{/* L160: */}
00418         \}
00419         \}
00420     \}
00421     \}
00422 
00423     \textcolor{keywordflow}{return} 0;
00424 
00425 \textcolor{comment}{/*     End of DTBMV . */}
00426 
00427 \} \textcolor{comment}{/* dtbmv\_ */}
00428 
\end{DoxyCode}
