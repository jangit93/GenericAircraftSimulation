\hypertarget{visual__studio_2zlib_2examples_2gzlog_8h_source}{}\section{visual\+\_\+studio/zlib/examples/gzlog.h}
\label{visual__studio_2zlib_2examples_2gzlog_8h_source}\index{gzlog.\+h@{gzlog.\+h}}

\begin{DoxyCode}
00001 \textcolor{comment}{/* gzlog.h}
00002 \textcolor{comment}{  Copyright (C) 2004, 2008, 2012 Mark Adler, all rights reserved}
00003 \textcolor{comment}{  version 2.2, 14 Aug 2012}
00004 \textcolor{comment}{}
00005 \textcolor{comment}{  This software is provided 'as-is', without any express or implied}
00006 \textcolor{comment}{  warranty.  In no event will the author be held liable for any damages}
00007 \textcolor{comment}{  arising from the use of this software.}
00008 \textcolor{comment}{}
00009 \textcolor{comment}{  Permission is granted to anyone to use this software for any purpose,}
00010 \textcolor{comment}{  including commercial applications, and to alter it and redistribute it}
00011 \textcolor{comment}{  freely, subject to the following restrictions:}
00012 \textcolor{comment}{}
00013 \textcolor{comment}{  1. The origin of this software must not be misrepresented; you must not}
00014 \textcolor{comment}{     claim that you wrote the original software. If you use this software}
00015 \textcolor{comment}{     in a product, an acknowledgment in the product documentation would be}
00016 \textcolor{comment}{     appreciated but is not required.}
00017 \textcolor{comment}{  2. Altered source versions must be plainly marked as such, and must not be}
00018 \textcolor{comment}{     misrepresented as being the original software.}
00019 \textcolor{comment}{  3. This notice may not be removed or altered from any source distribution.}
00020 \textcolor{comment}{}
00021 \textcolor{comment}{  Mark Adler    madler@alumni.caltech.edu}
00022 \textcolor{comment}{ */}
00023 
00024 \textcolor{comment}{/* Version History:}
00025 \textcolor{comment}{   1.0  26 Nov 2004  First version}
00026 \textcolor{comment}{   2.0  25 Apr 2008  Complete redesign for recovery of interrupted operations}
00027 \textcolor{comment}{                     Interface changed slightly in that now path is a prefix}
00028 \textcolor{comment}{                     Compression now occurs as needed during gzlog\_write()}
00029 \textcolor{comment}{                     gzlog\_write() now always leaves the log file as valid gzip}
00030 \textcolor{comment}{   2.1   8 Jul 2012  Fix argument checks in gzlog\_compress() and gzlog\_write()}
00031 \textcolor{comment}{   2.2  14 Aug 2012  Clean up signed comparisons}
00032 \textcolor{comment}{ */}
00033 
00034 \textcolor{comment}{/*}
00035 \textcolor{comment}{   The gzlog object allows writing short messages to a gzipped log file,}
00036 \textcolor{comment}{   opening the log file locked for small bursts, and then closing it.  The log}
00037 \textcolor{comment}{   object works by appending stored (uncompressed) data to the gzip file until}
00038 \textcolor{comment}{   1 MB has been accumulated.  At that time, the stored data is compressed, and}
00039 \textcolor{comment}{   replaces the uncompressed data in the file.  The log file is truncated to}
00040 \textcolor{comment}{   its new size at that time.  After each write operation, the log file is a}
00041 \textcolor{comment}{   valid gzip file that can decompressed to recover what was written.}
00042 \textcolor{comment}{}
00043 \textcolor{comment}{   The gzlog operations can be interupted at any point due to an application or}
00044 \textcolor{comment}{   system crash, and the log file will be recovered the next time the log is}
00045 \textcolor{comment}{   opened with gzlog\_open().}
00046 \textcolor{comment}{ */}
00047 
00048 \textcolor{preprocessor}{#ifndef GZLOG\_H}
00049 \textcolor{preprocessor}{#define GZLOG\_H}
00050 
00051 \textcolor{comment}{/* gzlog object type */}
00052 \textcolor{keyword}{typedef} \textcolor{keywordtype}{void} gzlog;
00053 
00054 \textcolor{comment}{/* Open a gzlog object, creating the log file if it does not exist.  Return}
00055 \textcolor{comment}{   NULL on error.  Note that gzlog\_open() could take a while to complete if it}
00056 \textcolor{comment}{   has to wait to verify that a lock is stale (possibly for five minutes), or}
00057 \textcolor{comment}{   if there is significant contention with other instantiations of this object}
00058 \textcolor{comment}{   when locking the resource.  path is the prefix of the file names created by}
00059 \textcolor{comment}{   this object.  If path is "foo", then the log file will be "foo.gz", and}
00060 \textcolor{comment}{   other auxiliary files will be created and destroyed during the process:}
00061 \textcolor{comment}{   "foo.dict" for a compression dictionary, "foo.temp" for a temporary (next)}
00062 \textcolor{comment}{   dictionary, "foo.add" for data being added or compressed, "foo.lock" for the}
00063 \textcolor{comment}{   lock file, and "foo.repairs" to log recovery operations performed due to}
00064 \textcolor{comment}{   interrupted gzlog operations.  A gzlog\_open() followed by a gzlog\_close()}
00065 \textcolor{comment}{   will recover a previously interrupted operation, if any. */}
00066 gzlog *gzlog\_open(\textcolor{keywordtype}{char} *path);
00067 
00068 \textcolor{comment}{/* Write to a gzlog object.  Return zero on success, -1 if there is a file i/o}
00069 \textcolor{comment}{   error on any of the gzlog files (this should not happen if gzlog\_open()}
00070 \textcolor{comment}{   succeeded, unless the device has run out of space or leftover auxiliary}
00071 \textcolor{comment}{   files have permissions or ownership that prevent their use), -2 if there is}
00072 \textcolor{comment}{   a memory allocation failure, or -3 if the log argument is invalid (e.g. if}
00073 \textcolor{comment}{   it was not created by gzlog\_open()).  This function will write data to the}
00074 \textcolor{comment}{   file uncompressed, until 1 MB has been accumulated, at which time that data}
00075 \textcolor{comment}{   will be compressed.  The log file will be a valid gzip file upon successful}
00076 \textcolor{comment}{   return. */}
00077 \textcolor{keywordtype}{int} gzlog\_write(gzlog *\hyperlink{structlog}{log}, \textcolor{keywordtype}{void} *data, \textcolor{keywordtype}{size\_t} len);
00078 
00079 \textcolor{comment}{/* Force compression of any uncompressed data in the log.  This should be used}
00080 \textcolor{comment}{   sparingly, if at all.  The main application would be when a log file will}
00081 \textcolor{comment}{   not be appended to again.  If this is used to compress frequently while}
00082 \textcolor{comment}{   appending, it will both significantly increase the execution time and}
00083 \textcolor{comment}{   reduce the compression ratio.  The return codes are the same as for}
00084 \textcolor{comment}{   gzlog\_write(). */}
00085 \textcolor{keywordtype}{int} gzlog\_compress(gzlog *\hyperlink{structlog}{log});
00086 
00087 \textcolor{comment}{/* Close a gzlog object.  Return zero on success, -3 if the log argument is}
00088 \textcolor{comment}{   invalid.  The log object is freed, and so cannot be referenced again. */}
00089 \textcolor{keywordtype}{int} gzlog\_close(gzlog *\hyperlink{structlog}{log});
00090 
00091 \textcolor{preprocessor}{#endif}
\end{DoxyCode}
