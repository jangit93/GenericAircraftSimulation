\hypertarget{visual__studio_2_h_d_f5_21_810_81_2include_2_h5_packet_table_8h_source}{}\section{visual\+\_\+studio/\+H\+D\+F5/1.10.1/include/\+H5\+Packet\+Table.h}
\label{visual__studio_2_h_d_f5_21_810_81_2include_2_h5_packet_table_8h_source}\index{H5\+Packet\+Table.\+h@{H5\+Packet\+Table.\+h}}

\begin{DoxyCode}
00001 \textcolor{comment}{/* * * * * * * * * * * * * * * * * * * * * * * * * * * * * * * * * * * * * * *}
00002 \textcolor{comment}{ * Copyright by The HDF Group.                                               *}
00003 \textcolor{comment}{ * Copyright by the Board of Trustees of the University of Illinois.         *}
00004 \textcolor{comment}{ * All rights reserved.                                                      *}
00005 \textcolor{comment}{ *                                                                           *}
00006 \textcolor{comment}{ * This file is part of HDF5.  The full HDF5 copyright notice, including     *}
00007 \textcolor{comment}{ * terms governing use, modification, and redistribution, is contained in    *}
00008 \textcolor{comment}{ * the COPYING file, which can be found at the root of the source code       *}
00009 \textcolor{comment}{ * distribution tree, or in https://support.hdfgroup.org/ftp/HDF5/releases.  *}
00010 \textcolor{comment}{ * If you do not have access to either file, you may request a copy from     *}
00011 \textcolor{comment}{ * help@hdfgroup.org.                                                        *}
00012 \textcolor{comment}{ * * * * * * * * * * * * * * * * * * * * * * * * * * * * * * * * * * * * * * */}
00013 
00014 \textcolor{comment}{/* Packet Table wrapper classes}
00015 \textcolor{comment}{ *}
00016 \textcolor{comment}{ * Wraps the H5PT Packet Table C functions in C++ objects}
00017 \textcolor{comment}{ *}
00018 \textcolor{comment}{ * Nat Furrer and James Laird}
00019 \textcolor{comment}{ * February 2004}
00020 \textcolor{comment}{ */}
00021 
00022 \textcolor{preprocessor}{#ifndef H5PTWRAP\_H}
00023 \textcolor{preprocessor}{#define H5PTWRAP\_H}
00024 
00025 \textcolor{comment}{/* Public HDF5 header */}
00026 \textcolor{preprocessor}{#include "hdf5.h"}
00027 
00028 \textcolor{preprocessor}{#include "H5PTpublic.h"}
00029 \textcolor{preprocessor}{#include "H5api\_adpt.h"}
00030 
00031 \textcolor{keyword}{class }H5\_HLCPPDLL  \hyperlink{class_packet_table}{PacketTable}
00032 \{
00033 \textcolor{keyword}{public}:
00034     \textcolor{comment}{/* Null constructor}
00035 \textcolor{comment}{     * Sets table\_id to "invalid"}
00036 \textcolor{comment}{     */}
00037     \hyperlink{class_packet_table}{PacketTable}() \{table\_id = H5I\_BADID;\}
00038 
00039     \textcolor{comment}{/* "Open" Constructor}
00040 \textcolor{comment}{     * Opens an existing packet table, which can contain either fixed-length or}
00041 \textcolor{comment}{     * variable-length packets.}
00042 \textcolor{comment}{     */}
00043     \hyperlink{class_packet_table}{PacketTable}(hid\_t fileID, \textcolor{keyword}{const} \textcolor{keywordtype}{char}* name);
00044 
00045     \textcolor{comment}{/* "Open" Constructor - will be deprecated because of char* name */}
00046     \hyperlink{class_packet_table}{PacketTable}(hid\_t fileID, \textcolor{keywordtype}{char}* name);
00047 
00048     \textcolor{comment}{/* Destructor}
00049 \textcolor{comment}{     * Cleans up the packet table}
00050 \textcolor{comment}{     */}
00051     \textcolor{keyword}{virtual} ~\hyperlink{class_packet_table}{PacketTable}();
00052 
00053     \textcolor{comment}{/* IsValid}
00054 \textcolor{comment}{     * Returns true if this packet table is valid, false otherwise.}
00055 \textcolor{comment}{     * Use this after the constructor to ensure HDF did not have}
00056 \textcolor{comment}{     * any trouble making or opening the packet table.}
00057 \textcolor{comment}{     */}
00058     \textcolor{keywordtype}{bool} IsValid();
00059 
00060     \textcolor{comment}{/* IsVariableLength}
00061 \textcolor{comment}{     * Return 1 if this packet table uses variable-length datatype,}
00062 \textcolor{comment}{     * return 0 if it is Fixed Length.  Returns -1 if the table is}
00063 \textcolor{comment}{     * invalid (not open).}
00064 \textcolor{comment}{     */}
00065     \textcolor{keywordtype}{int} IsVariableLength();
00066 
00067     \textcolor{comment}{/* ResetIndex}
00068 \textcolor{comment}{     * Sets the "current packet" index to point to the first packet in the}
00069 \textcolor{comment}{     * packet table}
00070 \textcolor{comment}{     */}
00071     \textcolor{keywordtype}{void} ResetIndex();
00072 
00073     \textcolor{comment}{/* SetIndex}
00074 \textcolor{comment}{     * Sets the current packet to point to the packet specified by index.}
00075 \textcolor{comment}{     * Returns 0 on success, negative on failure (if index is out of bounds)}
00076 \textcolor{comment}{     */}
00077     \textcolor{keywordtype}{int} SetIndex(hsize\_t index);
00078 
00079     \textcolor{comment}{/* GetIndex}
00080 \textcolor{comment}{     * Returns the position of the current packet.}
00081 \textcolor{comment}{     * On failure, returns 0 and error is set to negative.}
00082 \textcolor{comment}{     */}
00083     hsize\_t GetIndex(\textcolor{keywordtype}{int}& error);
00084 
00085     \textcolor{comment}{/* GetPacketCount}
00086 \textcolor{comment}{     * Returns the number of packets in the packet table.  Error}
00087 \textcolor{comment}{     * is set to 0 on success.  On failure, returns 0 and}
00088 \textcolor{comment}{     * error is set to negative.}
00089 \textcolor{comment}{     */}
00090     hsize\_t GetPacketCount(\textcolor{keywordtype}{int}& error);
00091 
00092     hsize\_t GetPacketCount()
00093     \{
00094         \textcolor{keywordtype}{int} ignoreError;
00095         \textcolor{keywordflow}{return} GetPacketCount(ignoreError);
00096     \}
00097 
00098     \textcolor{comment}{/* GetTableId}
00099 \textcolor{comment}{     * Returns the identifier of the packet table.}
00100 \textcolor{comment}{     */}
00101     hid\_t GetTableId();
00102 
00103     \textcolor{comment}{/* GetDatatype}
00104 \textcolor{comment}{     * Returns the datatype identifier used by the packet table, on success,}
00105 \textcolor{comment}{     * or FAIL, on failure.}
00106 \textcolor{comment}{     * Note: it is best to avoid using this identifier in applications, unless}
00107 \textcolor{comment}{     * the desired functionality cannot be performed via the packet table ID.}
00108 \textcolor{comment}{     */}
00109     hid\_t GetDatatype();
00110 
00111     \textcolor{comment}{/* GetDataset}
00112 \textcolor{comment}{     * Returns the dataset identifier associated with the packet table, on}
00113 \textcolor{comment}{     * success, or FAIL, on failure.}
00114 \textcolor{comment}{     * Note: it is best to avoid using this identifier in applications, unless}
00115 \textcolor{comment}{     * the desired functionality cannot be performed via the packet table ID.}
00116 \textcolor{comment}{     */}
00117     hid\_t GetDataset();
00118 
00119     \textcolor{comment}{/* FreeBuff}
00120 \textcolor{comment}{     * Frees the buffers created when variable-length packets are read.}
00121 \textcolor{comment}{     * Takes the number of hvl\_t structs to be freed and a pointer to their}
00122 \textcolor{comment}{     * location in memory.}
00123 \textcolor{comment}{     * Returns 0 on success, negative on error.}
00124 \textcolor{comment}{     */}
00125     \textcolor{keywordtype}{int} FreeBuff(\textcolor{keywordtype}{size\_t} numStructs, \hyperlink{structhvl__t}{hvl\_t} * buffer);
00126 
00127 \textcolor{keyword}{protected}:
00128     hid\_t table\_id;
00129 \};
00130 
00131 \textcolor{keyword}{class }H5\_HLCPPDLL \hyperlink{class_f_l___packet_table}{FL\_PacketTable} : \textcolor{keyword}{virtual} \textcolor{keyword}{public} \hyperlink{class_packet_table}{PacketTable}
00132 \{
00133 \textcolor{keyword}{public}:
00134     \textcolor{comment}{/* Constructor}
00135 \textcolor{comment}{     * Creates a packet table to store either fixed- or variable-length packets.}
00136 \textcolor{comment}{     * Takes the ID of the file the packet table will be created in, the ID of}
00137 \textcolor{comment}{     * the property list to specify compression, the name of the packet table,}
00138 \textcolor{comment}{     * the ID of the datatype, and the size of a memory chunk used in chunking.}
00139 \textcolor{comment}{     */}
00140     \hyperlink{class_f_l___packet_table}{FL\_PacketTable}(hid\_t fileID, \textcolor{keyword}{const} \textcolor{keywordtype}{char}* name, hid\_t dtypeID, hsize\_t chunkSize = 0, 
      hid\_t plistID = H5P\_DEFAULT);
00141 
00142     \textcolor{comment}{/* Constructors - deprecated}
00143 \textcolor{comment}{     * Creates a packet table in which to store fixed length packets.}
00144 \textcolor{comment}{     * Takes the ID of the file the packet table will be created in, the name of}
00145 \textcolor{comment}{     * the packet table, the ID of the datatype of the set, the size}
00146 \textcolor{comment}{     * of a memory chunk used in chunking, and the desired compression level}
00147 \textcolor{comment}{     * (0-9, or -1 for no compression).}
00148 \textcolor{comment}{     * Note: these overloaded constructors will be deprecated in favor of the}
00149 \textcolor{comment}{     * constructor above.}
00150 \textcolor{comment}{     */}
00151     \hyperlink{class_f_l___packet_table}{FL\_PacketTable}(hid\_t fileID, hid\_t plist\_id, \textcolor{keyword}{const} \textcolor{keywordtype}{char}* name, hid\_t dtypeID, hsize\_t 
      chunkSize);
00152     \hyperlink{class_f_l___packet_table}{FL\_PacketTable}(hid\_t fileID, \textcolor{keywordtype}{char}* name, hid\_t dtypeID, hsize\_t chunkSize, \textcolor{keywordtype}{int} 
      compression = 0);
00153 
00154     \textcolor{comment}{/* "Open" Constructor}
00155 \textcolor{comment}{     * Opens an existing fixed-length packet table.}
00156 \textcolor{comment}{     * Fails if the packet table specified is variable-length.}
00157 \textcolor{comment}{     */}
00158     \hyperlink{class_f_l___packet_table}{FL\_PacketTable}(hid\_t fileID, \textcolor{keyword}{const} \textcolor{keywordtype}{char}* name);
00159 
00160     \textcolor{comment}{/* "Open" Constructor - will be deprecated because of char* name */}
00161     \hyperlink{class_f_l___packet_table}{FL\_PacketTable}(hid\_t fileID, \textcolor{keywordtype}{char}* name);
00162 
00163     \textcolor{comment}{/* Destructor}
00164 \textcolor{comment}{     * Cleans up the packet table}
00165 \textcolor{comment}{     */}
00166     \textcolor{keyword}{virtual} ~\hyperlink{class_f_l___packet_table}{FL\_PacketTable}() \{\};
00167 
00168     \textcolor{comment}{/* AppendPacket}
00169 \textcolor{comment}{     * Adds a single packet to the packet table.  Takes a pointer}
00170 \textcolor{comment}{     * to the location of the data in memory.}
00171 \textcolor{comment}{     * Returns 0 on success, negative on failure}
00172 \textcolor{comment}{     */}
00173     \textcolor{keywordtype}{int} AppendPacket(\textcolor{keywordtype}{void} * data);
00174 
00175     \textcolor{comment}{/* AppendPackets (multiple packets)}
00176 \textcolor{comment}{     * Adds multiple packets to the packet table.  Takes the number of packets}
00177 \textcolor{comment}{     * to be added and a pointer to their location in memory.}
00178 \textcolor{comment}{     * Returns 0 on success, -1 on failure.}
00179 \textcolor{comment}{     */}
00180     \textcolor{keywordtype}{int} AppendPackets(\textcolor{keywordtype}{size\_t} numPackets, \textcolor{keywordtype}{void} * data);
00181 
00182     \textcolor{comment}{/* GetPacket (indexed)}
00183 \textcolor{comment}{     * Gets a single packet from the packet table.  Takes the index}
00184 \textcolor{comment}{     * of the packet (with 0 being the first packet) and a pointer}
00185 \textcolor{comment}{     * to memory where the data should be stored.}
00186 \textcolor{comment}{     * Returns 0 on success, negative on failure}
00187 \textcolor{comment}{     */}
00188     \textcolor{keywordtype}{int} GetPacket(hsize\_t index, \textcolor{keywordtype}{void} * data);
00189 
00190     \textcolor{comment}{/* GetPackets (multiple packets)}
00191 \textcolor{comment}{     * Gets multiple packets at once, all packets between}
00192 \textcolor{comment}{     * startIndex and endIndex inclusive.  Also takes a pointer to}
00193 \textcolor{comment}{     * the memory where these packets should be stored.}
00194 \textcolor{comment}{     * Returns 0 on success, negative on failure.}
00195 \textcolor{comment}{     */}
00196     \textcolor{keywordtype}{int} GetPackets(hsize\_t startIndex, hsize\_t endIndex, \textcolor{keywordtype}{void} * data);
00197 
00198     \textcolor{comment}{/* GetNextPacket (single packet)}
00199 \textcolor{comment}{     * Gets the next packet in the packet table.  Takes a pointer to}
00200 \textcolor{comment}{     * memory where the packet should be stored.}
00201 \textcolor{comment}{     * Returns 0 on success, negative on failure.  Index}
00202 \textcolor{comment}{     * is not advanced to the next packet on failure.}
00203 \textcolor{comment}{     */}
00204     \textcolor{keywordtype}{int} GetNextPacket(\textcolor{keywordtype}{void} * data);
00205 
00206     \textcolor{comment}{/* GetNextPackets (multiple packets)}
00207 \textcolor{comment}{     * Gets the next numPackets packets in the packet table.  Takes a}
00208 \textcolor{comment}{     * pointer to memory where these packets should be stored.}
00209 \textcolor{comment}{     * Returns 0 on success, negative on failure.  Index}
00210 \textcolor{comment}{     * is not advanced on failure.}
00211 \textcolor{comment}{     */}
00212     \textcolor{keywordtype}{int} GetNextPackets(\textcolor{keywordtype}{size\_t} numPackets, \textcolor{keywordtype}{void} * data);
00213 \};
00214 
00215 \textcolor{comment}{/* Removed "#ifdef VLPT\_REMOVED" block.  03/08/2016, -BMR */}
00216 
00217 \textcolor{preprocessor}{#endif }\textcolor{comment}{/* H5PTWRAP\_H */}\textcolor{preprocessor}{}
\end{DoxyCode}
