\hypertarget{matio_2visual__studio_2test_2eigen_2unsupported_2_eigen_2_c_x_x11_2src_2_tensor_2_tensor_contraction_thread_pool_8h_source}{}\section{matio/visual\+\_\+studio/test/eigen/unsupported/\+Eigen/\+C\+X\+X11/src/\+Tensor/\+Tensor\+Contraction\+Thread\+Pool.h}
\label{matio_2visual__studio_2test_2eigen_2unsupported_2_eigen_2_c_x_x11_2src_2_tensor_2_tensor_contraction_thread_pool_8h_source}\index{Tensor\+Contraction\+Thread\+Pool.\+h@{Tensor\+Contraction\+Thread\+Pool.\+h}}

\begin{DoxyCode}
00001 \textcolor{comment}{// This file is part of Eigen, a lightweight C++ template library}
00002 \textcolor{comment}{// for linear algebra.}
00003 \textcolor{comment}{//}
00004 \textcolor{comment}{// Copyright (C) 2014 Benoit Steiner <benoit.steiner.goog@gmail.com>}
00005 \textcolor{comment}{//}
00006 \textcolor{comment}{// This Source Code Form is subject to the terms of the Mozilla}
00007 \textcolor{comment}{// Public License v. 2.0. If a copy of the MPL was not distributed}
00008 \textcolor{comment}{// with this file, You can obtain one at http://mozilla.org/MPL/2.0/.}
00009 
00010 \textcolor{preprocessor}{#ifndef EIGEN\_CXX11\_TENSOR\_TENSOR\_CONTRACTION\_THREAD\_POOL\_H}
00011 \textcolor{preprocessor}{#define EIGEN\_CXX11\_TENSOR\_TENSOR\_CONTRACTION\_THREAD\_POOL\_H}
00012 
00013 \textcolor{comment}{// evaluator for thread pool device}
00014 \textcolor{preprocessor}{#ifdef EIGEN\_USE\_THREADS}
00015 
00016 \textcolor{keyword}{namespace }\hyperlink{namespace_eigen}{Eigen} \{
00017 
00018 \textcolor{preprocessor}{#ifdef EIGEN\_USE\_SIMPLE\_THREAD\_POOL}
00019 \textcolor{keyword}{namespace }\hyperlink{namespaceinternal}{internal} \{
00020 
00021 \textcolor{keyword}{template}<\textcolor{keyword}{typename} LhsScalar, \textcolor{keyword}{typename} LhsMapper, \textcolor{keyword}{typename} Index>
00022 \textcolor{keyword}{struct }packLhsArg \{
00023   LhsScalar* blockA;
00024   \textcolor{keyword}{const} LhsMapper& lhs;
00025   \textcolor{keyword}{const} \hyperlink{namespace_eigen_a62e77e0933482dafde8fe197d9a2cfde}{Index} m\_start;
00026   \textcolor{keyword}{const} \hyperlink{namespace_eigen_a62e77e0933482dafde8fe197d9a2cfde}{Index} k\_start;
00027   \textcolor{keyword}{const} \hyperlink{namespace_eigen_a62e77e0933482dafde8fe197d9a2cfde}{Index} mc;
00028   \textcolor{keyword}{const} \hyperlink{namespace_eigen_a62e77e0933482dafde8fe197d9a2cfde}{Index} kc;
00029 \};
00030 
00031 \textcolor{keyword}{template}<\textcolor{keyword}{typename} LhsScalar, \textcolor{keyword}{typename} RhsScalar, \textcolor{keyword}{typename} RhsMapper, \textcolor{keyword}{typename} OutputMapper, \textcolor{keyword}{typename} Index>
00032 \textcolor{keyword}{struct }packRhsAndKernelArg \{
00033   \textcolor{keyword}{const} MaxSizeVector<LhsScalar*>* blockAs;
00034   RhsScalar* blockB;
00035   \textcolor{keyword}{const} RhsMapper& rhs;
00036   OutputMapper& output;
00037   \textcolor{keyword}{const} \hyperlink{namespace_eigen_a62e77e0933482dafde8fe197d9a2cfde}{Index} m;
00038   \textcolor{keyword}{const} \hyperlink{namespace_eigen_a62e77e0933482dafde8fe197d9a2cfde}{Index} k;
00039   \textcolor{keyword}{const} \hyperlink{namespace_eigen_a62e77e0933482dafde8fe197d9a2cfde}{Index} n;
00040   \textcolor{keyword}{const} \hyperlink{namespace_eigen_a62e77e0933482dafde8fe197d9a2cfde}{Index} mc;
00041   \textcolor{keyword}{const} \hyperlink{namespace_eigen_a62e77e0933482dafde8fe197d9a2cfde}{Index} kc;
00042   \textcolor{keyword}{const} \hyperlink{namespace_eigen_a62e77e0933482dafde8fe197d9a2cfde}{Index} nc;
00043   \textcolor{keyword}{const} \hyperlink{namespace_eigen_a62e77e0933482dafde8fe197d9a2cfde}{Index} num\_threads;
00044   \textcolor{keyword}{const} \hyperlink{namespace_eigen_a62e77e0933482dafde8fe197d9a2cfde}{Index} num\_blockAs;
00045   \textcolor{keyword}{const} \hyperlink{namespace_eigen_a62e77e0933482dafde8fe197d9a2cfde}{Index} max\_m;
00046   \textcolor{keyword}{const} \hyperlink{namespace_eigen_a62e77e0933482dafde8fe197d9a2cfde}{Index} k\_block\_idx;
00047   \textcolor{keyword}{const} \hyperlink{namespace_eigen_a62e77e0933482dafde8fe197d9a2cfde}{Index} m\_block\_idx;
00048   \textcolor{keyword}{const} \hyperlink{namespace_eigen_a62e77e0933482dafde8fe197d9a2cfde}{Index} n\_block\_idx;
00049   \textcolor{keyword}{const} \hyperlink{namespace_eigen_a62e77e0933482dafde8fe197d9a2cfde}{Index} m\_blocks;
00050   \textcolor{keyword}{const} \hyperlink{namespace_eigen_a62e77e0933482dafde8fe197d9a2cfde}{Index} n\_blocks;
00051   MaxSizeVector<Notification*>* kernel\_notifications;
00052   \textcolor{keyword}{const} MaxSizeVector<Notification*>* lhs\_notifications;
00053   \textcolor{keyword}{const} \textcolor{keywordtype}{bool} need\_to\_pack;
00054 \};
00055 
00056 \}  \textcolor{comment}{// end namespace internal}
00057 \textcolor{preprocessor}{#endif  // EIGEN\_USE\_SIMPLE\_THREAD\_POOL}
00058 
00059 \textcolor{keyword}{template}<\textcolor{keyword}{typename} Indices, \textcolor{keyword}{typename} LeftArgType, \textcolor{keyword}{typename} RightArgType>
00060 \textcolor{keyword}{struct }TensorEvaluator<const TensorContractionOp<Indices, LeftArgType, RightArgType>, ThreadPoolDevice> :
00061     \textcolor{keyword}{public} TensorContractionEvaluatorBase<TensorEvaluator<const TensorContractionOp<Indices, LeftArgType, R
      ightArgType>, ThreadPoolDevice> > \{
00062 
00063   \textcolor{keyword}{typedef} ThreadPoolDevice Device;
00064 
00065   \textcolor{keyword}{typedef} TensorEvaluator<const TensorContractionOp<Indices, LeftArgType, RightArgType>, Device> Self;
00066   \textcolor{keyword}{typedef} TensorContractionEvaluatorBase<Self> Base;
00067 
00068   \textcolor{keyword}{typedef} TensorContractionOp<Indices, LeftArgType, RightArgType> XprType;
00069   \textcolor{keyword}{typedef} \textcolor{keyword}{typename} internal::remove\_const<typename XprType::Scalar>::type Scalar;
00070   \textcolor{keyword}{typedef} \textcolor{keyword}{typename} XprType::Index \hyperlink{namespace_eigen_a62e77e0933482dafde8fe197d9a2cfde}{Index};
00071   \textcolor{keyword}{typedef} \textcolor{keyword}{typename} XprType::CoeffReturnType CoeffReturnType;
00072   \textcolor{keyword}{typedef} \textcolor{keyword}{typename} PacketType<CoeffReturnType, Device>::type PacketReturnType;
00073 
00074   \textcolor{keyword}{enum} \{
00075     Layout = TensorEvaluator<LeftArgType, Device>::Layout,
00076   \};
00077 
00078   \textcolor{comment}{// Most of the code is assuming that both input tensors are ColMajor. If the}
00079   \textcolor{comment}{// inputs are RowMajor, we will "cheat" by swapping the LHS and RHS:}
00080   \textcolor{comment}{// If we want to compute A * B = C, where A is LHS and B is RHS, the code}
00081   \textcolor{comment}{// will pretend B is LHS and A is RHS.}
00082   \textcolor{keyword}{typedef} \textcolor{keyword}{typename} internal::conditional<
00083     \textcolor{keyword}{static\_cast<}\textcolor{keywordtype}{int}\textcolor{keyword}{>}(Layout) == static\_cast<int>(\hyperlink{group__enums_ggaacded1a18ae58b0f554751f6cdf9eb13a0cbd4bdd0abcfc0224c5fcb5e4f6669a}{ColMajor}), LeftArgType, RightArgType>::type 
      EvalLeftArgType;
00084   \textcolor{keyword}{typedef} \textcolor{keyword}{typename} internal::conditional<
00085     \textcolor{keyword}{static\_cast<}\textcolor{keywordtype}{int}\textcolor{keyword}{>}(Layout) == static\_cast<int>(\hyperlink{group__enums_ggaacded1a18ae58b0f554751f6cdf9eb13a0cbd4bdd0abcfc0224c5fcb5e4f6669a}{ColMajor}), RightArgType, LeftArgType>::type 
      EvalRightArgType;
00086 
00087   \textcolor{keyword}{static} \textcolor{keyword}{const} \textcolor{keywordtype}{int} LDims =
00088       internal::array\_size<typename TensorEvaluator<EvalLeftArgType, Device>::Dimensions>::value;
00089   \textcolor{keyword}{static} \textcolor{keyword}{const} \textcolor{keywordtype}{int} RDims =
00090       internal::array\_size<typename TensorEvaluator<EvalRightArgType, Device>::Dimensions>::value;
00091   \textcolor{keyword}{static} \textcolor{keyword}{const} \textcolor{keywordtype}{int} ContractDims = internal::array\_size<Indices>::value;
00092 
00093   \textcolor{keyword}{typedef} array<Index, LDims> left\_dim\_mapper\_t;
00094   \textcolor{keyword}{typedef} array<Index, RDims> right\_dim\_mapper\_t;
00095 
00096   \textcolor{keyword}{typedef} array<Index, ContractDims> contract\_t;
00097   \textcolor{keyword}{typedef} array<\hyperlink{namespace_eigen_a62e77e0933482dafde8fe197d9a2cfde}{Index}, LDims - ContractDims> left\_nocontract\_t;
00098   \textcolor{keyword}{typedef} array<\hyperlink{namespace_eigen_a62e77e0933482dafde8fe197d9a2cfde}{Index}, RDims - ContractDims> right\_nocontract\_t;
00099 
00100   \textcolor{keyword}{static} \textcolor{keyword}{const} \textcolor{keywordtype}{int} NumDims = LDims + RDims - 2 * ContractDims;
00101 
00102   \textcolor{keyword}{typedef} DSizes<Index, NumDims> Dimensions;
00103 
00104   \textcolor{comment}{// typedefs needed in evalTo}
00105   \textcolor{keyword}{typedef} \textcolor{keyword}{typename} internal::remove\_const<typename EvalLeftArgType::Scalar>::type LhsScalar;
00106   \textcolor{keyword}{typedef} \textcolor{keyword}{typename} internal::remove\_const<typename EvalRightArgType::Scalar>::type RhsScalar;
00107   \textcolor{keyword}{typedef} \textcolor{keyword}{typename} internal::gebp\_traits<LhsScalar, RhsScalar> Traits;
00108 
00109   \textcolor{keyword}{typedef} TensorEvaluator<EvalLeftArgType, Device> LeftEvaluator;
00110   \textcolor{keyword}{typedef} TensorEvaluator<EvalRightArgType, Device> RightEvaluator;
00111 
00112   TensorEvaluator(\textcolor{keyword}{const} XprType& op, \textcolor{keyword}{const} Device& device) :
00113       Base(op, device) \{\}
00114 
00115 \textcolor{preprocessor}{#ifndef EIGEN\_USE\_SIMPLE\_THREAD\_POOL}
00116   \textcolor{keyword}{template} <\textcolor{keywordtype}{bool} lhs\_inner\_dim\_contiguous, \textcolor{keywordtype}{bool} rhs\_inner\_dim\_contiguous,
00117             \textcolor{keywordtype}{bool} rhs\_inner\_dim\_reordered, \textcolor{keywordtype}{int} Alignment>
00118   \textcolor{keywordtype}{void} evalProduct(Scalar* buffer)\textcolor{keyword}{ const }\{
00119     \textcolor{keyword}{typedef}
00120         \textcolor{keyword}{typename} internal::remove\_const<typename EvalLeftArgType::Scalar>::type
00121             LhsScalar;
00122     \textcolor{keyword}{typedef}
00123         \textcolor{keyword}{typename} internal::remove\_const<typename EvalRightArgType::Scalar>::type
00124             RhsScalar;
00125     \textcolor{keyword}{typedef} \textcolor{keyword}{typename} internal::gebp\_traits<LhsScalar, RhsScalar> Traits;
00126     \textcolor{keyword}{typedef} TensorEvaluator<EvalLeftArgType, Device> LeftEvaluator;
00127     \textcolor{keyword}{typedef} TensorEvaluator<EvalRightArgType, Device> RightEvaluator;
00128     \textcolor{keyword}{typedef} internal::TensorContractionInputMapper<
00129         LhsScalar, \hyperlink{namespace_eigen_a62e77e0933482dafde8fe197d9a2cfde}{Index}, internal::Lhs, LeftEvaluator, left\_nocontract\_t,
00130         contract\_t, internal::packet\_traits<LhsScalar>::size,
00131         lhs\_inner\_dim\_contiguous, \textcolor{keyword}{false}, \hyperlink{group__enums_gga45fe06e29902b7a2773de05ba27b47a1ac935220b4c844108e183ebe30a4d5204}{Unaligned}>
00132         LhsMapper;
00133     \textcolor{keyword}{typedef} internal::TensorContractionInputMapper<
00134         RhsScalar, \hyperlink{namespace_eigen_a62e77e0933482dafde8fe197d9a2cfde}{Index}, internal::Rhs, RightEvaluator, right\_nocontract\_t,
00135         contract\_t, internal::packet\_traits<RhsScalar>::size,
00136         rhs\_inner\_dim\_contiguous, rhs\_inner\_dim\_reordered, \hyperlink{group__enums_gga45fe06e29902b7a2773de05ba27b47a1ac935220b4c844108e183ebe30a4d5204}{Unaligned}>
00137         RhsMapper;
00138     \textcolor{keyword}{typedef} internal::blas\_data\_mapper<Scalar, Index, ColMajor> OutputMapper;
00139     \textcolor{keyword}{typedef} internal::gemm\_pack\_lhs<LhsScalar, \hyperlink{namespace_eigen_a62e77e0933482dafde8fe197d9a2cfde}{Index},
00140                                     \textcolor{keyword}{typename} LhsMapper::SubMapper, Traits::mr,
00141                                     Traits::LhsProgress, \hyperlink{group__enums_ggaacded1a18ae58b0f554751f6cdf9eb13a0cbd4bdd0abcfc0224c5fcb5e4f6669a}{ColMajor}>
00142         LhsPacker;
00143     \textcolor{keyword}{typedef} internal::gemm\_pack\_rhs<
00144         RhsScalar, \hyperlink{namespace_eigen_a62e77e0933482dafde8fe197d9a2cfde}{Index}, \textcolor{keyword}{typename} RhsMapper::SubMapper, Traits::nr, 
      \hyperlink{group__enums_ggaacded1a18ae58b0f554751f6cdf9eb13a0cbd4bdd0abcfc0224c5fcb5e4f6669a}{ColMajor}>
00145         RhsPacker;
00146     \textcolor{keyword}{typedef} internal::gebp\_kernel<LhsScalar, RhsScalar, \hyperlink{namespace_eigen_a62e77e0933482dafde8fe197d9a2cfde}{Index}, OutputMapper,
00147                                   Traits::mr, Traits::nr, \textcolor{keyword}{false}, \textcolor{keyword}{false}>
00148         GebpKernel;
00149 
00150     \textcolor{keyword}{const} Index m = this->m\_i\_size;
00151     \textcolor{keyword}{const} Index n = this->m\_j\_size;
00152     \textcolor{keyword}{const} Index k = this->m\_k\_size;
00153     \textcolor{keywordflow}{if} (m == 0 || n == 0 || k == 0) \textcolor{keywordflow}{return};
00154 
00155     \textcolor{comment}{// Compute a set of algorithm parameters:}
00156     \textcolor{comment}{// - kernel block sizes (bm, bn, bk)}
00157     \textcolor{comment}{// - task grain sizes (number of kernels executed per task: gm, gn)}
00158     \textcolor{comment}{// - number of threads}
00159     \textcolor{comment}{// - sharding by row/column}
00160     \textcolor{comment}{// - parallel packing or first lhs then rhs}
00161     \textcolor{comment}{// and some derived parameters:}
00162     \textcolor{comment}{// - number of tasks (nm, nn, nk)}
00163     \textcolor{comment}{// - number of kernels (nm0, nn0)}
00164     \textcolor{comment}{// Unfortunately, all these parameters are tightly interdependent.}
00165     \textcolor{comment}{// So in some cases we first compute approximate values, then compute other}
00166     \textcolor{comment}{// values based on these approximations and then refine the approximations.}
00167 
00168     \textcolor{comment}{// There are lots of heuristics here. There is some reasoning behind them,}
00169     \textcolor{comment}{// but ultimately they are just tuned on contraction benchmarks for}
00170     \textcolor{comment}{// different input configurations, thread counts and instruction sets.}
00171     \textcolor{comment}{// So feel free to question any of them.}
00172 
00173     \textcolor{comment}{// Compute whether we want to shard by row or by column.}
00174     \textcolor{comment}{// This is a first approximation, it will be refined later. Since we don't}
00175     \textcolor{comment}{// know number of threads yet we use 2, because what's we are most}
00176     \textcolor{comment}{// interested in at this point is whether it makes sense to use}
00177     \textcolor{comment}{// parallelization at all or not.}
00178     \textcolor{keywordtype}{bool} shard\_by\_col = shardByCol(m, n, 2);
00179 
00180     \textcolor{comment}{// First approximation of kernel blocking sizes.}
00181     \textcolor{comment}{// Again, we don't know number of threads yet, so we use 2.}
00182     Index bm, bn, bk;
00183     \textcolor{keywordflow}{if} (shard\_by\_col) \{
00184       internal::TensorContractionBlocking<LhsMapper, RhsMapper, \hyperlink{namespace_eigen_a62e77e0933482dafde8fe197d9a2cfde}{Index},
00185                                           internal::ShardByCol>
00186           blocking(k, m, n, 2);
00187       bm = blocking.mc();
00188       bn = blocking.nc();
00189       bk = blocking.kc();
00190     \} \textcolor{keywordflow}{else} \{
00191       internal::TensorContractionBlocking<LhsMapper, RhsMapper, \hyperlink{namespace_eigen_a62e77e0933482dafde8fe197d9a2cfde}{Index},
00192                                           internal::ShardByRow>
00193           blocking(k, m, n, 2);
00194       bm = blocking.mc();
00195       bn = blocking.nc();
00196       bk = blocking.kc();
00197     \}
00198 
00199     \textcolor{comment}{// Compute optimal number of threads.}
00200     \textcolor{comment}{// Note: we use bk instead of k here because we are interested in amount of}
00201     \textcolor{comment}{// \_parallelizable\_ computations, and computations are not parallelizable}
00202     \textcolor{comment}{// across k dimension.}
00203     \textcolor{keyword}{const} TensorOpCost cost =
00204         contractionCost(m, n, bm, bn, bk, shard\_by\_col, \textcolor{keyword}{false});
00205     \textcolor{keywordtype}{int} num\_threads = TensorCostModel<ThreadPoolDevice>::numThreads(
00206         static\_cast<double>(n) * m, cost, this->m\_device.numThreads());
00207 
00208     \textcolor{comment}{// TODO(dvyukov): this is a stop-gap to prevent regressions while the cost}
00209     \textcolor{comment}{// model is not tuned. Remove this when the cost model is tuned.}
00210     \textcolor{keywordflow}{if} (n == 1) num\_threads = 1;
00211 
00212     \textcolor{keywordflow}{if} (num\_threads == 1) \{
00213       \textcolor{comment}{// The single-threaded algorithm should be faster in this case.}
00214       \textcolor{keywordflow}{if} (n == 1)
00215         this->\textcolor{keyword}{template} evalGemv<lhs\_inner\_dim\_contiguous,
00216                                 rhs\_inner\_dim\_contiguous,
00217                                 rhs\_inner\_dim\_reordered, Alignment>(buffer);
00218       \textcolor{keywordflow}{else}
00219         this->\textcolor{keyword}{template} evalGemm<lhs\_inner\_dim\_contiguous,
00220                                 rhs\_inner\_dim\_contiguous,
00221                                 rhs\_inner\_dim\_reordered, Alignment>(buffer);
00222       \textcolor{keywordflow}{return};
00223     \}
00224 
00225     \textcolor{comment}{// Now that we know number of threads, recalculate sharding and blocking.}
00226     shard\_by\_col = shardByCol(m, n, num\_threads);
00227     \textcolor{keywordflow}{if} (shard\_by\_col) \{
00228       internal::TensorContractionBlocking<LhsMapper, RhsMapper, \hyperlink{namespace_eigen_a62e77e0933482dafde8fe197d9a2cfde}{Index},
00229                                           internal::ShardByCol>
00230           blocking(k, m, n, num\_threads);
00231       bm = blocking.mc();
00232       bn = blocking.nc();
00233       bk = blocking.kc();
00234     \} \textcolor{keywordflow}{else} \{
00235       internal::TensorContractionBlocking<LhsMapper, RhsMapper, \hyperlink{namespace_eigen_a62e77e0933482dafde8fe197d9a2cfde}{Index},
00236                                           internal::ShardByRow>
00237           blocking(k, m, n, num\_threads);
00238       bm = blocking.mc();
00239       bn = blocking.nc();
00240       bk = blocking.kc();
00241     \}
00242 
00243     \textcolor{comment}{// Number of kernels for each dimension.}
00244     Index nm0 = divup(m, bm);
00245     Index nn0 = divup(n, bn);
00246     Index nk = divup(k, bk);
00247 
00248     \textcolor{comment}{// Calculate task grain size (number of kernels executed per task).}
00249     \textcolor{comment}{// This task size coarsening serves two purposes:}
00250     \textcolor{comment}{// 1. It reduces per-task overheads including synchronization overheads.}
00251     \textcolor{comment}{// 2. It allows to use caches better (reuse the same packed rhs in several}
00252     \textcolor{comment}{// consecutive kernels).}
00253     Index gm = 1;
00254     Index gn = 1;
00255     \textcolor{comment}{// If we are sharding by column, then we prefer to reduce rows first.}
00256     \textcolor{keywordflow}{if} (shard\_by\_col) \{
00257       gm = coarsenM(m, n, bm, bn, bk, gn, num\_threads, shard\_by\_col);
00258       gn = coarsenN(m, n, bm, bn, bk, gm, num\_threads, shard\_by\_col);
00259     \} \textcolor{keywordflow}{else} \{
00260       gn = coarsenN(m, n, bm, bn, bk, gm, num\_threads, shard\_by\_col);
00261       gm = coarsenM(m, n, bm, bn, bk, gn, num\_threads, shard\_by\_col);
00262     \}
00263     \textcolor{comment}{// Number of tasks in each dimension.}
00264     Index nm = divup(nm0, gm);
00265     Index nn = divup(nn0, gn);
00266 
00267     \textcolor{comment}{// Last by not least, decide whether we want to issue both lhs and rhs}
00268     \textcolor{comment}{// packing in parallel; or issue lhs packing first, and then issue rhs}
00269     \textcolor{comment}{// packing when lhs packing completes (for !shard\_by\_col lhs and rhs are}
00270     \textcolor{comment}{// swapped). Parallel packing allows more parallelism (for both packing and}
00271     \textcolor{comment}{// kernels), while sequential packing provides better locality (once}
00272     \textcolor{comment}{// a thread finishes rhs packing it proceed to kernels with that rhs).}
00273     \textcolor{comment}{// First, we are interested in parallel packing if there are few tasks.}
00274     \textcolor{keywordtype}{bool} parallel\_pack = num\_threads >= nm * nn;
00275     \textcolor{comment}{// Also do parallel packing if all data fits into L2$.}
00276     \textcolor{keywordflow}{if} (m * bk * \hyperlink{namespace_eigen_a62e77e0933482dafde8fe197d9a2cfde}{Index}(\textcolor{keyword}{sizeof}(LhsScalar)) + n * bk * \hyperlink{namespace_eigen_a62e77e0933482dafde8fe197d9a2cfde}{Index}(\textcolor{keyword}{sizeof}(RhsScalar)) <=
00277         \hyperlink{namespace_eigen_a2cfc0330ba567d63a496be1cac8427ae}{l2CacheSize}() * num\_threads)
00278       parallel\_pack = \textcolor{keyword}{true};
00279     \textcolor{comment}{// But don't do it if we will use each rhs only once. Locality seems to be}
00280     \textcolor{comment}{// more important in this case.}
00281     \textcolor{keywordflow}{if} ((shard\_by\_col ? nm : nn) == 1) parallel\_pack = \textcolor{keyword}{false};
00282 
00283     LhsMapper lhs(this->m\_leftImpl, this->m\_left\_nocontract\_strides,
00284                   this->m\_i\_strides, this->m\_left\_contracting\_strides,
00285                   this->m\_k\_strides);
00286 
00287     RhsMapper rhs(this->m\_rightImpl, this->m\_right\_nocontract\_strides,
00288                   this->m\_j\_strides, this->m\_right\_contracting\_strides,
00289                   this->m\_k\_strides);
00290 
00291     Context<LhsPacker, RhsPacker, GebpKernel, LhsMapper, RhsMapper,
00292             OutputMapper>(this->m\_device, num\_threads, lhs, rhs, buffer, m, n,
00293                           k, bm, bn, bk, nm, nn, nk, gm, gn, nm0, nn0,
00294                           shard\_by\_col, parallel\_pack)
00295         .run();
00296   \}
00297 
00298   \textcolor{comment}{// Context coordinates a single parallel gemm operation.}
00299   \textcolor{keyword}{template} <\textcolor{keyword}{typename} LhsPacker, \textcolor{keyword}{typename} RhsPacker, \textcolor{keyword}{typename} GebpKernel,
00300             \textcolor{keyword}{typename} LhsMapper, \textcolor{keyword}{typename} RhsMapper, \textcolor{keyword}{typename} OutputMapper>
00301   \textcolor{keyword}{class }Context \{
00302    \textcolor{keyword}{public}:
00303     Context(\textcolor{keyword}{const} Device& device, \textcolor{keywordtype}{int} num\_threads, LhsMapper& lhs,
00304             RhsMapper& rhs, Scalar* buffer, Index tm, Index tn, Index tk, Index bm,
00305             Index bn, Index bk, Index nm, Index nn, Index nk, Index gm,
00306             Index gn, Index nm0, Index nn0, \textcolor{keywordtype}{bool} shard\_by\_col,
00307             \textcolor{keywordtype}{bool} parallel\_pack)
00308         : device\_(device),
00309           lhs\_(lhs),
00310           rhs\_(rhs),
00311           buffer\_(buffer),
00312           output\_(buffer, tm),
00313           num\_threads\_(num\_threads),
00314           shard\_by\_col\_(shard\_by\_col),
00315           parallel\_pack\_(parallel\_pack),
00316           m\_(tm),
00317           n\_(tn),
00318           k\_(tk),
00319           bm\_(bm),
00320           bn\_(bn),
00321           bk\_(bk),
00322           nm\_(nm),
00323           nn\_(nn),
00324           nk\_(nk),
00325           gm\_(gm),
00326           gn\_(gn),
00327           nm0\_(nm0),
00328           nn0\_(nn0)
00329   \{
00330       \textcolor{keywordflow}{for} (Index x = 0; x < P; x++) \{
00331         \textcolor{comment}{// Normal number of notifications for k slice switch is}
00332         \textcolor{comment}{// nm\_ + nn\_ + nm\_ * nn\_. However, first P - 1 slices will receive only}
00333         \textcolor{comment}{// nm\_ + nn\_ notifications, because they will not receive notifications}
00334         \textcolor{comment}{// from preceeding kernels.}
00335         state\_switch\_[x] =
00336             x == 0
00337                 ? 1
00338                 : (parallel\_pack\_ ? nn\_ + nm\_ : (shard\_by\_col\_ ? nn\_ : nm\_)) +
00339                       (x == P - 1 ? nm\_ * nn\_ : 0);
00340         state\_packing\_ready\_[x] =
00341             parallel\_pack\_ ? 0 : (shard\_by\_col\_ ? nm\_ : nn\_);
00342         state\_kernel\_[x] = \textcolor{keyword}{new} std::atomic<uint8\_t>*[nm\_];
00343         \textcolor{keywordflow}{for} (Index m = 0; m < nm\_; m++) \{
00344           state\_kernel\_[x][m] = \textcolor{keyword}{new} std::atomic<uint8\_t>[nn\_];
00345           \textcolor{comment}{// Kernels generally receive 3 notifications (previous kernel + 2}
00346           \textcolor{comment}{// packing), but the first slice won't get notifications from previous}
00347           \textcolor{comment}{// kernels.}
00348           \textcolor{keywordflow}{for} (Index n = 0; n < nn\_; n++)
00349             state\_kernel\_[x][m][n].store(
00350                 (x == 0 ? 0 : 1) + (parallel\_pack\_ ? 2 : 1),
00351                 std::memory\_order\_relaxed);
00352         \}
00353       \}
00354 
00355       \textcolor{comment}{// Allocate memory for packed rhs/lhs matrices.}
00356       \textcolor{keywordtype}{size\_t} align = numext::maxi(EIGEN\_MAX\_ALIGN\_BYTES, 1);
00357       \textcolor{keywordtype}{size\_t} lhs\_size =
00358           divup<size\_t>(bm\_ * bk\_ * \textcolor{keyword}{sizeof}(LhsScalar), align) * align;
00359       \textcolor{keywordtype}{size\_t} rhs\_size =
00360           divup<size\_t>(bn\_ * bk\_ * \textcolor{keyword}{sizeof}(RhsScalar), align) * align;
00361       packed\_mem\_ = \textcolor{keyword}{static\_cast<}\textcolor{keywordtype}{char}*\textcolor{keyword}{>}(internal::aligned\_malloc(
00362           (nm0\_ * lhs\_size + nn0\_ * rhs\_size) * std::min<size\_t>(nk\_, P - 1)));
00363       \textcolor{keywordtype}{char}* mem = \textcolor{keyword}{static\_cast<}\textcolor{keywordtype}{char}*\textcolor{keyword}{>}(packed\_mem\_);
00364       \textcolor{keywordflow}{for} (Index x = 0; x < numext::mini<Index>(nk\_, P - 1); x++) \{
00365         packed\_lhs\_[x].resize(nm0\_);
00366         \textcolor{keywordflow}{for} (Index m = 0; m < nm0\_; m++) \{
00367           packed\_lhs\_[x][m] = \textcolor{keyword}{reinterpret\_cast<}LhsScalar*\textcolor{keyword}{>}(mem);
00368           mem += lhs\_size;
00369         \}
00370         packed\_rhs\_[x].resize(nn0\_);
00371         \textcolor{keywordflow}{for} (Index n = 0; n < nn0\_; n++) \{
00372           packed\_rhs\_[x][n] = \textcolor{keyword}{reinterpret\_cast<}RhsScalar*\textcolor{keyword}{>}(mem);
00373           mem += rhs\_size;
00374         \}
00375       \}
00376     \}
00377 
00378     ~Context() \{
00379       \textcolor{keywordflow}{for} (Index x = 0; x < P; x++) \{
00380         \textcolor{keywordflow}{for} (Index m = 0; m < nm\_; m++) \textcolor{keyword}{delete}[] state\_kernel\_[x][m];
00381         \textcolor{keyword}{delete}[] state\_kernel\_[x];
00382       \}
00383       internal::aligned\_free(packed\_mem\_);
00384     \}
00385 
00386     \textcolor{keywordtype}{void} run() \{
00387       \textcolor{comment}{// Kick off packing of the first slice.}
00388       signal\_switch(0, 1);
00389       \textcolor{comment}{// Wait for overall completion.}
00390       \textcolor{comment}{// TODO(dvyukov): this wait can lead to deadlock.}
00391       \textcolor{comment}{// If nthreads contractions are concurrently submitted from worker}
00392       \textcolor{comment}{// threads, this wait will block all worker threads and the system will}
00393       \textcolor{comment}{// deadlock.}
00394       done\_.Wait();
00395     \}
00396 
00397    \textcolor{keyword}{private}:
00398     Notification done\_;
00399     \textcolor{keyword}{const} Device& device\_;
00400     LhsMapper& lhs\_;
00401     RhsMapper& rhs\_;
00402     Scalar* \textcolor{keyword}{const} buffer\_;
00403     OutputMapper output\_;
00404     \textcolor{keyword}{const} \textcolor{keywordtype}{int} num\_threads\_;
00405     \textcolor{keyword}{const} \textcolor{keywordtype}{bool} shard\_by\_col\_;
00406     \textcolor{keyword}{const} \textcolor{keywordtype}{bool} parallel\_pack\_;
00407     \textcolor{comment}{// Matrix sizes.}
00408     \textcolor{keyword}{const} Index m\_;
00409     \textcolor{keyword}{const} Index n\_;
00410     \textcolor{keyword}{const} Index k\_;
00411     \textcolor{comment}{// Block sizes.}
00412     \textcolor{keyword}{const} Index bm\_;
00413     \textcolor{keyword}{const} Index bn\_;
00414     \textcolor{keyword}{const} Index bk\_;
00415     \textcolor{comment}{// Number of tasks.}
00416     \textcolor{keyword}{const} Index nm\_;
00417     \textcolor{keyword}{const} Index nn\_;
00418     \textcolor{keyword}{const} Index nk\_;
00419     \textcolor{comment}{// Task grain sizes (number of kernels executed per task).}
00420     \textcolor{keyword}{const} Index gm\_;
00421     \textcolor{keyword}{const} Index gn\_;
00422     \textcolor{comment}{// Number of blocks (this is different from ni\_/nn\_ because of task size}
00423     \textcolor{comment}{// coarsening).}
00424     \textcolor{keyword}{const} Index nm0\_;
00425     \textcolor{keyword}{const} Index nn0\_;
00426 
00427     \textcolor{comment}{// Parallelization strategy.}
00428     \textcolor{comment}{//}
00429     \textcolor{comment}{// Blocks related to the same k block can run in parallel because they write}
00430     \textcolor{comment}{// to different output blocks. So we parallelize within k slices, this}
00431     \textcolor{comment}{// gives us parallelism level of m x n. Before we can start any kernels}
00432     \textcolor{comment}{// related to k-th slice, we need to issue m lhs packing tasks and n rhs}
00433     \textcolor{comment}{// packing tasks.}
00434     \textcolor{comment}{//}
00435     \textcolor{comment}{// However, there is a bottleneck when we are finishing kernels for k-th}
00436     \textcolor{comment}{// slice (at the very end there is only 1 runnable kernel). To mitigate this}
00437     \textcolor{comment}{// bottleneck we allow kernels from k-th and k+1-th slices to run in}
00438     \textcolor{comment}{// parallel. Note that (m, n, k) and (m, n, k+1) kernels write to the same}
00439     \textcolor{comment}{// output block, so they must not run in parallel.}
00440     \textcolor{comment}{//}
00441     \textcolor{comment}{// This gives us the following dependency graph.}
00442     \textcolor{comment}{// On each k slice we have m x n kernel tasks, m lhs paking tasks and n rhs}
00443     \textcolor{comment}{// packing tasks.}
00444     \textcolor{comment}{// Kernel (m, n, k) can start when:}
00445     \textcolor{comment}{//  - kernel (m, n, k-1) has finished}
00446     \textcolor{comment}{//  - lhs packing (m, k) has finished}
00447     \textcolor{comment}{//  - rhs packing (n, k) has finished}
00448     \textcolor{comment}{// Lhs/rhs packing can start when:}
00449     \textcolor{comment}{//  - all k-1 packing has finished (artificially imposed to limit amount of}
00450     \textcolor{comment}{//  parallel packing)}
00451     \textcolor{comment}{//}
00452     \textcolor{comment}{// On top of that we limit runnable tasks to two consecutive k slices.}
00453     \textcolor{comment}{// This is done to limit amount of memory we need for packed lhs/rhs}
00454     \textcolor{comment}{// (for each k slice we need m*bk + n*bk memory in packed\_lhs\_/packed\_rhs\_).}
00455     \textcolor{comment}{//}
00456     \textcolor{comment}{// state\_switch\_ tracks when we are ready to switch to the next k slice.}
00457     \textcolor{comment}{// state\_kernel\_[m][n] tracks when we are ready to kick off kernel (m, n).}
00458     \textcolor{comment}{// These variable are rolling over 3 consecutive k slices: first two we are}
00459     \textcolor{comment}{// actively executing + one to track completion of kernels in the second}
00460     \textcolor{comment}{// slice.}
00461     \textcolor{keyword}{static} \textcolor{keyword}{const} Index P = 3;
00462     \textcolor{keywordtype}{void}* packed\_mem\_;
00463     std::vector<LhsScalar*> packed\_lhs\_[P - 1];
00464     std::vector<RhsScalar*> packed\_rhs\_[P - 1];
00465     std::atomic<uint8\_t>** state\_kernel\_[P];
00466     \textcolor{comment}{// state\_switch\_ is frequently modified by worker threads, while other}
00467     \textcolor{comment}{// fields are read-only after constructor. Let's move it to a separate cache}
00468     \textcolor{comment}{// line to reduce cache-coherency traffic.}
00469     \textcolor{keywordtype}{char} pad\_[128];
00470     std::atomic<Index> state\_packing\_ready\_[P];
00471     std::atomic<Index> state\_switch\_[P];
00472 
00473     \textcolor{keywordtype}{void} pack\_lhs(Index m, Index k) \{
00474       \textcolor{keyword}{const} Index mend = m * gm\_ + gm(m);
00475       \textcolor{keywordflow}{for} (Index m1 = m * gm\_; m1 < mend; m1++)
00476         LhsPacker()(packed\_lhs\_[k % (P - 1)][m1],
00477                     lhs\_.getSubMapper(m1 * bm\_, k * bk\_), bk(k), bm(m1));
00478 
00479       \textcolor{keywordflow}{if} (!parallel\_pack\_ && shard\_by\_col\_) \{
00480         signal\_packing(k);
00481       \} \textcolor{keywordflow}{else} \{
00482         signal\_switch(k + 1);
00483         \textcolor{keywordflow}{for} (Index n = nn\_ - 1; n >= 0; n--) signal\_kernel(m, n, k, n == 0);
00484       \}
00485     \}
00486 
00487     \textcolor{keywordtype}{void} pack\_rhs(Index n, Index k) \{
00488       \textcolor{keyword}{const} Index nend = n * gn\_ + gn(n);
00489       \textcolor{keywordflow}{for} (Index n1 = n * gn\_; n1 < nend; n1++) \{
00490         \textcolor{keywordflow}{if} (k == 0) \{
00491           \textcolor{comment}{// Zero the output memory in parallel.}
00492           \textcolor{comment}{// On 10000x2x10000 mm zeroing can easily take half of time.}
00493           \textcolor{comment}{// Zero (bn x m) row. Safe to do here because all kernels that will}
00494           \textcolor{comment}{// write to this memory depend on completion of this task.}
00495           \textcolor{comment}{// Note: don't call device\_.memset() here. device\_.memset() blocks on}
00496           \textcolor{comment}{// thread pool worker thread, which can lead to underutilization and}
00497           \textcolor{comment}{// deadlocks.}
00498           memset(buffer\_ + n1 * bn\_ * m\_, 0, bn(n1) * m\_ * \textcolor{keyword}{sizeof}(Scalar));
00499         \}
00500         RhsPacker()(packed\_rhs\_[k % (P - 1)][n1],
00501                     rhs\_.getSubMapper(k * bk\_, n1 * bn\_), bk(k), bn(n1));
00502       \}
00503 
00504       \textcolor{keywordflow}{if} (parallel\_pack\_ || shard\_by\_col\_) \{
00505         signal\_switch(k + 1);
00506         \textcolor{keywordflow}{for} (Index m = nm\_ - 1; m >= 0; m--) signal\_kernel(m, n, k, m == 0);
00507       \} \textcolor{keywordflow}{else} \{
00508         signal\_packing(k);
00509       \}
00510     \}
00511 
00512     \textcolor{keywordtype}{void} kernel(Index m, Index n, Index k) \{
00513       \textcolor{comment}{// Note: order of iteration matters here. Iteration over m is innermost}
00514       \textcolor{comment}{// because we want to reuse the same packed rhs in consequetive tasks}
00515       \textcolor{comment}{// (rhs fits into L2$ while lhs only into L3$).}
00516       \textcolor{keyword}{const} Index nend = n * gn\_ + gn(n);
00517       \textcolor{keyword}{const} Index mend = m * gm\_ + gm(m);
00518       \textcolor{keywordflow}{if} (shard\_by\_col\_) \{
00519         \textcolor{keywordflow}{for} (Index n1 = n * gn\_; n1 < nend; n1++) \{
00520           \textcolor{keywordflow}{for} (Index m1 = m * gm\_; m1 < mend; m1++)
00521             GebpKernel()(output\_.getSubMapper(m1 * bm\_, n1 * bn\_),
00522                          packed\_lhs\_[k % (P - 1)][m1],
00523                          packed\_rhs\_[k % (P - 1)][n1], bm(m1), bk(k), bn(n1),
00524                          Scalar(1), -1, -1, 0, 0);
00525         \}
00526       \} \textcolor{keywordflow}{else} \{
00527         \textcolor{keywordflow}{for} (Index m1 = m * gm\_; m1 < mend; m1++)
00528           \textcolor{keywordflow}{for} (Index n1 = n * gn\_; n1 < nend; n1++) \{
00529             GebpKernel()(output\_.getSubMapper(m1 * bm\_, n1 * bn\_),
00530                          packed\_lhs\_[k % (P - 1)][m1],
00531                          packed\_rhs\_[k % (P - 1)][n1], bm(m1), bk(k), bn(n1),
00532                          Scalar(1), -1, -1, 0, 0);
00533           \}
00534       \}
00535       signal\_kernel(m, n, k + 1, \textcolor{keyword}{false});
00536       signal\_switch(k + 2);
00537     \}
00538 
00539     \textcolor{keywordtype}{void} signal\_packing(Index k) \{
00540       eigen\_assert(!parallel\_pack\_);
00541       Index s = state\_packing\_ready\_[k % P].fetch\_sub(1);
00542       eigen\_assert(s > 0);
00543       \textcolor{keywordflow}{if} (s != 1) \textcolor{keywordflow}{return};
00544       state\_packing\_ready\_[k % P] = shard\_by\_col\_ ? nm\_ : nn\_;
00545       enqueue\_packing(k, shard\_by\_col\_);
00546     \}
00547 
00548     \textcolor{keywordtype}{void} signal\_kernel(Index m, Index n, Index k, \textcolor{keywordtype}{bool} sync) \{
00549       std::atomic<uint8\_t>* \hyperlink{structstate}{state} = &state\_kernel\_[k % P][m][n];
00550       Index s = state->load();
00551       eigen\_assert(s > 0);
00552       \textcolor{keywordflow}{if} (s != 1 && state->fetch\_sub(1) != 1) \textcolor{keywordflow}{return};
00553       state->store(parallel\_pack\_ ? 3 : 2, std::memory\_order\_relaxed);
00554       \textcolor{keywordflow}{if} (sync)
00555         kernel(m, n, k);
00556       \textcolor{keywordflow}{else}
00557         device\_.enqueueNoNotification([=]() \{ kernel(m, n, k); \});
00558     \}
00559 
00560     \textcolor{keywordtype}{void} signal\_switch(Index k, Index v = 1) \{
00561       Index s = state\_switch\_[k % P].fetch\_sub(v);
00562       eigen\_assert(s >= v);
00563       \textcolor{keywordflow}{if} (s != v) \textcolor{keywordflow}{return};
00564 
00565       \textcolor{comment}{// Ready to switch to the next k slice.}
00566       \textcolor{comment}{// Reset counter for the next iteration.}
00567       state\_switch\_[k % P] =
00568           (parallel\_pack\_ ? nm\_ + nn\_ : (shard\_by\_col\_ ? nn\_ : nm\_)) +
00569           nm\_ * nn\_;
00570       \textcolor{keywordflow}{if} (k < nk\_) \{
00571         \textcolor{comment}{// Issue lhs/rhs packing. Their completion will in turn kick off}
00572         \textcolor{comment}{// kernels.}
00573         \textcolor{keywordflow}{if} (parallel\_pack\_) \{
00574           enqueue\_packing(k, !shard\_by\_col\_);
00575           enqueue\_packing(k, shard\_by\_col\_);
00576         \} \textcolor{keywordflow}{else} \textcolor{keywordflow}{if} (shard\_by\_col\_) \{
00577           enqueue\_packing(k, \textcolor{keyword}{false});
00578         \} \textcolor{keywordflow}{else} \{
00579           enqueue\_packing(k, \textcolor{keyword}{true});
00580         \}
00581 
00582         \textcolor{comment}{// Termination handling.}
00583         \textcolor{comment}{// Because kernel completion signals k + 2 switch, we need to finish nk}
00584         \textcolor{comment}{// + 2 slices without issuing any tasks on nk + 1 slice. So here we}
00585         \textcolor{comment}{// pretend that all nk + 1 packing tasks just finish instantly; so that}
00586         \textcolor{comment}{// nk + 2 switch only waits for completion of nk kernels.}
00587       \} \textcolor{keywordflow}{else} \textcolor{keywordflow}{if} (k == nk\_) \{
00588         signal\_switch(k + 1,
00589                       parallel\_pack\_ ? nm\_ + nn\_ : (shard\_by\_col\_ ? nn\_ : nm\_));
00590       \} \textcolor{keywordflow}{else} \{
00591         done\_.Notify();
00592       \}
00593     \}
00594 
00595     \textcolor{comment}{// Enqueue all rhs/lhs packing for k-th slice.}
00596     \textcolor{keywordtype}{void} enqueue\_packing(Index k, \textcolor{keywordtype}{bool} rhs) \{
00597       enqueue\_packing\_helper(0, rhs ? nn\_ : nm\_, k, rhs);
00598     \}
00599 
00600     \textcolor{keywordtype}{void} enqueue\_packing\_helper(Index start, Index end, Index k, \textcolor{keywordtype}{bool} rhs) \{
00601       \textcolor{keywordflow}{if} (end - start == 1) \{
00602         \textcolor{keywordflow}{if} (rhs)
00603           pack\_rhs(start, k);
00604         \textcolor{keywordflow}{else}
00605           pack\_lhs(start, k);
00606       \} \textcolor{keywordflow}{else} \{
00607         Index mid = (start + end) / 2;
00608         device\_.enqueueNoNotification(
00609             [=]() \{ enqueue\_packing\_helper(mid, end, k, rhs); \});
00610         device\_.enqueueNoNotification(
00611             [=]() \{ enqueue\_packing\_helper(start, mid, k, rhs); \});
00612       \}
00613     \}
00614 
00615     \textcolor{comment}{// Block sizes with accounting for potentially incomplete last block.}
00616     Index bm(Index m)\textcolor{keyword}{ const }\{ \textcolor{keywordflow}{return} m + 1 < nm0\_ ? bm\_ : m\_ + bm\_ - bm\_ * nm0\_; \}
00617     Index bn(Index n)\textcolor{keyword}{ const }\{ \textcolor{keywordflow}{return} n + 1 < nn0\_ ? bn\_ : n\_ + bn\_ - bn\_ * nn0\_; \}
00618     Index bk(Index k)\textcolor{keyword}{ const }\{ \textcolor{keywordflow}{return} k + 1 < nk\_ ? bk\_ : k\_ + bk\_ - bk\_ * nk\_; \}
00619     \textcolor{comment}{// Task grain sizes accounting for potentially incomplete last task.}
00620     Index gm(Index m)\textcolor{keyword}{ const }\{ \textcolor{keywordflow}{return} m + 1 < nm\_ ? gm\_ : nm0\_ + gm\_ - gm\_ * nm\_; \}
00621     Index gn(Index n)\textcolor{keyword}{ const }\{ \textcolor{keywordflow}{return} n + 1 < nn\_ ? gn\_ : nn0\_ + gn\_ - gn\_ * nn\_; \}
00622 
00623     Context(\textcolor{keyword}{const} Context&) = \textcolor{keyword}{delete};
00624     \textcolor{keywordtype}{void} operator=(\textcolor{keyword}{const} Context&) = \textcolor{keyword}{delete};
00625   \};
00626 
00627   \textcolor{comment}{// Decide whether we want to shard m x n contraction by columns or by rows.}
00628   \textcolor{keyword}{static} \textcolor{keywordtype}{bool} shardByCol(Index m, Index n, Index num\_threads) \{
00629     \textcolor{comment}{// Note: we are comparing both n and m against Traits::nr, it is not}
00630     \textcolor{comment}{// a mistake. We are trying to figure out how both n and m will fit into}
00631     \textcolor{comment}{// the main sharding dimension.}
00632 
00633     \textcolor{comment}{// Sharding by column is the default}
00634     \textcolor{comment}{// ... unless there is enough data for vectorization over rows}
00635     \textcolor{keywordflow}{if} (m / num\_threads >= Traits::nr &&
00636         \textcolor{comment}{// and not enough data for vectorization over columns}
00637         (n / num\_threads < Traits::nr ||
00638          \textcolor{comment}{// ... or barely enough data for vectorization over columns,}
00639          \textcolor{comment}{// but it is not evenly dividable across threads}
00640          (n / num\_threads < 4 * Traits::nr &&
00641           (n % (num\_threads * Traits::nr)) != 0 &&
00642           \textcolor{comment}{// ... and it is evenly dividable across threads for rows}
00643           ((m % (num\_threads * Traits::nr)) == 0 ||
00644            \textcolor{comment}{// .. or it is not evenly dividable for both dimensions but}
00645            \textcolor{comment}{// there is much more data over rows so that corner effects are}
00646            \textcolor{comment}{// mitigated.}
00647            (m / n >= 6)))))
00648       \textcolor{keywordflow}{return} \textcolor{keyword}{false};
00649     \textcolor{comment}{// Wait, or if matrices are just substantially prolonged over the other}
00650     \textcolor{comment}{// dimension.}
00651     \textcolor{keywordflow}{if} (n / num\_threads < 16 * Traits::nr && m > n * 32) \textcolor{keywordflow}{return} \textcolor{keyword}{false};
00652     \textcolor{keywordflow}{return} \textcolor{keyword}{true};
00653   \}
00654 
00655   Index coarsenM(Index m, Index n, Index bm, Index bn, Index bk, Index gn,
00656                  \textcolor{keywordtype}{int} num\_threads, \textcolor{keywordtype}{bool} shard\_by\_col)\textcolor{keyword}{ const }\{
00657     Index gm = 1;
00658     Index gm1 = 1;
00659     Index nm0 = divup(m, bm);
00660     Index nm1 = nm0;
00661     \textcolor{keywordflow}{for} (;;) \{
00662       \textcolor{comment}{// Find the next candidate for m grain size. It needs to result in}
00663       \textcolor{comment}{// different number of blocks. E.g. if we have 10 kernels, we want to try}
00664       \textcolor{comment}{// 5 and 10, but not 6, 7, 8 and 9.}
00665       \textcolor{keywordflow}{while} (gm1 <= nm0 && nm1 == divup(nm0, gm1)) gm1++;
00666       \textcolor{keywordflow}{if} (gm1 > nm0) \textcolor{keywordflow}{break};
00667       \textcolor{comment}{// Check the candidate.}
00668       \textcolor{keywordtype}{int} res = checkGrain(m, n, bm, bn, bk, gm1, gn, gm, gn, num\_threads,
00669                            shard\_by\_col);
00670       \textcolor{keywordflow}{if} (res < 0) \textcolor{keywordflow}{break};
00671       nm1 = divup(nm0, gm1);
00672       \textcolor{keywordflow}{if} (res == 0) \textcolor{keywordflow}{continue};
00673       \textcolor{comment}{// Commit new grain size.}
00674       gm = gm1;
00675     \}
00676     \textcolor{keywordflow}{return} gm;
00677   \}
00678 
00679   Index coarsenN(Index m, Index n, Index bm, Index bn, Index bk, Index gm,
00680                  \textcolor{keywordtype}{int} num\_threads, \textcolor{keywordtype}{bool} shard\_by\_col)\textcolor{keyword}{ const }\{
00681     Index gn = 1;
00682     Index gn1 = 1;
00683     Index nn0 = divup(n, bn);
00684     Index nn1 = nn0;
00685     \textcolor{keywordflow}{for} (;;) \{
00686       \textcolor{keywordflow}{while} (gn1 <= nn0 && nn1 == divup(nn0, gn1)) gn1++;
00687       \textcolor{keywordflow}{if} (gn1 > nn0) \textcolor{keywordflow}{break};
00688       \textcolor{keywordtype}{int} res = checkGrain(m, n, bm, bn, bk, gm, gn1, gm, gn, num\_threads,
00689                            shard\_by\_col);
00690       \textcolor{keywordflow}{if} (res < 0) \textcolor{keywordflow}{break};
00691       nn1 = divup(nn0, gn1);
00692       \textcolor{keywordflow}{if} (res == 0) \textcolor{keywordflow}{continue};
00693       gn = gn1;
00694     \}
00695     \textcolor{keywordflow}{return} gn;
00696   \}
00697 
00698   \textcolor{comment}{// checkGrain checks whether grain (gm, gn) is suitable and is better than}
00699   \textcolor{comment}{// (oldgm, oldgn).}
00700   \textcolor{keywordtype}{int} checkGrain(Index m, Index n, Index bm, Index bn, Index bk, Index gm,
00701                  Index gn, Index oldgm, Index oldgn, \textcolor{keywordtype}{int} num\_threads,
00702                  \textcolor{keywordtype}{bool} shard\_by\_col)\textcolor{keyword}{ const }\{
00703     \textcolor{keyword}{const} TensorOpCost cost =
00704         contractionCost(bm * gm, bn * gn, bm, bn, bk, shard\_by\_col, \textcolor{keyword}{true});
00705     \textcolor{keywordtype}{double} taskSize = TensorCostModel<ThreadPoolDevice>::taskSize(
00706         static\_cast<double>(bm) * gm * bn * gn, cost);
00707     \textcolor{comment}{// If the task is too small, then we agree on it regardless of anything}
00708     \textcolor{comment}{// else. Otherwise synchronization overheads will dominate.}
00709     \textcolor{keywordflow}{if} (taskSize < 1) \textcolor{keywordflow}{return} 1;
00710     \textcolor{comment}{// If it is too large, then we reject it and all larger tasks.}
00711     \textcolor{keywordflow}{if} (taskSize > 2) \textcolor{keywordflow}{return} -1;
00712     \textcolor{comment}{// Now we are in presumably good task size range.}
00713     \textcolor{comment}{// The main deciding factor here is parallelism. Consider that we have 12}
00714     \textcolor{comment}{// kernels and 4 threads. Grains of 2, 3 and 4 all yield good task sizes.}
00715     \textcolor{comment}{// But 2/4 yield 6/3 tasks, which gives us parallelism of 0.75 (at most 3/4}
00716     \textcolor{comment}{// of cores will be busy). While grain size 3 gives us 4 tasks, which gives}
00717     \textcolor{comment}{// us parallelism of 1 (we can load all cores).}
00718     Index nm0 = divup(m, bm);
00719     Index nn0 = divup(n, bn);
00720     Index new\_tasks = divup(nm0, gm) * divup(nn0, gn);
00721     \textcolor{keywordtype}{double} new\_parallelism = \textcolor{keyword}{static\_cast<}\textcolor{keywordtype}{double}\textcolor{keyword}{>}(new\_tasks) /
00722                              (divup<int>(new\_tasks, num\_threads) * num\_threads);
00723     Index old\_tasks = divup(nm0, oldgm) * divup(nn0, oldgn);
00724     \textcolor{keywordtype}{double} old\_parallelism = \textcolor{keyword}{static\_cast<}\textcolor{keywordtype}{double}\textcolor{keyword}{>}(old\_tasks) /
00725                              (divup<int>(old\_tasks, num\_threads) * num\_threads);
00726     \textcolor{keywordflow}{if} (new\_parallelism > old\_parallelism || new\_parallelism == 1) \textcolor{keywordflow}{return} 1;
00727     \textcolor{keywordflow}{return} 0;
00728   \}
00729 
00730 \textcolor{preprocessor}{#else  // EIGEN\_USE\_SIMPLE\_THREAD\_POOL}
00731 
00732   \textcolor{keyword}{template} <\textcolor{keywordtype}{bool} lhs\_inner\_dim\_contiguous, \textcolor{keywordtype}{bool} rhs\_inner\_dim\_contiguous, \textcolor{keywordtype}{bool} rhs\_inner\_dim\_reordered, \textcolor{keywordtype}{int}
       Alignment>
00733   \textcolor{keywordtype}{void} evalProduct(Scalar* buffer)\textcolor{keyword}{ const }\{
00734     \textcolor{keywordflow}{if} (this->m\_j\_size == 1) \{
00735       this->\textcolor{keyword}{template} evalGemv<lhs\_inner\_dim\_contiguous, rhs\_inner\_dim\_contiguous, rhs\_inner\_dim\_reordered,
       Alignment>(buffer);
00736       \textcolor{keywordflow}{return};
00737     \}
00738 
00739     evalGemm<lhs\_inner\_dim\_contiguous, rhs\_inner\_dim\_contiguous, rhs\_inner\_dim\_reordered, Alignment>(buffer
      );
00740   \}
00741 
00742   \textcolor{keyword}{template} <\textcolor{keywordtype}{bool} lhs\_inner\_dim\_contiguous, \textcolor{keywordtype}{bool} rhs\_inner\_dim\_contiguous, \textcolor{keywordtype}{bool} rhs\_inner\_dim\_reordered, \textcolor{keywordtype}{int}
       Alignment>
00743   \textcolor{keywordtype}{void} evalGemm(Scalar* buffer)\textcolor{keyword}{ const }\{
00744     \textcolor{comment}{// columns in left side, rows in right side}
00745     \textcolor{keyword}{const} Index k = this->m\_k\_size;
00746 
00747     \textcolor{comment}{// rows in left side}
00748     \textcolor{keyword}{const} Index m = this->m\_i\_size;
00749 
00750     \textcolor{comment}{// columns in right side}
00751     \textcolor{keyword}{const} Index n = this->m\_j\_size;
00752 
00753     \textcolor{comment}{// zero out the result buffer (which must be of size at least m * n * sizeof(Scalar)}
00754     this->m\_device.memset(buffer, 0, m * n * \textcolor{keyword}{sizeof}(Scalar));
00755 
00756 
00757     \textcolor{keyword}{const} \textcolor{keywordtype}{int} lhs\_packet\_size = internal::unpacket\_traits<typename LeftEvaluator::PacketReturnType>::size;
00758     \textcolor{keyword}{const} \textcolor{keywordtype}{int} rhs\_packet\_size = internal::unpacket\_traits<typename RightEvaluator::PacketReturnType>::size;
00759 
00760     \textcolor{keyword}{typedef} internal::TensorContractionInputMapper<LhsScalar, \hyperlink{namespace_eigen_a62e77e0933482dafde8fe197d9a2cfde}{Index}, internal::Lhs,
00761                                                    LeftEvaluator, left\_nocontract\_t,
00762                                                    contract\_t, lhs\_packet\_size,
00763                                                    lhs\_inner\_dim\_contiguous,
00764                                                    \textcolor{keyword}{false}, \hyperlink{group__enums_gga45fe06e29902b7a2773de05ba27b47a1ac935220b4c844108e183ebe30a4d5204}{Unaligned}> LhsMapper;
00765 
00766     \textcolor{keyword}{typedef} internal::TensorContractionInputMapper<RhsScalar, \hyperlink{namespace_eigen_a62e77e0933482dafde8fe197d9a2cfde}{Index}, internal::Rhs,
00767                                                    RightEvaluator, right\_nocontract\_t,
00768                                                    contract\_t, rhs\_packet\_size,
00769                                                    rhs\_inner\_dim\_contiguous,
00770                                                    rhs\_inner\_dim\_reordered, 
      \hyperlink{group__enums_gga45fe06e29902b7a2773de05ba27b47a1ac935220b4c844108e183ebe30a4d5204}{Unaligned}> RhsMapper;
00771 
00772     \textcolor{keyword}{typedef} internal::blas\_data\_mapper<Scalar, Index, ColMajor> OutputMapper;
00773 
00774     \textcolor{comment}{// TODO: packing could be faster sometimes if we supported row major tensor mappers}
00775     \textcolor{keyword}{typedef} internal::gemm\_pack\_lhs<LhsScalar, \hyperlink{namespace_eigen_a62e77e0933482dafde8fe197d9a2cfde}{Index}, \textcolor{keyword}{typename} LhsMapper::SubMapper, Traits::mr,
00776                                     Traits::LhsProgress, \hyperlink{group__enums_ggaacded1a18ae58b0f554751f6cdf9eb13a0cbd4bdd0abcfc0224c5fcb5e4f6669a}{ColMajor}> LhsPacker;
00777     \textcolor{keyword}{typedef} internal::gemm\_pack\_rhs<RhsScalar, Index, typename RhsMapper::SubMapper, Traits::nr, ColMajor> 
      RhsPacker;
00778 
00779     \textcolor{comment}{// TODO: replace false, false with conjugate values?}
00780     \textcolor{keyword}{typedef} internal::gebp\_kernel<LhsScalar, RhsScalar, \hyperlink{namespace_eigen_a62e77e0933482dafde8fe197d9a2cfde}{Index}, OutputMapper,
00781                                   Traits::mr, Traits::nr, \textcolor{keyword}{false}, \textcolor{keyword}{false}> GebpKernel;
00782 
00783     \textcolor{keyword}{typedef} internal::packLhsArg<LhsScalar, LhsMapper, Index> packLArg;
00784     \textcolor{keyword}{typedef} internal::packRhsAndKernelArg<LhsScalar, RhsScalar, RhsMapper, OutputMapper, Index> packRKArg;
00785 
00786     \textcolor{comment}{// initialize data mappers}
00787     LhsMapper lhs(this->m\_leftImpl, this->m\_left\_nocontract\_strides, this->m\_i\_strides,
00788                   this->m\_left\_contracting\_strides, this->m\_k\_strides);
00789 
00790     RhsMapper rhs(this->m\_rightImpl, this->m\_right\_nocontract\_strides, this->m\_j\_strides,
00791                   this->m\_right\_contracting\_strides, this->m\_k\_strides);
00792 
00793     OutputMapper output(buffer, m);
00794 
00795     \textcolor{comment}{// compute block sizes (which depend on number of threads)}
00796     \textcolor{keyword}{const} Index num\_threads = this->m\_device.numThreads();
00797     internal::TensorContractionBlocking<LhsMapper, RhsMapper, Index, internal::ShardByCol> blocking(k, m, n
      , num\_threads);
00798     Index mc = blocking.mc();
00799     Index nc = blocking.nc();
00800     Index kc = blocking.kc();
00801     eigen\_assert(mc <= m);
00802     eigen\_assert(nc <= n);
00803     eigen\_assert(kc <= k);
00804 
00805 \textcolor{preprocessor}{#define CEIL\_DIV(a, b) (((a) + (b) - 1) / (b))}
00806     \textcolor{keyword}{const} Index k\_blocks = CEIL\_DIV(k, kc);
00807     \textcolor{keyword}{const} Index n\_blocks = CEIL\_DIV(n, nc);
00808     \textcolor{keyword}{const} Index m\_blocks = CEIL\_DIV(m, mc);
00809     \textcolor{keyword}{const} Index sizeA = mc * kc;
00810     \textcolor{keyword}{const} Index sizeB = kc * nc;
00811 
00812     \textcolor{comment}{/*    cout << "m: " << m << " n: " << n << " k: " << k << endl;}
00813 \textcolor{comment}{    cout << "mc: " << mc << " nc: " << nc << " kc: " << kc << endl;}
00814 \textcolor{comment}{    cout << "m\_blocks: " << m\_blocks << " n\_blocks: " << n\_blocks << " k\_blocks: " << k\_blocks << endl;}
00815 \textcolor{comment}{    cout << "num threads: " << num\_threads << endl;}
00816 \textcolor{comment}{    */}
00817 
00818     \textcolor{comment}{// note: m\_device.allocate should return 16 byte aligned pointers, but if blockA and blockB}
00819     \textcolor{comment}{//       aren't 16 byte aligned segfaults will happen due to SIMD instructions}
00820     \textcolor{comment}{// note: You can get away with allocating just a single blockA and offsets and meet the}
00821     \textcolor{comment}{//       the alignment requirements with the assumption that}
00822     \textcolor{comment}{//       (Traits::mr * sizeof(ResScalar)) % 16 == 0}
00823     \textcolor{keyword}{const} Index numBlockAs = numext::mini(num\_threads, m\_blocks);
00824     MaxSizeVector<LhsScalar *> blockAs(num\_threads);
00825     \textcolor{keywordflow}{for} (\textcolor{keywordtype}{int} i = 0; i < num\_threads; i++) \{
00826       blockAs.push\_back(static\_cast<LhsScalar *>(this->m\_device.allocate(sizeA * \textcolor{keyword}{sizeof}(LhsScalar))));
00827     \}
00828 
00829     \textcolor{comment}{// To circumvent alignment issues, I'm just going to separately allocate the memory for each thread}
00830     \textcolor{comment}{// TODO: is this too much memory to allocate? This simplifies coding a lot, but is wasteful.}
00831     \textcolor{comment}{//       Other options: (1) reuse memory when a thread finishes. con: tricky}
00832     \textcolor{comment}{//                      (2) allocate block B memory in each thread. con: overhead}
00833     MaxSizeVector<RhsScalar *> blockBs(n\_blocks);
00834     \textcolor{keywordflow}{for} (\textcolor{keywordtype}{int} i = 0; i < n\_blocks; i++) \{
00835       blockBs.push\_back(static\_cast<RhsScalar *>(this->m\_device.allocate(sizeB * \textcolor{keyword}{sizeof}(RhsScalar))));
00836     \}
00837 
00838     \textcolor{comment}{// lhs\_notifications starts with all null Notifications}
00839     MaxSizeVector<Notification*> lhs\_notifications(num\_threads, \textcolor{keyword}{nullptr});
00840 
00841     \textcolor{comment}{// this should really be numBlockAs * n\_blocks;}
00842     \textcolor{keyword}{const} Index num\_kernel\_notifications = num\_threads * n\_blocks;
00843     MaxSizeVector<Notification*> kernel\_notifications(num\_kernel\_notifications,
00844                                                     \textcolor{keyword}{nullptr});
00845 
00846     \textcolor{keywordflow}{for} (Index k\_block\_idx = 0; k\_block\_idx < k\_blocks; k\_block\_idx++) \{
00847       \textcolor{keyword}{const} Index k\_start = k\_block\_idx * kc;
00848       \textcolor{comment}{// make sure we don't overshoot right edge of left matrix}
00849       \textcolor{keyword}{const} Index actual\_kc = numext::mini(k\_start + kc, k) - k\_start;
00850 
00851       \textcolor{keywordflow}{for} (Index m\_block\_idx = 0; m\_block\_idx < m\_blocks; m\_block\_idx += numBlockAs) \{
00852         \textcolor{keyword}{const} Index num\_blocks = numext::mini(m\_blocks-m\_block\_idx, numBlockAs);
00853 
00854         \textcolor{keywordflow}{for} (Index mt\_block\_idx = m\_block\_idx; mt\_block\_idx < m\_block\_idx+num\_blocks; mt\_block\_idx++) \{
00855           \textcolor{keyword}{const} Index m\_start = mt\_block\_idx * mc;
00856           \textcolor{keyword}{const} Index actual\_mc = numext::mini(m\_start + mc, m) - m\_start;
00857           eigen\_assert(actual\_mc > 0);
00858 
00859           Index blockAId = (k\_block\_idx * m\_blocks + mt\_block\_idx) % num\_threads;
00860 
00861           \textcolor{keywordflow}{for} (\textcolor{keywordtype}{int} i = 0; i < n\_blocks; ++i) \{
00862             Index notification\_id = (blockAId * n\_blocks + i);
00863             \textcolor{comment}{// Wait for any current kernels using this slot to complete}
00864             \textcolor{comment}{// before using it.}
00865             \textcolor{keywordflow}{if} (kernel\_notifications[notification\_id]) \{
00866               wait\_until\_ready(kernel\_notifications[notification\_id]);
00867               \textcolor{keyword}{delete} kernel\_notifications[notification\_id];
00868             \}
00869             kernel\_notifications[notification\_id] = \textcolor{keyword}{new} Notification();
00870           \}
00871           \textcolor{keyword}{const} packLArg arg = \{
00872             blockAs[blockAId], \textcolor{comment}{// blockA}
00873             lhs,        \textcolor{comment}{// lhs}
00874             m\_start,    \textcolor{comment}{// m}
00875             k\_start,    \textcolor{comment}{// k}
00876             actual\_mc,  \textcolor{comment}{// mc}
00877             actual\_kc,  \textcolor{comment}{// kc}
00878           \};
00879 
00880           \textcolor{comment}{// Delete any existing notification since we may be}
00881           \textcolor{comment}{// replacing it.  The algorithm should ensure that there are}
00882           \textcolor{comment}{// no existing waiters on this notification.}
00883           \textcolor{keyword}{delete} lhs\_notifications[blockAId];
00884           lhs\_notifications[blockAId] =
00885           this->m\_device.enqueue(&Self::packLhs<packLArg, LhsPacker>, arg);
00886         \}
00887 
00888         \textcolor{comment}{// now start kernels.}
00889         \textcolor{keyword}{const} Index m\_base\_start = m\_block\_idx * mc;
00890         \textcolor{keyword}{const} \textcolor{keywordtype}{bool} need\_to\_pack = m\_block\_idx == 0;
00891 
00892         \textcolor{keywordflow}{for} (Index n\_block\_idx = 0; n\_block\_idx < n\_blocks; n\_block\_idx++) \{
00893           \textcolor{keyword}{const} Index n\_start = n\_block\_idx * nc;
00894           \textcolor{keyword}{const} Index actual\_nc = numext::mini(n\_start + nc, n) - n\_start;
00895 
00896           \textcolor{comment}{// first make sure the previous kernels are all done before overwriting rhs. Also wait if}
00897           \textcolor{comment}{// we're going to start new k. In both cases need\_to\_pack is true.}
00898           \textcolor{keywordflow}{if} (need\_to\_pack) \{
00899             \textcolor{keywordflow}{for} (Index i = num\_blocks; i < num\_threads; ++i) \{
00900               Index blockAId = (k\_block\_idx * m\_blocks + i + m\_block\_idx) % num\_threads;
00901               Index future\_id = (blockAId * n\_blocks + n\_block\_idx);
00902               wait\_until\_ready(kernel\_notifications[future\_id]);
00903             \}
00904           \}
00905 
00906           packRKArg arg = \{
00907             &blockAs, \textcolor{comment}{// blockA}
00908             blockBs[n\_block\_idx], \textcolor{comment}{// blockB}
00909             rhs,          \textcolor{comment}{// rhs}
00910             output,       \textcolor{comment}{// output}
00911             m\_base\_start, \textcolor{comment}{// m}
00912             k\_start,      \textcolor{comment}{// k}
00913             n\_start,      \textcolor{comment}{// n}
00914             mc,           \textcolor{comment}{// mc}
00915             actual\_kc,    \textcolor{comment}{// kc}
00916             actual\_nc,    \textcolor{comment}{// nc}
00917             num\_threads,
00918             numBlockAs,
00919             m,
00920             k\_block\_idx,
00921             m\_block\_idx,
00922             n\_block\_idx, \textcolor{comment}{// n\_block\_idx}
00923             m\_blocks, \textcolor{comment}{// m\_blocks}
00924             n\_blocks, \textcolor{comment}{// n\_blocks}
00925             &kernel\_notifications, \textcolor{comment}{// kernel notifications}
00926             &lhs\_notifications,    \textcolor{comment}{// lhs notifications}
00927             need\_to\_pack, \textcolor{comment}{// need\_to\_pack}
00928           \};
00929 
00930           \textcolor{comment}{// We asynchronously kick off this function, which ends up}
00931           \textcolor{comment}{// notifying the appropriate kernel\_notifications objects,}
00932           \textcolor{comment}{// which this thread waits on before exiting.}
00933           this->m\_device.enqueueNoNotification(&Self::packRhsAndKernel<packRKArg, RhsPacker, GebpKernel>, 
      arg);
00934         \}
00935       \}
00936     \}
00937 
00938     \textcolor{comment}{// Make sure all the kernels are done.}
00939     \textcolor{keywordflow}{for} (\textcolor{keywordtype}{size\_t} i = 0; i < kernel\_notifications.size(); ++i) \{
00940       wait\_until\_ready(kernel\_notifications[i]);
00941       \textcolor{keyword}{delete} kernel\_notifications[i];
00942     \}
00943 
00944     \textcolor{comment}{// No need to wait for lhs notifications since they should have}
00945     \textcolor{comment}{// already been waited on.  Just clean them up.}
00946     \textcolor{keywordflow}{for} (\textcolor{keywordtype}{size\_t} i = 0; i < lhs\_notifications.size(); ++i) \{
00947       \textcolor{keyword}{delete} lhs\_notifications[i];
00948     \}
00949 
00950     \textcolor{comment}{// deallocate all of the memory for both A and B's}
00951     \textcolor{keywordflow}{for} (\textcolor{keywordtype}{size\_t} i = 0; i < blockAs.size(); i++) \{
00952       this->m\_device.deallocate(blockAs[i]);
00953     \}
00954     \textcolor{keywordflow}{for} (\textcolor{keywordtype}{size\_t} i = 0; i < blockBs.size(); i++) \{
00955       this->m\_device.deallocate(blockBs[i]);
00956     \}
00957 
00958 \textcolor{preprocessor}{#undef CEIL\_DIV}
00959   \}
00960 
00961   \textcolor{comment}{/*}
00962 \textcolor{comment}{   * Packs a LHS block of size (mt, kc) starting at lhs(m, k). Before packing}
00963 \textcolor{comment}{   * the LHS block, check that all of the kernels that worked on the same}
00964 \textcolor{comment}{   * mt\_block\_idx in the previous m\_block are done.}
00965 \textcolor{comment}{   */}
00966   \textcolor{keyword}{template} <\textcolor{keyword}{typename} packLArg, \textcolor{keyword}{typename} LhsPacker>
00967   \textcolor{keyword}{static} \textcolor{keywordtype}{void} packLhs(\textcolor{keyword}{const} packLArg arg) \{
00968     \textcolor{comment}{// perform actual packing}
00969     LhsPacker pack\_lhs;
00970     pack\_lhs(arg.blockA, arg.lhs.getSubMapper(arg.m\_start, arg.k\_start), arg.kc, arg.mc);
00971   \}
00972 
00973   \textcolor{comment}{/*}
00974 \textcolor{comment}{   * Packs a RHS block of size (kc, nc) starting at (k, n) after checking that}
00975 \textcolor{comment}{   * all kernels in the previous block are done.}
00976 \textcolor{comment}{   * Then for each LHS future, we wait on the future and then call GEBP}
00977 \textcolor{comment}{   * on the area packed by the future (which starts at}
00978 \textcolor{comment}{   * blockA + future\_idx * mt * kc) on the LHS and with the full packed}
00979 \textcolor{comment}{   * RHS block.}
00980 \textcolor{comment}{   * The output of this GEBP is written to output(m + i * mt, n).}
00981 \textcolor{comment}{   */}
00982   \textcolor{keyword}{template} <\textcolor{keyword}{typename} packRKArg, \textcolor{keyword}{typename} RhsPacker, \textcolor{keyword}{typename} GebpKernel>
00983   \textcolor{keyword}{static} \textcolor{keywordtype}{void} packRhsAndKernel(packRKArg arg) \{
00984     \textcolor{keywordflow}{if} (arg.need\_to\_pack) \{
00985       RhsPacker pack\_rhs;
00986       pack\_rhs(arg.blockB, arg.rhs.getSubMapper(arg.k, arg.n), arg.kc, arg.nc);
00987     \}
00988 
00989     GebpKernel gebp;
00990     \textcolor{keywordflow}{for} (Index mt\_block\_idx = 0; mt\_block\_idx < arg.num\_blockAs; mt\_block\_idx++) \{
00991       \textcolor{keyword}{const} Index m\_base\_start = arg.m + arg.mc*mt\_block\_idx;
00992       \textcolor{keywordflow}{if} (m\_base\_start < arg.max\_m) \{
00993         Index blockAId = (arg.k\_block\_idx * arg.m\_blocks + mt\_block\_idx + arg.m\_block\_idx) % arg.
      num\_threads;
00994         wait\_until\_ready((*arg.lhs\_notifications)[blockAId]);
00995         \textcolor{keyword}{const} Index actual\_mc = numext::mini(m\_base\_start + arg.mc, arg.max\_m) - m\_base\_start;
00996         gebp(arg.output.getSubMapper(m\_base\_start, arg.n),
00997              (*arg.blockAs)[blockAId], arg.blockB,
00998              actual\_mc, arg.kc, arg.nc, Scalar(1), -1, -1, 0, 0);
00999 
01000         \textcolor{comment}{// Notify that the kernel is done.}
01001         \textcolor{keyword}{const} Index set\_idx = blockAId * arg.n\_blocks + arg.n\_block\_idx;
01002         (*arg.kernel\_notifications)[set\_idx]->Notify();
01003       \}
01004     \}
01005   \}
01006 \textcolor{preprocessor}{#endif  // EIGEN\_USE\_SIMPLE\_THREAD\_POOL}
01007 
01008   TensorOpCost contractionCost(Index m, Index n, Index bm, Index bn, Index bk,
01009                                \textcolor{keywordtype}{bool} shard\_by\_col, \textcolor{keywordtype}{bool} prepacked)\textcolor{keyword}{ const }\{
01010     \textcolor{keyword}{const} \textcolor{keywordtype}{int} packed\_size = std::min<int>(PacketType<LhsScalar, Device>::size,
01011                                           PacketType<RhsScalar, Device>::size);
01012     \textcolor{keyword}{const} \textcolor{keywordtype}{int} output\_packet\_size = internal::unpacket\_traits<PacketReturnType>::size;
01013     \textcolor{keyword}{const} \textcolor{keywordtype}{double} kd = \textcolor{keyword}{static\_cast<}\textcolor{keywordtype}{double}\textcolor{keyword}{>}(bk);
01014     \textcolor{comment}{// Peak VFMA bandwidth is 0.5. However if we have not enough data for}
01015     \textcolor{comment}{// vectorization bandwidth drops. The 4.0 and 2.0 bandwidth is determined}
01016     \textcolor{comment}{// experimentally.}
01017     \textcolor{keywordtype}{double} computeBandwidth = bk == 1 ? 4.0 :
01018           (shard\_by\_col ? bn : bm) < Traits::nr ||
01019           (shard\_by\_col ? bm : bn) < Traits::mr ? 2.0 : 0.5;
01020 \textcolor{preprocessor}{#ifndef EIGEN\_VECTORIZE\_FMA}
01021     \textcolor{comment}{// Bandwidth of all of VFMA/MULPS/ADDPS is 0.5 on latest Intel processors.}
01022     \textcolor{comment}{// However for MULPS/ADDPS we have dependent sequence of 2 such instructions,}
01023     \textcolor{comment}{// so overall bandwidth is 1.0.}
01024     \textcolor{keywordflow}{if} (computeBandwidth == 0.5) computeBandwidth = 1.0;
01025 \textcolor{preprocessor}{#endif}
01026     \textcolor{comment}{// Computations.}
01027     TensorOpCost cost = TensorOpCost(0, 0, kd * computeBandwidth, \textcolor{keyword}{true}, packed\_size);
01028     \textcolor{comment}{// Output stores.}
01029     cost += TensorOpCost(0, \textcolor{keyword}{sizeof}(CoeffReturnType), 0, \textcolor{keyword}{true}, output\_packet\_size);
01030     \textcolor{keywordflow}{if} (prepacked) \{
01031       \textcolor{comment}{// Packing and kernels are executed in different tasks. When we calculate}
01032       \textcolor{comment}{// task grain size we look only at kernel cost assuming that kernel}
01033       \textcolor{comment}{// is more expensive than packing.}
01034       \textcolor{keywordflow}{return} cost;
01035     \}
01036     \textcolor{comment}{// Lhs/rhs loads + computations.}
01037     TensorOpCost lhsCost = this->m\_leftImpl.costPerCoeff(\textcolor{keyword}{true}) * (kd / n);
01038     TensorOpCost rhsCost = this->m\_rightImpl.costPerCoeff(\textcolor{keyword}{true}) * (kd / m);
01039     \textcolor{comment}{// Lhs packing memory cost does not contribute considerably to overall}
01040     \textcolor{comment}{// execution time because lhs is prefetched early and accessed sequentially.}
01041     \textcolor{keywordflow}{if} (shard\_by\_col)
01042       lhsCost.dropMemoryCost();
01043     \textcolor{keywordflow}{else}
01044       rhsCost.dropMemoryCost();
01045     \textcolor{keywordflow}{return} cost + lhsCost + rhsCost;
01046   \}
01047 \};
01048 
01049 \} \textcolor{comment}{// end namespace Eigen}
01050 
01051 \textcolor{preprocessor}{#endif  // EIGEN\_USE\_THREADS}
01052 \textcolor{preprocessor}{#endif // EIGEN\_CXX11\_TENSOR\_TENSOR\_CONTRACTION\_THREAD\_POOL\_H}
\end{DoxyCode}
