\hypertarget{_h_d_f5_21_810_81_2_h_d_f5_examples_2_f_o_r_t_r_a_n_2_h5_t_2h5ex__t__vlen___f03_8f90_source}{}\section{H\+D\+F5/1.10.1/\+H\+D\+F5\+Examples/\+F\+O\+R\+T\+R\+A\+N/\+H5\+T/h5ex\+\_\+t\+\_\+vlen\+\_\+\+F03.f90}
\label{_h_d_f5_21_810_81_2_h_d_f5_examples_2_f_o_r_t_r_a_n_2_h5_t_2h5ex__t__vlen___f03_8f90_source}\index{h5ex\+\_\+t\+\_\+vlen\+\_\+\+F03.\+f90@{h5ex\+\_\+t\+\_\+vlen\+\_\+\+F03.\+f90}}

\begin{DoxyCode}
00001 \textcolor{comment}{!************************************************************}
00002 \textcolor{comment}{!}
00003 \textcolor{comment}{!  This example shows how to read and write variable-length}
00004 \textcolor{comment}{!  datatypes to a dataset.  The program first writes two}
00005 \textcolor{comment}{!  variable-length integer arrays to a dataset then closes}
00006 \textcolor{comment}{!  the file.  Next, it reopens the file, reads back the data,}
00007 \textcolor{comment}{!  and outputs it to the screen.}
00008 \textcolor{comment}{!}
00009 \textcolor{comment}{!  The data structure is a matrix which is has 2 rows}
00010 \textcolor{comment}{!  and the number of columns varies in each row, for }
00011 \textcolor{comment}{!  this example row 1 has LEN0 columns and row 2 has LEN1 columns}
00012 \textcolor{comment}{!}
00013 \textcolor{comment}{!  This file is intended for use with HDF5 Library version 1.8}
00014 \textcolor{comment}{!  with --enable-fortran2003 }
00015 \textcolor{comment}{!}
00016 \textcolor{comment}{!************************************************************}
00017 
00018 \textcolor{keyword}{PROGRAM} main
00019 
00020   \textcolor{keywordtype}{USE }hdf5
00021   \textcolor{keywordtype}{USE }iso\_c\_binding
00022   
00023   \textcolor{keywordtype}{IMPLICIT NONE}
00024 
00025   \textcolor{keywordtype}{CHARACTER(LEN=18)}, \textcolor{keywordtype}{PARAMETER} :: filename  = \textcolor{stringliteral}{"h5ex\_t\_vlen\_F03.h5"}
00026   \textcolor{keywordtype}{CHARACTER(LEN=3)} , \textcolor{keywordtype}{PARAMETER} :: dataset   = \textcolor{stringliteral}{"DS1"}
00027   \textcolor{keywordtype}{INTEGER}, \textcolor{keywordtype}{PARAMETER} :: len0 = 3
00028   \textcolor{keywordtype}{INTEGER}, \textcolor{keywordtype}{PARAMETER} :: len1 = 12
00029 
00030   \textcolor{keywordtype}{INTEGER(HID\_T)}  :: \hyperlink{structfile}{file}, filetype, memtype, space, dset \textcolor{comment}{! Handles}
00031   \textcolor{keywordtype}{INTEGER} :: hdferr
00032   \textcolor{keywordtype}{INTEGER(HSIZE\_T)}, \textcolor{keywordtype}{DIMENSION(1:2)}   :: maxdims
00033   \textcolor{keywordtype}{INTEGER} :: i, j
00034 
00035   \textcolor{comment}{! vl data}
\Hypertarget{_h_d_f5_21_810_81_2_h_d_f5_examples_2_f_o_r_t_r_a_n_2_h5_t_2h5ex__t__vlen___f03_8f90_source_l00036}\hyperlink{structvl}{00036}   \textcolor{keyword}{TYPE} \hyperlink{structvl}{vl}
00037      \textcolor{keywordtype}{INTEGER}, \textcolor{keywordtype}{DIMENSION(:)}, \textcolor{keywordtype}{POINTER} :: data
00038 \textcolor{keyword}{  END TYPE }\hyperlink{structvl}{vl}
00039   \textcolor{keywordtype}{TYPE}(\hyperlink{structvl}{vl}), \textcolor{keywordtype}{DIMENSION(:)}, \textcolor{keywordtype}{ALLOCATABLE} :: ptr
00040 
00041   \textcolor{keywordtype}{TYPE}(\hyperlink{structhvl__t}{hvl\_t}), \textcolor{keywordtype}{DIMENSION(1:2)}, \textcolor{keywordtype}{TARGET} :: wdata \textcolor{comment}{! Array of vlen structures}
00042   \textcolor{keywordtype}{TYPE}(\hyperlink{structhvl__t}{hvl\_t}), \textcolor{keywordtype}{DIMENSION(1:2)}, \textcolor{keywordtype}{TARGET} :: rdata \textcolor{comment}{! Pointer to vlen structures}
00043 
00044   \textcolor{keywordtype}{INTEGER(hsize\_t)}, \textcolor{keywordtype}{DIMENSION(1:1)} :: dims = (/2/)
00045   \textcolor{keywordtype}{INTEGER}, \textcolor{keywordtype}{DIMENSION(:)}, \textcolor{keywordtype}{POINTER} :: ptr\_r 
00046   \textcolor{keywordtype}{TYPE}(c\_ptr) :: f\_ptr
00047   
00048   \textcolor{comment}{!}
00049   \textcolor{comment}{! Initialize FORTRAN interface.}
00050   \textcolor{comment}{!}
00051   \textcolor{keyword}{CALL }h5open\_f(hdferr)
00052   \textcolor{comment}{!}
00053   \textcolor{comment}{! Initialize variable-length data.  wdata(1) is a countdown of}
00054   \textcolor{comment}{! length LEN0, wdata(2) is a Fibonacci sequence of length LEN1.}
00055   \textcolor{comment}{!}
00056   wdata(1)%len = len0
00057   wdata(2)%len = len1
00058 
00059   \textcolor{keyword}{ALLOCATE}( ptr(1:2) )
00060   \textcolor{keyword}{ALLOCATE}( ptr(1)%data(1:wdata(1)%len) )
00061   \textcolor{keyword}{ALLOCATE}( ptr(2)%data(1:wdata(2)%len) )
00062 
00063   \textcolor{keywordflow}{DO} i=1, wdata(1)%len
00064      ptr(1)%data(i) = wdata(1)%len - i + 1 \textcolor{comment}{! 3 2 1}
00065 \textcolor{keywordflow}{  ENDDO}
00066   wdata(1)%p = c\_loc(ptr(1)%data(1))
00067 
00068   ptr(2)%data(1:2) = 1
00069   \textcolor{keywordflow}{DO} i = 3, wdata(2)%len
00070      ptr(2)%data(i) = ptr(2)%data(i-1) + ptr(2)%data(i-2) \textcolor{comment}{! (1 1 2 3 5 8 etc.)}
00071 \textcolor{keywordflow}{  ENDDO}
00072   wdata(2)%p = c\_loc(ptr(2)%data(1))
00073 
00074   \textcolor{comment}{!}
00075   \textcolor{comment}{! Create a new file using the default properties.}
00076   \textcolor{comment}{!}
00077   \textcolor{keyword}{CALL }h5fcreate\_f(filename, h5f\_acc\_trunc\_f, \hyperlink{structfile}{file}, hdferr)
00078   \textcolor{comment}{!}
00079   \textcolor{comment}{! Create variable-length datatype for file and memory.}
00080   \textcolor{comment}{!}
00081   \textcolor{keyword}{CALL }h5tvlen\_create\_f(h5t\_std\_i32le, filetype, hdferr)
00082   \textcolor{keyword}{CALL }h5tvlen\_create\_f(h5t\_native\_integer, memtype, hdferr)
00083   \textcolor{comment}{!}
00084   \textcolor{comment}{! Create dataspace.}
00085   \textcolor{comment}{!}
00086   \textcolor{keyword}{CALL }h5screate\_simple\_f(1, dims, space, hdferr)
00087   \textcolor{comment}{!}
00088   \textcolor{comment}{! Create the dataset and write the variable-length data to it.}
00089   \textcolor{comment}{!}
00090   \textcolor{keyword}{CALL }h5dcreate\_f(\hyperlink{structfile}{file}, dataset, filetype, space, dset, hdferr)
00091  
00092   f\_ptr = c\_loc(wdata(1))
00093   \textcolor{keyword}{CALL }h5dwrite\_f(dset, memtype, f\_ptr, hdferr)
00094 
00095   \textcolor{keyword}{CALL }h5dclose\_f(dset , hdferr)
00096   \textcolor{keyword}{CALL }h5sclose\_f(space, hdferr)
00097   \textcolor{keyword}{CALL }h5tclose\_f(filetype, hdferr)
00098   \textcolor{keyword}{CALL }h5tclose\_f(memtype, hdferr)
00099   \textcolor{keyword}{CALL }h5fclose\_f(\hyperlink{structfile}{file} , hdferr)
00100   \textcolor{keyword}{DEALLOCATE}(ptr)
00101 
00102   \textcolor{comment}{!}
00103   \textcolor{comment}{! Now we begin the read section of this example.}
00104 
00105   \textcolor{comment}{!}
00106   \textcolor{comment}{! Open file and dataset.}
00107   \textcolor{comment}{!}
00108   \textcolor{keyword}{CALL }h5fopen\_f(filename, h5f\_acc\_rdonly\_f, \hyperlink{structfile}{file}, hdferr)
00109   \textcolor{keyword}{CALL }h5dopen\_f(\hyperlink{structfile}{file}, dataset, dset, hdferr)
00110 
00111   \textcolor{comment}{!}
00112   \textcolor{comment}{! Get dataspace and allocate memory for array of vlen structures.}
00113   \textcolor{comment}{! This does not actually allocate memory for the vlen data, that}
00114   \textcolor{comment}{! will be done by the library.}
00115   \textcolor{comment}{!}
00116   \textcolor{keyword}{CALL }h5dget\_space\_f(dset, space, hdferr)
00117   \textcolor{keyword}{CALL }h5sget\_simple\_extent\_dims\_f(space, dims, maxdims, hdferr) 
00118 
00119   \textcolor{comment}{!}
00120   \textcolor{comment}{! Create the memory datatype.}
00121   \textcolor{comment}{!}
00122   \textcolor{keyword}{CALL }h5tvlen\_create\_f(h5t\_native\_integer, memtype, hdferr)
00123 
00124   \textcolor{comment}{!}
00125   \textcolor{comment}{! Read the data.}
00126   \textcolor{comment}{!}
00127   f\_ptr = c\_loc(rdata(1))
00128   \textcolor{keyword}{CALL }h5dread\_f(dset, memtype, f\_ptr, hdferr)
00129   \textcolor{comment}{!}
00130   \textcolor{comment}{! Output the variable-length data to the screen.}
00131   \textcolor{comment}{!}
00132   \textcolor{keywordflow}{DO} i = 1, dims(1)
00133      \textcolor{keyword}{WRITE}(*,\textcolor{stringliteral}{'(A,"(",I0,"):",/,"\{")'}, advance=\textcolor{stringliteral}{"no"}) dataset,i
00134      \textcolor{keyword}{CALL }c\_f\_pointer(rdata(i)%p, ptr\_r, [rdata(i)%len] )
00135      \textcolor{keywordflow}{DO} j = 1, rdata(i)%len
00136         \textcolor{keyword}{WRITE}(*,\textcolor{stringliteral}{'(1X,I0)'}, advance=\textcolor{stringliteral}{'no'}) ptr\_r(j)
00137         \textcolor{keywordflow}{IF} ( j .LT. rdata(i)%len) \textcolor{keyword}{WRITE}(*,\textcolor{stringliteral}{'(",")'}, advance=\textcolor{stringliteral}{'no'})
00138 \textcolor{keywordflow}{     ENDDO}
00139      \textcolor{keyword}{WRITE}(*,\textcolor{stringliteral}{'( " \}")'})
00140 \textcolor{keywordflow}{  ENDDO}
00141 
00142   \textcolor{comment}{!}
00143   \textcolor{comment}{! Close and release resources.  Note the use of H5Dvlen\_reclaim}
00144   \textcolor{comment}{! removes the need to manually deallocate the previously allocated}
00145   \textcolor{comment}{! data.}
00146   \textcolor{comment}{!}
00147   \textcolor{keyword}{CALL }h5dvlen\_reclaim\_f(memtype, space, h5p\_default\_f, f\_ptr, hdferr)
00148   \textcolor{keyword}{CALL }h5dclose\_f(dset , hdferr)
00149   \textcolor{keyword}{CALL }h5sclose\_f(space, hdferr)
00150   \textcolor{keyword}{CALL }h5tclose\_f(memtype, hdferr)
00151   \textcolor{keyword}{CALL }h5fclose\_f(\hyperlink{structfile}{file} , hdferr)
00152 
00153 \textcolor{keyword}{END PROGRAM }main
\end{DoxyCode}
