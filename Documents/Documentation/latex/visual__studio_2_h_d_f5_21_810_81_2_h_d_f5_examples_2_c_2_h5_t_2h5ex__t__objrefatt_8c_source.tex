\hypertarget{visual__studio_2_h_d_f5_21_810_81_2_h_d_f5_examples_2_c_2_h5_t_2h5ex__t__objrefatt_8c_source}{}\section{visual\+\_\+studio/\+H\+D\+F5/1.10.1/\+H\+D\+F5\+Examples/\+C/\+H5\+T/h5ex\+\_\+t\+\_\+objrefatt.c}
\label{visual__studio_2_h_d_f5_21_810_81_2_h_d_f5_examples_2_c_2_h5_t_2h5ex__t__objrefatt_8c_source}\index{h5ex\+\_\+t\+\_\+objrefatt.\+c@{h5ex\+\_\+t\+\_\+objrefatt.\+c}}

\begin{DoxyCode}
00001 \textcolor{comment}{/************************************************************}
00002 \textcolor{comment}{}
00003 \textcolor{comment}{  This example shows how to read and write object references}
00004 \textcolor{comment}{  to an attribute.  The program first creates objects in the}
00005 \textcolor{comment}{  file and writes references to those objects to an}
00006 \textcolor{comment}{  attribute with a dataspace of DIM0, then closes the file.}
00007 \textcolor{comment}{  Next, it reopens the file, dereferences the references,}
00008 \textcolor{comment}{  and outputs the names of their targets to the screen.}
00009 \textcolor{comment}{}
00010 \textcolor{comment}{  This file is intended for use with HDF5 Library version 1.8}
00011 \textcolor{comment}{}
00012 \textcolor{comment}{ ************************************************************/}
00013 
00014 \textcolor{preprocessor}{#include "hdf5.h"}
00015 \textcolor{preprocessor}{#include <stdio.h>}
00016 \textcolor{preprocessor}{#include <stdlib.h>}
00017 
00018 \textcolor{preprocessor}{#define FILE            "h5ex\_t\_objrefatt.h5"}
00019 \textcolor{preprocessor}{#define DATASET         "DS1"}
00020 \textcolor{preprocessor}{#define ATTRIBUTE       "A1"}
00021 \textcolor{preprocessor}{#define DIM0            2}
00022 
00023 \textcolor{keywordtype}{int}
00024 main (\textcolor{keywordtype}{void})
00025 \{
00026     hid\_t       \hyperlink{structfile}{file}, space, dset, obj, attr;   \textcolor{comment}{/* Handles */}
00027     herr\_t      status;
00028     hsize\_t     dims[1] = \{DIM0\};
00029     hobj\_ref\_t  wdata[DIM0],                    \textcolor{comment}{/* Write buffer */}
00030                 *rdata;                         \textcolor{comment}{/* Read buffer */}
00031     H5O\_type\_t  objtype;
00032     ssize\_t     size;
00033     \textcolor{keywordtype}{char}        *name;
00034     \textcolor{keywordtype}{int}         ndims,
00035                 i;
00036 
00037     \textcolor{comment}{/*}
00038 \textcolor{comment}{     * Create a new file using the default properties.}
00039 \textcolor{comment}{     */}
00040     file = H5Fcreate (FILE, H5F\_ACC\_TRUNC, H5P\_DEFAULT, H5P\_DEFAULT);
00041 
00042     \textcolor{comment}{/*}
00043 \textcolor{comment}{     * Create a dataset with a null dataspace.}
00044 \textcolor{comment}{     */}
00045     space = H5Screate (H5S\_NULL);
00046     obj = H5Dcreate (file, \textcolor{stringliteral}{"DS2"}, H5T\_STD\_I32LE, space, H5P\_DEFAULT,
00047                 H5P\_DEFAULT, H5P\_DEFAULT);
00048     status = H5Dclose (obj);
00049     status = H5Sclose (space);
00050 
00051     \textcolor{comment}{/*}
00052 \textcolor{comment}{     * Create a group.}
00053 \textcolor{comment}{     */}
00054     obj = H5Gcreate (file, \textcolor{stringliteral}{"G1"}, H5P\_DEFAULT, H5P\_DEFAULT, H5P\_DEFAULT);
00055     status = H5Gclose (obj);
00056 
00057     \textcolor{comment}{/*}
00058 \textcolor{comment}{     * Create references to the previously created objects.  Passing -1}
00059 \textcolor{comment}{     * as space\_id causes this parameter to be ignored.  Other values}
00060 \textcolor{comment}{     * besides valid dataspaces result in an error.}
00061 \textcolor{comment}{     */}
00062     status = H5Rcreate (&wdata[0], file, \textcolor{stringliteral}{"G1"}, H5R\_OBJECT, -1);
00063     status = H5Rcreate (&wdata[1], file, \textcolor{stringliteral}{"DS2"}, H5R\_OBJECT, -1);
00064 
00065     \textcolor{comment}{/*}
00066 \textcolor{comment}{     * Create dataset with a null dataspace to serve as the parent for}
00067 \textcolor{comment}{     * the attribute.}
00068 \textcolor{comment}{     */}
00069     space = H5Screate (H5S\_NULL);
00070     dset = H5Dcreate (file, DATASET, H5T\_STD\_I32LE, space, H5P\_DEFAULT,
00071                 H5P\_DEFAULT, H5P\_DEFAULT);
00072     status = H5Sclose (space);
00073 
00074     \textcolor{comment}{/*}
00075 \textcolor{comment}{     * Create dataspace.  Setting maximum size to NULL sets the maximum}
00076 \textcolor{comment}{     * size to be the current size.}
00077 \textcolor{comment}{     */}
00078     space = H5Screate\_simple (1, dims, NULL);
00079 
00080     \textcolor{comment}{/*}
00081 \textcolor{comment}{     * Create the attribute and write the object references to it.}
00082 \textcolor{comment}{     */}
00083     attr = H5Acreate (dset, ATTRIBUTE, H5T\_STD\_REF\_OBJ, space, H5P\_DEFAULT,
00084                     H5P\_DEFAULT);
00085     status = H5Awrite (attr, H5T\_STD\_REF\_OBJ, wdata);
00086 
00087     \textcolor{comment}{/*}
00088 \textcolor{comment}{     * Close and release resources.}
00089 \textcolor{comment}{     */}
00090     status = H5Aclose (attr);
00091     status = H5Dclose (dset);
00092     status = H5Sclose (space);
00093     status = H5Fclose (file);
00094 
00095 
00096     \textcolor{comment}{/*}
00097 \textcolor{comment}{     * Now we begin the read section of this example.  Here we assume}
00098 \textcolor{comment}{     * the attribute has the same name and rank, but can have any size.}
00099 \textcolor{comment}{     * Therefore we must allocate a new array to read in data using}
00100 \textcolor{comment}{     * malloc().}
00101 \textcolor{comment}{     */}
00102 
00103     \textcolor{comment}{/*}
00104 \textcolor{comment}{     * Open file, dataset, and attribute.}
00105 \textcolor{comment}{     */}
00106     file = H5Fopen (FILE, H5F\_ACC\_RDONLY, H5P\_DEFAULT);
00107     dset = H5Dopen (file, DATASET, H5P\_DEFAULT);
00108     attr = H5Aopen (dset, ATTRIBUTE, H5P\_DEFAULT);
00109 
00110     \textcolor{comment}{/*}
00111 \textcolor{comment}{     * Get dataspace and allocate memory for read buffer.}
00112 \textcolor{comment}{     */}
00113     space = H5Aget\_space (attr);
00114     ndims = H5Sget\_simple\_extent\_dims (space, dims, NULL);
00115     rdata = (hobj\_ref\_t *) malloc (dims[0] * \textcolor{keyword}{sizeof} (hobj\_ref\_t));
00116 
00117     \textcolor{comment}{/*}
00118 \textcolor{comment}{     * Read the data.}
00119 \textcolor{comment}{     */}
00120     status = H5Aread (attr, H5T\_STD\_REF\_OBJ, rdata);
00121 
00122     \textcolor{comment}{/*}
00123 \textcolor{comment}{     * Output the data to the screen.}
00124 \textcolor{comment}{     */}
00125     \textcolor{keywordflow}{for} (i=0; i<dims[0]; i++) \{
00126         printf (\textcolor{stringliteral}{"%s[%d]:\(\backslash\)n  ->"}, ATTRIBUTE, i);
00127 
00128         \textcolor{comment}{/*}
00129 \textcolor{comment}{         * Open the referenced object, get its name and type.}
00130 \textcolor{comment}{         */}
00131         obj = H5Rdereference (dset, H5P\_DEFAULT, H5R\_OBJECT, &rdata[i]);
00132         status = H5Rget\_obj\_type (dset, H5R\_OBJECT, &rdata[i], &objtype);
00133 
00134         \textcolor{comment}{/*}
00135 \textcolor{comment}{         * Get the length of the name, allocate space, then retrieve}
00136 \textcolor{comment}{         * the name.}
00137 \textcolor{comment}{         */}
00138         size = 1 + H5Iget\_name (obj, NULL, 0);
00139         name = (\textcolor{keywordtype}{char} *) malloc (size);
00140         size = H5Iget\_name (obj, name, size);
00141 
00142         \textcolor{comment}{/*}
00143 \textcolor{comment}{         * Print the object type and close the object.}
00144 \textcolor{comment}{         */}
00145         \textcolor{keywordflow}{switch} (objtype) \{
00146             \textcolor{keywordflow}{case} H5O\_TYPE\_GROUP:
00147                 printf (\textcolor{stringliteral}{"Group"});
00148                 \textcolor{keywordflow}{break};
00149             \textcolor{keywordflow}{case} H5O\_TYPE\_DATASET:
00150                 printf (\textcolor{stringliteral}{"Dataset"});
00151                 \textcolor{keywordflow}{break};
00152             \textcolor{keywordflow}{case} H5O\_TYPE\_NAMED\_DATATYPE:
00153                 printf (\textcolor{stringliteral}{"Named Datatype"});
00154                 \textcolor{keywordflow}{break};
00155             \textcolor{keywordflow}{case} H5O\_TYPE\_UNKNOWN:
00156             \textcolor{keywordflow}{case} H5O\_TYPE\_NTYPES:
00157                 printf (\textcolor{stringliteral}{"Unknown"});
00158         \}
00159         status = H5Oclose (obj);
00160 
00161         \textcolor{comment}{/*}
00162 \textcolor{comment}{         * Print the name and deallocate space for the name.}
00163 \textcolor{comment}{         */}
00164         printf (\textcolor{stringliteral}{": %s\(\backslash\)n"}, name);
00165         free (name);
00166     \}
00167 
00168     \textcolor{comment}{/*}
00169 \textcolor{comment}{     * Close and release resources.}
00170 \textcolor{comment}{     */}
00171     free (rdata);
00172     status = H5Aclose (attr);
00173     status = H5Dclose (dset);
00174     status = H5Sclose (space);
00175     status = H5Fclose (file);
00176 
00177     \textcolor{keywordflow}{return} 0;
00178 \}
\end{DoxyCode}
