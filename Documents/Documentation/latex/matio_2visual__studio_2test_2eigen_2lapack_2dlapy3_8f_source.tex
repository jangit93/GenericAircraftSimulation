\hypertarget{matio_2visual__studio_2test_2eigen_2lapack_2dlapy3_8f_source}{}\section{matio/visual\+\_\+studio/test/eigen/lapack/dlapy3.f}
\label{matio_2visual__studio_2test_2eigen_2lapack_2dlapy3_8f_source}\index{dlapy3.\+f@{dlapy3.\+f}}

\begin{DoxyCode}
00001 \textcolor{comment}{*> \(\backslash\)brief \(\backslash\)b DLAPY3}
00002 \textcolor{comment}{*}
00003 \textcolor{comment}{*  =========== DOCUMENTATION ===========}
00004 \textcolor{comment}{*}
00005 \textcolor{comment}{* Online html documentation available at }
00006 \textcolor{comment}{*            http://www.netlib.org/lapack/explore-html/ }
00007 \textcolor{comment}{*}
00008 \textcolor{comment}{*> \(\backslash\)htmlonly}
00009 \textcolor{comment}{*> Download DLAPY3 + dependencies }
00010 \textcolor{comment}{*> <a
       href="http://www.netlib.org/cgi-bin/netlibfiles.tgz?format=tgz&filename=/lapack/lapack\_routine/dlapy3.f"> }
00011 \textcolor{comment}{*> [TGZ]</a> }
00012 \textcolor{comment}{*> <a
       href="http://www.netlib.org/cgi-bin/netlibfiles.zip?format=zip&filename=/lapack/lapack\_routine/dlapy3.f"> }
00013 \textcolor{comment}{*> [ZIP]</a> }
00014 \textcolor{comment}{*> <a
       href="http://www.netlib.org/cgi-bin/netlibfiles.txt?format=txt&filename=/lapack/lapack\_routine/dlapy3.f"> }
00015 \textcolor{comment}{*> [TXT]</a>}
00016 \textcolor{comment}{*> \(\backslash\)endhtmlonly }
00017 \textcolor{comment}{*}
00018 \textcolor{comment}{*  Definition:}
00019 \textcolor{comment}{*  ===========}
00020 \textcolor{comment}{*}
00021 \textcolor{comment}{*       DOUBLE PRECISION FUNCTION DLAPY3( X, Y, Z )}
00022 \textcolor{comment}{* }
00023 \textcolor{comment}{*       .. Scalar Arguments ..}
00024 \textcolor{comment}{*       DOUBLE PRECISION   X, Y, Z}
00025 \textcolor{comment}{*       ..}
00026 \textcolor{comment}{*  }
00027 \textcolor{comment}{*}
00028 \textcolor{comment}{*> \(\backslash\)par Purpose:}
00029 \textcolor{comment}{*  =============}
00030 \textcolor{comment}{*>}
00031 \textcolor{comment}{*> \(\backslash\)verbatim}
00032 \textcolor{comment}{*>}
00033 \textcolor{comment}{*> DLAPY3 returns sqrt(x**2+y**2+z**2), taking care not to cause}
00034 \textcolor{comment}{*> unnecessary overflow.}
00035 \textcolor{comment}{*> \(\backslash\)endverbatim}
00036 \textcolor{comment}{*}
00037 \textcolor{comment}{*  Arguments:}
00038 \textcolor{comment}{*  ==========}
00039 \textcolor{comment}{*}
00040 \textcolor{comment}{*> \(\backslash\)param[in] X}
00041 \textcolor{comment}{*> \(\backslash\)verbatim}
00042 \textcolor{comment}{*>          X is DOUBLE PRECISION}
00043 \textcolor{comment}{*> \(\backslash\)endverbatim}
00044 \textcolor{comment}{*>}
00045 \textcolor{comment}{*> \(\backslash\)param[in] Y}
00046 \textcolor{comment}{*> \(\backslash\)verbatim}
00047 \textcolor{comment}{*>          Y is DOUBLE PRECISION}
00048 \textcolor{comment}{*> \(\backslash\)endverbatim}
00049 \textcolor{comment}{*>}
00050 \textcolor{comment}{*> \(\backslash\)param[in] Z}
00051 \textcolor{comment}{*> \(\backslash\)verbatim}
00052 \textcolor{comment}{*>          Z is DOUBLE PRECISION}
00053 \textcolor{comment}{*>          X, Y and Z specify the values x, y and z.}
00054 \textcolor{comment}{*> \(\backslash\)endverbatim}
00055 \textcolor{comment}{*}
00056 \textcolor{comment}{*  Authors:}
00057 \textcolor{comment}{*  ========}
00058 \textcolor{comment}{*}
00059 \textcolor{comment}{*> \(\backslash\)author Univ. of Tennessee }
00060 \textcolor{comment}{*> \(\backslash\)author Univ. of California Berkeley }
00061 \textcolor{comment}{*> \(\backslash\)author Univ. of Colorado Denver }
00062 \textcolor{comment}{*> \(\backslash\)author NAG Ltd. }
00063 \textcolor{comment}{*}
00064 \textcolor{comment}{*> \(\backslash\)date November 2011}
00065 \textcolor{comment}{*}
00066 \textcolor{comment}{*> \(\backslash\)ingroup auxOTHERauxiliary}
00067 \textcolor{comment}{*}
00068 \textcolor{comment}{*  =====================================================================}
00069 \textcolor{keyword}{      DOUBLE PRECISION }\textcolor{keyword}{FUNCTION }dlapy3( X, Y, Z )
00070 \textcolor{comment}{*}
00071 \textcolor{comment}{*  -- LAPACK auxiliary routine (version 3.4.0) --}
00072 \textcolor{comment}{*  -- LAPACK is a software package provided by Univ. of Tennessee,    --}
00073 \textcolor{comment}{*  -- Univ. of California Berkeley, Univ. of Colorado Denver and NAG Ltd..--}
00074 \textcolor{comment}{*     November 2011}
00075 \textcolor{comment}{*}
00076 \textcolor{comment}{*     .. Scalar Arguments ..}
00077       \textcolor{keywordtype}{DOUBLE PRECISION}   x, y, z
00078 \textcolor{comment}{*     ..}
00079 \textcolor{comment}{*}
00080 \textcolor{comment}{*  =====================================================================}
00081 \textcolor{comment}{*}
00082 \textcolor{comment}{*     .. Parameters ..}
00083       \textcolor{keywordtype}{DOUBLE PRECISION}   zero
00084       parameter( zero = 0.0d0 )
00085 \textcolor{comment}{*     ..}
00086 \textcolor{comment}{*     .. Local Scalars ..}
00087       \textcolor{keywordtype}{DOUBLE PRECISION}   w, xabs, yabs, zabs
00088 \textcolor{comment}{*     ..}
00089 \textcolor{comment}{*     .. Intrinsic Functions ..}
00090       \textcolor{keywordtype}{INTRINSIC}          abs, max, sqrt
00091 \textcolor{comment}{*     ..}
00092 \textcolor{comment}{*     .. Executable Statements ..}
00093 \textcolor{comment}{*}
00094       xabs = abs( x )
00095       yabs = abs( y )
00096       zabs = abs( z )
00097       w = max( xabs, yabs, zabs )
00098       \textcolor{keywordflow}{IF}( w.EQ.zero ) \textcolor{keywordflow}{THEN}
00099 \textcolor{comment}{*     W can be zero for max(0,nan,0)}
00100 \textcolor{comment}{*     adding all three entries together will make sure}
00101 \textcolor{comment}{*     NaN will not disappear.}
00102          dlapy3 =  xabs + yabs + zabs
00103       \textcolor{keywordflow}{ELSE}
00104          dlapy3 = w*sqrt( ( xabs / w )**2+( yabs / w )**2+
00105      $            ( zabs / w )**2 )
00106 \textcolor{keywordflow}{      END IF}
00107       \textcolor{keywordflow}{RETURN}
00108 \textcolor{comment}{*}
00109 \textcolor{comment}{*     End of DLAPY3}
00110 \textcolor{comment}{*}
00111 \textcolor{keyword}{      END}
\end{DoxyCode}
