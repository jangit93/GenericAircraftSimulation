\hypertarget{visual__studio_2_h_d_f5_21_810_81_2_h_d_f5_examples_2_c_2_h5_d_2h5ex__d__unlimgzip_8c_source}{}\section{visual\+\_\+studio/\+H\+D\+F5/1.10.1/\+H\+D\+F5\+Examples/\+C/\+H5\+D/h5ex\+\_\+d\+\_\+unlimgzip.c}
\label{visual__studio_2_h_d_f5_21_810_81_2_h_d_f5_examples_2_c_2_h5_d_2h5ex__d__unlimgzip_8c_source}\index{h5ex\+\_\+d\+\_\+unlimgzip.\+c@{h5ex\+\_\+d\+\_\+unlimgzip.\+c}}

\begin{DoxyCode}
00001 \textcolor{comment}{/************************************************************}
00002 \textcolor{comment}{}
00003 \textcolor{comment}{  This example shows how to create and extend an unlimited}
00004 \textcolor{comment}{  dataset with gzip compression.  The program first writes}
00005 \textcolor{comment}{  integers to a gzip compressed dataset with dataspace}
00006 \textcolor{comment}{  dimensions of DIM0xDIM1, then closes the file.  Next, it}
00007 \textcolor{comment}{  reopens the file, reads back the data, outputs it to the}
00008 \textcolor{comment}{  screen, extends the dataset, and writes new data to the}
00009 \textcolor{comment}{  extended portions of the dataset.  Finally it reopens the}
00010 \textcolor{comment}{  file again, reads back the data, and outputs it to the}
00011 \textcolor{comment}{  screen.}
00012 \textcolor{comment}{}
00013 \textcolor{comment}{  This file is intended for use with HDF5 Library version 1.8}
00014 \textcolor{comment}{}
00015 \textcolor{comment}{ ************************************************************/}
00016 
00017 \textcolor{preprocessor}{#include "hdf5.h"}
00018 \textcolor{preprocessor}{#include <stdio.h>}
00019 \textcolor{preprocessor}{#include <stdlib.h>}
00020 
00021 \textcolor{preprocessor}{#define FILE            "h5ex\_d\_unlimgzip.h5"}
00022 \textcolor{preprocessor}{#define DATASET         "DS1"}
00023 \textcolor{preprocessor}{#define DIM0            4}
00024 \textcolor{preprocessor}{#define DIM1            7}
00025 \textcolor{preprocessor}{#define EDIM0           6}
00026 \textcolor{preprocessor}{#define EDIM1           10}
00027 \textcolor{preprocessor}{#define CHUNK0          4}
00028 \textcolor{preprocessor}{#define CHUNK1          4}
00029 
00030 \textcolor{keywordtype}{int}
00031 main (\textcolor{keywordtype}{void})
00032 \{
00033     hid\_t           \hyperlink{structfile}{file}, space, dset, dcpl;    \textcolor{comment}{/* Handles */}
00034     herr\_t          status;
00035     htri\_t          avail;
00036     H5Z\_filter\_t    filter\_type;
00037     hsize\_t         dims[2] = \{DIM0, DIM1\},
00038                     extdims[2] = \{EDIM0, EDIM1\},
00039                     maxdims[2],
00040                     chunk[2] = \{CHUNK0, CHUNK1\},
00041                     start[2],
00042                     count[2];
00043     \textcolor{keywordtype}{size\_t}          nelmts;
00044     \textcolor{keywordtype}{unsigned} \textcolor{keywordtype}{int}    flags,
00045                     filter\_info;
00046     \textcolor{keywordtype}{int}             wdata[DIM0][DIM1],          \textcolor{comment}{/* Write buffer */}
00047                     wdata2[EDIM0][EDIM1],       \textcolor{comment}{/* Write buffer for}
00048 \textcolor{comment}{                                                   extension */}
00049                     **rdata,                    \textcolor{comment}{/* Read buffer */}
00050                     ndims,
00051                     i, j;
00052 
00053     \textcolor{comment}{/*}
00054 \textcolor{comment}{     * Check if gzip compression is available and can be used for both}
00055 \textcolor{comment}{     * compression and decompression.  Normally we do not perform error}
00056 \textcolor{comment}{     * checking in these examples for the sake of clarity, but in this}
00057 \textcolor{comment}{     * case we will make an exception because this filter is an}
00058 \textcolor{comment}{     * optional part of the hdf5 library.}
00059 \textcolor{comment}{     */}
00060     avail = H5Zfilter\_avail(H5Z\_FILTER\_DEFLATE);
00061     \textcolor{keywordflow}{if} (!avail) \{
00062         printf (\textcolor{stringliteral}{"gzip filter not available.\(\backslash\)n"});
00063         \textcolor{keywordflow}{return} 1;
00064     \}
00065     status = H5Zget\_filter\_info (H5Z\_FILTER\_DEFLATE, &filter\_info);
00066     \textcolor{keywordflow}{if} ( !(filter\_info & H5Z\_FILTER\_CONFIG\_ENCODE\_ENABLED) ||
00067                 !(filter\_info & H5Z\_FILTER\_CONFIG\_DECODE\_ENABLED) ) \{
00068         printf (\textcolor{stringliteral}{"gzip filter not available for encoding and decoding.\(\backslash\)n"});
00069         \textcolor{keywordflow}{return} 1;
00070     \}
00071 
00072     \textcolor{comment}{/*}
00073 \textcolor{comment}{     * Initialize data.}
00074 \textcolor{comment}{     */}
00075     \textcolor{keywordflow}{for} (i=0; i<DIM0; i++)
00076         \textcolor{keywordflow}{for} (j=0; j<DIM1; j++)
00077             wdata[i][j] = i * j - j;
00078 
00079     \textcolor{comment}{/*}
00080 \textcolor{comment}{     * Create a new file using the default properties.}
00081 \textcolor{comment}{     */}
00082     file = H5Fcreate (FILE, H5F\_ACC\_TRUNC, H5P\_DEFAULT, H5P\_DEFAULT);
00083 
00084     \textcolor{comment}{/*}
00085 \textcolor{comment}{     * Create dataspace with unlimited dimensions.}
00086 \textcolor{comment}{     */}
00087     maxdims[0] = H5S\_UNLIMITED;
00088     maxdims[1] = H5S\_UNLIMITED;
00089     space = H5Screate\_simple (2, dims, maxdims);
00090 
00091     \textcolor{comment}{/*}
00092 \textcolor{comment}{     * Create the dataset creation property list, add the gzip}
00093 \textcolor{comment}{     * compression filter and set the chunk size.}
00094 \textcolor{comment}{     */}
00095     dcpl = H5Pcreate (H5P\_DATASET\_CREATE);
00096     status = H5Pset\_deflate (dcpl, 9);
00097     status = H5Pset\_chunk (dcpl, 2, chunk);
00098 
00099     \textcolor{comment}{/*}
00100 \textcolor{comment}{     * Create the compressed unlimited dataset.}
00101 \textcolor{comment}{     */}
00102     dset = H5Dcreate (file, DATASET, H5T\_STD\_I32LE, space, H5P\_DEFAULT, dcpl,
00103                 H5P\_DEFAULT);
00104 
00105     \textcolor{comment}{/*}
00106 \textcolor{comment}{     * Write the data to the dataset.}
00107 \textcolor{comment}{     */}
00108     status = H5Dwrite (dset, H5T\_NATIVE\_INT, H5S\_ALL, H5S\_ALL, H5P\_DEFAULT,
00109                 wdata[0]);
00110 
00111     \textcolor{comment}{/*}
00112 \textcolor{comment}{     * Close and release resources.}
00113 \textcolor{comment}{     */}
00114     status = H5Pclose (dcpl);
00115     status = H5Dclose (dset);
00116     status = H5Sclose (space);
00117     status = H5Fclose (file);
00118 
00119 
00120     \textcolor{comment}{/*}
00121 \textcolor{comment}{     * In this next section we read back the data, extend the dataset,}
00122 \textcolor{comment}{     * and write new data to the extended portions.}
00123 \textcolor{comment}{     */}
00124 
00125     \textcolor{comment}{/*}
00126 \textcolor{comment}{     * Open file and dataset using the default properties.}
00127 \textcolor{comment}{     */}
00128     file = H5Fopen (FILE, H5F\_ACC\_RDWR, H5P\_DEFAULT);
00129     dset = H5Dopen (file, DATASET, H5P\_DEFAULT);
00130 
00131     \textcolor{comment}{/*}
00132 \textcolor{comment}{     * Get dataspace and allocate memory for read buffer.  This is a}
00133 \textcolor{comment}{     * two dimensional dataset so the dynamic allocation must be done}
00134 \textcolor{comment}{     * in steps.}
00135 \textcolor{comment}{     */}
00136     space = H5Dget\_space (dset);
00137     ndims = H5Sget\_simple\_extent\_dims (space, dims, NULL);
00138 
00139     \textcolor{comment}{/*}
00140 \textcolor{comment}{     * Allocate array of pointers to rows.}
00141 \textcolor{comment}{     */}
00142     rdata = (\textcolor{keywordtype}{int} **) malloc (dims[0] * \textcolor{keyword}{sizeof} (\textcolor{keywordtype}{int} *));
00143 
00144     \textcolor{comment}{/*}
00145 \textcolor{comment}{     * Allocate space for integer data.}
00146 \textcolor{comment}{     */}
00147     rdata[0] = (\textcolor{keywordtype}{int} *) malloc (dims[0] * dims[1] * \textcolor{keyword}{sizeof} (\textcolor{keywordtype}{int}));
00148 
00149     \textcolor{comment}{/*}
00150 \textcolor{comment}{     * Set the rest of the pointers to rows to the correct addresses.}
00151 \textcolor{comment}{     */}
00152     \textcolor{keywordflow}{for} (i=1; i<dims[0]; i++)
00153         rdata[i] = rdata[0] + i * dims[1];
00154 
00155     \textcolor{comment}{/*}
00156 \textcolor{comment}{     * Read the data using the default properties.}
00157 \textcolor{comment}{     */}
00158     status = H5Dread (dset, H5T\_NATIVE\_INT, H5S\_ALL, H5S\_ALL, H5P\_DEFAULT,
00159                 rdata[0]);
00160 
00161     \textcolor{comment}{/*}
00162 \textcolor{comment}{     * Output the data to the screen.}
00163 \textcolor{comment}{     */}
00164     printf (\textcolor{stringliteral}{"Dataset before extension:\(\backslash\)n"});
00165     \textcolor{keywordflow}{for} (i=0; i<dims[0]; i++) \{
00166         printf (\textcolor{stringliteral}{" ["});
00167         \textcolor{keywordflow}{for} (j=0; j<dims[1]; j++)
00168             printf (\textcolor{stringliteral}{" %3d"}, rdata[i][j]);
00169         printf (\textcolor{stringliteral}{"]\(\backslash\)n"});
00170     \}
00171 
00172     status = H5Sclose (space);
00173 
00174     \textcolor{comment}{/*}
00175 \textcolor{comment}{     * Extend the dataset.}
00176 \textcolor{comment}{     */}
00177     status = H5Dset\_extent (dset, extdims);
00178 
00179     \textcolor{comment}{/*}
00180 \textcolor{comment}{     * Retrieve the dataspace for the newly extended dataset.}
00181 \textcolor{comment}{     */}
00182     space = H5Dget\_space (dset);
00183 
00184     \textcolor{comment}{/*}
00185 \textcolor{comment}{     * Initialize data for writing to the extended dataset.}
00186 \textcolor{comment}{     */}
00187     \textcolor{keywordflow}{for} (i=0; i<EDIM0; i++)
00188         \textcolor{keywordflow}{for} (j=0; j<EDIM1; j++)
00189             wdata2[i][j] = j;
00190 
00191     \textcolor{comment}{/*}
00192 \textcolor{comment}{     * Select the entire dataspace.}
00193 \textcolor{comment}{     */}
00194     status = H5Sselect\_all (space);
00195 
00196     \textcolor{comment}{/*}
00197 \textcolor{comment}{     * Subtract a hyperslab reflecting the original dimensions from the}
00198 \textcolor{comment}{     * selection.  The selection now contains only the newly extended}
00199 \textcolor{comment}{     * portions of the dataset.}
00200 \textcolor{comment}{     */}
00201     start[0] = 0;
00202     start[1] = 0;
00203     count[0] = dims[0];
00204     count[1] = dims[1];
00205     status = H5Sselect\_hyperslab (space, H5S\_SELECT\_NOTB, start, NULL, count,
00206                 NULL);
00207 
00208     \textcolor{comment}{/*}
00209 \textcolor{comment}{     * Write the data to the selected portion of the dataset.}
00210 \textcolor{comment}{     */}
00211     status = H5Dwrite (dset, H5T\_NATIVE\_INT, H5S\_ALL, space, H5P\_DEFAULT,
00212                 wdata2[0]);
00213 
00214     \textcolor{comment}{/*}
00215 \textcolor{comment}{     * Close and release resources.}
00216 \textcolor{comment}{     */}
00217     free (rdata[0]);
00218     free(rdata);
00219     status = H5Dclose (dset);
00220     status = H5Sclose (space);
00221     status = H5Fclose (file);
00222 
00223 
00224     \textcolor{comment}{/*}
00225 \textcolor{comment}{     * Now we simply read back the data and output it to the screen.}
00226 \textcolor{comment}{     */}
00227 
00228     \textcolor{comment}{/*}
00229 \textcolor{comment}{     * Open file and dataset using the default properties.}
00230 \textcolor{comment}{     */}
00231     file = H5Fopen (FILE, H5F\_ACC\_RDONLY, H5P\_DEFAULT);
00232     dset = H5Dopen (file, DATASET, H5P\_DEFAULT);
00233 
00234     \textcolor{comment}{/*}
00235 \textcolor{comment}{     * Retrieve dataset creation property list.}
00236 \textcolor{comment}{     */}
00237     dcpl = H5Dget\_create\_plist (dset);
00238 
00239     \textcolor{comment}{/*}
00240 \textcolor{comment}{     * Retrieve and print the filter type.  Here we only retrieve the}
00241 \textcolor{comment}{     * first filter because we know that we only added one filter.}
00242 \textcolor{comment}{     */}
00243     nelmts = 0;
00244     filter\_type = H5Pget\_filter (dcpl, 0, &flags, &nelmts, NULL, 0, NULL,
00245                 &filter\_info);
00246     printf (\textcolor{stringliteral}{"\(\backslash\)nFilter type is: "});
00247     \textcolor{keywordflow}{switch} (filter\_type) \{
00248         \textcolor{keywordflow}{case} H5Z\_FILTER\_DEFLATE:
00249             printf (\textcolor{stringliteral}{"H5Z\_FILTER\_DEFLATE\(\backslash\)n"});
00250             \textcolor{keywordflow}{break};
00251         \textcolor{keywordflow}{case} H5Z\_FILTER\_SHUFFLE:
00252             printf (\textcolor{stringliteral}{"H5Z\_FILTER\_SHUFFLE\(\backslash\)n"});
00253             \textcolor{keywordflow}{break};
00254         \textcolor{keywordflow}{case} H5Z\_FILTER\_FLETCHER32:
00255             printf (\textcolor{stringliteral}{"H5Z\_FILTER\_FLETCHER32\(\backslash\)n"});
00256             \textcolor{keywordflow}{break};
00257         \textcolor{keywordflow}{case} H5Z\_FILTER\_SZIP:
00258             printf (\textcolor{stringliteral}{"H5Z\_FILTER\_SZIP\(\backslash\)n"});
00259             \textcolor{keywordflow}{break};
00260         \textcolor{keywordflow}{case} H5Z\_FILTER\_NBIT:
00261             printf (\textcolor{stringliteral}{"H5Z\_FILTER\_NBIT\(\backslash\)n"});
00262             \textcolor{keywordflow}{break};
00263         \textcolor{keywordflow}{case} H5Z\_FILTER\_SCALEOFFSET:
00264             printf (\textcolor{stringliteral}{"H5Z\_FILTER\_SCALEOFFSET\(\backslash\)n"});
00265     \}
00266 
00267     \textcolor{comment}{/*}
00268 \textcolor{comment}{     * Get dataspace and allocate memory for the read buffer as before.}
00269 \textcolor{comment}{     */}
00270     space = H5Dget\_space (dset);
00271     ndims = H5Sget\_simple\_extent\_dims (space, dims, NULL);
00272     rdata = (\textcolor{keywordtype}{int} **) malloc (dims[0] * \textcolor{keyword}{sizeof} (\textcolor{keywordtype}{int} *));
00273     rdata[0] = (\textcolor{keywordtype}{int} *) malloc (dims[0] * dims[1] * \textcolor{keyword}{sizeof} (\textcolor{keywordtype}{int}));
00274     \textcolor{keywordflow}{for} (i=1; i<dims[0]; i++)
00275         rdata[i] = rdata[0] + i * dims[1];
00276 
00277     \textcolor{comment}{/*}
00278 \textcolor{comment}{     * Read the data using the default properties.}
00279 \textcolor{comment}{     */}
00280     status = H5Dread (dset, H5T\_NATIVE\_INT, H5S\_ALL, H5S\_ALL, H5P\_DEFAULT,
00281                 rdata[0]);
00282 
00283     \textcolor{comment}{/*}
00284 \textcolor{comment}{     * Output the data to the screen.}
00285 \textcolor{comment}{     */}
00286     printf (\textcolor{stringliteral}{"Dataset after extension:\(\backslash\)n"});
00287     \textcolor{keywordflow}{for} (i=0; i<dims[0]; i++) \{
00288         printf (\textcolor{stringliteral}{" ["});
00289         \textcolor{keywordflow}{for} (j=0; j<dims[1]; j++)
00290             printf (\textcolor{stringliteral}{" %3d"}, rdata[i][j]);
00291         printf (\textcolor{stringliteral}{"]\(\backslash\)n"});
00292     \}
00293 
00294     \textcolor{comment}{/*}
00295 \textcolor{comment}{     * Close and release resources.}
00296 \textcolor{comment}{     */}
00297     free (rdata[0]);
00298     free(rdata);
00299     status = H5Pclose (dcpl);
00300     status = H5Dclose (dset);
00301     status = H5Sclose (space);
00302     status = H5Fclose (file);
00303 
00304     \textcolor{keywordflow}{return} 0;
00305 \}
\end{DoxyCode}
