\hypertarget{_h_d_f5_21_810_81_2include_2_h5_l_tparse_8h_source}{}\section{H\+D\+F5/1.10.1/include/\+H5\+L\+Tparse.h}
\label{_h_d_f5_21_810_81_2include_2_h5_l_tparse_8h_source}\index{H5\+L\+Tparse.\+h@{H5\+L\+Tparse.\+h}}

\begin{DoxyCode}
00001 \textcolor{comment}{/* A Bison parser, made by GNU Bison 3.0.2.  */}
00002 
00003 \textcolor{comment}{/* Bison interface for Yacc-like parsers in C}
00004 \textcolor{comment}{}
00005 \textcolor{comment}{   Copyright (C) 1984, 1989-1990, 2000-2013 Free Software Foundation, Inc.}
00006 \textcolor{comment}{}
00007 \textcolor{comment}{   This program is free software: you can redistribute it and/or modify}
00008 \textcolor{comment}{   it under the terms of the GNU General Public License as published by}
00009 \textcolor{comment}{   the Free Software Foundation, either version 3 of the License, or}
00010 \textcolor{comment}{   (at your option) any later version.}
00011 \textcolor{comment}{}
00012 \textcolor{comment}{   This program is distributed in the hope that it will be useful,}
00013 \textcolor{comment}{   but WITHOUT ANY WARRANTY; without even the implied warranty of}
00014 \textcolor{comment}{   MERCHANTABILITY or FITNESS FOR A PARTICULAR PURPOSE.  See the}
00015 \textcolor{comment}{   GNU General Public License for more details.}
00016 \textcolor{comment}{}
00017 \textcolor{comment}{   You should have received a copy of the GNU General Public License}
00018 \textcolor{comment}{   along with this program.  If not, see <http://www.gnu.org/licenses/>.  */}
00019 
00020 \textcolor{comment}{/* As a special exception, you may create a larger work that contains}
00021 \textcolor{comment}{   part or all of the Bison parser skeleton and distribute that work}
00022 \textcolor{comment}{   under terms of your choice, so long as that work isn't itself a}
00023 \textcolor{comment}{   parser generator using the skeleton or a modified version thereof}
00024 \textcolor{comment}{   as a parser skeleton.  Alternatively, if you modify or redistribute}
00025 \textcolor{comment}{   the parser skeleton itself, you may (at your option) remove this}
00026 \textcolor{comment}{   special exception, which will cause the skeleton and the resulting}
00027 \textcolor{comment}{   Bison output files to be licensed under the GNU General Public}
00028 \textcolor{comment}{   License without this special exception.}
00029 \textcolor{comment}{}
00030 \textcolor{comment}{   This special exception was added by the Free Software Foundation in}
00031 \textcolor{comment}{   version 2.2 of Bison.  */}
00032 
00033 \textcolor{preprocessor}{#ifndef YY\_H5LTYY\_HL\_SRC\_H5LTPARSE\_H\_INCLUDED}
00034 \textcolor{preprocessor}{# define YY\_H5LTYY\_HL\_SRC\_H5LTPARSE\_H\_INCLUDED}
00035 \textcolor{comment}{/* Debug traces.  */}
00036 \textcolor{preprocessor}{#ifndef YYDEBUG}
00037 \textcolor{preprocessor}{# define YYDEBUG 0}
00038 \textcolor{preprocessor}{#endif}
00039 \textcolor{preprocessor}{#if YYDEBUG}
00040 \textcolor{keyword}{extern} \textcolor{keywordtype}{int} H5LTyydebug;
00041 \textcolor{preprocessor}{#endif}
00042 
00043 \textcolor{comment}{/* Token type.  */}
00044 \textcolor{preprocessor}{#ifndef YYTOKENTYPE}
00045 \textcolor{preprocessor}{# define YYTOKENTYPE}
00046   \textcolor{keyword}{enum} yytokentype
00047   \{
00048     H5T\_STD\_I8BE\_TOKEN = 258,
00049     H5T\_STD\_I8LE\_TOKEN = 259,
00050     H5T\_STD\_I16BE\_TOKEN = 260,
00051     H5T\_STD\_I16LE\_TOKEN = 261,
00052     H5T\_STD\_I32BE\_TOKEN = 262,
00053     H5T\_STD\_I32LE\_TOKEN = 263,
00054     H5T\_STD\_I64BE\_TOKEN = 264,
00055     H5T\_STD\_I64LE\_TOKEN = 265,
00056     H5T\_STD\_U8BE\_TOKEN = 266,
00057     H5T\_STD\_U8LE\_TOKEN = 267,
00058     H5T\_STD\_U16BE\_TOKEN = 268,
00059     H5T\_STD\_U16LE\_TOKEN = 269,
00060     H5T\_STD\_U32BE\_TOKEN = 270,
00061     H5T\_STD\_U32LE\_TOKEN = 271,
00062     H5T\_STD\_U64BE\_TOKEN = 272,
00063     H5T\_STD\_U64LE\_TOKEN = 273,
00064     H5T\_NATIVE\_CHAR\_TOKEN = 274,
00065     H5T\_NATIVE\_SCHAR\_TOKEN = 275,
00066     H5T\_NATIVE\_UCHAR\_TOKEN = 276,
00067     H5T\_NATIVE\_SHORT\_TOKEN = 277,
00068     H5T\_NATIVE\_USHORT\_TOKEN = 278,
00069     H5T\_NATIVE\_INT\_TOKEN = 279,
00070     H5T\_NATIVE\_UINT\_TOKEN = 280,
00071     H5T\_NATIVE\_LONG\_TOKEN = 281,
00072     H5T\_NATIVE\_ULONG\_TOKEN = 282,
00073     H5T\_NATIVE\_LLONG\_TOKEN = 283,
00074     H5T\_NATIVE\_ULLONG\_TOKEN = 284,
00075     H5T\_IEEE\_F32BE\_TOKEN = 285,
00076     H5T\_IEEE\_F32LE\_TOKEN = 286,
00077     H5T\_IEEE\_F64BE\_TOKEN = 287,
00078     H5T\_IEEE\_F64LE\_TOKEN = 288,
00079     H5T\_NATIVE\_FLOAT\_TOKEN = 289,
00080     H5T\_NATIVE\_DOUBLE\_TOKEN = 290,
00081     H5T\_NATIVE\_LDOUBLE\_TOKEN = 291,
00082     H5T\_STRING\_TOKEN = 292,
00083     STRSIZE\_TOKEN = 293,
00084     STRPAD\_TOKEN = 294,
00085     CSET\_TOKEN = 295,
00086     CTYPE\_TOKEN = 296,
00087     H5T\_VARIABLE\_TOKEN = 297,
00088     H5T\_STR\_NULLTERM\_TOKEN = 298,
00089     H5T\_STR\_NULLPAD\_TOKEN = 299,
00090     H5T\_STR\_SPACEPAD\_TOKEN = 300,
00091     H5T\_CSET\_ASCII\_TOKEN = 301,
00092     H5T\_CSET\_UTF8\_TOKEN = 302,
00093     H5T\_C\_S1\_TOKEN = 303,
00094     H5T\_FORTRAN\_S1\_TOKEN = 304,
00095     H5T\_OPAQUE\_TOKEN = 305,
00096     OPQ\_SIZE\_TOKEN = 306,
00097     OPQ\_TAG\_TOKEN = 307,
00098     H5T\_COMPOUND\_TOKEN = 308,
00099     H5T\_ENUM\_TOKEN = 309,
00100     H5T\_ARRAY\_TOKEN = 310,
00101     H5T\_VLEN\_TOKEN = 311,
00102     STRING = 312,
00103     NUMBER = 313
00104   \};
00105 \textcolor{preprocessor}{#endif}
00106 
00107 \textcolor{comment}{/* Value type.  */}
00108 \textcolor{preprocessor}{#if ! defined YYSTYPE && ! defined YYSTYPE\_IS\_DECLARED}
00109 \textcolor{keyword}{typedef} \textcolor{keyword}{union }\hyperlink{union_y_y_s_t_y_p_e}{YYSTYPE} \hyperlink{union_y_y_s_t_y_p_e}{YYSTYPE};
\Hypertarget{_h_d_f5_21_810_81_2include_2_h5_l_tparse_8h_source_l00110}\hyperlink{union_y_y_s_t_y_p_e}{00110} \textcolor{keyword}{union }\hyperlink{union_y_y_s_t_y_p_e}{YYSTYPE}
00111 \{
00112 \textcolor{preprocessor}{#line 74 "hl/src/H5LTparse.y" }\textcolor{comment}{/* yacc.c:1909  */}\textcolor{preprocessor}{}
00113 
00114     \textcolor{keywordtype}{int}     ival;         \textcolor{comment}{/*for integer token*/}
00115     \textcolor{keywordtype}{char}    *sval;        \textcolor{comment}{/*for name string*/}
00116     hid\_t   hid;          \textcolor{comment}{/*for hid\_t token*/}
00117 
00118 \textcolor{preprocessor}{#line 119 "hl/src/H5LTparse.h" }\textcolor{comment}{/* yacc.c:1909  */}\textcolor{preprocessor}{}
00119 \};
00120 \textcolor{preprocessor}{# define YYSTYPE\_IS\_TRIVIAL 1}
00121 \textcolor{preprocessor}{# define YYSTYPE\_IS\_DECLARED 1}
00122 \textcolor{preprocessor}{#endif}
00123 
00124 
00125 \textcolor{keyword}{extern} \hyperlink{union_y_y_s_t_y_p_e}{YYSTYPE} H5LTyylval;
00126 
00127 \textcolor{keywordtype}{int} H5LTyyparse (\textcolor{keywordtype}{void});
00128 
00129 \textcolor{preprocessor}{#endif }\textcolor{comment}{/* !YY\_H5LTYY\_HL\_SRC\_H5LTPARSE\_H\_INCLUDED  */}\textcolor{preprocessor}{}
\end{DoxyCode}
