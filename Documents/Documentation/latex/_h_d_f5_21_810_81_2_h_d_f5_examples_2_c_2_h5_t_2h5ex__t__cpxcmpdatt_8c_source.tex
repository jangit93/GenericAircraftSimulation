\hypertarget{_h_d_f5_21_810_81_2_h_d_f5_examples_2_c_2_h5_t_2h5ex__t__cpxcmpdatt_8c_source}{}\section{H\+D\+F5/1.10.1/\+H\+D\+F5\+Examples/\+C/\+H5\+T/h5ex\+\_\+t\+\_\+cpxcmpdatt.c}
\label{_h_d_f5_21_810_81_2_h_d_f5_examples_2_c_2_h5_t_2h5ex__t__cpxcmpdatt_8c_source}\index{h5ex\+\_\+t\+\_\+cpxcmpdatt.\+c@{h5ex\+\_\+t\+\_\+cpxcmpdatt.\+c}}

\begin{DoxyCode}
00001 \textcolor{comment}{/************************************************************}
00002 \textcolor{comment}{}
00003 \textcolor{comment}{  This example shows how to read and write a complex}
00004 \textcolor{comment}{  compound datatype to an attribute.  The program first}
00005 \textcolor{comment}{  writes complex compound structures to an attribute with a}
00006 \textcolor{comment}{  dataspace of DIM0, then closes the file.  Next, it reopens}
00007 \textcolor{comment}{  the file, reads back selected fields in the structure, and}
00008 \textcolor{comment}{  outputs them to the screen.}
00009 \textcolor{comment}{}
00010 \textcolor{comment}{  Unlike the other datatype examples, in this example we}
00011 \textcolor{comment}{  save to the file using native datatypes to simplify the}
00012 \textcolor{comment}{  type definitions here.  To save using standard types you}
00013 \textcolor{comment}{  must manually calculate the sizes and offsets of compound}
00014 \textcolor{comment}{  types as shown in h5ex\_t\_cmpd.c, and convert enumerated}
00015 \textcolor{comment}{  values as shown in h5ex\_t\_enum.c.}
00016 \textcolor{comment}{}
00017 \textcolor{comment}{  The datatype defined here consists of a compound}
00018 \textcolor{comment}{  containing a variable-length list of compound types, as}
00019 \textcolor{comment}{  well as a variable-length string, enumeration, double}
00020 \textcolor{comment}{  array, object reference and region reference.  The nested}
00021 \textcolor{comment}{  compound type contains an int, variable-length string and}
00022 \textcolor{comment}{  two doubles.}
00023 \textcolor{comment}{}
00024 \textcolor{comment}{  This file is intended for use with HDF5 Library version 1.8}
00025 \textcolor{comment}{}
00026 \textcolor{comment}{ ************************************************************/}
00027 
00028 \textcolor{preprocessor}{#include "hdf5.h"}
00029 \textcolor{preprocessor}{#include <stdio.h>}
00030 \textcolor{preprocessor}{#include <stdlib.h>}
00031 
00032 \textcolor{preprocessor}{#define FILE            "h5ex\_t\_cpxcmpdatt.h5"}
00033 \textcolor{preprocessor}{#define DATASET         "DS1"}
00034 \textcolor{preprocessor}{#define ATTRIBUTE       "A1"}
00035 \textcolor{preprocessor}{#define DIM0            2}
00036 
00037 \textcolor{keyword}{typedef} \textcolor{keyword}{struct }\{
00038     \textcolor{keywordtype}{int}     serial\_no;
00039     \textcolor{keywordtype}{char}    *location;
00040     \textcolor{keywordtype}{double}  temperature;
00041     \textcolor{keywordtype}{double}  pressure;
00042 \} \hyperlink{structsensor__t}{sensor\_t};                                 \textcolor{comment}{/* Nested compound type */}
00043 
00044 \textcolor{keyword}{typedef} \textcolor{keyword}{enum} \{
00045     RED,
00046     GREEN,
00047     BLUE
00048 \} color\_t  ;                                \textcolor{comment}{/* Enumerated type */}
00049 
00050 \textcolor{keyword}{typedef} \textcolor{keyword}{struct }\{
00051     \hyperlink{structhvl__t}{hvl\_t}               sensors;
00052     \textcolor{keywordtype}{char}                *name;
00053     color\_t             color;
00054     \textcolor{keywordtype}{double}              location[3];
00055     hobj\_ref\_t          group;
00056     hdset\_reg\_ref\_t     surveyed\_areas;
00057 \} \hyperlink{structvehicle__t}{vehicle\_t};                                \textcolor{comment}{/* Main compound type */}
00058 
00059 \textcolor{keyword}{typedef} \textcolor{keyword}{struct }\{
00060     \hyperlink{structhvl__t}{hvl\_t}       sensors;
00061     \textcolor{keywordtype}{char}        *name;
00062 \} \hyperlink{structrvehicle__t}{rvehicle\_t};                               \textcolor{comment}{/* Read type */}
00063 
00064 \textcolor{keywordtype}{int}
00065 main (\textcolor{keywordtype}{void})
00066 \{
00067     hid\_t       \hyperlink{structfile}{file}, vehicletype, colortype, sensortype, sensorstype, loctype,
00068                 strtype, rvehicletype, rsensortype, rsensorstype, space, dset,
00069                 group, attr;
00070                                             \textcolor{comment}{/* Handles */}
00071     herr\_t      status;
00072     hsize\_t     dims[1] = \{DIM0\},
00073                 adims[1] = \{3\},
00074                 adims2[2] = \{32, 32\},
00075                 start[2] = \{8, 26\},
00076                 count[2] = \{4, 3\},
00077                 coords[3][2] = \{ \{3, 2\},
00078                                  \{3, 3\},
00079                                  \{4, 4\} \};
00080     \hyperlink{structvehicle__t}{vehicle\_t}   wdata[2];                   \textcolor{comment}{/* Write buffer */}
00081     \hyperlink{structrvehicle__t}{rvehicle\_t}  *rdata;                     \textcolor{comment}{/* Read buffer */}
00082     color\_t     val;
00083     \hyperlink{structsensor__t}{sensor\_t}    *ptr;
00084     \textcolor{keywordtype}{double}      wdata2[32][32];
00085     \textcolor{keywordtype}{int}         ndims,
00086                 i, j;
00087 
00088     \textcolor{comment}{/*}
00089 \textcolor{comment}{     * Create a new file using the default properties.}
00090 \textcolor{comment}{     */}
00091     file = H5Fcreate (FILE, H5F\_ACC\_TRUNC, H5P\_DEFAULT, H5P\_DEFAULT);
00092 
00093     \textcolor{comment}{/*}
00094 \textcolor{comment}{     * Create dataset to use for region references.}
00095 \textcolor{comment}{     */}
00096     \textcolor{keywordflow}{for} (i=0; i<32; i++)
00097         \textcolor{keywordflow}{for} (j=0; j<32; j++)
00098             wdata2[i][j]= 70. + 0.1 * (i - 16.) + 0.1 * (j - 16.);
00099     space = H5Screate\_simple (2, adims2, NULL);
00100     dset = H5Dcreate (file, \textcolor{stringliteral}{"Ambient\_Temperature"}, H5T\_NATIVE\_DOUBLE, space,
00101                 H5P\_DEFAULT, H5P\_DEFAULT, H5P\_DEFAULT);
00102     status = H5Dwrite (dset, H5T\_NATIVE\_DOUBLE, H5S\_ALL, H5S\_ALL, H5P\_DEFAULT,
00103                 wdata2[0]);
00104     status = H5Dclose (dset);
00105 
00106     \textcolor{comment}{/*}
00107 \textcolor{comment}{     * Create groups to use for object references.}
00108 \textcolor{comment}{     */}
00109     group = H5Gcreate (file, \textcolor{stringliteral}{"Land\_Vehicles"}, H5P\_DEFAULT, H5P\_DEFAULT,
00110                 H5P\_DEFAULT);
00111     status = H5Gclose (group);
00112     group = H5Gcreate (file, \textcolor{stringliteral}{"Air\_Vehicles"}, H5P\_DEFAULT, H5P\_DEFAULT,
00113                 H5P\_DEFAULT);
00114     status = H5Gclose (group);
00115 
00116     \textcolor{comment}{/*}
00117 \textcolor{comment}{     * Initialize variable-length compound in the first data element.}
00118 \textcolor{comment}{     */}
00119     wdata[0].sensors.len = 4;
00120     ptr = (\hyperlink{structsensor__t}{sensor\_t} *) malloc (wdata[0].sensors.len * sizeof (\hyperlink{structsensor__t}{sensor\_t}));
00121     ptr[0].serial\_no = 1153;
00122     ptr[0].location = \textcolor{stringliteral}{"Exterior (static)"};
00123     ptr[0].temperature = 53.23;
00124     ptr[0].pressure = 24.57;
00125     ptr[1].serial\_no = 1184;
00126     ptr[1].location = \textcolor{stringliteral}{"Intake"};
00127     ptr[1].temperature = 55.12;
00128     ptr[1].pressure = 22.95;
00129     ptr[2].serial\_no = 1027;
00130     ptr[2].location = \textcolor{stringliteral}{"Intake manifold"};
00131     ptr[2].temperature = 103.55;
00132     ptr[2].pressure = 31.23;
00133     ptr[3].serial\_no = 1313;
00134     ptr[3].location = \textcolor{stringliteral}{"Exhaust manifold"};
00135     ptr[3].temperature = 1252.89;
00136     ptr[3].pressure = 84.11;
00137     wdata[0].sensors.p = (\textcolor{keywordtype}{void} *) ptr;
00138 
00139     \textcolor{comment}{/*}
00140 \textcolor{comment}{     * Initialize other fields in the first data element.}
00141 \textcolor{comment}{     */}
00142     wdata[0].name = \textcolor{stringliteral}{"Airplane"};
00143     wdata[0].color = GREEN;
00144     wdata[0].location[0] = -103234.21;
00145     wdata[0].location[1] = 422638.78;
00146     wdata[0].location[2] = 5996.43;
00147     status = H5Rcreate (&wdata[0].group, file, \textcolor{stringliteral}{"Air\_Vehicles"}, H5R\_OBJECT, -1);
00148     status = H5Sselect\_elements (space, H5S\_SELECT\_SET, 3, coords[0]);
00149     status = H5Rcreate (&wdata[0].surveyed\_areas, file, \textcolor{stringliteral}{"Ambient\_Temperature"},
00150                 H5R\_DATASET\_REGION, space);
00151 
00152     \textcolor{comment}{/*}
00153 \textcolor{comment}{     * Initialize variable-length compound in the second data element.}
00154 \textcolor{comment}{     */}
00155     wdata[1].sensors.len = 1;
00156     ptr = (\hyperlink{structsensor__t}{sensor\_t} *) malloc (wdata[1].sensors.len * sizeof (\hyperlink{structsensor__t}{sensor\_t}));
00157     ptr[0].serial\_no = 3244;
00158     ptr[0].location = \textcolor{stringliteral}{"Roof"};
00159     ptr[0].temperature = 83.82;
00160     ptr[0].pressure = 29.92;
00161     wdata[1].sensors.p = (\textcolor{keywordtype}{void} *) ptr;
00162 
00163     \textcolor{comment}{/*}
00164 \textcolor{comment}{     * Initialize other fields in the second data element.}
00165 \textcolor{comment}{     */}
00166     wdata[1].name = \textcolor{stringliteral}{"Automobile"};
00167     wdata[1].color = RED;
00168     wdata[1].location[0] = 326734.36;
00169     wdata[1].location[1] = 221568.23;
00170     wdata[1].location[2] = 432.36;
00171     status = H5Rcreate (&wdata[1].group, file, \textcolor{stringliteral}{"Land\_Vehicles"}, H5R\_OBJECT, -1);
00172     status = H5Sselect\_hyperslab (space, H5S\_SELECT\_SET, start, NULL, count,
00173                 NULL);
00174     status = H5Rcreate (&wdata[1].surveyed\_areas, file, \textcolor{stringliteral}{"Ambient\_Temperature"},
00175                 H5R\_DATASET\_REGION, space);
00176 
00177     status = H5Sclose (space);
00178 
00179     \textcolor{comment}{/*}
00180 \textcolor{comment}{     * Create variable-length string datatype.}
00181 \textcolor{comment}{     */}
00182     strtype = H5Tcopy (H5T\_C\_S1);
00183     status = H5Tset\_size (strtype, H5T\_VARIABLE);
00184 
00185     \textcolor{comment}{/*}
00186 \textcolor{comment}{     * Create the nested compound datatype.}
00187 \textcolor{comment}{     */}
00188     sensortype = H5Tcreate (H5T\_COMPOUND, \textcolor{keyword}{sizeof} (\hyperlink{structsensor__t}{sensor\_t}));
00189     status = H5Tinsert (sensortype, \textcolor{stringliteral}{"Serial number"},
00190                 HOFFSET (\hyperlink{structsensor__t}{sensor\_t}, serial\_no), H5T\_NATIVE\_INT);
00191     status = H5Tinsert (sensortype, \textcolor{stringliteral}{"Location"}, HOFFSET (\hyperlink{structsensor__t}{sensor\_t}, location),
00192                 strtype);
00193     status = H5Tinsert (sensortype, \textcolor{stringliteral}{"Temperature (F)"},
00194                 HOFFSET (\hyperlink{structsensor__t}{sensor\_t}, temperature), H5T\_NATIVE\_DOUBLE);
00195     status = H5Tinsert (sensortype, \textcolor{stringliteral}{"Pressure (inHg)"},
00196                 HOFFSET (\hyperlink{structsensor__t}{sensor\_t}, pressure), H5T\_NATIVE\_DOUBLE);
00197 
00198     \textcolor{comment}{/*}
00199 \textcolor{comment}{     * Create the variable-length datatype.}
00200 \textcolor{comment}{     */}
00201     sensorstype = H5Tvlen\_create (sensortype);
00202 
00203     \textcolor{comment}{/*}
00204 \textcolor{comment}{     * Create the enumerated datatype.}
00205 \textcolor{comment}{     */}
00206     colortype = H5Tenum\_create (H5T\_NATIVE\_INT);
00207     val = (color\_t) RED;
00208     status = H5Tenum\_insert (colortype, \textcolor{stringliteral}{"Red"}, &val);
00209     val = (color\_t) GREEN;
00210     status = H5Tenum\_insert (colortype, \textcolor{stringliteral}{"Green"}, &val);
00211     val = (color\_t) BLUE;
00212     status = H5Tenum\_insert (colortype, \textcolor{stringliteral}{"Blue"}, &val);
00213 
00214     \textcolor{comment}{/*}
00215 \textcolor{comment}{     * Create the array datatype.}
00216 \textcolor{comment}{     */}
00217     loctype = H5Tarray\_create (H5T\_NATIVE\_DOUBLE, 1, adims);
00218 
00219     \textcolor{comment}{/*}
00220 \textcolor{comment}{     * Create the main compound datatype.}
00221 \textcolor{comment}{     */}
00222     vehicletype = H5Tcreate (H5T\_COMPOUND, \textcolor{keyword}{sizeof} (\hyperlink{structvehicle__t}{vehicle\_t}));
00223     status = H5Tinsert (vehicletype, \textcolor{stringliteral}{"Sensors"}, HOFFSET (\hyperlink{structvehicle__t}{vehicle\_t}, sensors),
00224                 sensorstype);
00225     status = H5Tinsert (vehicletype, \textcolor{stringliteral}{"Name"}, HOFFSET (\hyperlink{structvehicle__t}{vehicle\_t}, name),
00226                 strtype);
00227     status = H5Tinsert (vehicletype, \textcolor{stringliteral}{"Color"}, HOFFSET (\hyperlink{structvehicle__t}{vehicle\_t}, color),
00228                 colortype);
00229     status = H5Tinsert (vehicletype, \textcolor{stringliteral}{"Location"}, HOFFSET (\hyperlink{structvehicle__t}{vehicle\_t}, location),
00230                 loctype);
00231     status = H5Tinsert (vehicletype, \textcolor{stringliteral}{"Group"}, HOFFSET (\hyperlink{structvehicle__t}{vehicle\_t}, group),
00232                 H5T\_STD\_REF\_OBJ);
00233     status = H5Tinsert (vehicletype, \textcolor{stringliteral}{"Surveyed areas"},
00234                 HOFFSET (\hyperlink{structvehicle__t}{vehicle\_t}, surveyed\_areas), H5T\_STD\_REF\_DSETREG);
00235 
00236     \textcolor{comment}{/*}
00237 \textcolor{comment}{     * Create dataset with a null dataspace. to serve as the parent for}
00238 \textcolor{comment}{     * the attribute.}
00239 \textcolor{comment}{     */}
00240     space = H5Screate (H5S\_NULL);
00241     dset = H5Dcreate (file, DATASET, H5T\_STD\_I32LE, space, H5P\_DEFAULT,
00242                 H5P\_DEFAULT, H5P\_DEFAULT);
00243     status = H5Sclose (space);
00244 
00245     \textcolor{comment}{/*}
00246 \textcolor{comment}{     * Create dataspace.  Setting maximum size to NULL sets the maximum}
00247 \textcolor{comment}{     * size to be the current size.}
00248 \textcolor{comment}{     */}
00249     space = H5Screate\_simple (1, dims, NULL);
00250 
00251     \textcolor{comment}{/*}
00252 \textcolor{comment}{     * Create the attribute and write the compound data to it.}
00253 \textcolor{comment}{     */}
00254     attr = H5Acreate (dset, ATTRIBUTE, vehicletype, space, H5P\_DEFAULT,
00255                 H5P\_DEFAULT);
00256     status = H5Awrite (attr, vehicletype, wdata);
00257 
00258 
00259     \textcolor{comment}{/*}
00260 \textcolor{comment}{     * Close and release resources.  Note that we cannot use}
00261 \textcolor{comment}{     * H5Dvlen\_reclaim as it would attempt to free() the string}
00262 \textcolor{comment}{     * constants used to initialize the name fields in wdata.  We must}
00263 \textcolor{comment}{     * therefore manually free() only the data previously allocated}
00264 \textcolor{comment}{     * through malloc().}
00265 \textcolor{comment}{     */}
00266     \textcolor{keywordflow}{for} (i=0; i<dims[0]; i++)
00267         free (wdata[i].sensors.p);
00268     status = H5Aclose (attr);
00269     status = H5Dclose (dset);
00270     status = H5Sclose (space);
00271     status = H5Tclose (strtype);
00272     status = H5Tclose (sensortype);
00273     status = H5Tclose (sensorstype);
00274     status = H5Tclose (colortype);
00275     status = H5Tclose (loctype);
00276     status = H5Tclose (vehicletype);
00277     status = H5Fclose (file);
00278 
00279     \textcolor{comment}{/*}
00280 \textcolor{comment}{     * Now we begin the read section of this example.  Here we assume}
00281 \textcolor{comment}{     * the attribute has the same name and rank, but can have any size.}
00282 \textcolor{comment}{     * Therefore we must allocate a new array to read in data using}
00283 \textcolor{comment}{     * malloc().  We will only read back the variable length strings.}
00284 \textcolor{comment}{     */}
00285 
00286     \textcolor{comment}{/*}
00287 \textcolor{comment}{     * Open file, dataset, and attribute.}
00288 \textcolor{comment}{     */}
00289     file = H5Fopen (FILE, H5F\_ACC\_RDONLY, H5P\_DEFAULT);
00290     dset = H5Dopen (file, DATASET, H5P\_DEFAULT);
00291     attr = H5Aopen (dset, ATTRIBUTE, H5P\_DEFAULT);
00292 
00293     \textcolor{comment}{/*}
00294 \textcolor{comment}{     * Create variable-length string datatype.}
00295 \textcolor{comment}{     */}
00296     strtype = H5Tcopy (H5T\_C\_S1);
00297     status = H5Tset\_size (strtype, H5T\_VARIABLE);
00298 
00299     \textcolor{comment}{/*}
00300 \textcolor{comment}{     * Create the nested compound datatype for reading.  Even though it}
00301 \textcolor{comment}{     * has only one field, it must still be defined as a compound type}
00302 \textcolor{comment}{     * so the library can match the correct field in the file type.}
00303 \textcolor{comment}{     * This matching is done by name.  However, we do not need to}
00304 \textcolor{comment}{     * define a structure for the read buffer as we can simply treat it}
00305 \textcolor{comment}{     * as a char *.}
00306 \textcolor{comment}{     */}
00307     rsensortype = H5Tcreate (H5T\_COMPOUND, \textcolor{keyword}{sizeof} (\textcolor{keywordtype}{char} *));
00308     status = H5Tinsert (rsensortype, \textcolor{stringliteral}{"Location"}, 0, strtype);
00309 
00310     \textcolor{comment}{/*}
00311 \textcolor{comment}{     * Create the variable-length datatype for reading.}
00312 \textcolor{comment}{     */}
00313     rsensorstype = H5Tvlen\_create (rsensortype);
00314 
00315     \textcolor{comment}{/*}
00316 \textcolor{comment}{     * Create the main compound datatype for reading.}
00317 \textcolor{comment}{     */}
00318     rvehicletype = H5Tcreate (H5T\_COMPOUND, \textcolor{keyword}{sizeof} (\hyperlink{structrvehicle__t}{rvehicle\_t}));
00319     status = H5Tinsert (rvehicletype, \textcolor{stringliteral}{"Sensors"}, HOFFSET (\hyperlink{structrvehicle__t}{rvehicle\_t}, sensors),
00320                 rsensorstype);
00321     status = H5Tinsert (rvehicletype, \textcolor{stringliteral}{"Name"}, HOFFSET (\hyperlink{structrvehicle__t}{rvehicle\_t}, name),
00322                 strtype);
00323 
00324     \textcolor{comment}{/*}
00325 \textcolor{comment}{     * Get dataspace and allocate memory for read buffer.}
00326 \textcolor{comment}{     */}
00327     space = H5Aget\_space (attr);
00328     ndims = H5Sget\_simple\_extent\_dims (space, dims, NULL);
00329     rdata = (\hyperlink{structrvehicle__t}{rvehicle\_t} *) malloc (dims[0] * \textcolor{keyword}{sizeof} (\hyperlink{structrvehicle__t}{rvehicle\_t}));
00330 
00331     \textcolor{comment}{/*}
00332 \textcolor{comment}{     * Read the data.}
00333 \textcolor{comment}{     */}
00334     status = H5Aread (attr, rvehicletype, rdata);
00335 
00336     \textcolor{comment}{/*}
00337 \textcolor{comment}{     * Output the data to the screen.}
00338 \textcolor{comment}{     */}
00339     \textcolor{keywordflow}{for} (i=0; i<dims[0]; i++) \{
00340         printf (\textcolor{stringliteral}{"%s[%d]:\(\backslash\)n"}, ATTRIBUTE, i);
00341         printf (\textcolor{stringliteral}{"   Vehicle name :\(\backslash\)n      %s\(\backslash\)n"}, rdata[i].name);
00342         printf (\textcolor{stringliteral}{"   Sensor locations :\(\backslash\)n"});
00343         \textcolor{keywordflow}{for} (j=0; j<rdata[i].sensors.len; j++)
00344             printf (\textcolor{stringliteral}{"      %s\(\backslash\)n"}, ( (\textcolor{keywordtype}{char} **) rdata[i].sensors.p )[j] );
00345     \}
00346 
00347     \textcolor{comment}{/*}
00348 \textcolor{comment}{     * Close and release resources.  H5Dvlen\_reclaim will automatically}
00349 \textcolor{comment}{     * traverse the structure and free any vlen data (including}
00350 \textcolor{comment}{     * strings).}
00351 \textcolor{comment}{     */}
00352     status = H5Dvlen\_reclaim (rvehicletype, space, H5P\_DEFAULT, rdata);
00353     free (rdata);
00354     status = H5Aclose (attr);
00355     status = H5Dclose (dset);
00356     status = H5Sclose (space);
00357     status = H5Tclose (strtype);
00358     status = H5Tclose (rsensortype);
00359     status = H5Tclose (rsensorstype);
00360     status = H5Tclose (rvehicletype);
00361     status = H5Fclose (file);
00362 
00363     \textcolor{keywordflow}{return} 0;
00364 \}
\end{DoxyCode}
