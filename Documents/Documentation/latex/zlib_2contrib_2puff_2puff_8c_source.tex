\hypertarget{zlib_2contrib_2puff_2puff_8c_source}{}\section{zlib/contrib/puff/puff.c}
\label{zlib_2contrib_2puff_2puff_8c_source}\index{puff.\+c@{puff.\+c}}

\begin{DoxyCode}
00001 \textcolor{comment}{/*}
00002 \textcolor{comment}{ * puff.c}
00003 \textcolor{comment}{ * Copyright (C) 2002-2013 Mark Adler}
00004 \textcolor{comment}{ * For conditions of distribution and use, see copyright notice in puff.h}
00005 \textcolor{comment}{ * version 2.3, 21 Jan 2013}
00006 \textcolor{comment}{ *}
00007 \textcolor{comment}{ * puff.c is a simple inflate written to be an unambiguous way to specify the}
00008 \textcolor{comment}{ * deflate format.  It is not written for speed but rather simplicity.  As a}
00009 \textcolor{comment}{ * side benefit, this code might actually be useful when small code is more}
00010 \textcolor{comment}{ * important than speed, such as bootstrap applications.  For typical deflate}
00011 \textcolor{comment}{ * data, zlib's inflate() is about four times as fast as puff().  zlib's}
00012 \textcolor{comment}{ * inflate compiles to around 20K on my machine, whereas puff.c compiles to}
00013 \textcolor{comment}{ * around 4K on my machine (a PowerPC using GNU cc).  If the faster decode()}
00014 \textcolor{comment}{ * function here is used, then puff() is only twice as slow as zlib's}
00015 \textcolor{comment}{ * inflate().}
00016 \textcolor{comment}{ *}
00017 \textcolor{comment}{ * All dynamically allocated memory comes from the stack.  The stack required}
00018 \textcolor{comment}{ * is less than 2K bytes.  This code is compatible with 16-bit int's and}
00019 \textcolor{comment}{ * assumes that long's are at least 32 bits.  puff.c uses the short data type,}
00020 \textcolor{comment}{ * assumed to be 16 bits, for arrays in order to conserve memory.  The code}
00021 \textcolor{comment}{ * works whether integers are stored big endian or little endian.}
00022 \textcolor{comment}{ *}
00023 \textcolor{comment}{ * In the comments below are "Format notes" that describe the inflate process}
00024 \textcolor{comment}{ * and document some of the less obvious aspects of the format.  This source}
00025 \textcolor{comment}{ * code is meant to supplement RFC 1951, which formally describes the deflate}
00026 \textcolor{comment}{ * format:}
00027 \textcolor{comment}{ *}
00028 \textcolor{comment}{ *    http://www.zlib.org/rfc-deflate.html}
00029 \textcolor{comment}{ */}
00030 
00031 \textcolor{comment}{/*}
00032 \textcolor{comment}{ * Change history:}
00033 \textcolor{comment}{ *}
00034 \textcolor{comment}{ * 1.0  10 Feb 2002     - First version}
00035 \textcolor{comment}{ * 1.1  17 Feb 2002     - Clarifications of some comments and notes}
00036 \textcolor{comment}{ *                      - Update puff() dest and source pointers on negative}
00037 \textcolor{comment}{ *                        errors to facilitate debugging deflators}
00038 \textcolor{comment}{ *                      - Remove longest from struct huffman -- not needed}
00039 \textcolor{comment}{ *                      - Simplify offs[] index in construct()}
00040 \textcolor{comment}{ *                      - Add input size and checking, using longjmp() to}
00041 \textcolor{comment}{ *                        maintain easy readability}
00042 \textcolor{comment}{ *                      - Use short data type for large arrays}
00043 \textcolor{comment}{ *                      - Use pointers instead of long to specify source and}
00044 \textcolor{comment}{ *                        destination sizes to avoid arbitrary 4 GB limits}
00045 \textcolor{comment}{ * 1.2  17 Mar 2002     - Add faster version of decode(), doubles speed (!),}
00046 \textcolor{comment}{ *                        but leave simple version for readabilty}
00047 \textcolor{comment}{ *                      - Make sure invalid distances detected if pointers}
00048 \textcolor{comment}{ *                        are 16 bits}
00049 \textcolor{comment}{ *                      - Fix fixed codes table error}
00050 \textcolor{comment}{ *                      - Provide a scanning mode for determining size of}
00051 \textcolor{comment}{ *                        uncompressed data}
00052 \textcolor{comment}{ * 1.3  20 Mar 2002     - Go back to lengths for puff() parameters [Gailly]}
00053 \textcolor{comment}{ *                      - Add a puff.h file for the interface}
00054 \textcolor{comment}{ *                      - Add braces in puff() for else do [Gailly]}
00055 \textcolor{comment}{ *                      - Use indexes instead of pointers for readability}
00056 \textcolor{comment}{ * 1.4  31 Mar 2002     - Simplify construct() code set check}
00057 \textcolor{comment}{ *                      - Fix some comments}
00058 \textcolor{comment}{ *                      - Add FIXLCODES #define}
00059 \textcolor{comment}{ * 1.5   6 Apr 2002     - Minor comment fixes}
00060 \textcolor{comment}{ * 1.6   7 Aug 2002     - Minor format changes}
00061 \textcolor{comment}{ * 1.7   3 Mar 2003     - Added test code for distribution}
00062 \textcolor{comment}{ *                      - Added zlib-like license}
00063 \textcolor{comment}{ * 1.8   9 Jan 2004     - Added some comments on no distance codes case}
00064 \textcolor{comment}{ * 1.9  21 Feb 2008     - Fix bug on 16-bit integer architectures [Pohland]}
00065 \textcolor{comment}{ *                      - Catch missing end-of-block symbol error}
00066 \textcolor{comment}{ * 2.0  25 Jul 2008     - Add #define to permit distance too far back}
00067 \textcolor{comment}{ *                      - Add option in TEST code for puff to write the data}
00068 \textcolor{comment}{ *                      - Add option in TEST code to skip input bytes}
00069 \textcolor{comment}{ *                      - Allow TEST code to read from piped stdin}
00070 \textcolor{comment}{ * 2.1   4 Apr 2010     - Avoid variable initialization for happier compilers}
00071 \textcolor{comment}{ *                      - Avoid unsigned comparisons for even happier compilers}
00072 \textcolor{comment}{ * 2.2  25 Apr 2010     - Fix bug in variable initializations [Oberhumer]}
00073 \textcolor{comment}{ *                      - Add const where appropriate [Oberhumer]}
00074 \textcolor{comment}{ *                      - Split if's and ?'s for coverage testing}
00075 \textcolor{comment}{ *                      - Break out test code to separate file}
00076 \textcolor{comment}{ *                      - Move NIL to puff.h}
00077 \textcolor{comment}{ *                      - Allow incomplete code only if single code length is 1}
00078 \textcolor{comment}{ *                      - Add full code coverage test to Makefile}
00079 \textcolor{comment}{ * 2.3  21 Jan 2013     - Check for invalid code length codes in dynamic blocks}
00080 \textcolor{comment}{ */}
00081 
00082 \textcolor{preprocessor}{#include <setjmp.h>}             \textcolor{comment}{/* for setjmp(), longjmp(), and jmp\_buf */}
00083 \textcolor{preprocessor}{#include "puff.h"}               \textcolor{comment}{/* prototype for puff() */}
00084 
00085 \textcolor{preprocessor}{#define local static            }\textcolor{comment}{/* for local function definitions */}\textcolor{preprocessor}{}
00086 
00087 \textcolor{comment}{/*}
00088 \textcolor{comment}{ * Maximums for allocations and loops.  It is not useful to change these --}
00089 \textcolor{comment}{ * they are fixed by the deflate format.}
00090 \textcolor{comment}{ */}
00091 \textcolor{preprocessor}{#define MAXBITS 15              }\textcolor{comment}{/* maximum bits in a code */}\textcolor{preprocessor}{}
00092 \textcolor{preprocessor}{#define MAXLCODES 286           }\textcolor{comment}{/* maximum number of literal/length codes */}\textcolor{preprocessor}{}
00093 \textcolor{preprocessor}{#define MAXDCODES 30            }\textcolor{comment}{/* maximum number of distance codes */}\textcolor{preprocessor}{}
00094 \textcolor{preprocessor}{#define MAXCODES (MAXLCODES+MAXDCODES)  }\textcolor{comment}{/* maximum codes lengths to read */}\textcolor{preprocessor}{}
00095 \textcolor{preprocessor}{#define FIXLCODES 288           }\textcolor{comment}{/* number of fixed literal/length codes */}\textcolor{preprocessor}{}
00096 
00097 \textcolor{comment}{/* input and output state */}
00098 \textcolor{keyword}{struct }\hyperlink{structstate}{state} \{
00099     \textcolor{comment}{/* output state */}
00100     \textcolor{keywordtype}{unsigned} \textcolor{keywordtype}{char} *out;         \textcolor{comment}{/* output buffer */}
00101     \textcolor{keywordtype}{unsigned} \textcolor{keywordtype}{long} outlen;       \textcolor{comment}{/* available space at out */}
00102     \textcolor{keywordtype}{unsigned} \textcolor{keywordtype}{long} outcnt;       \textcolor{comment}{/* bytes written to out so far */}
00103 
00104     \textcolor{comment}{/* input state */}
00105     \textcolor{keyword}{const} \textcolor{keywordtype}{unsigned} \textcolor{keywordtype}{char} *in;    \textcolor{comment}{/* input buffer */}
00106     \textcolor{keywordtype}{unsigned} \textcolor{keywordtype}{long} inlen;        \textcolor{comment}{/* available input at in */}
00107     \textcolor{keywordtype}{unsigned} \textcolor{keywordtype}{long} incnt;        \textcolor{comment}{/* bytes read so far */}
00108     \textcolor{keywordtype}{int} bitbuf;                 \textcolor{comment}{/* bit buffer */}
00109     \textcolor{keywordtype}{int} bitcnt;                 \textcolor{comment}{/* number of bits in bit buffer */}
00110 
00111     \textcolor{comment}{/* input limit error return state for bits() and decode() */}
00112     jmp\_buf env;
00113 \};
00114 
00115 \textcolor{comment}{/*}
00116 \textcolor{comment}{ * Return need bits from the input stream.  This always leaves less than}
00117 \textcolor{comment}{ * eight bits in the buffer.  bits() works properly for need == 0.}
00118 \textcolor{comment}{ *}
00119 \textcolor{comment}{ * Format notes:}
00120 \textcolor{comment}{ *}
00121 \textcolor{comment}{ * - Bits are stored in bytes from the least significant bit to the most}
00122 \textcolor{comment}{ *   significant bit.  Therefore bits are dropped from the bottom of the bit}
00123 \textcolor{comment}{ *   buffer, using shift right, and new bytes are appended to the top of the}
00124 \textcolor{comment}{ *   bit buffer, using shift left.}
00125 \textcolor{comment}{ */}
00126 local \textcolor{keywordtype}{int} bits(\textcolor{keyword}{struct} \hyperlink{structstate}{state} *s, \textcolor{keywordtype}{int} need)
00127 \{
00128     \textcolor{keywordtype}{long} val;           \textcolor{comment}{/* bit accumulator (can use up to 20 bits) */}
00129 
00130     \textcolor{comment}{/* load at least need bits into val */}
00131     val = s->bitbuf;
00132     \textcolor{keywordflow}{while} (s->bitcnt < need) \{
00133         \textcolor{keywordflow}{if} (s->incnt == s->inlen)
00134             longjmp(s->env, 1);         \textcolor{comment}{/* out of input */}
00135         val |= (long)(s->in[s->incnt++]) << s->bitcnt;  \textcolor{comment}{/* load eight bits */}
00136         s->bitcnt += 8;
00137     \}
00138 
00139     \textcolor{comment}{/* drop need bits and update buffer, always zero to seven bits left */}
00140     s->bitbuf = (int)(val >> need);
00141     s->bitcnt -= need;
00142 
00143     \textcolor{comment}{/* return need bits, zeroing the bits above that */}
00144     \textcolor{keywordflow}{return} (\textcolor{keywordtype}{int})(val & ((1L << need) - 1));
00145 \}
00146 
00147 \textcolor{comment}{/*}
00148 \textcolor{comment}{ * Process a stored block.}
00149 \textcolor{comment}{ *}
00150 \textcolor{comment}{ * Format notes:}
00151 \textcolor{comment}{ *}
00152 \textcolor{comment}{ * - After the two-bit stored block type (00), the stored block length and}
00153 \textcolor{comment}{ *   stored bytes are byte-aligned for fast copying.  Therefore any leftover}
00154 \textcolor{comment}{ *   bits in the byte that has the last bit of the type, as many as seven, are}
00155 \textcolor{comment}{ *   discarded.  The value of the discarded bits are not defined and should not}
00156 \textcolor{comment}{ *   be checked against any expectation.}
00157 \textcolor{comment}{ *}
00158 \textcolor{comment}{ * - The second inverted copy of the stored block length does not have to be}
00159 \textcolor{comment}{ *   checked, but it's probably a good idea to do so anyway.}
00160 \textcolor{comment}{ *}
00161 \textcolor{comment}{ * - A stored block can have zero length.  This is sometimes used to byte-align}
00162 \textcolor{comment}{ *   subsets of the compressed data for random access or partial recovery.}
00163 \textcolor{comment}{ */}
00164 local \textcolor{keywordtype}{int} stored(\textcolor{keyword}{struct} \hyperlink{structstate}{state} *s)
00165 \{
00166     \textcolor{keywordtype}{unsigned} len;       \textcolor{comment}{/* length of stored block */}
00167 
00168     \textcolor{comment}{/* discard leftover bits from current byte (assumes s->bitcnt < 8) */}
00169     s->bitbuf = 0;
00170     s->bitcnt = 0;
00171 
00172     \textcolor{comment}{/* get length and check against its one's complement */}
00173     \textcolor{keywordflow}{if} (s->incnt + 4 > s->inlen)
00174         \textcolor{keywordflow}{return} 2;                               \textcolor{comment}{/* not enough input */}
00175     len = s->in[s->incnt++];
00176     len |= s->in[s->incnt++] << 8;
00177     \textcolor{keywordflow}{if} (s->in[s->incnt++] != (~len & 0xff) ||
00178         s->in[s->incnt++] != ((~len >> 8) & 0xff))
00179         \textcolor{keywordflow}{return} -2;                              \textcolor{comment}{/* didn't match complement! */}
00180 
00181     \textcolor{comment}{/* copy len bytes from in to out */}
00182     \textcolor{keywordflow}{if} (s->incnt + len > s->inlen)
00183         \textcolor{keywordflow}{return} 2;                               \textcolor{comment}{/* not enough input */}
00184     \textcolor{keywordflow}{if} (s->out != NIL) \{
00185         \textcolor{keywordflow}{if} (s->outcnt + len > s->outlen)
00186             \textcolor{keywordflow}{return} 1;                           \textcolor{comment}{/* not enough output space */}
00187         \textcolor{keywordflow}{while} (len--)
00188             s->out[s->outcnt++] = s->in[s->incnt++];
00189     \}
00190     \textcolor{keywordflow}{else} \{                                      \textcolor{comment}{/* just scanning */}
00191         s->outcnt += len;
00192         s->incnt += len;
00193     \}
00194 
00195     \textcolor{comment}{/* done with a valid stored block */}
00196     \textcolor{keywordflow}{return} 0;
00197 \}
00198 
00199 \textcolor{comment}{/*}
00200 \textcolor{comment}{ * Huffman code decoding tables.  count[1..MAXBITS] is the number of symbols of}
00201 \textcolor{comment}{ * each length, which for a canonical code are stepped through in order.}
00202 \textcolor{comment}{ * symbol[] are the symbol values in canonical order, where the number of}
00203 \textcolor{comment}{ * entries is the sum of the counts in count[].  The decoding process can be}
00204 \textcolor{comment}{ * seen in the function decode() below.}
00205 \textcolor{comment}{ */}
00206 \textcolor{keyword}{struct }\hyperlink{structhuffman}{huffman} \{
00207     \textcolor{keywordtype}{short} *count;       \textcolor{comment}{/* number of symbols of each length */}
00208     \textcolor{keywordtype}{short} *symbol;      \textcolor{comment}{/* canonically ordered symbols */}
00209 \};
00210 
00211 \textcolor{comment}{/*}
00212 \textcolor{comment}{ * Decode a code from the stream s using huffman table h.  Return the symbol or}
00213 \textcolor{comment}{ * a negative value if there is an error.  If all of the lengths are zero, i.e.}
00214 \textcolor{comment}{ * an empty code, or if the code is incomplete and an invalid code is received,}
00215 \textcolor{comment}{ * then -10 is returned after reading MAXBITS bits.}
00216 \textcolor{comment}{ *}
00217 \textcolor{comment}{ * Format notes:}
00218 \textcolor{comment}{ *}
00219 \textcolor{comment}{ * - The codes as stored in the compressed data are bit-reversed relative to}
00220 \textcolor{comment}{ *   a simple integer ordering of codes of the same lengths.  Hence below the}
00221 \textcolor{comment}{ *   bits are pulled from the compressed data one at a time and used to}
00222 \textcolor{comment}{ *   build the code value reversed from what is in the stream in order to}
00223 \textcolor{comment}{ *   permit simple integer comparisons for decoding.  A table-based decoding}
00224 \textcolor{comment}{ *   scheme (as used in zlib) does not need to do this reversal.}
00225 \textcolor{comment}{ *}
00226 \textcolor{comment}{ * - The first code for the shortest length is all zeros.  Subsequent codes of}
00227 \textcolor{comment}{ *   the same length are simply integer increments of the previous code.  When}
00228 \textcolor{comment}{ *   moving up a length, a zero bit is appended to the code.  For a complete}
00229 \textcolor{comment}{ *   code, the last code of the longest length will be all ones.}
00230 \textcolor{comment}{ *}
00231 \textcolor{comment}{ * - Incomplete codes are handled by this decoder, since they are permitted}
00232 \textcolor{comment}{ *   in the deflate format.  See the format notes for fixed() and dynamic().}
00233 \textcolor{comment}{ */}
00234 \textcolor{preprocessor}{#ifdef SLOW}
00235 local \textcolor{keywordtype}{int} decode(\textcolor{keyword}{struct} \hyperlink{structstate}{state} *s, \textcolor{keyword}{const} \textcolor{keyword}{struct} \hyperlink{structhuffman}{huffman} *h)
00236 \{
00237     \textcolor{keywordtype}{int} len;            \textcolor{comment}{/* current number of bits in code */}
00238     \textcolor{keywordtype}{int} \hyperlink{structcode}{code};           \textcolor{comment}{/* len bits being decoded */}
00239     \textcolor{keywordtype}{int} first;          \textcolor{comment}{/* first code of length len */}
00240     \textcolor{keywordtype}{int} count;          \textcolor{comment}{/* number of codes of length len */}
00241     \textcolor{keywordtype}{int} index;          \textcolor{comment}{/* index of first code of length len in symbol table */}
00242 
00243     code = first = index = 0;
00244     \textcolor{keywordflow}{for} (len = 1; len <= MAXBITS; len++) \{
00245         code |= bits(s, 1);             \textcolor{comment}{/* get next bit */}
00246         count = h->count[len];
00247         \textcolor{keywordflow}{if} (code - count < first)       \textcolor{comment}{/* if length len, return symbol */}
00248             \textcolor{keywordflow}{return} h->symbol[index + (code - first)];
00249         index += count;                 \textcolor{comment}{/* else update for next length */}
00250         first += count;
00251         first <<= 1;
00252         code <<= 1;
00253     \}
00254     \textcolor{keywordflow}{return} -10;                         \textcolor{comment}{/* ran out of codes */}
00255 \}
00256 
00257 \textcolor{comment}{/*}
00258 \textcolor{comment}{ * A faster version of decode() for real applications of this code.   It's not}
00259 \textcolor{comment}{ * as readable, but it makes puff() twice as fast.  And it only makes the code}
00260 \textcolor{comment}{ * a few percent larger.}
00261 \textcolor{comment}{ */}
00262 \textcolor{preprocessor}{#else }\textcolor{comment}{/* !SLOW */}\textcolor{preprocessor}{}
00263 local \textcolor{keywordtype}{int} decode(\textcolor{keyword}{struct} \hyperlink{structstate}{state} *s, \textcolor{keyword}{const} \textcolor{keyword}{struct} \hyperlink{structhuffman}{huffman} *h)
00264 \{
00265     \textcolor{keywordtype}{int} len;            \textcolor{comment}{/* current number of bits in code */}
00266     \textcolor{keywordtype}{int} code;           \textcolor{comment}{/* len bits being decoded */}
00267     \textcolor{keywordtype}{int} first;          \textcolor{comment}{/* first code of length len */}
00268     \textcolor{keywordtype}{int} count;          \textcolor{comment}{/* number of codes of length len */}
00269     \textcolor{keywordtype}{int} index;          \textcolor{comment}{/* index of first code of length len in symbol table */}
00270     \textcolor{keywordtype}{int} bitbuf;         \textcolor{comment}{/* bits from stream */}
00271     \textcolor{keywordtype}{int} left;           \textcolor{comment}{/* bits left in next or left to process */}
00272     \textcolor{keywordtype}{short} *next;        \textcolor{comment}{/* next number of codes */}
00273 
00274     bitbuf = s->bitbuf;
00275     left = s->bitcnt;
00276     code = first = index = 0;
00277     len = 1;
00278     next = h->count + 1;
00279     \textcolor{keywordflow}{while} (1) \{
00280         \textcolor{keywordflow}{while} (left--) \{
00281             code |= bitbuf & 1;
00282             bitbuf >>= 1;
00283             count = *next++;
00284             \textcolor{keywordflow}{if} (code - count < first) \{ \textcolor{comment}{/* if length len, return symbol */}
00285                 s->bitbuf = bitbuf;
00286                 s->bitcnt = (s->bitcnt - len) & 7;
00287                 \textcolor{keywordflow}{return} h->symbol[index + (code - first)];
00288             \}
00289             index += count;             \textcolor{comment}{/* else update for next length */}
00290             first += count;
00291             first <<= 1;
00292             code <<= 1;
00293             len++;
00294         \}
00295         left = (MAXBITS+1) - len;
00296         \textcolor{keywordflow}{if} (left == 0)
00297             \textcolor{keywordflow}{break};
00298         \textcolor{keywordflow}{if} (s->incnt == s->inlen)
00299             longjmp(s->env, 1);         \textcolor{comment}{/* out of input */}
00300         bitbuf = s->in[s->incnt++];
00301         \textcolor{keywordflow}{if} (left > 8)
00302             left = 8;
00303     \}
00304     \textcolor{keywordflow}{return} -10;                         \textcolor{comment}{/* ran out of codes */}
00305 \}
00306 \textcolor{preprocessor}{#endif }\textcolor{comment}{/* SLOW */}\textcolor{preprocessor}{}
00307 
00308 \textcolor{comment}{/*}
00309 \textcolor{comment}{ * Given the list of code lengths length[0..n-1] representing a canonical}
00310 \textcolor{comment}{ * Huffman code for n symbols, construct the tables required to decode those}
00311 \textcolor{comment}{ * codes.  Those tables are the number of codes of each length, and the symbols}
00312 \textcolor{comment}{ * sorted by length, retaining their original order within each length.  The}
00313 \textcolor{comment}{ * return value is zero for a complete code set, negative for an over-}
00314 \textcolor{comment}{ * subscribed code set, and positive for an incomplete code set.  The tables}
00315 \textcolor{comment}{ * can be used if the return value is zero or positive, but they cannot be used}
00316 \textcolor{comment}{ * if the return value is negative.  If the return value is zero, it is not}
00317 \textcolor{comment}{ * possible for decode() using that table to return an error--any stream of}
00318 \textcolor{comment}{ * enough bits will resolve to a symbol.  If the return value is positive, then}
00319 \textcolor{comment}{ * it is possible for decode() using that table to return an error for received}
00320 \textcolor{comment}{ * codes past the end of the incomplete lengths.}
00321 \textcolor{comment}{ *}
00322 \textcolor{comment}{ * Not used by decode(), but used for error checking, h->count[0] is the number}
00323 \textcolor{comment}{ * of the n symbols not in the code.  So n - h->count[0] is the number of}
00324 \textcolor{comment}{ * codes.  This is useful for checking for incomplete codes that have more than}
00325 \textcolor{comment}{ * one symbol, which is an error in a dynamic block.}
00326 \textcolor{comment}{ *}
00327 \textcolor{comment}{ * Assumption: for all i in 0..n-1, 0 <= length[i] <= MAXBITS}
00328 \textcolor{comment}{ * This is assured by the construction of the length arrays in dynamic() and}
00329 \textcolor{comment}{ * fixed() and is not verified by construct().}
00330 \textcolor{comment}{ *}
00331 \textcolor{comment}{ * Format notes:}
00332 \textcolor{comment}{ *}
00333 \textcolor{comment}{ * - Permitted and expected examples of incomplete codes are one of the fixed}
00334 \textcolor{comment}{ *   codes and any code with a single symbol which in deflate is coded as one}
00335 \textcolor{comment}{ *   bit instead of zero bits.  See the format notes for fixed() and dynamic().}
00336 \textcolor{comment}{ *}
00337 \textcolor{comment}{ * - Within a given code length, the symbols are kept in ascending order for}
00338 \textcolor{comment}{ *   the code bits definition.}
00339 \textcolor{comment}{ */}
00340 local \textcolor{keywordtype}{int} construct(\textcolor{keyword}{struct} \hyperlink{structhuffman}{huffman} *h, \textcolor{keyword}{const} \textcolor{keywordtype}{short} *length, \textcolor{keywordtype}{int} n)
00341 \{
00342     \textcolor{keywordtype}{int} symbol;         \textcolor{comment}{/* current symbol when stepping through length[] */}
00343     \textcolor{keywordtype}{int} len;            \textcolor{comment}{/* current length when stepping through h->count[] */}
00344     \textcolor{keywordtype}{int} left;           \textcolor{comment}{/* number of possible codes left of current length */}
00345     \textcolor{keywordtype}{short} offs[MAXBITS+1];      \textcolor{comment}{/* offsets in symbol table for each length */}
00346 
00347     \textcolor{comment}{/* count number of codes of each length */}
00348     \textcolor{keywordflow}{for} (len = 0; len <= MAXBITS; len++)
00349         h->count[len] = 0;
00350     for (symbol = 0; symbol < n; symbol++)
00351         (h->count[length[symbol]])++;   \textcolor{comment}{/* assumes lengths are within bounds */}
00352     \textcolor{keywordflow}{if} (h->count[0] == n)               \textcolor{comment}{/* no codes! */}
00353         \textcolor{keywordflow}{return} 0;                       \textcolor{comment}{/* complete, but decode() will fail */}
00354 
00355     \textcolor{comment}{/* check for an over-subscribed or incomplete set of lengths */}
00356     left = 1;                           \textcolor{comment}{/* one possible code of zero length */}
00357     \textcolor{keywordflow}{for} (len = 1; len <= MAXBITS; len++) \{
00358         left <<= 1;                     \textcolor{comment}{/* one more bit, double codes left */}
00359         left -= h->count[len];          \textcolor{comment}{/* deduct count from possible codes */}
00360         \textcolor{keywordflow}{if} (left < 0)
00361             \textcolor{keywordflow}{return} left;                \textcolor{comment}{/* over-subscribed--return negative */}
00362     \}                                   \textcolor{comment}{/* left > 0 means incomplete */}
00363 
00364     \textcolor{comment}{/* generate offsets into symbol table for each length for sorting */}
00365     offs[1] = 0;
00366     \textcolor{keywordflow}{for} (len = 1; len < MAXBITS; len++)
00367         offs[len + 1] = offs[len] + h->count[len];
00368 
00369     \textcolor{comment}{/*}
00370 \textcolor{comment}{     * put symbols in table sorted by length, by symbol order within each}
00371 \textcolor{comment}{     * length}
00372 \textcolor{comment}{     */}
00373     for (symbol = 0; symbol < n; symbol++)
00374         \textcolor{keywordflow}{if} (length[symbol] != 0)
00375             h->symbol[offs[length[symbol]]++] = symbol;
00376 
00377     \textcolor{comment}{/* return zero for complete set, positive for incomplete set */}
00378     \textcolor{keywordflow}{return} left;
00379 \}
00380 
00381 \textcolor{comment}{/*}
00382 \textcolor{comment}{ * Decode literal/length and distance codes until an end-of-block code.}
00383 \textcolor{comment}{ *}
00384 \textcolor{comment}{ * Format notes:}
00385 \textcolor{comment}{ *}
00386 \textcolor{comment}{ * - Compressed data that is after the block type if fixed or after the code}
00387 \textcolor{comment}{ *   description if dynamic is a combination of literals and length/distance}
00388 \textcolor{comment}{ *   pairs terminated by and end-of-block code.  Literals are simply Huffman}
00389 \textcolor{comment}{ *   coded bytes.  A length/distance pair is a coded length followed by a}
00390 \textcolor{comment}{ *   coded distance to represent a string that occurs earlier in the}
00391 \textcolor{comment}{ *   uncompressed data that occurs again at the current location.}
00392 \textcolor{comment}{ *}
00393 \textcolor{comment}{ * - Literals, lengths, and the end-of-block code are combined into a single}
00394 \textcolor{comment}{ *   code of up to 286 symbols.  They are 256 literals (0..255), 29 length}
00395 \textcolor{comment}{ *   symbols (257..285), and the end-of-block symbol (256).}
00396 \textcolor{comment}{ *}
00397 \textcolor{comment}{ * - There are 256 possible lengths (3..258), and so 29 symbols are not enough}
00398 \textcolor{comment}{ *   to represent all of those.  Lengths 3..10 and 258 are in fact represented}
00399 \textcolor{comment}{ *   by just a length symbol.  Lengths 11..257 are represented as a symbol and}
00400 \textcolor{comment}{ *   some number of extra bits that are added as an integer to the base length}
00401 \textcolor{comment}{ *   of the length symbol.  The number of extra bits is determined by the base}
00402 \textcolor{comment}{ *   length symbol.  These are in the static arrays below, lens[] for the base}
00403 \textcolor{comment}{ *   lengths and lext[] for the corresponding number of extra bits.}
00404 \textcolor{comment}{ *}
00405 \textcolor{comment}{ * - The reason that 258 gets its own symbol is that the longest length is used}
00406 \textcolor{comment}{ *   often in highly redundant files.  Note that 258 can also be coded as the}
00407 \textcolor{comment}{ *   base value 227 plus the maximum extra value of 31.  While a good deflate}
00408 \textcolor{comment}{ *   should never do this, it is not an error, and should be decoded properly.}
00409 \textcolor{comment}{ *}
00410 \textcolor{comment}{ * - If a length is decoded, including its extra bits if any, then it is}
00411 \textcolor{comment}{ *   followed a distance code.  There are up to 30 distance symbols.  Again}
00412 \textcolor{comment}{ *   there are many more possible distances (1..32768), so extra bits are added}
00413 \textcolor{comment}{ *   to a base value represented by the symbol.  The distances 1..4 get their}
00414 \textcolor{comment}{ *   own symbol, but the rest require extra bits.  The base distances and}
00415 \textcolor{comment}{ *   corresponding number of extra bits are below in the static arrays dist[]}
00416 \textcolor{comment}{ *   and dext[].}
00417 \textcolor{comment}{ *}
00418 \textcolor{comment}{ * - Literal bytes are simply written to the output.  A length/distance pair is}
00419 \textcolor{comment}{ *   an instruction to copy previously uncompressed bytes to the output.  The}
00420 \textcolor{comment}{ *   copy is from distance bytes back in the output stream, copying for length}
00421 \textcolor{comment}{ *   bytes.}
00422 \textcolor{comment}{ *}
00423 \textcolor{comment}{ * - Distances pointing before the beginning of the output data are not}
00424 \textcolor{comment}{ *   permitted.}
00425 \textcolor{comment}{ *}
00426 \textcolor{comment}{ * - Overlapped copies, where the length is greater than the distance, are}
00427 \textcolor{comment}{ *   allowed and common.  For example, a distance of one and a length of 258}
00428 \textcolor{comment}{ *   simply copies the last byte 258 times.  A distance of four and a length of}
00429 \textcolor{comment}{ *   twelve copies the last four bytes three times.  A simple forward copy}
00430 \textcolor{comment}{ *   ignoring whether the length is greater than the distance or not implements}
00431 \textcolor{comment}{ *   this correctly.  You should not use memcpy() since its behavior is not}
00432 \textcolor{comment}{ *   defined for overlapped arrays.  You should not use memmove() or bcopy()}
00433 \textcolor{comment}{ *   since though their behavior -is- defined for overlapping arrays, it is}
00434 \textcolor{comment}{ *   defined to do the wrong thing in this case.}
00435 \textcolor{comment}{ */}
00436 local \textcolor{keywordtype}{int} codes(\textcolor{keyword}{struct} \hyperlink{structstate}{state} *s,
00437                 \textcolor{keyword}{const} \textcolor{keyword}{struct} \hyperlink{structhuffman}{huffman} *lencode,
00438                 \textcolor{keyword}{const} \textcolor{keyword}{struct} \hyperlink{structhuffman}{huffman} *distcode)
00439 \{
00440     \textcolor{keywordtype}{int} symbol;         \textcolor{comment}{/* decoded symbol */}
00441     \textcolor{keywordtype}{int} len;            \textcolor{comment}{/* length for copy */}
00442     \textcolor{keywordtype}{unsigned} dist;      \textcolor{comment}{/* distance for copy */}
00443     \textcolor{keyword}{static} \textcolor{keyword}{const} \textcolor{keywordtype}{short} lens[29] = \{ \textcolor{comment}{/* Size base for length codes 257..285 */}
00444         3, 4, 5, 6, 7, 8, 9, 10, 11, 13, 15, 17, 19, 23, 27, 31,
00445         35, 43, 51, 59, 67, 83, 99, 115, 131, 163, 195, 227, 258\};
00446     \textcolor{keyword}{static} \textcolor{keyword}{const} \textcolor{keywordtype}{short} lext[29] = \{ \textcolor{comment}{/* Extra bits for length codes 257..285 */}
00447         0, 0, 0, 0, 0, 0, 0, 0, 1, 1, 1, 1, 2, 2, 2, 2,
00448         3, 3, 3, 3, 4, 4, 4, 4, 5, 5, 5, 5, 0\};
00449     \textcolor{keyword}{static} \textcolor{keyword}{const} \textcolor{keywordtype}{short} dists[30] = \{ \textcolor{comment}{/* Offset base for distance codes 0..29 */}
00450         1, 2, 3, 4, 5, 7, 9, 13, 17, 25, 33, 49, 65, 97, 129, 193,
00451         257, 385, 513, 769, 1025, 1537, 2049, 3073, 4097, 6145,
00452         8193, 12289, 16385, 24577\};
00453     \textcolor{keyword}{static} \textcolor{keyword}{const} \textcolor{keywordtype}{short} dext[30] = \{ \textcolor{comment}{/* Extra bits for distance codes 0..29 */}
00454         0, 0, 0, 0, 1, 1, 2, 2, 3, 3, 4, 4, 5, 5, 6, 6,
00455         7, 7, 8, 8, 9, 9, 10, 10, 11, 11,
00456         12, 12, 13, 13\};
00457 
00458     \textcolor{comment}{/* decode literals and length/distance pairs */}
00459     \textcolor{keywordflow}{do} \{
00460         symbol = decode(s, lencode);
00461         \textcolor{keywordflow}{if} (symbol < 0)
00462             \textcolor{keywordflow}{return} symbol;              \textcolor{comment}{/* invalid symbol */}
00463         \textcolor{keywordflow}{if} (symbol < 256) \{             \textcolor{comment}{/* literal: symbol is the byte */}
00464             \textcolor{comment}{/* write out the literal */}
00465             \textcolor{keywordflow}{if} (s->out != NIL) \{
00466                 \textcolor{keywordflow}{if} (s->outcnt == s->outlen)
00467                     \textcolor{keywordflow}{return} 1;
00468                 s->out[s->outcnt] = symbol;
00469             \}
00470             s->outcnt++;
00471         \}
00472         \textcolor{keywordflow}{else} \textcolor{keywordflow}{if} (symbol > 256) \{        \textcolor{comment}{/* length */}
00473             \textcolor{comment}{/* get and compute length */}
00474             symbol -= 257;
00475             \textcolor{keywordflow}{if} (symbol >= 29)
00476                 \textcolor{keywordflow}{return} -10;             \textcolor{comment}{/* invalid fixed code */}
00477             len = lens[symbol] + bits(s, lext[symbol]);
00478 
00479             \textcolor{comment}{/* get and check distance */}
00480             symbol = decode(s, distcode);
00481             \textcolor{keywordflow}{if} (symbol < 0)
00482                 \textcolor{keywordflow}{return} symbol;          \textcolor{comment}{/* invalid symbol */}
00483             dist = dists[symbol] + bits(s, dext[symbol]);
00484 \textcolor{preprocessor}{#ifndef INFLATE\_ALLOW\_INVALID\_DISTANCE\_TOOFAR\_ARRR}
00485             \textcolor{keywordflow}{if} (dist > s->outcnt)
00486                 \textcolor{keywordflow}{return} -11;     \textcolor{comment}{/* distance too far back */}
00487 \textcolor{preprocessor}{#endif}
00488 
00489             \textcolor{comment}{/* copy length bytes from distance bytes back */}
00490             \textcolor{keywordflow}{if} (s->out != NIL) \{
00491                 \textcolor{keywordflow}{if} (s->outcnt + len > s->outlen)
00492                     \textcolor{keywordflow}{return} 1;
00493                 \textcolor{keywordflow}{while} (len--) \{
00494                     s->out[s->outcnt] =
00495 \textcolor{preprocessor}{#ifdef INFLATE\_ALLOW\_INVALID\_DISTANCE\_TOOFAR\_ARRR}
00496                         dist > s->outcnt ?
00497                             0 :
00498 \textcolor{preprocessor}{#endif}
00499                             s->out[s->outcnt - dist];
00500                     s->outcnt++;
00501                 \}
00502             \}
00503             \textcolor{keywordflow}{else}
00504                 s->outcnt += len;
00505         \}
00506     \} \textcolor{keywordflow}{while} (symbol != 256);            \textcolor{comment}{/* end of block symbol */}
00507 
00508     \textcolor{comment}{/* done with a valid fixed or dynamic block */}
00509     \textcolor{keywordflow}{return} 0;
00510 \}
00511 
00512 \textcolor{comment}{/*}
00513 \textcolor{comment}{ * Process a fixed codes block.}
00514 \textcolor{comment}{ *}
00515 \textcolor{comment}{ * Format notes:}
00516 \textcolor{comment}{ *}
00517 \textcolor{comment}{ * - This block type can be useful for compressing small amounts of data for}
00518 \textcolor{comment}{ *   which the size of the code descriptions in a dynamic block exceeds the}
00519 \textcolor{comment}{ *   benefit of custom codes for that block.  For fixed codes, no bits are}
00520 \textcolor{comment}{ *   spent on code descriptions.  Instead the code lengths for literal/length}
00521 \textcolor{comment}{ *   codes and distance codes are fixed.  The specific lengths for each symbol}
00522 \textcolor{comment}{ *   can be seen in the "for" loops below.}
00523 \textcolor{comment}{ *}
00524 \textcolor{comment}{ * - The literal/length code is complete, but has two symbols that are invalid}
00525 \textcolor{comment}{ *   and should result in an error if received.  This cannot be implemented}
00526 \textcolor{comment}{ *   simply as an incomplete code since those two symbols are in the "middle"}
00527 \textcolor{comment}{ *   of the code.  They are eight bits long and the longest literal/length\(\backslash\)}
00528 \textcolor{comment}{ *   code is nine bits.  Therefore the code must be constructed with those}
00529 \textcolor{comment}{ *   symbols, and the invalid symbols must be detected after decoding.}
00530 \textcolor{comment}{ *}
00531 \textcolor{comment}{ * - The fixed distance codes also have two invalid symbols that should result}
00532 \textcolor{comment}{ *   in an error if received.  Since all of the distance codes are the same}
00533 \textcolor{comment}{ *   length, this can be implemented as an incomplete code.  Then the invalid}
00534 \textcolor{comment}{ *   codes are detected while decoding.}
00535 \textcolor{comment}{ */}
00536 local \textcolor{keywordtype}{int} fixed(\textcolor{keyword}{struct} \hyperlink{structstate}{state} *s)
00537 \{
00538     \textcolor{keyword}{static} \textcolor{keywordtype}{int} virgin = 1;
00539     \textcolor{keyword}{static} \textcolor{keywordtype}{short} lencnt[MAXBITS+1], lensym[FIXLCODES];
00540     \textcolor{keyword}{static} \textcolor{keywordtype}{short} distcnt[MAXBITS+1], distsym[MAXDCODES];
00541     \textcolor{keyword}{static} \textcolor{keyword}{struct }\hyperlink{structhuffman}{huffman} lencode, distcode;
00542 
00543     \textcolor{comment}{/* build fixed huffman tables if first call (may not be thread safe) */}
00544     \textcolor{keywordflow}{if} (virgin) \{
00545         \textcolor{keywordtype}{int} symbol;
00546         \textcolor{keywordtype}{short} lengths[FIXLCODES];
00547 
00548         \textcolor{comment}{/* construct lencode and distcode */}
00549         lencode.count = lencnt;
00550         lencode.symbol = lensym;
00551         distcode.count = distcnt;
00552         distcode.symbol = distsym;
00553 
00554         \textcolor{comment}{/* literal/length table */}
00555         \textcolor{keywordflow}{for} (symbol = 0; symbol < 144; symbol++)
00556             lengths[symbol] = 8;
00557         \textcolor{keywordflow}{for} (; symbol < 256; symbol++)
00558             lengths[symbol] = 9;
00559         \textcolor{keywordflow}{for} (; symbol < 280; symbol++)
00560             lengths[symbol] = 7;
00561         \textcolor{keywordflow}{for} (; symbol < FIXLCODES; symbol++)
00562             lengths[symbol] = 8;
00563         construct(&lencode, lengths, FIXLCODES);
00564 
00565         \textcolor{comment}{/* distance table */}
00566         \textcolor{keywordflow}{for} (symbol = 0; symbol < MAXDCODES; symbol++)
00567             lengths[symbol] = 5;
00568         construct(&distcode, lengths, MAXDCODES);
00569 
00570         \textcolor{comment}{/* do this just once */}
00571         virgin = 0;
00572     \}
00573 
00574     \textcolor{comment}{/* decode data until end-of-block code */}
00575     \textcolor{keywordflow}{return} codes(s, &lencode, &distcode);
00576 \}
00577 
00578 \textcolor{comment}{/*}
00579 \textcolor{comment}{ * Process a dynamic codes block.}
00580 \textcolor{comment}{ *}
00581 \textcolor{comment}{ * Format notes:}
00582 \textcolor{comment}{ *}
00583 \textcolor{comment}{ * - A dynamic block starts with a description of the literal/length and}
00584 \textcolor{comment}{ *   distance codes for that block.  New dynamic blocks allow the compressor to}
00585 \textcolor{comment}{ *   rapidly adapt to changing data with new codes optimized for that data.}
00586 \textcolor{comment}{ *}
00587 \textcolor{comment}{ * - The codes used by the deflate format are "canonical", which means that}
00588 \textcolor{comment}{ *   the actual bits of the codes are generated in an unambiguous way simply}
00589 \textcolor{comment}{ *   from the number of bits in each code.  Therefore the code descriptions}
00590 \textcolor{comment}{ *   are simply a list of code lengths for each symbol.}
00591 \textcolor{comment}{ *}
00592 \textcolor{comment}{ * - The code lengths are stored in order for the symbols, so lengths are}
00593 \textcolor{comment}{ *   provided for each of the literal/length symbols, and for each of the}
00594 \textcolor{comment}{ *   distance symbols.}
00595 \textcolor{comment}{ *}
00596 \textcolor{comment}{ * - If a symbol is not used in the block, this is represented by a zero as}
00597 \textcolor{comment}{ *   as the code length.  This does not mean a zero-length code, but rather}
00598 \textcolor{comment}{ *   that no code should be created for this symbol.  There is no way in the}
00599 \textcolor{comment}{ *   deflate format to represent a zero-length code.}
00600 \textcolor{comment}{ *}
00601 \textcolor{comment}{ * - The maximum number of bits in a code is 15, so the possible lengths for}
00602 \textcolor{comment}{ *   any code are 1..15.}
00603 \textcolor{comment}{ *}
00604 \textcolor{comment}{ * - The fact that a length of zero is not permitted for a code has an}
00605 \textcolor{comment}{ *   interesting consequence.  Normally if only one symbol is used for a given}
00606 \textcolor{comment}{ *   code, then in fact that code could be represented with zero bits.  However}
00607 \textcolor{comment}{ *   in deflate, that code has to be at least one bit.  So for example, if}
00608 \textcolor{comment}{ *   only a single distance base symbol appears in a block, then it will be}
00609 \textcolor{comment}{ *   represented by a single code of length one, in particular one 0 bit.  This}
00610 \textcolor{comment}{ *   is an incomplete code, since if a 1 bit is received, it has no meaning,}
00611 \textcolor{comment}{ *   and should result in an error.  So incomplete distance codes of one symbol}
00612 \textcolor{comment}{ *   should be permitted, and the receipt of invalid codes should be handled.}
00613 \textcolor{comment}{ *}
00614 \textcolor{comment}{ * - It is also possible to have a single literal/length code, but that code}
00615 \textcolor{comment}{ *   must be the end-of-block code, since every dynamic block has one.  This}
00616 \textcolor{comment}{ *   is not the most efficient way to create an empty block (an empty fixed}
00617 \textcolor{comment}{ *   block is fewer bits), but it is allowed by the format.  So incomplete}
00618 \textcolor{comment}{ *   literal/length codes of one symbol should also be permitted.}
00619 \textcolor{comment}{ *}
00620 \textcolor{comment}{ * - If there are only literal codes and no lengths, then there are no distance}
00621 \textcolor{comment}{ *   codes.  This is represented by one distance code with zero bits.}
00622 \textcolor{comment}{ *}
00623 \textcolor{comment}{ * - The list of up to 286 length/literal lengths and up to 30 distance lengths}
00624 \textcolor{comment}{ *   are themselves compressed using Huffman codes and run-length encoding.  In}
00625 \textcolor{comment}{ *   the list of code lengths, a 0 symbol means no code, a 1..15 symbol means}
00626 \textcolor{comment}{ *   that length, and the symbols 16, 17, and 18 are run-length instructions.}
00627 \textcolor{comment}{ *   Each of 16, 17, and 18 are follwed by extra bits to define the length of}
00628 \textcolor{comment}{ *   the run.  16 copies the last length 3 to 6 times.  17 represents 3 to 10}
00629 \textcolor{comment}{ *   zero lengths, and 18 represents 11 to 138 zero lengths.  Unused symbols}
00630 \textcolor{comment}{ *   are common, hence the special coding for zero lengths.}
00631 \textcolor{comment}{ *}
00632 \textcolor{comment}{ * - The symbols for 0..18 are Huffman coded, and so that code must be}
00633 \textcolor{comment}{ *   described first.  This is simply a sequence of up to 19 three-bit values}
00634 \textcolor{comment}{ *   representing no code (0) or the code length for that symbol (1..7).}
00635 \textcolor{comment}{ *}
00636 \textcolor{comment}{ * - A dynamic block starts with three fixed-size counts from which is computed}
00637 \textcolor{comment}{ *   the number of literal/length code lengths, the number of distance code}
00638 \textcolor{comment}{ *   lengths, and the number of code length code lengths (ok, you come up with}
00639 \textcolor{comment}{ *   a better name!) in the code descriptions.  For the literal/length and}
00640 \textcolor{comment}{ *   distance codes, lengths after those provided are considered zero, i.e. no}
00641 \textcolor{comment}{ *   code.  The code length code lengths are received in a permuted order (see}
00642 \textcolor{comment}{ *   the order[] array below) to make a short code length code length list more}
00643 \textcolor{comment}{ *   likely.  As it turns out, very short and very long codes are less likely}
00644 \textcolor{comment}{ *   to be seen in a dynamic code description, hence what may appear initially}
00645 \textcolor{comment}{ *   to be a peculiar ordering.}
00646 \textcolor{comment}{ *}
00647 \textcolor{comment}{ * - Given the number of literal/length code lengths (nlen) and distance code}
00648 \textcolor{comment}{ *   lengths (ndist), then they are treated as one long list of nlen + ndist}
00649 \textcolor{comment}{ *   code lengths.  Therefore run-length coding can and often does cross the}
00650 \textcolor{comment}{ *   boundary between the two sets of lengths.}
00651 \textcolor{comment}{ *}
00652 \textcolor{comment}{ * - So to summarize, the code description at the start of a dynamic block is}
00653 \textcolor{comment}{ *   three counts for the number of code lengths for the literal/length codes,}
00654 \textcolor{comment}{ *   the distance codes, and the code length codes.  This is followed by the}
00655 \textcolor{comment}{ *   code length code lengths, three bits each.  This is used to construct the}
00656 \textcolor{comment}{ *   code length code which is used to read the remainder of the lengths.  Then}
00657 \textcolor{comment}{ *   the literal/length code lengths and distance lengths are read as a single}
00658 \textcolor{comment}{ *   set of lengths using the code length codes.  Codes are constructed from}
00659 \textcolor{comment}{ *   the resulting two sets of lengths, and then finally you can start}
00660 \textcolor{comment}{ *   decoding actual compressed data in the block.}
00661 \textcolor{comment}{ *}
00662 \textcolor{comment}{ * - For reference, a "typical" size for the code description in a dynamic}
00663 \textcolor{comment}{ *   block is around 80 bytes.}
00664 \textcolor{comment}{ */}
00665 local \textcolor{keywordtype}{int} dynamic(\textcolor{keyword}{struct} \hyperlink{structstate}{state} *s)
00666 \{
00667     \textcolor{keywordtype}{int} nlen, ndist, ncode;             \textcolor{comment}{/* number of lengths in descriptor */}
00668     \textcolor{keywordtype}{int} index;                          \textcolor{comment}{/* index of lengths[] */}
00669     \textcolor{keywordtype}{int} err;                            \textcolor{comment}{/* construct() return value */}
00670     \textcolor{keywordtype}{short} lengths[MAXCODES];            \textcolor{comment}{/* descriptor code lengths */}
00671     \textcolor{keywordtype}{short} lencnt[MAXBITS+1], lensym[MAXLCODES];         \textcolor{comment}{/* lencode memory */}
00672     \textcolor{keywordtype}{short} distcnt[MAXBITS+1], distsym[MAXDCODES];       \textcolor{comment}{/* distcode memory */}
00673     \textcolor{keyword}{struct }\hyperlink{structhuffman}{huffman} lencode, distcode;   \textcolor{comment}{/* length and distance codes */}
00674     \textcolor{keyword}{static} \textcolor{keyword}{const} \textcolor{keywordtype}{short} order[19] =      \textcolor{comment}{/* permutation of code length codes */}
00675         \{16, 17, 18, 0, 8, 7, 9, 6, 10, 5, 11, 4, 12, 3, 13, 2, 14, 1, 15\};
00676 
00677     \textcolor{comment}{/* construct lencode and distcode */}
00678     lencode.count = lencnt;
00679     lencode.symbol = lensym;
00680     distcode.count = distcnt;
00681     distcode.symbol = distsym;
00682 
00683     \textcolor{comment}{/* get number of lengths in each table, check lengths */}
00684     nlen = bits(s, 5) + 257;
00685     ndist = bits(s, 5) + 1;
00686     ncode = bits(s, 4) + 4;
00687     \textcolor{keywordflow}{if} (nlen > MAXLCODES || ndist > MAXDCODES)
00688         \textcolor{keywordflow}{return} -3;                      \textcolor{comment}{/* bad counts */}
00689 
00690     \textcolor{comment}{/* read code length code lengths (really), missing lengths are zero */}
00691     \textcolor{keywordflow}{for} (index = 0; index < ncode; index++)
00692         lengths[order[index]] = bits(s, 3);
00693     \textcolor{keywordflow}{for} (; index < 19; index++)
00694         lengths[order[index]] = 0;
00695 
00696     \textcolor{comment}{/* build huffman table for code lengths codes (use lencode temporarily) */}
00697     err = construct(&lencode, lengths, 19);
00698     \textcolor{keywordflow}{if} (err != 0)               \textcolor{comment}{/* require complete code set here */}
00699         \textcolor{keywordflow}{return} -4;
00700 
00701     \textcolor{comment}{/* read length/literal and distance code length tables */}
00702     index = 0;
00703     \textcolor{keywordflow}{while} (index < nlen + ndist) \{
00704         \textcolor{keywordtype}{int} symbol;             \textcolor{comment}{/* decoded value */}
00705         \textcolor{keywordtype}{int} len;                \textcolor{comment}{/* last length to repeat */}
00706 
00707         symbol = decode(s, &lencode);
00708         \textcolor{keywordflow}{if} (symbol < 0)
00709             \textcolor{keywordflow}{return} symbol;          \textcolor{comment}{/* invalid symbol */}
00710         \textcolor{keywordflow}{if} (symbol < 16)                \textcolor{comment}{/* length in 0..15 */}
00711             lengths[index++] = symbol;
00712         \textcolor{keywordflow}{else} \{                          \textcolor{comment}{/* repeat instruction */}
00713             len = 0;                    \textcolor{comment}{/* assume repeating zeros */}
00714             \textcolor{keywordflow}{if} (symbol == 16) \{         \textcolor{comment}{/* repeat last length 3..6 times */}
00715                 \textcolor{keywordflow}{if} (index == 0)
00716                     \textcolor{keywordflow}{return} -5;          \textcolor{comment}{/* no last length! */}
00717                 len = lengths[index - 1];       \textcolor{comment}{/* last length */}
00718                 symbol = 3 + bits(s, 2);
00719             \}
00720             \textcolor{keywordflow}{else} \textcolor{keywordflow}{if} (symbol == 17)      \textcolor{comment}{/* repeat zero 3..10 times */}
00721                 symbol = 3 + bits(s, 3);
00722             \textcolor{keywordflow}{else}                        \textcolor{comment}{/* == 18, repeat zero 11..138 times */}
00723                 symbol = 11 + bits(s, 7);
00724             \textcolor{keywordflow}{if} (index + symbol > nlen + ndist)
00725                 \textcolor{keywordflow}{return} -6;              \textcolor{comment}{/* too many lengths! */}
00726             \textcolor{keywordflow}{while} (symbol--)            \textcolor{comment}{/* repeat last or zero symbol times */}
00727                 lengths[index++] = len;
00728         \}
00729     \}
00730 
00731     \textcolor{comment}{/* check for end-of-block code -- there better be one! */}
00732     \textcolor{keywordflow}{if} (lengths[256] == 0)
00733         \textcolor{keywordflow}{return} -9;
00734 
00735     \textcolor{comment}{/* build huffman table for literal/length codes */}
00736     err = construct(&lencode, lengths, nlen);
00737     \textcolor{keywordflow}{if} (err && (err < 0 || nlen != lencode.count[0] + lencode.count[1]))
00738         \textcolor{keywordflow}{return} -7;      \textcolor{comment}{/* incomplete code ok only for single length 1 code */}
00739 
00740     \textcolor{comment}{/* build huffman table for distance codes */}
00741     err = construct(&distcode, lengths + nlen, ndist);
00742     \textcolor{keywordflow}{if} (err && (err < 0 || ndist != distcode.count[0] + distcode.count[1]))
00743         \textcolor{keywordflow}{return} -8;      \textcolor{comment}{/* incomplete code ok only for single length 1 code */}
00744 
00745     \textcolor{comment}{/* decode data until end-of-block code */}
00746     \textcolor{keywordflow}{return} codes(s, &lencode, &distcode);
00747 \}
00748 
00749 \textcolor{comment}{/*}
00750 \textcolor{comment}{ * Inflate source to dest.  On return, destlen and sourcelen are updated to the}
00751 \textcolor{comment}{ * size of the uncompressed data and the size of the deflate data respectively.}
00752 \textcolor{comment}{ * On success, the return value of puff() is zero.  If there is an error in the}
00753 \textcolor{comment}{ * source data, i.e. it is not in the deflate format, then a negative value is}
00754 \textcolor{comment}{ * returned.  If there is not enough input available or there is not enough}
00755 \textcolor{comment}{ * output space, then a positive error is returned.  In that case, destlen and}
00756 \textcolor{comment}{ * sourcelen are not updated to facilitate retrying from the beginning with the}
00757 \textcolor{comment}{ * provision of more input data or more output space.  In the case of invalid}
00758 \textcolor{comment}{ * inflate data (a negative error), the dest and source pointers are updated to}
00759 \textcolor{comment}{ * facilitate the debugging of deflators.}
00760 \textcolor{comment}{ *}
00761 \textcolor{comment}{ * puff() also has a mode to determine the size of the uncompressed output with}
00762 \textcolor{comment}{ * no output written.  For this dest must be (unsigned char *)0.  In this case,}
00763 \textcolor{comment}{ * the input value of *destlen is ignored, and on return *destlen is set to the}
00764 \textcolor{comment}{ * size of the uncompressed output.}
00765 \textcolor{comment}{ *}
00766 \textcolor{comment}{ * The return codes are:}
00767 \textcolor{comment}{ *}
00768 \textcolor{comment}{ *   2:  available inflate data did not terminate}
00769 \textcolor{comment}{ *   1:  output space exhausted before completing inflate}
00770 \textcolor{comment}{ *   0:  successful inflate}
00771 \textcolor{comment}{ *  -1:  invalid block type (type == 3)}
00772 \textcolor{comment}{ *  -2:  stored block length did not match one's complement}
00773 \textcolor{comment}{ *  -3:  dynamic block code description: too many length or distance codes}
00774 \textcolor{comment}{ *  -4:  dynamic block code description: code lengths codes incomplete}
00775 \textcolor{comment}{ *  -5:  dynamic block code description: repeat lengths with no first length}
00776 \textcolor{comment}{ *  -6:  dynamic block code description: repeat more than specified lengths}
00777 \textcolor{comment}{ *  -7:  dynamic block code description: invalid literal/length code lengths}
00778 \textcolor{comment}{ *  -8:  dynamic block code description: invalid distance code lengths}
00779 \textcolor{comment}{ *  -9:  dynamic block code description: missing end-of-block code}
00780 \textcolor{comment}{ * -10:  invalid literal/length or distance code in fixed or dynamic block}
00781 \textcolor{comment}{ * -11:  distance is too far back in fixed or dynamic block}
00782 \textcolor{comment}{ *}
00783 \textcolor{comment}{ * Format notes:}
00784 \textcolor{comment}{ *}
00785 \textcolor{comment}{ * - Three bits are read for each block to determine the kind of block and}
00786 \textcolor{comment}{ *   whether or not it is the last block.  Then the block is decoded and the}
00787 \textcolor{comment}{ *   process repeated if it was not the last block.}
00788 \textcolor{comment}{ *}
00789 \textcolor{comment}{ * - The leftover bits in the last byte of the deflate data after the last}
00790 \textcolor{comment}{ *   block (if it was a fixed or dynamic block) are undefined and have no}
00791 \textcolor{comment}{ *   expected values to check.}
00792 \textcolor{comment}{ */}
00793 \textcolor{keywordtype}{int} puff(\textcolor{keywordtype}{unsigned} \textcolor{keywordtype}{char} *dest,           \textcolor{comment}{/* pointer to destination pointer */}
00794          \textcolor{keywordtype}{unsigned} \textcolor{keywordtype}{long} *destlen,        \textcolor{comment}{/* amount of output space */}
00795          \textcolor{keyword}{const} \textcolor{keywordtype}{unsigned} \textcolor{keywordtype}{char} *source,   \textcolor{comment}{/* pointer to source data pointer */}
00796          \textcolor{keywordtype}{unsigned} \textcolor{keywordtype}{long} *sourcelen)      \textcolor{comment}{/* amount of input available */}
00797 \{
00798     \textcolor{keyword}{struct }\hyperlink{structstate}{state} s;             \textcolor{comment}{/* input/output state */}
00799     \textcolor{keywordtype}{int} last, type;             \textcolor{comment}{/* block information */}
00800     \textcolor{keywordtype}{int} err;                    \textcolor{comment}{/* return value */}
00801 
00802     \textcolor{comment}{/* initialize output state */}
00803     s.out = dest;
00804     s.outlen = *destlen;                \textcolor{comment}{/* ignored if dest is NIL */}
00805     s.outcnt = 0;
00806 
00807     \textcolor{comment}{/* initialize input state */}
00808     s.in = source;
00809     s.inlen = *sourcelen;
00810     s.incnt = 0;
00811     s.bitbuf = 0;
00812     s.bitcnt = 0;
00813 
00814     \textcolor{comment}{/* return if bits() or decode() tries to read past available input */}
00815     \textcolor{keywordflow}{if} (setjmp(s.env) != 0)             \textcolor{comment}{/* if came back here via longjmp() */}
00816         err = 2;                        \textcolor{comment}{/* then skip do-loop, return error */}
00817     \textcolor{keywordflow}{else} \{
00818         \textcolor{comment}{/* process blocks until last block or error */}
00819         \textcolor{keywordflow}{do} \{
00820             last = bits(&s, 1);         \textcolor{comment}{/* one if last block */}
00821             type = bits(&s, 2);         \textcolor{comment}{/* block type 0..3 */}
00822             err = type == 0 ?
00823                     stored(&s) :
00824                     (type == 1 ?
00825                         fixed(&s) :
00826                         (type == 2 ?
00827                             dynamic(&s) :
00828                             -1));       \textcolor{comment}{/* type == 3, invalid */}
00829             \textcolor{keywordflow}{if} (err != 0)
00830                 \textcolor{keywordflow}{break};                  \textcolor{comment}{/* return with error */}
00831         \} \textcolor{keywordflow}{while} (!last);
00832     \}
00833 
00834     \textcolor{comment}{/* update the lengths and return */}
00835     \textcolor{keywordflow}{if} (err <= 0) \{
00836         *destlen = s.outcnt;
00837         *sourcelen = s.incnt;
00838     \}
00839     \textcolor{keywordflow}{return} err;
00840 \}
\end{DoxyCode}
