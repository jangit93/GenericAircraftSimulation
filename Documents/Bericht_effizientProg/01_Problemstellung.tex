\chapter{Einleitung}
\section{Ausgangslage}
\label{sec:ausgangslage}
Der Prozess des Flugzeugentwurfs erstreckt sich über mehrere Iterationen in denen das Flugzeug immer detaillierter ausgearbeitet wird. Um das Leistungsvermögen früh im Entwicklungsprozess zu beurteilen, können numerische Simulationen genutzt werden. Zum Einen werden mathematische Modelle benötigt, um das physikalische Verhalten von spezifischen Domänen des Flugzeuges abzubilden. Zum Anderen werden Parameter benötigt, um besagte Modelle zu beschreiben. Viele dieser Parameter sind erst im Laufe des Entwurfsprozesses verfügbar. Um das Flugverhalten dennoch abzubilden, wird eine Simulation mit verschiedenen Ausbaustufen und Fehlermodellen benötigt.\\\\
In der Arbeit nach  \cite{Olucak.15.02.2017} wurde eine Matlab basierendes Programm entwickelt, mit dessen Hilfe Simulationsmodelle für Flugzeuge modelliert werden können. Die fertigen Modelle werden sowohl in Text-Files, als auch komprimierten mat-files gespeichert. Um die Modelle zu validieren wurde zudem eine Simulation mit 6 Freiheitsgraden implementiert. Das Programm Aircraft Designer kann jederzeit um weitere Methoden ergänzt werden, um so den Entwurfsprozess fortzusetzen. Mittels dem semi-empirischen Tool DataCompendium (DATCOM) wird die Aerodynamik des Flugzeuges für zuvor definierte Flugzustände berechnet. In einem automatisierten Prozess wird ein Vorgaberegler für die Flugzustände entworfen. Ist der Entwurfsprozess abgeschlossen kann durch die zuvor erwähnte Simulation die Güte der Modelle beurteilt werden und gegebenenfalls durch Anpassung der Parameter modifiziert werden. Die Laufzeit von Matlab ist mitunter eine sehr kritische Größe. Zudem ist eine objektorientierte Programmierung nur bedingt möglich, was den Ausbau zu einer generischen Simulation erschwert.
\newpage
\section{Zielsetzung}
Innerhalb dieser Ausarbeitung soll eine generische Flugzeugsimulation in der Hochsprache C++ implementiert werden. Generisch bedeutet in diesem Zusammenhang, dass die Simulationsumgebung nicht auf einen spezifischen Entwurf bezogen ist, sondern verschiedene Entwürfe mit dem in \cite{Olucak.15.02.2017} entstandenen Aircraft Designer simulierbar sind. Somit sollen die Methoden sehr allgemein gehalten werden.  Jede Ausbaustufe des Entwurfsprozesses soll simulierbar sein, ohne dass dabei der Code verändert wird. Über Input-Files sollen verschiedene Domänen-Modelle ausgewählt und die Simulation gesteuert werden können.  Ein modularer Aufbau soll zudem eine einfache Erweiterung des Simulation gewährleisten. Zudem soll die Performance der Simulation hinsichtlich Rechenzeit gegenüber Matlab verbesert und im Nachgang weiter optimiert werden. Um das Simulations-Framework zu validieren, wird die in \cite{Olucak.15.02.2017} implementierte Simulation in C++ übersetzt. Der gesamte Entwicklungsprozess soll sich dabei auf die in \cite{Kessler.Sommersemester2017} und \cite{Kessler.Wintersemester201718} vorgestellten Methoden und Anregungen stützen.
 
