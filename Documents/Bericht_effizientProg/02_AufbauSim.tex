\chapter{Aufbau der Simulation}
\section{Grundidee des Simulation-Frameworks}
Für die Umsetzung einer generischen Simulation werden die einzelnen Domänen des Flugzeuges in Modulen implementiert. Innerhalb der Module werden die verschiedenen Ausbaustufen oder mathematischen Modelle der Domänen implementiert. Um nun einen generischen Aufruf zu gewährleisten wird eine Klasse gleichnamig zu dem jeweiligen Modul implementiert, in der zum Einen die Auswahl für das jeweilige Modell getroffen wird, zum Anderen der eigentlichen Funktionsaufruf. Die Auswahl wird über ein Input File durch Veränderung eines Parameters getroffen. \\
Um nun einen generischen Aufruf zu gewährleisten wird in jedem Modul einen Basis-Klasse definiert. Neue Modelle können von der Basis-Klasse selbst oder von einem Kind der Basis-Klasse erben. Wird nun über das Input-File ein Modell ausgewählt, wird über eine switch-Funktion die Basis-Klasse initialisiert, wobei der Zeiger auf das jeweilige Modell gerichtet wird. \\
\begin{figure}[h]
\centering\includegraphics[width=0.3\linewidth]{UML_modul.PNG}	
\caption{UML Diagramm des grundlegenden Funktionsaufruf}
\label{fig:UML_modul}
\end{figure}

Somit kann das Modul jederzeit um neue Modelle erweitert werden, ohne den Funktionsaufruf selbst anzupassen.  
\section{Ablauf der Simulation}
