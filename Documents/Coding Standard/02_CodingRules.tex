\chapter{Allgemein}
\section{Projektaufbau}
\begin{table}[h]
	\begin{tabular}{lp{10cm}}
		\textbf{Standard 1} & Das Projekt soll in der Entwicklungsumgebung in Unterordner aufgeteilt werden. \\\\
		\textbf{Standard 2} & Die Klassen sollten inhaltlich in gleichen Unterordnern zusammengefasst werden. \\\\
		\textbf{Standard 3} & Header-Files sollen die Endung.hpp und Source-Files die Endung .cpp erhalten
	\end{tabular}
\end{table}

\section{Kommentare}
\begin{table}[h]
	\begin{tabular}{lp{10cm}}
	    	\textbf{Standard 4} & Jedes File erhält einen Top-Kommentar mit den wichtigsten Informationen über das jeweilige File\\\\
	    	\textbf{Standard 5} & Kommentare sollten richtig und genau sein und auch bleiben\\\\
	    	\textbf{Standard 6} & Kommentare und Quellcode sollten klar voneinander getrennt sein.\\\\
	    	\textbf{Standard 7} & Beschreibung von Funktionen im Header File sollten ausführlich genug sein, dass der Nutzer rein durch das lesen des Header-Files den Aufbau der Klasse versteht.\\\\
	        \textbf{Standard 8} & Hinter Variablen Deklarationen sollte, wenn es sich um eine physikalische Größe handelt, die SI-Einheit als Kommentar stehen.
	\end{tabular}
\end{table}

\chapter{Code Layout}
\section{Formatierung}
\begin{table}[h]
	\begin{tabular}{lp{10cm}}
      \textbf{Standard 9}       & Code/Funktionen sollte(n) in Blöcke eingeteilt werden.\\\\
      \textbf{Standard 10}      & Jede(r) Block/Funktion sollte einen eigenen Kommentar erhalten.\\\\
      \textbf{Standard 11}     & Jede Variablendeklaration sollte in einer neuen Zeile starten.\\\\
      \textbf{Standard 12} & Der Code sollte in der Vertikalen angeglichen werden\\\\ 
      \textbf{Standard 13} & Zwischen Kommentar und Funktion sollte eine Zeile Platz gelassen werden. \\\\
      \textbf{Standard 14} & Wenn möglich sollten Klammern genutzt werden, um Zusammenhänge im Code besser zu erkennen.\\\\
      \textbf{Standard 15} & Ein File sollte 300 Zeilen exklusive Kommentare nicht überschreiten.\\\\
      \textbf{Standard 16} & An Funktion übergebene Variablen sollten untereinander geschrieben werden.
	\end{tabular}
\end{table}
\section{Namensgebung}
\begin{table}[h]
	\begin{tabular}{lp{10cm}}
     \textbf{Standard 17} & Namen sollte klar und einzigartig sein.\\\\
   \textbf{Standard 18}   & Namen sollten aufschluss über deren nutzen geben. 
	\end{tabular}
\end{table}\newpage
\section{Deklaration von Variablen}
\begin{table}[h]
	\begin{tabular}{lp{10cm}}
     \textbf{Standard 19} & Variablen erhalten entsprechend Tabelle ... Präfixe \\\\
\textbf{Standard 20} & Die Deklaration von Variablen sollte im Header-File stattfinden.\\\\  
	\end{tabular}
\end{table}
\chapter{Aufbau von Klassen und Funktionen}
\section{Konstruktor und Destruktor}
\section{Initialisierung}
\section{Methoden}
\chapter{Beispiel für eine Klasse}


