%%%%%%%%%%%%%%%%%%%%%%%%%%%%%%%%%%%%%%%%%%%%%%%%%%%%%%%%%%%%%%%%%%%%%%%%%%%
% Formelsammlung Flugzeugentwurf I
%%%%%%%%%%%%%%%%%%%%%%%%%%%%%%%%%%%%%%%%%%%%%%%%%%%%%%%%%%%%%%%%%%%%%%%%%%%

%/////////////////////////////////////////////////////////////////////////////
%
% Definition des Dokuments
%
%/////////////////////////////////////////////////////////////////////////////
\documentclass[
11pt,				% Schriftgröße
a4paper,			% Seitengröße
DIV12,		     	% Satzspiegel
liststotoc,			% Inhalsverzeichnis in Inhaltsverzeichnis
bibliography=totoc, % Quellenverzeichnis in Inhaltsverzeichnis
listof=entryprefix, % Abbildungsverzeichnis in Inhaltsverzeichnis
listot=entryprefix, % Tabellenverzeichnis in Inhaltsverzeichnis
appendixprefix=true,
pointlessnumbers,
fleqn,
%	DIV=calc,		% Satzspiegel autom. berechnen, widerspricht sich mit vorherigem
%	twocolumn,		% Zweispaltiger Text
oneside,		% Doppelseitig, während dem Arbeiten ausschalten, dann besser zum arbeiten, für digitale Variante auch ausschalten, 
%	BROC=10mm		% Bindeausgleich, Achtung links oder rechts anpassen
]{scrbook}

% Zeilenabstand
\usepackage{setspace}

% Absatzeinrückung und -abstand
\parindent 	= 0pt
\parskip 	= 6pt

% Dateicodierung
\usepackage[utf8]{inputenc} 
%\usepackage[latin1]{inputenc}

% Spracheinstellung
\usepackage[ngerman]{babel}
\usepackage[ngerman]{translator}

%Besondere Trennungen
\hyphenation{De-zi-mal-tren-nung}

% Zeichencodierung
\usepackage[T1]{fontenc}
\usepackage{titlesec}

% Schriften einstellen
\usepackage{lmodern}		%Vektorschrift, keine Pixel
\usepackage{microtype}		%Verbessert microtypographie, Verbessert auch warnings 	(overfull boxes)
\renewcommand{\familydefault}{\sfdefault} 	% {1}Standard Schrift, {2}entfernt Serifen, 													Schrift dann ähnlich Calibri
\usepackage{sansmath}		% Mathemodus ohne Serifen
%\sansmath					% Mathemodus aktivieren

\setcounter{secnumdepth}{3}
\setcounter{tocdepth}{3}

% Bilder einbinden
\usepackage{graphicx}
\usepackage{subfigure} 
\graphicspath{{Bilder/}}  % können mehrere Pfade eingebunden werden, von aktuellen Dokument ausgehend


% Dynamisch verlinktes Dokument
\usepackage[%pdfborder={0 0 0 }, 
colorlinks = true,
linkcolor = blue,
pdftitle={Titel der Abschlussarbeit},
pdfsubject={},
pdfauthor={Euer Name},
pdfkeywords={},	
]{hyperref}		%[]Art und Weise für die Anzeige

%--------------Große Tabellen---------------------------------------
\usepackage{longtable}
%------------------------Mathe----------------------------
\usepackage{amssymb}
\usepackage{nicefrac}
\usepackage{amsfonts}
\usepackage{amsmath}
%-----------------------------------------------------------


%------------- Definition der Titelseite ---------------------
\title{Effizient Programmieren I $\&$ II}
\subtitle{
	\vspace{0.5cm}					% fügt Vertikalen Abstand ein
	\Large 							% skaliert die Schriftgröße
Coding Standard für das Programmierprojekt Generische Flugzeugsimulation}

%--------------- Kopf und Fusszeile --------------
\usepackage{scrpage2}
\pagestyle{scrheadings}
\clearscrheadfoot
\ofoot{\pagemark}

\automark[section]{chapter}
\ihead{\headmark}

\setheadsepline{0.2pt}


\begin{document}
\maketitle
\tableofcontents

\chapter{Vorwort}
Im Zuge der Vorlesungsreihe Effizient Programmieren I $\&$ II soll innerhalb einer Semesterarbeit ein vollständiges Programmierprojekt erstellt werden. Ein Teil dieser Ausarbeitung stellt dieses Dokument, welches Regeln  für die Programm Erstellung und Formatierung des Codes bereitstellt. \\
Hierbei liegt der Schwerpunkt am Code Layout und weniger am an der Umsetzung der Programmierung. Ziel ist es einen einheitlich gut leserlichen und aufgebauten Code zu erhalten.


\chapter{Allgemein}
\section{Projektaufbau}
\begin{table}[h]
	\begin{tabular}{lp{10cm}}
		\textbf{Standard 1} & Das Projekt soll in der Entwicklungsumgebung in Unterordner aufgeteilt werden. \\\\
		\textbf{Standard 2} & Die Klassen sollten inhaltlich in gleichen Unterordnern zusammengefasst werden. \\\\
		\textbf{Standard 3} & Header-Files sollen die Endung.hpp und Source-Files die Endung .cpp erhalten
	\end{tabular}
\end{table}

\section{Kommentare}
\begin{table}[h]
	\begin{tabular}{lp{10cm}}
	    	\textbf{Standard 4} & Jedes File erhält einen Top-Kommentar mit den wichtigsten Informationen über das jeweilige File\\\\
	    	\textbf{Standard 5} & Kommentare sollten richtig und genau sein und auch bleiben\\\\
	    	\textbf{Standard 6} & Kommentare und Quellcode sollten klar voneinander getrennt sein.\\\\
	    	\textbf{Standard 7} & Beschreibung von Funktionen im Header File sollten ausführlich genug sein, dass der Nutzer rein durch das lesen des Header-Files den Aufbau der Klasse versteht.\\\\
	        \textbf{Standard 8} & Hinter Variablen Deklarationen sollte, wenn es sich um eine physikalische Größe handelt, die SI-Einheit als Kommentar stehen.
	\end{tabular}
\end{table}

\chapter{Code Layout}
\section{Formatierung}
\begin{table}[h]
	\begin{tabular}{lp{10cm}}
      \textbf{Standard 9}       & Code/Funktionen sollte(n) in Blöcke eingeteilt werden.\\\\
      \textbf{Standard 10}      & Jede(r) Block/Funktion sollte einen eigenen Kommentar erhalten.\\\\
      \textbf{Standard 11}     & Jede Variablendeklaration sollte in einer neuen Zeile starten.\\\\
      \textbf{Standard 12} & Der Code sollte in der Vertikalen angeglichen werden\\\\ 
      \textbf{Standard 13} & Zwischen Kommentar und Funktion sollte eine Zeile Platz gelassen werden. \\\\
      \textbf{Standard 14} & Wenn möglich sollten Klammern genutzt werden, um Zusammenhänge im Code besser zu erkennen.\\\\
      \textbf{Standard 15} & Ein File sollte 300 Zeilen exklusive Kommentare nicht überschreiten.\\\\
      \textbf{Standard 16} & An Funktion übergebene Variablen sollten untereinander geschrieben werden.
	\end{tabular}
\end{table}
\section{Namensgebung}
\begin{table}[h]
	\begin{tabular}{lp{10cm}}
     \textbf{Standard 17} & Namen sollte klar und einzigartig sein.\\\\
   \textbf{Standard 18}   & Namen sollten aufschluss über deren nutzen geben. 
	\end{tabular}
\end{table}\newpage
\section{Deklaration von Variablen}
\begin{table}[h]
	\begin{tabular}{lp{10cm}}
     \textbf{Standard 19} & Variablen erhalten entsprechend Tabelle ... Präfixe \\\\
\textbf{Standard 20} & Die Deklaration von Variablen sollte im Header-File stattfinden.\\\\  
	\end{tabular}
\end{table}
\chapter{Aufbau von Klassen und Funktionen}
\section{Konstruktor und Destruktor}
\section{Initialisierung}
\section{Methoden}
\chapter{Beispiel für eine Klasse}




\end{document}

